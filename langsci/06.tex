\chapter{Zusammenfassung und Ausblick}
\label{chapter:schluss}
Anhand der detaillierten Untersuchung der MP-Kombinationen aus \textit{ja} und \textit{doch}, \textit{halt} und \textit{eben} sowie \textit{doch} und \textit{auch} mache ich in dieser Arbeit einen Vorschlag, wie sich Reihungsbeschränkungen von MPn in Kombinationen unter Bezug auf ihre Interpretation ableiten lassen.

Ich führe die Idee aus, dass die Abfolgen von MPn in Kombinationen für Diskurs\-verläufe unabhängig gültige Prinzipien widerspiegeln. Es liegt folglich ein Form-Funktionszusammenhang vor, bei dem erstere durch die MP-Abfolge und letztere durch den Diskursbeitrag der Partikeln realisiert wird. Diese Relation fasse ich als Form von diskursstruktureller Ikonizität \is{diskursstrukturelle Ikonizität} auf. In allen drei Fällen zeige ich zudem auf, dass Grund zur Annahme besteht, dass es nicht eine einzige grammatische Abfolge gibt und deren Ableitung so beschaffen sein sollte, dass sie die Umkehr der MP-Reihung kategorisch ausschließt. Im Rahmen der mir derzeitig vorliegenden empirischen Erkenntnisse argumentiere ich, dass man es mit einem Markiertheitsphänomen \is{Markiertheit} zu tun hat. Eine Erklärung der unmarkierten Ordnung sollte deshalb die markierte Abfolge erlauben, wenn sich die Konstellationen im Vergleich zur unmarkierten Abfolge entsprechend der für die unmarkierte Ordnung verantwortlich gemachten Faktoren verändern. Die im Folgenden kurz zusammengefassten Beobachtungen und Erkenntnisse sind die entscheidenden Schritte dieser Argumentation.\\

\noindent
Kapitel~\ref{chapter:hintergrund} dient der Einführung von Hintergrundkonzepten und der Einordnung einiger eigener Annahmen. 

Abschnitt~\ref{sec:forschung} zeigt hier auf, dass aus nahezu allen systematisch-linguistischen Per\-spektiven (Phonologie, Syntax, Semantik, Pragmatik, Informationsstruktur) Ansätze zu den Beschränkungen über die Reihung in MP-Kombinationen vorliegen, sowie, dass auch diachrone \is{Diachronie} Aspekte für Erklärungen zunutze gemacht worden sind. Im Mittelpunkt stehen hier weniger spezielle Kombinationen als die Konzep\-te und Ideen, die für Erklärungen der Reihungsbeschränkungen herangezogen worden sind. 

Abschnitt~\ref{sec:transparenz} führt die Möglichkeiten der Interpretation von MP-Kombinationen aus. Geht man von der Möglichkeit einer transparenten Bedeutungszuschreibung aus (die gegenteilige Ansicht wird hier auch angeführt), stellt sich die Frage nach den Skopusverläufen \is{Skopus} der Partikeln: Nimmt die eine MP Skopus über die andere oder beziehen sich beide gleichermaßen auf die in der Äußerung enthaltene Proposition? Es zeigt sich hier auch, dass Ansätze, die von einer transparenten Bedeutung ausgehen, wenn nicht explizit, so doch implizit Annahmen vertreten. 

Abschnitt~\ref{sec:diskursmodell} führt das Diskursmodell nach \citet{Farkas2010} ein, vor dessen Hintergrund ich in späteren Kapiteln den diskursstrukturellen Beitrag der Partikeln beschreibe. Insbesondere mache ich mir hieraus später zunutze, dass in diesem Modell Sprecher- und Hörerannahmen getrennt (und unterschieden vom cg) registriert und die zur Diskussion stehenden Themen erfasst werden können. Während der Effekt von MP-losen Äußerungen auf den Diskurskontext stets vorwärts gerichtet ist, schlage ich für MPn schließlich die gegenteilige Betrachtungsrichtung im Diskurs ein und nehme an, dass sie Anforderungen an den vorweggehenden Kontext stellen. Diese Modellierung entwickle ich unter Bezug auf die Konzeption des Beitrags von MPn nach \citet{Diewald2006, Diewald2007} bzw. \citet{Diewald1998}, die vertritt/vertreten, dass MPn (simulierte) reaktive Züge markieren. Diese Auffassung wird in Abschnitt~\ref{sec:zugang} vorgestellt.

Da ich für einen Form-Funktionszusammenhang zwischen der MP-Abfolge und dem diskursstrukturellen Beitrag der beteiligten MPn argumentiere, führt Abschnitt~\ref{sec:ikonizität} einige Aspekte zum Konzept der Ikonizität \is{Ikonizität} ein. Der relevante Punkt ist hier, aufzuzeigen, dass es sich bei der von mir angenommenen Art von ikoni\-schem Zusammenhang mit \textit{ikonischer Motivierung} \is{ikonische Motivierung} (als Fall \textit{struktureller diagrammatischer Ikonizität}) \is{strukturelle diagrammatische Ikonizität} (vgl. \citealt{Haiman1980}) um einen etablierten Typ von Ikonizität handelt, der meines Wissens bisher allerdings nicht auf diskursstrukturelle Verhältnisse bezogen worden ist.\\

\noindent
In Kapitel~\ref{chapter:jud} untersuche ich Kombinationen aus \textit{ja} und \textit{doch}. 

Abschnitt~\ref{sec:distributionjd} dient dem Zweck, selbständige und unselbständige assertive Kontexte als die relevante Domäne des kombinierten Vorkommens auszumachen. 

Abschnitt~\ref{sec:abfolgejd} stellt bestehende Ansätze zu dieser konkreten Kombination vor, mit dem Ziel, aufzuzeigen, dass in Analysen stets angelegt ist, eine grammatische Abfolge abzuleiten und die umgekehrte Abfolge als ungrammatisch und non-existent herauszufiltern. 

Abschnitt~\ref{sec:distributiondj} schlägt hier die gegenteilige Argumentation ein, indem drei Satzkon\-texte eingeführt werden, in denen die Anordnung \textit{doch ja} zu belegen ist. Hierbei handelt es sich um (lexikalisch als solche ausgezeichnete) \is{Bewertung} Bewertungen, epistemische Modalisierungen \is{epistemische Modalisierung} und \is{modaler Kausalsatz} modale (d.h. epistemisch oder illokutionär interpretierte) Kausalsätze. 

Abschnitt~\ref{sec:interpretationjd} führt meine Modellierung des Beitrags von \textit{doch} und \textit{ja} ein und argumentiert auf der Basis der Interpretation von authentischen Belegen dafür, dass sich beide Partikeln gleichermaßen auf dieselbe Proposition beziehen. Konkret fordert eine \textit{ja doch}-Assertion einen Kontextzustand, in dem dem Gesprächspartner ein Bekenntnis zu p zugeschrieben werden kann (\textit{ja}) und die in der MP-Äußerung enthaltene Proposition bereits zur Diskussion steht (\textit{doch}). 

Abschnitt~\ref{sec:unmarkiert} präsentiert meine Erklärung der unmarkierten Abfolge. Kommunikation wird \citet{Farkas2010} zufolge generell durch zwei Züge angetrieben: Diskursteilnehmer beabsichtigen, das Wissen im cg anzureichern sowie einen (sogenannten \textit{stabilen}) Kontextzustand zu erreichen, in dem alle Themen gelöst sind. Da \textit{ja} meiner Modellierung nach stets einen stabilen Kontextzustand herstellt (p ist/wird cg) und \textit{doch} anders die Offenheit von p voraussetzt (und auch nicht beseitigen kann), motiviere ich die unmarkierte Abfolge \textit{ja doch} dadurch, dass sie diesem obersten Ziel von Kommunikation direkter nachkommt als \textit{doch ja}. Die Partikel, die cg herstellen kann, wird frühestmöglich zur Anwendung gebracht, um den gewünschten diskursiven Zustand zu erreichen. Da die Partikeln den gleichen Skopus \is{Skopus} nehmen, leistet die Reihung \textit{doch ja} grundsätzlich den gleichen Beitrag, sie bildet das Diskursziel aber nicht isomorph \is{Isomorphie} ab. \citet{Farkas2010} formulieren auch Eigenschaften \is{prototypische Assertion} prototypischer Assertionen: I.E. gilt für diese u.a., dass sie ihren Inhalt zu cg-Wissen zu machen beabsichtigen. Bei einer solchen Assertion decken sich die Eigenschaften des Äußerungstyps mit dem generellen Ziel von Kommunikation (s.o.). Die Abfolge sollte folglich erst recht \textit{ja doch} sein.  

Für die markierten Abfolgen vertrete ich dann in Abschnitt~\ref{sec:markiert}, dass man es bei ihren Auftretenskontexten nicht mit prototypischen Assertionen zu tun hat, in dem Sinne, dass es nicht ihr Hauptziel ist, p zum Teil des cg zu machen. Be\-wertungen, epistemisch modalisierte Äußerungen und epistemische/illokutionäre Kausalsätze beabsichtigen nicht in erster Linie, sich mit ihrem Gegenüber hinsichtlich dieser Aspekte zu einigen. Da sie nicht zuoberst darauf abheben, einen stabilen Kontextzustand zu erreichen, kann die Partikel \textit{ja}, die dies bezweckt, auch später zur Applikation gebracht werden. Der Rückgriff auf das allgemeine Diskursprinzip ist aber auch für diese weniger prototypischen Assertionen mög\-lich. Die gewählte Abfolge ist dann \textit{ja doch}. 

Abschnitt~\ref{sec:status} diskutiert einige Aspekte rund um meine kontroverse Behauptung, dass die Abfolge \textit{doch ja} existiert. Ich weise hier die Vorwürfe zurück, es handle sich um schlampige Sprache, die für Webdaten generell gelte, und es lägen Konstruktionen vor, nicht kombinierte Partikelauftreten. Ich präsentiere hier auch ältere Belege und führe das Vorkommen in den Webdaten auf die Menge der zugänglichen Daten zurück. Einige weitere empirische Fragen schätze ich als offen ein (insbesondere den Nachweis, dass man es mit genuinen \textit{doch ja}-Kontexten zu tun hat). Die Ergebnisse einer Akzeptabilitätsstudie beantworten diese Frage für illokutionäre Kausalsätze positiv.\\

\noindent		
Kapitel~\ref{chapter:hue} behandelt Kombinationen aus \textit{halt} und \textit{eben}. 

In Abschnitt~\ref{sec:hueinliteratur} stelle ich hier zunächst Ansichten aus der Literatur zu den beiden Einzelpartikeln dar. Wenngleich es dialektale Studien gibt, die darauf hindeuten, dass die MPn einmal regional verschieden verwendet wurden (\textit{halt} Süd-/Westdeutsch, \textit{eben} Nord-/Ostdeutsch), scheint dieser Gebrauchsunterschied im Gegenwartsdeutschen (und zwar schon seit den 80er Jahren) abgebaut zu sein. Ich weise hier auch auf mögliche Einflussfaktoren auf die Ergebnisse derartiger Studien hin. Hinsichtlich der Bedeutung von \textit{halt} und \textit{eben} spalten sich Autoren, die sich mit dieser Frage beschäftigt haben, in zwei Lager. Die eine Gruppe vertritt, die Partikeln wiesen eine identische Bedeutung auf. Die andere geht davon aus, dass sie sich subtil voneinander unterscheiden und \textit{halt} die Effekte, die \textit{eben} zugeschrieben werden, in abgeschwächter Form kodiert. Äußerungen, in denen die Partikeln auftreten, dienen entweder der Begründung eines vorweggehenden Beitrags oder zeigen an, dass die durch sie ausgedrückte Proposition in der Relation der Folge zum Vorgängerbeitrag steht. \textit{Eben} wird dabei mit Eigenschaften wie \textit{evident}, \textit{klar} und \textit{bekannt}, \textit{halt} mit Attributen wie \textit{plausibel} und \textit{denkbar} belegt. 

Ich schließe mich dieser Auffassung, dass es Interpretations- und Verwendungs\-unterschiede gibt, an, und mache in Abschnitt~\ref{sec:modellierung} einen Vorschlag zur Erfassung dieser deskriptiven Beobachtungen im Diskursmodell aus \citet{Farkas2010}. Anders als in Kapitel~\ref{chapter:jud} zu \textit{ja} und \textit{doch} kann sich die Darstellung hier nicht auf Assertionen beschränken. \textit{Halt} und \textit{eben} kombinieren sich auch in \is{Direktiv} Direktiven. Ich gehe davon aus, dass im Kontext vor einer \textit{eben}-Assertion mit einer Proposition p, die als Begründung fungiert, eine Inferenzrelation \is{Inferenzrelation} p $>$ q im cg enthalten sein muss sowie p sich unter den Diskursbekenntnissen von B befinden muss. Im Falle von \textit{halt} ist die Relation Teil der Sprecherbekenntnisse. Beide Partikeln fordern zudem, dass ein Bekenntnis von A oder B gegenüber der Proposition, die die MP-Assertion anschließend begründet, vorliegt. Die gleichen Verhältnisse müssen für eine angemessene \textit{halt}-/\textit{eben}-Folge vorliegen. In diesem Fall kehrt sich die Inferenzrelation um. Ich argumentiere zudem für das Vorliegen der prinzipiell gleichen Konstellation in \textit{halt}-/\textit{eben}-Direktiven, die stets die Folge markieren. Um auch den Beitrag dieser Äußerungstypen erfassen zu können, habe ich das Diskursmodell um die Komponente der To-Do-Liste \is{To-Do-Liste} erweitert. Generell habe ich \textit{halt} mit Assertivität \is{Assertivität} (bzw. verwandten Eigenschaften wie \is{Rhematizität} Rhematizität, \is{Vordergrund} Vordergrundierung) und \textit{eben} mit \is{Präsupposition} Präsupponiertheit (Thematizität, Hintergrundierung) \is{Thematizität} \is{Hintergrund} assoziiert.

Abschnitt~\ref{sec:empirie} präsentiert sich widersprechende Annahmen aus der Literatur zur Verteilung von \textit{halt eben} und \textit{eben halt}. Auf der Basis von Korpusuntersuchungen und Sprecherurteilen lässt sich keine Evidenz dafür finden, dass die Abfolgen regional spezifisch oder satzmodusabhängig sind. Unter Bezug auf meine empirischen Studien stufe ich die Abfolge \textit{halt eben} als unmarkiert und \textit{eben halt} als markiert ein. 

Die Untersuchung der Interpretation der Kombinationen in Abschnitt~\ref{sec:interpretationkombi} kommt zu dem Ergebnis, dass die Partikeln sich beide gleichermaßen auf die Proposition beziehen und kein Skopusverhältnis \is{Skopus} zwischen ihnen vorliegt. 

Für meine Erklärung des Markiertheitsunterschieds \is{Markiertheit} mache ich mir in Abschnitt~\ref{sec:erklärunghe} vor allem zunutze, dass zwischen \textit{eben} und \textit{halt} ein Implikationsverhältnis \is{Implikation} besteht. Da Elemente, die Implikationen auslösen, und die implizierten Inhalte unabhängig der Ordnung Implikation $>$ Implikationsauslöser folgen und die Umkehr dieses Verhältnisses (\textit{Implikationsverstärkung}) \is{Implikationsverstärkung} zu markierten Strukturen führt, folgen die beiden Partikelabfolgen dieser generell (dis)präferierten Konstellation. Letztere kann ich auch für Hyperonyme und Hyponyme im Rahmen der Akzeptabilitätsstudien klar nachweisen. Die Ableitung des Markierts\-heitsunterschieds basiert m.E. folglich auch im Falle dieser Partikelkombinationen auf Bedeutungsas\-pekten. An dieser These ändern auch die Ergebnisse einer wei\-teren Studie zum potenziellen Einfluss des Rhythmus nichts. 

Da in der Literatur beobachtet worden ist, dass es Kontexte gibt, in denen Im\-plikationsverstärkungen möglich sind, widmet sich Abschnitt~\ref{sec:gebrauchheeh} der Frage, ob es auch Umgebungen gibt, in denen die markierte Partikelfolge, die meiner Argumentation nach diesem Muster folgt, (eher) verwendet wird. Eine Korpusuntersuchung zur Verteilung von \textit{halt}, \textit{eben}, \textit{halt eben} und \textit{eben halt} in Relativsätzen zeigt, dass die Tendenz besteht, dass \textit{eben halt} (genauso wie \textit{eben}) die Domäne des restriktiven Relativsatzes meidet, während \textit{halt eben} (wie auch \textit{halt}) hier eine weitere Distribution aufweist und in beiden Typen von Relativsätzen vorzufinden ist. Diese grundsätzliche Beobachtung bleibt auch unter weiteren Differenzierungen der Relativsätze bestehen. Ich erkläre diese Verteilungsunterschiede auf die Art, dass das mit Assertivität, Rhematizität und Vordergrund assoziierte \textit{halt} im thematischen, hintergrundierenden appositiven Relativsatz weniger dominant ist als im restriktiven Relativsatz. Die normalerweise zu Markiertheit führende Im\-plikationsverstärkung tritt in diesen Kontexten deshalb nicht im gleichen Ausmaß auf, als wenn es sich um einen vordergrundierten Satzkontext handelt – was für Assertionen im Standardfall anzunehmen ist.\\

\noindent
In Kapitel~\ref{chapter:dua} behandle ich Sequenzen aus \textit{doch} und \textit{auch}. 

Abschnitt~\ref{sec:präferenz} dient unter Bezug auf Annahmen aus der Literatur und der Angabe von Korpusfrequenzen dem Zweck, für die weitere Analyse den markierten Status von \textit{auch doch} und die Unmarkiertheit von \textit{doch auch} festzulegen.

Abschnitt~\ref{sec:distributionda} bestimmt Assertionen, Direktive sowie Exklamative als die wesent\-lichen Domänen des möglichen kombinierten Vorkommens und führt hier auch die assertiven Randtypen \is{V1-Deklarativsatz} \is{Wo-VL-Deklarativsatz} des V1- und \textit{Wo}-VL-Deklarativsatzes ein. Auch hier stellt sich folglich die Situation ein, dass der Anwendungsbereich des Kriteriums, das die Reihungsbeschränkung erfassen sollte, so weit sein muss, dass diese Äußerungs\-typen gleichermaßen erfassbar werden.

Abschnitt~\ref{sec:V2} entwickelt die Ableitung der unmarkierten Reihung \textit{doch auch} zunächst für Standard-V2-Deklarativsätze. Die Modellierung von \textit{doch} aus Kapitel~\ref{chapter:jud}, Abschnitt~\ref{sec:doch1} wird beibehalten: Eine \textit{doch}-Assertion setzt die Offenheit des Themas voraus, zu dem sie einen Beitrag leistet. Im Kontext vor einer \textit{auch}-Assertion ist eine \is{Inferenzrelation} Inferenzrelation p $>$ q im cg enthalten. Die zu begründende Proposition q ist Teil der Diskursbekenntnisse von Sprecher oder Hörer. Auch im Falle der Kombination \textit{doch auch} fällt die Untersuchung der Interpretation der Kombination anhand authentischer Belege zugunsten der additiven Lesart aus. Beide Partikeln beziehen sich gleichermaßen auf die enthaltene Proposition.

Unter Bezug auf die von \citet{Farkas2010} formulierten generellen Antriebe von Kommunikation ( a) cg-Anreicherung, b) Herstellung eines stabilen Kontextzustandes) verläuft die Argumentation so, dass eine Voraussetzung für das Erreichen dieser Ziele die Adressierung des aktuellen Themas ist. Da \textit{doch} anzeigt, dass die ausgedrückte Proposition einen Beitrag zum sich auf dem Tisch be\-findenden Thema macht, und \textit{auch} vermittelt, dass die Proposition die Begründung einer anderen Proposition ausmacht, bildet die Sequenz \textit{doch auch} das Diskurs\-ziel direkter ab als die Reihung \textit{auch doch}. 

Abschnitt~\ref{sec:Rand} betrachtet dann die assertiven Randtypen des V1- und \textit{Wo}-VL-Satzes mit dem Ziel, festzustellen, ob meine Erklärung auch auf diese Äußerungs\-typen übertragbar ist. Hier steht insbesondere die Frage im Mittelpunkt, welchen Beitrag das für diese Sätze sehr typische bzw. obligatorische \textit{doch} leistet. In der Litera\-tur liegen Meinungen vor, die vertreten, dass \textit{doch} in dieser Umgebung nicht transparent verwendet wird. Zur Beantwortung dieser Frage ist es vonnöten, verschiedene (vorgeschlagene) Eigenschaften der Sätze zu untersuchen. Mein Schluss aus dieser Untersuchung ist, dass kein Grund besteht, \textit{doch} seine Transparenz abzusprechen, wenn man annimmt, dass \textit{doch} auf das offene, zur Diskussion stehende Thema reagiert, und nicht – wie in anderen Arbeiten angenommen – Bekanntheit und i.e.S. Kontrast oder Widerspruch kodiert.

Abschnitt~\ref{sec:direktive} weitet die Betrachtung auf Direktive \is{Direktiv} aus. Anhand authentischer Belege zeige ich, dass grundsätzlich die gleichen Kontextanforderungen vorliegen müssen wie in Assertionen, damit die Partikel-Äußerungen angemessen erfolgen können. Unterschiede zwischen den Modellierungen für \textit{doch}/\textit{auch}-Asser\-tionen und -Direktive ergeben sich aus der unterschiedlichen Natur der Äußerungs\-typen. Bei p handelt es sich um eine noch zu erfüllende Proposition und mit !p wird die TDL affiziert und nicht das DC-System. Im Falle von \textit{doch} steht das Thema der Realisierung des geforderten Sachverhalts zur Debatte. \textit{Auch} drückt auf der Basis einer \is{Inferenzrelation} Inferenzrelation q $>$ !p und des beim Adressaten vorhandenen Bekennt\-nisses zu q aus, dass die angeordnete Handlung eine abzuleitende Forderung ist. 

Abschnitt~\ref{sec:distributionad} geht der Frage nach, ob sich auch bei dieser Kombination ein Nachweis für die Existenz der umgekehrten Abfolge erbringen lässt. Trotz weiterer empirischer Fragen und einer eher kleinen Datenmenge ergeben meine bisherigen Recherchen hier, dass auch für die Reihung \textit{auch doch} zwei Muster feststellbar sind. Die Abfolge tritt wiederholt in Verbindung mit \textit{ja} und in kausalen Nebensätzen auf. Für beide Domänen argumentiere ich hier, dass sich annehmen lässt, dass der Aspekt der Themaadressierung weniger dominant ist als außerhalb dieser Umgebung. \textit{Ja} hat bereits cg hergestellt, wenn es am linken Rand der Kombination früh appliziert, und auch Kausalsätze haben im Zentrum ihres Interesses gerade den Ausdruck des kausalen Zusammenhangs. Tritt dieses ansonsten hochrelevante Diskursziel in den Hintergrund, ermöglicht dies die spätere Ap\-plikation von \textit{doch}. Treten \textit{doch} und \textit{auch} gemeinsam auf, scheint auch hier die passendere Interpretation einzutreten, wenn die Partikeln den gleichen Skopus \is{Skopus} nehmen und ihre Bedeutung sich somit addiert. Im Rahmen der Möglichkeiten eines Direktivs, Einfluss auf ein im Raum stehendes Thema zu nehmen, kommt auch die Applikation der beiden Partikeln in der Reihenfolge \textit{doch auch} diesem Diskursziel direkter nach als die umgekehrte Ordnung \textit{auch doch}, die zuerst ausdrückt, dass die geforderte Handlung klar ist und anschließend mitteilt, dass sie sich auf das zu lösende Thema bezieht.\\

\noindent
Neben den drei Detailuntersuchungen zu den Kombinationen aus \textit{ja} und \textit{doch}, \textit{halt} und \textit{eben} sowie \textit{doch} und \textit{auch} und den im Rahmen der allgemeinen These um Ikonizität dort jeweils als applizierend angenommenen diskursstrukturellen Prinzipien leistet die Arbeit auch einige allgemeinere Beiträge bzw. eröffnet wei\-tere Fragen, die mit der hier behandelten Thematik in Zusammenhang stehen.\\
\noindent
Ich beabsichtige mit dieser Arbeit, bei einem Thema, das entweder eher deskriptiv bei detaillierter Datenauswertung oder theoretisch mit ausbleibender Datenar\-beit behandelt worden ist, Empirie und Theorie besser zusammenzuführen, als es in meinen Augen bisher geschehen ist. 

Aus Sicht grammatiktheoretischer Beschäftigung sollte eine präzise, abstrakte Modellierung angestrebt werden, um den Beitrag der MPn zu erfassen. Empirisch sollten gerade bei diesem Phänomen, das aus konzeptioneller Mündlichkeit nicht wegzudenken ist und bei dem die theoretische Modellierung letztlich in der Formulierung von Gebrauchsbedingungen besteht, Belege und Sprecherurteile in die Analyse mit einbezogen werden. Meiner Meinung nach zeigen die Ausführungen, dass diese Verbindung in diesem mitunter subtilen Phänomenbereich als gewinnbringend einzuschätzen ist. Bei der Korpusarbeit erweist sich der Datentyp der Webdaten als hilfreich, da hier große Datenmengen, die medial schriftlich vorliegen, konzeptionelle Mündlichkeit – für meine Begriffe realistisch – abbilden. Im Bereich von Akzeptabilitätsurteilen vermögen Paarvergleiche es, auch in subtilen Bereichen Asymmetrien aufzudecken. Die Studien sprechen dafür, dass sich derartige Methoden auch für semantisch-pragmatische Fragestellungen eignen – bei deren Bearbeitung sie weniger zur Anwendung kommen. Allerdings zeigt sich an verschiedenen Stellen ebenfalls, dass auch empirischem Vorgehen und seinen Erkenntnissen Grenzen gesetzt sind. Beispielsweise stellen sich manche Datenmengen, die es für die Beantwortung der Frage\-stellung zu untersuchen gälte, schlicht als zu groß heraus, um jeden Beleg zu betrachten und die Non-MP-Verwendungen herauszufiltern. Das gleiche Problem tritt ggf. auf, wenn Erwartungswerte bestimmt werden müssen, um bestimmte Verteilungen verlässlich bewerten zu können. Auch gilt für die inferenzstatistische Auswertung der meisten meiner Studien, dass nur kleine Effekte nachzuweisen sind.

Ich halte die empirische Arbeit zu diesem Phänomenbereich dennoch für loh\-nenswert. Man gewinnt auch oder gerade bei den eher subtilen Fragen auf diese Weise einen realistischeren Einblick in die Verhältnisse, als wenn die Analyse von MP-Äußerungen allein auf konstruierten Beispielen basiert oder mit Belegen rein qualitativ umgegangen wird und die Untersuchung deshalb meist auf sehr begrenzten Belegzahlen basiert.\\

\noindent
Neben der allgemeineren Frage nach geeigneten empirischen Methoden bei der Untersuchung semantisch-pragmatischer Phänomene gibt es zwei weitere Berei\-che, für die sich die Frage nach der Verallgemeinerbarkeit des prinzipiellen Zugangs und der konkreten Ableitung stellt.\\

\noindent
Mit drei speziellen Reihungen untersuche ich ohne Zweifel nur einen kleinen Ausschnitt möglicher (Zweier-)Kombinationen. \citet[280]{Thurmair1989} geht von 171 möglichen Kombinationen aus, von denen ca. 50 verwendet werden. Meine Ausführungen zeigen, wie viele Aspekte es jeweils im Detail zu klären gilt. Dennoch möchte ich einige Überlegungen zu generellen Tendenzen des Partikelvor\-kommens in Sequenzen anstellen und diskutieren, um aufzuzeigen, dass meine Annahmen sich in das Gesamtbild einfügen (vgl. auch schon \citealt[236]{Mueller2017b}).

Die Partikel \textit{ja} steht beispielsweise generell am linken Rand einer Kombination, d.h. sie geht allen anderen an der Sequenz beteiligten Partikeln voran. Da \textit{ja}, das ausschließlich in Assertionen vorkommt, sich auch nur in Assertionen mit anderen Partikeln kombinieren kann, bietet sich meine Erklärung, warum es \textit{doch} vorangeht, dafür an, dieses generelle Stellungsmuster abzuleiten: Es ist ein übergeordnetes Diskursziel (von Assertionen), die ausgedrückte Proposition zu einem cg-Inhalt zu machen. Da es keine andere Partikel gibt, die (ausschließlich) dieses Ziel verfolgt, sollte \textit{ja} meiner Argumentation nach auch immer früh zur Applikation gebracht werden (und nicht nur in Kombination mit \textit{doch}). 

Der Partikel \textit{doch} schreibe ich zu, anzuzeigen, dass die Äußerung auf das offene Thema des Gesprächs verweist/reagiert. Wenn nicht andere (noch) höher geordnete Diskursziele zusätzlich kodiert werden (wie eben die Auszeichnung als cg-Inhalt), sollte \textit{doch} ebenfalls generell früh zur Applikation gebracht werden. Wie ich ausgeführt habe, kann die Themaadressierung als Voraussetzung für die Absicht, einen stabilen Kontextzustand herzustellen und p zum Teil des cg zu machen, angesehen werden. Schaut man sich andere Kombinationen an, in denen \textit{doch} auftritt, trifft dies in der Tat zu. Diese Partikel geht in verschiedenen Illokutionstypen den anderen beteiligten Partikeln voran (\textit{doch auch}, \textit{doch eben}, \textit{doch halt}, \textit{doch schon}, \textit{doch wohl}, \textit{doch eh}, \textit{doch sowieso}, \textit{doch einfach}, \textit{doch bloß}, \textit{doch nur}, \textit{doch mal}, \textit{doch ruhig}, \textit{doch nicht etwa}). Eine Ausnahme ist \textit{denn doch} in Assertionen.

Die beiden Diskursprinzipien (cg-Herstellung, Themaadressierung), die ich für die präferierten Reihungen \textit{ja doch} und \textit{doch auch} verantwortlich mache, scheinen folglich auf die generellen Stellungsmuster von \textit{ja} und \textit{doch} übertragbar zu sein.

Fraglicher ist dies für meine Annahme, dass die unerwünschte Konstellation einer verstärkten Implikation, über die ich die Präferenz von \textit{halt eben} gegenüber \textit{eben halt} ableite, Einfluss auf Partikelfolgen nimmt. Oben habe ich angeführt, dass \textit{ja} stets am linken Rand auftritt. Da es keine andere Partikel gibt, die ebenfalls ausschließlich auf den (ggf. akkommodierten) präsupponierten Status der Proposition verweist, fügt sich beispielsweise auch die Abfolge \textit{ja eben} in diese Analyse gut ein. \textit{Eben} zeigt nicht nur die Bekanntheit von p an, sondern gibt den Sachverhalt unter Bezug auf eine Inferenzrelation im cg zusätzlich als ableitbar aus. \textit{Ja} ist in diesem Sinne weniger informativ als \textit{eben} und sollte dieser Partikel deshalb vorangehen. Die Abfolge von \textit{halt} und \textit{auch} erklärt sich auf diese Art ebenfalls: Beide Äußerungen setzen p nicht voraus, d.h. p wird assertiert. Während \textit{halt} fordert, dass die Inferenzrelation p $>$ q Teil des DC-Systems des Sprechers ist, liegt sie bei \textit{auch}-Äußerungen als cg-Information vor. Aufgrund des Implikationsverhältnisses zwischen dem cg und den DC-Systemen, sollte das \textit{halt} dem \textit{auch} vorangehen. Im Falle der Ordnung von \textit{eben} und \textit{auch} (unmarkiert ist \textit{eben auch}) macht das Kriterium der zu vermeidenden Implikationsverstärkung allerdings falsche Vorhersagen: Die beiden Partikeln unterscheiden sich nur darin, dass \textit{eben} auf die Bekanntheit von p verweist, während \textit{auch} von dieser nicht ausgeht. In beiden Fällen ist die Inferenzrelation p $>$ q im cg enthalten. Man könnte jetzt sagen, dass hier nicht das Prinzip greift, hinsichtlich der Abfolge von Implikationsauslöser und Implikation die angemessene Konstellation zu erzielen, sondern das für die Voranstellung von \textit{ja} verantwortliche Prinzip, cg herzustellen, sobald dies möglich ist. Dann könnte man aber wiederum einwenden, dass dieses Prinzip auch die Abfolge \textit{eben halt} voraussagen sollte. Meine (tentative) Lösung dieses Verhältnisses ist (unter Festhalten an meiner Ableitung), dass derartige Daten möglicherweise darauf hinweisen, dass die Kriterien, die ich formuliert habe, ggf. auch in Interaktion treten können. Diskursprinzipien wären dieser Sicht nach unterschiedlich relevant. Innerhalb meiner Modellierung gehe ich von diesem Umstand ohnehin bereits aus (z.B. die Themaadressierung ist \glq dringlicher\grqq{} als eine kausale Einstufung). Wenn die Implikation nun durch genau anderweitig als beteiligt angesehene Verhältnisse zustande kommt, kommt es natürlich zu Interaktionen der Beschränkungen.\footnote{Ein Gutachter verweist darauf, dass man sich ähnliche Fragen auch für die Interaktion meiner Erklärung der Abfolge \textit{ja doch} und \textit{halt eben} stellen könne, da auch mit \textit{eben} cg hergestellt werde.}

Diese Überlegung erklärt natürlich noch nicht, \underline{warum} dann bei \textit{halt eben} die Implikationsverstärkung greift und bei \textit{eben auch} die frühe cg-Herstellung. Ich möchte hierzu den Vorschlag machen, dass dies im vorliegenden Fall damit zu tun hat, zwischen welchen Bedeutungsanteilen sich die Implikation ergibt. Bei \textit{halt} und \textit{eben} kommt die Implikation sowohl durch die Verankerung der Infe\-renzrelation (\textit{eben}: p $>$ q in cg, \textit{halt}: p $>$ q in DC$_{\textrm{Sprecher}}$) als auch durch die Zuweisung der Proposition (\textit{eben}: p ist/wird cg, \textit{halt}: p wird Teil von DC$_{\textrm{Sprecher}}$) zustande. Zwischen \textit{eben} und \textit{auch} tritt sie allerdings allein aufgrund der unterschiedlichen Verankerung von p ein (\textit{eben}: p ist/wird cg, \textit{auch}: p wird Teil von DC$_{\textrm{Sprecher}}$). Die Inferenzrelation p $>$ q ist bei beiden Partikeln im cg enthalten. Ich möchte deshalb den Unterschied, dass die Implikation keine \underline{weiteren} Bedeutungskomponenten betrifft als den (nicht) präsupponierten Status von p (der gerade das einzige Kriterium der konkurrierenden Beschränkung $[$cg-Herstellung$]$ ist) dafür verantwortlich machen, dass im Falle von \textit{eben auch} das Kriterium greift, p möglichst direkt zu cg-Inhalt zu machen. Es bietet sich in diesem Fall somit eine Erklärung für das zutreffende Prinzip an, weil die Implikationsverhältnisse nicht identisch sind. Sicherlich gilt es aber, zu prüfen, ob es andere bedeutungsähnliche Partikeln gibt, zwischen denen sich Implikationen einstellen und die dann entlang meiner Vorhersage geordnet werden.

Eine weitere Frage, die sich für das Thema der MP-Reihungen an sich stellt, ist, ob für alle Kombinationen anzunehmen ist, dass die traditionelle Ansicht fester MP-Abfolgen generell zu strikt ist. Ich kann an dieser Stelle keinen Nachweis bringen, dass dies so ist. Ich möchte aber einige Daten anführen, die darauf hinweisen, dass die umgekehrten Abfolgen auch für andere in meiner Arbeit nicht weiter betrachtete MP-Kombinationen zu belegen sind (vgl. (1) bis (5)) (vgl. auch \citealt[251-252]{Mueller2017a}, \citeyear[237]{Mueller2017b}). Ich habe mir zu diesen Sequenzen weiter keine Verteilungen angeschaut, ich halte allerdings keine der auftretende Abfolgen für besonders abweichend.

\begin{exe}
	\ex\label{1196}
		\begin{xlist}	
			\ex\label{1196a}
			\scriptsize 
			{Die Avesta hat quasi die Mythologie der Avesta $[$...$]$ mit den Archämiden vermischt und $[$...$]$ sich quasi über die Avesta 			$				[$...$]$ legitimiert. Und sorry, \textbf{hört \underline{langsam mal} auf alle Namen mit \glqq glänzend und hell\grqq{} zu übersetzen.}
			\hfill\hbox {(Wikipedia-Diskussion-Arash (Mythologie))}}		
			\ex\label{1196b} 
			\scriptsize
			{Karl-Heinrich Langspecht [CDU]: Kommen Sie \underline{\textbf{mal langsam}} zum Haushalt!
			\newline
			\hbox{}\hfill\hbox {(Protokoll der Sitzung des Parlaments Landtag}
			\newline
			\hbox{}\hfill\hbox {Niedersachsen am 22.12.2010)}}
		\end{xlist}
\end{exe}

\begin{exe}
	\ex\label{1197} 
		\begin{xlist}	
			\ex\label{1197a} 
			\scriptsize 
			{ob da nicht vielleicht doch auch was anderes noch eine rolle spielt?\\
			\textbf{was könnte das \underline{nur bloß} gewesen sein?} *grübel, grübel*
			\hfill\hbox {(DECOW2012-03X: 622063131)}}		
			\ex\label{1197b} 
			\scriptsize 
			{\textbf{Auwei was ist \underline{bloß nur} aus den Ruhrbaronen geworden?} Wollt ihr alle einen Job bei der BILD ergattern oder warum ist die 				Qualität in letzter Zeit so massiv gesunken?      
			\newline
			\hbox{}\hfill\hbox {(DECOW2012-06X: 709617545)}
			\newline
			\hbox{}\hfill\hbox {\citet[14]{Mueller2014b}}}
		\end{xlist}
\end{exe}		  										         
		
\begin{exe}
	\ex\label{1198} 
		\begin{xlist}
			\ex\label{1198a} 
			\scriptsize 
			{Weiterhin viel Spaß und Erfolg beim Stricken und falls es noch Fragen gibt, \textbf{\underline{einfach ruhig} noch mal melden!}	
			\hfill\hbox {(DECOW2014AX)}			
			\newline
			\hbox{}\hfill\hbox {(http://www.abc-kinder.de/familie/strickanleitung-fur-babysockchen-in-grose-62-teil-3/)}}
			
			\ex\label{1198b} 
			\scriptsize 
			{Und bei allem anderen, sehe ich in deinem angewandten Weniger ein Mehr für die Flüssigkeit des Textes. Und das mit dem Warum des KALPs habe 				ich ja schon erläutert. \textbf{Also mach \underline{ruhig einfach} wie bisher weiter} – bislang scheint sich ja sonst eh kein anderer für den 				Artikel zu interessieren. Was mich allerdings auch ein wenig irritiert ... --HerrZog 19:27, 10. Okt. 2011 (CEST)	     
			\newline
			\hbox{}\hfill\hbox {(WDD11/F14.19129: Diskussion: Fürstpropstei Berchtesgaden)}}
		\end{xlist}
\end{exe}	

\begin{exe}
	\ex\label{1199} 
		\begin{xlist}	
			\ex\label{1199a} 
			\scriptsize 
			{heute morgen standen sie an der kreuzung och/lindenstraße richtung oranienstraße 8. \textbf{wat is \underline{denn eigentlich} so furchtbar 				schlimm an fahrradkontrollen?}	
			\hfill\hbox {(DECOW2014AX)}			
			\newline
			\hbox{}\hfill\hbox {(http://blogs.taz.de/hausmeisterblog/2006/10/page/2/)}}
			\ex\label{1199b} 
			\scriptsize 
			{Dieses Eintippen und warten, was wohl für ein Vorschlag kommt und nicht im Überblick gucken können: \textbf{Was hab ich \underline{eigentlich 				denn} schon? }    
			\hfill\hbox {(DECOW2012-03X: 622063131)}}	
			\newline
			\hbox{}\hfill\hbox {(http://www.zielpublikum.de/2008/07/31/neues-in-sachen-tags-und-wordpress/)}
		\end{xlist}
\end{exe}
	           	
\begin{exe}
	\ex\label{1200} 
		\begin{xlist}
			\ex\label{1200a} 
			\scriptsize 
			{Und auch die eher ruhigen Momente gehen beinahe unter die Haut – ein verwundeter Soldat kriecht weg, im Schutz eines Kellers oder Bunkers 					warten Soldaten auf etwas. \textbf{Es wird \underline{eben wohl} eine ganz besondere Erfahrung.} 	
			\hfill\hbox {(DECOW2014AX)}			
			\newline
			\hbox{}\hfill\hbox {(http://www.gamersunity.de/modern-warfare-2/trail}	
			\newline
			\hbox{}\hfill\hbox {er-zeigt-washington-in-flammen.t9380.html)}}			
			\ex\label{1200b}
			\scriptsize  
			{\textbf{In der Regel ist ein Bandscheibenvorfall \underline{wohl eben} etwas für ältere} – obwohl der natürliche Verschleiß der Bandscheiben 				bereits zwischen 15 und 20 Jahren beginnt.    
			\hfill\hbox {(DECOW2014AX)}	
			\newline
			\hbox{}\hfill\hbox {(http://www.med1.de/Forum/Orthopaedie/434615/)}}
		\end{xlist}
\end{exe}				           	
Auch hier plädiere ich dafür, die Belege ernst zu nehmen und keine Ableitungen zu formulieren, in denen allein \underline{eine} akzeptable Reihung vorhergesagt wird. Denkbar (bzw. sehr wahrscheinlich) ist, dass nicht alle MP-Kombinationen in verschiedenen Abfolgen vorkommen. Dieser Frage muss man jedoch im Detail nachgehen.\\

\noindent
Einen dritten Bereich, den ich hervorheben möchte, weil meine Ausführungen hierzu m.E. einen prinzipielleren Beitrag leisten, betrifft die so genannten \textit{Wurzel\-phänomene} \is{Wurzelphänomen} bzw. die Diskussion, ob MPn als ein solches aufzufassen sind. Ich habe diese Fragen in Abschnitt~\ref{sec:rs} in Kapitel~\ref{chapter:hue} schon aufgeworfen. Wie ich gezeigt habe, zeigen \textit{halt} und \textit{eben} (und auch \textit{halt eben} und \textit{eben halt}) nicht das gleiche Verhalten in restriktiven und appositiven Relativsätzen. Zunächst ist die Annahme, dass MPn generell in restriktiven Relativsätzen nicht auftreten können, auf der Basis meiner Daten klar abzulehnen. Vielmehr zeigt sich, dass es entscheidend ist, \underline{welche} Partikel untersucht wird. In einschlägigen Arbeiten, in denen vertreten wird, dass MPn (weil sie Wurzelphänomene sind) aus restriktiven Relativsätzen (weil sie zentrale Nebensätze \is{zentraler Nebensatz} sind) ausgeschlossen sind, werden meist nur \textit{ja} und \textit{doch} betrachtet. Die von mir in Abschnitt~\ref{sec:rs} untersuchten Partikeln und Partikelkombinationen treten hier aber auf und weisen – je nach MP – Präferenzen auf. Der Blick auf weitere zentrale Nebensätze zeigt, dass damit zu rechnen ist, auch dort Unterschiede im Auftreten verschiedener Partikeln festzustellen. In jedem Fall besteht hier die Aufgabe, das Vorkommen einer größeren Menge von Partikeln in verschiedenen als Wurzelphänomene ausgegebenen Domänen zu untersuchen.

Die Konsequenz aus diesen Beobachtungen ist meiner Meinung nach, dass man die Einstufung von MPn als Wurzelphänomen sowie/bzw. die semantisch-pragmatische Fassung von Wurzelkontexten überdenken bzw. differenzieren muss. Mein Eindruck ist, dass sich Arbeiten darauf beschränken, den Wurzelkontexten (eher vage) eine gewisse illokutionäre Selbständigkeit zuzuschreiben, die dann aber eher syntaktisch (und gerade nicht semantisch-pragmatisch) in Form von Force-/Sprecherdeixis-Projektionen in Baumstrukturen ausbuchstabiert werden, obwohl ein genuin semantisch-pragmatisches Konzept beteiligt ist. Da verschiedene Partikeln (nicht) in den beiden Typen von Relativsätzen auftreten können, gilt es hier (und je nach den Ergebnissen der oben gewünschten Untersuchung auch für andere Umgebungen), das Konzept von \glq illokutionärer (Un)selb\-ständigkeit\grq {} zu präzisieren bzw. überhaupt genauer zu fassen. 
	
Auch bei diesem (aktuellen) Thema steuern meine spezifischen Studien zu den von mir betrachteten MPn folglich einen Beitrag bei.\\

\noindent
In erster Linie leistet diese Arbeit eine auf empirischen Ergebnissen basierende diskursstrukturelle Ableitung der Reihungsbeschränkungen bei MP-Kombinatio\-nen aus \textit{ja}/\textit{doch}, \textit{halt}/\textit{eben} und \textit{doch}/\textit{auch}. Ich hoffe, außerdem Ideen und Er\-kenntnisse vorgestellt zu haben, die für das Thema der MP-Sequenzen generell re\-levant sind und die auch darüber hinaus einen Beitrag zu angrenzenden Phänomenen an der Syntax-Pragmatik-Schnittstelle (wie z.B. Satz-/Äußerungstypen, Wur\-zelphänomenen, Relativsätzen oder dem Verständnis von Illokution) zu leisten vermögen.
















