\chapter{Kombinationen aus \textit{ja} und \textit{doch}}
\label{chapter:jud}
\section{Distribution von \textit{ja}, \textit{doch} und \textit{ja doch}}
\label{sec:distributionjd}
Es ist aus der Literatur bekannt, dass sich bestimmte MPn nur in bestimmten Domänen kombinieren lassen, wobei \textit{Domäne} hier im Sinne von \textit{Satztyp}, \textit{Satzmodus} bzw. \textit{Illokutionstyp} zu verstehen ist. Welches der drei genannten Konzepte das Kriterium ist, ist unklar. Die Annahmen unterscheiden sich je nach betrachte\-ten MPn. Dazu kommt, dass es in der Literatur keinen Konsens in Bezug auf die Auffassung der drei Konzepte gibt sowie dass im Einzelfall Uneinigkeit darüber besteht, wie bestimmte Konstruktionstypen zu klassifizieren sind. Auf konkrete Fälle dieser generellen Problematik verweise ich im Laufe des Kapitels. Trotz dieser Problematik muss einer Analyse der Abfolge von MPn in Kombinationen aber eine Bestimmung der beteiligten sprachlichen Kontexte, in denen die Kombination der betrachteten MPn überhaupt möglich ist, vorangehen. Neben der Festlegung derartiger (eher) struktureller Domänen ist dazu verschiedentlich beobachtet worden, dass auch semantische und pragmatische Faktoren Einfluss auf zulässige Domänen der Kombination nehmen. 

\subsection{Syntaktische Schnittmengenbedingung}
\label{sec:synschnitt}
\citet[20]{Thurmair1991} definiert die Beobachtung, dass sich MPn nur in bestimmten Umgebungen miteinander kombinieren lassen, über eine satzmodale Schnittmengenbedingung \is{satzmodale Schnittmengenbedingung}:
\begin{quotation}
[...] a modal particle A, which may only appear with sentence mood Z, and a modal particle B, which may only appear with sentence mood Y (Z $\neq$ Y), should not be combinable, and [...] a modal particle A, which appears in sentence mood X and Y, may co-occur with a modal particle B, which appears in sentence moods Y and Z, only in sentence mood Y, that is in the intersection. 
\end{quotation}
MPn können demnach nur in den Satzmodi \is{Satzmodus} miteinander kombiniert werden, in denen sie auch allein auftreten können. Thurmair geht hierbei von der Satzmoduskonzeption nach \citet{Altmann1984, Altmann1987} aus. Er trennt bei der Beschreibung von Sätzen/Äußerungen zwischen Form und Funktion; der Satzmodus ist die regelmäßige Verbindung aus Formtyp \is{Formtyp} und Funktionstyp \is{Funktionstyp} (vgl. \citealt[22]{Altmann1987}). Die von ihm angenommenen Formtypen lassen sich entlang grammatischer Eigenschaften (wie Verbstellung, Verbmorphologie, Tonmuster, Obligatorizität eines w-Elements, Exklamativakzent) beschreiben. Diesen Formtypen wird je\-weils ein Funktionstyp zugeordnet.\footnote{Ich verwende hier im Folgenden Thurmairs Terminologie, da ich mich an ihrer Darstellung orientiere, wähle ansonsten in der Arbeit aber selbst die üblichen Satzmodusbezeichnungen \textit{Deklarativ}-, \textit{Interrogativ}-, \textit{Imperativ}-, \textit{Optativ}- und \textit{Exklamativsatz} bzw. nehme die Betrachtung ab Abschnitt~\ref{sec:nonkan} aus der Perspektive des Diskursbeitrags, und damit der Illokution, der Äußerungen vor. Für die Fälle, in denen man es mit einer 1:1-Zuordnung von Satztyp, Satzmodus und Illokution zu tun hat, ist die Unterscheidung zwischen den Konzepten auch unerheblich für meine Argumentation.}

\begin{exe}
\ex\label{262}
\begin{tabular}[t]{lll}
  	\textbf{Formtypen} & & \textbf{Funktionstypen}\\
  	Aussagesatz & \scriptsize{\textit{Anja hat eine schöne Wohnung.}} & Assertion\\
  	E-Fragesatz & \scriptsize{\textit{Wohnt Anja alleine hier?}} & Frage\\
  	w-Fragesatz & \scriptsize{\textit{Wer wohnt in dieser Wohnung?}} & Frage\\
  	Imperativsatz & \scriptsize{\textit{Putz die Wohnung!}} & Aufforderung\\
  	Wunschsatz & \scriptsize{\textit{Hätte ich doch auch so eine Wohnung!}} & Wunsch\\
  	(Satz)Exklamativsatz & \scriptsize{\textit{Hast DU eine schöne Wohnung!}} & Exklamativ\\
  	w-Exklamativsatz & \scriptsize{\textit{Wie HELL ist die Wohnung!}} & Exklamativ
\end{tabular}
\end{exe}
Basierend auf den Formtypen setzt \citet[49]{Thurmair1989} die Verteilung von \textit{ja} und \textit{doch} (und damit das gemeinsame Auftreten) wie in (\ref{263}) an.\footnote{Übersichten dieser Art finden sich z.B. auch in \citet[59]{Karagjosova2004} und \citet[183]{Kwon2005}. Wie anfänglich erwähnt, können derartige Klassifikationen je nach Arbeit anders aussehen. Zu den Gründen, die ich dafür anführe, s.o. Am Beispiel der Verteilung von \textit{ja} und \textit{doch} lassen sich konkrete Unterschiede in der Zuordnung illustrieren: Bei \citet{Karagjosova2004} beispielsweise tritt \textit{ja} (anders als bei \citealt{Thurmair1989}) in Satz- und \textit{dass}-Exklamativsätzen auf. Bei \citet[157]{Hentschel1986} stellen hingegen nicht Formtypen, sondern Funktionstypen (Assertionen und Exkla\-mationen) die zulässige Domäne dar.}

\begin{exe}
	\ex\label{263}
	\tiny
     \begin{tabular}[t]{|l|l|l|l|l|l|l|l|}
     		\hline
     		& Aussagesatz & E-Fragesatz & w-Fragesatz & Imperativsatz & Wunschsatz & (Satz-)Exklamativsatz & w-Exklamativsatz\\
            \hline
            \textit{doch} & $\plus$ & $\minus$ & $\plus$ & $\plus$ & $\plus$ & $\minus$ & $\plus$\\
             \hline
             \textit{ja} & $\plus$ & $\minus$ & $\minus$ & $\minus$ & $\minus$ & $\minus$ & $\minus$\\
             \hline
      \end{tabular}\\
\end{exe}
Nach Thurmairs satzmodaler Schnittmengenbedingung (s.o.) sollten \textit{ja} und \textit{doch} gemeinsam nur im Aussagesatz auftreten können. Die Beispiele in den folgenden zwei Abschnitten bestätigen diese Vorhersage.

\subsubsection{Deklarativ-, Interrogativ-, Imperativ- und Optativ- und Exklamativsatz}
\label{sec:exkl}
Da weder \textit{doch} noch \textit{ja} im E-Fragesatz auftreten kann, ist auch die Kombination ausgeschlossen (vgl. (\ref{264})).

\begin{exe}
	\ex\label{264} 
	*Hast du \textbf{doch}/\textbf{ja}/\textbf{ja doch} am Wochenende Zeit?
\end{exe}
Der w-Fragesatz \is{w-Fragesatz}, Imperativsatz \is{Imperativsatz} und Wunschsatz \is{Wunschsatz} erlauben das Auftreten von \textit{doch}. Die Unzulässigkeit der Kombination folgt jedoch, da \textit{ja} in diesen satzmodalen Umgebungen nicht lizensiert ist (vgl. (\ref{265}) bis (\ref{267})).
	
\begin{exe}
	\ex\label{265} 
	Wie heißt \textbf{doch gleich}/*\textbf{ja gleich}/*\textbf{ja doch gleich} die Straße, in der du wohnst?
\end{exe}	
\vspace{-0.65cm}	
\begin{exe}
	\ex\label{266} 
	Mach' \textbf{doch}/*\textbf{ja}/*\textbf{ja doch} die Heizung an!
\end{exe}
\vspace{-0.65cm}	
\begin{exe}
	\ex\label{267} 
	Hätte ich \textbf{doch}/*\textbf{ja}/*\textbf{ja doch} am Gewinnspiel teilgenommen!
\end{exe}
In allen Fällen aus (\ref{265}) bis (\ref{267}) bleibt die Schnittmenge der Satzmodi, in denen die beiden MPn je in Isolation auftreten können, leer, weshalb die Kombination von \textit{ja} und \textit{doch} im Einvernehmen mit Thurmairs Beschränkung (s.o.) nicht akzep\-tabel ist.

Da \textit{ja} und \textit{doch} aber unabhängig voneinander in Aussagesätzen stehen können, führt auch ihr kombiniertes Auftreten zu einer wohlgeformten Struktur (vgl. (\ref{268})).

\begin{exe}
	\ex\label{268} 
	In Hamburg ist \textbf{doch}/\textbf{ja}/\textbf{ja doch} Hafenfest.
\end{exe}
Die in (\ref{265}) bis (\ref{268}) illustrierten Distributionsverhältnisse von \textit{ja}, \textit{doch} und \textit{ja doch} in E-Frage, w-Frage-, Imperativ-, Wunsch- und Aussagesätzen sind als unkontrovers einzustufen. Ausführlichere Ausführungen sind nötig in Bezug auf die Satz- und w-Exklamativsätze aus der Übersicht in (\ref{262}).\\

\noindent
In w-Exklamativsätzen, \is{w-Exklamativsatz} die nach \citet[45]{Thurmair1989} charakterisiert werden durch indikativischen Verbmodus, ein w-Element im Vorfeld, das mit einem wertenden Adjektiv/Adverb verbunden ist, das einen Akzent trägt, sowie fallenden Tonverlauf, kann \textit{doch}, nicht aber \textit{ja} auftreten. Dies gilt sowohl für w-V2-Exklamativsätze als auch für die VL-Variante dieses Konstruktionstyps (vgl. (\ref{269}) und (\ref{270})).

\begin{exe}
	\ex\label{269} 
		\begin{xlist}	
			\ex\label{269a} Was BIST du \textbf{doch}/*\textbf{ja bloß} für ein Mensch!
			\ex\label{269b} Was für eine Wohltat ist \textbf{doch}/*\textbf{ja} dieses Buch!
		\end{xlist}
\end{exe}

\begin{exe}
	\ex\label{270} 
		\begin{xlist}	
			\ex\label{270a} Wie SCHÖN du \textbf{doch}/*\textbf{ja} bist!	
			\hfill\hbox {\citet[218-219]{Rinas2006}}
			\ex\label{270b} Was für ein TOLLER KERL er \textbf{doch}/*\textbf{ja} ist!
			\newline
			\hbox{}\hfill\hbox {\citet[37]{Kwon2005}, (TAZ, 02.10.1996, VI)}
		\end{xlist}
\end{exe}
Da die Schnittmenge der Satzmodi hinsichtlich des Einzelauftretens von \textit{ja} und \textit{doch} leer ist, ist folglich auch die Kombination nicht möglich (vgl. (\ref{271})).

\begin{exe}
	\ex\label{271} 
		\begin{xlist}	
			\ex\label{271a} *Wie schön du \textbf{ja doch} bist!	
			\ex\label{271b} *Was für ein TOLLER KERL er \textbf{ja doch} ist!
			\ex\label{271c} *Was BIST du \textbf{ja doch bloß} für ein Mensch!	
			\ex\label{271d} *Was für eine Wohltat ist \textbf{ja doch} dieses Buch!
		\end{xlist}
\end{exe}
Zu den Satzexklamativsätzen \is{Satzexklamativsatz} zählen bei Thurmair Strukturen wie in (\ref{272}).

\begin{exe}
	\ex\label{272} 
		\begin{xlist}	
			\ex\label{272a} Hast DU eine schöne Wohnung!	
			\ex\label{271b} DU hast (vielleicht) eine schöne Wohnung!
		\end{xlist}
\end{exe}
Dieser Satztyp weist nach \citet[45]{Thurmair1989} die folgenden Charakteristika auf: V1- oder V2-Stellung, indikativischer Verbmodus, fallendes Tonmuster, Exklamativakzent. \citet[49]{Thurmair1989} vertritt die Annahme, dass in diesem satzmodalen Kontext weder \textit{ja} noch \textit{doch} auftreten können (vgl. auch \citealt[40]{Altmann1987}). 

Die Anwesenheit von \textit{doch} scheint wirklich ausgeschlossen (vgl. (\ref{273}) (vs. (\ref{274}))).

\begin{exe}
	\ex\label{273} 
	*DER hat \textbf{doch} einen Bart!
\end{exe}
\vspace{-0.65cm}	
\begin{exe}
	\ex\label{274} 
	DER hat \textbf{aber}/\textbf{vielleicht} einen Bart!	
			\hfill\hbox {\citet[218]{Rinas2006}}
\end{exe}																	     
\citet[141]{Hentschel1986} hält die Verwendung von \textit{doch} hier für veraltet. Nach \citet[26]{Weydt1969} ist (\ref{275}) grammatisch.

\begin{exe}
	\ex\label{275} 
	Kommst du \textbf{doch} spät!
\end{exe}
\citet[141]{Hentschel1986} hält die Sätze in (\ref{276}) für \glqq möglich\grqq{}, mit Akzent auf \textit{das} allerdings für  \glqq problematisch\grqq{}. 

\begin{exe}
	\ex\label{276} 
		\begin{xlist}	
			\ex\label{276a} War das \textbf{doch} ein Fest!	
			\ex\label{276b} Ist das \textbf{doch} schön!
		\end{xlist}
\end{exe}
M.E. sind die Strukturen in (\ref{273}), (\ref{275}) und (\ref{276}) alle ungrammatisch, wie auch schon von Thurmair und Rinas derart angenommen. Thurmairs Annahme widersprechend scheint mir \textit{ja} in diesem Kontext aber problemlos auftreten zu können (vgl. (\ref{277}) sowie den authentischen Beleg in (\ref{278a}), bei dem \textit{der} in diesem Kontext plausiblerweise akzentuiert wird).

\begin{exe}
	\ex\label{277} 
	DER hat \textbf{ja} einen Bart!	
			\hfill\hbox {\citet[221]{Rinas2006}}
\end{exe}
\vspace{-0.65cm}
\begin{exe}
	\ex\label{278a}
	\scriptsize
	 Bevor der Unterricht begann, mussten sich die Mädchen Kittelschürzen anziehen und die Jungen Westen, denn früher durften sich die Kinder nicht 			schmutzig machen. Außerdem bekamen wir alle Holzpantinen. Dann zeigte uns das Fräulein Lehrerin den Klassenraum. \glqq \textbf{Der ist \underline{ja} 		klein!}\grqq{}, riefen einige Kinder. 
	\hfill\hbox {(BRZ09/APR.06684 Braunschweiger Zeitung, 17.04.2009)} 
\end{exe}	
Im Einvernehmen mit Thurmairs satzmodaler Schnittmengenbedingung lassen sich \textit{ja} und \textit{doch} in dieser Domäne aufgrund der leeren Schnittmenge hinsichtlich ihres Einzelauftretens in Satzexklamativsätzen nicht kombinieren.\footnote{Interessanterweise findet man in diesem Typ von Exklamativsatz gelegentlich die Partikel \textit{mal} (vgl. (\ref{278})).

\begin{exe}
	\ex\label{278} 
	Dick sein ist doch gesund?\\
	Na, \textbf{das ist \underline{ja mal} eine Schlagzeile.} \glqq Übergewichtige Patienten leben länger\grqq{}.\\ Zu diesem Schluss kommt ein deutscher 		Arzt bei der diesjährigen Jahrestagung der \glqq Österreichischen Adipositas Gesellschaft\grqq{}. 	
	\newline
	\hbox{}\hfill\hbox {(BVZ08/NOV.02111 Burgenländische Volkszeitung, 26.11.2008)}
\end{exe}
Parallele Fälle findet man auch mit \textit{doch} (vgl. (\ref{279})).

\begin{exe}
	\ex\label{279} 
	STUTTGART Udo Jürgens statt Queen – \textbf{das ist \underline{doch mal} ein Tausch!} Ab Herbst 2010 soll nach Medienberichten \glqq Ich war noch 			niemals in New York\grqq{} in Stuttgart aufgeführt werden und die \glqq We Will Rock You\grqq{}-Show ersetzen [...].      	
	\newline
	\hbox{}\hfill\hbox {(HMP09/DEZ.00272 Hamburger Morgenpost, 03.12.2009)}
\end{exe}					             
Im Einklang mit Thurmairs Bedingung über zulässiges kombiniertes Auftreten halte ich die Kombination aus \textit{ja} $\plus$ \textit{doch} $\plus$ \textit{mal} (vgl. (\ref{280}) und (\ref{281})) wieder für grammatisch.

\begin{exe}
	\ex\label{280} 
	Na, \textbf{das ist \underline{ja doch mal} eine Schlagzeile.} \glqq Übergewichtige Patienten leben länger\grqq{}.
\end{exe}
\vspace{-0.4cm}
\begin{exe}
	\ex\label{281} 
	Udo Jürgens statt Queen – \textbf{das ist \underline{ja doch mal} ein Tausch!} 
\end{exe}
An dieser Stelle kann ich den Verweis auf derartige akzeptable Strukturen nur als Beobachtung stehen lassen und muss offen lassen, welchen Einfluss das \textit{mal} in diesem Kontext nimmt und welche Konsequenzen die Existenz dieser Struktur für die weitere Untersuchung der Abfolge von \textit{ja} und \textit{doch} in Kombinationen nimmt. Für die Betrachtung in Abschnitt~\ref{sec:unmarkiert} würde dies bedeuten, dass sie auch derartige exklamative Satztypen und damit deren Effekt auf den Äußerungskontext in die Analyse aufnehmen müsste.

Ähnliche Beobachtungen lassen sich dazu auch für Wunschsätze machen. Wie schon gezeigt, kann \textit{doch} im Gegensatz zu \textit{ja} in diesem Satzkontext auftreten (vgl. (\ref{282}) und (\ref{283})).

\begin{exe}
	\ex\label{282} 
	Wenn \textbf{doch} schon alles vorbei wäre!
	\hfill\hbox {\citet[140]{Hentschel1986}}
\end{exe}
\vspace{-0.5cm}
\begin{exe}
	\ex\label{283} 
	*Wenn \textbf{ja} schon alles vorbei wäre!
\end{exe}
Tritt \textit{mal} hinzu, scheint das Auftreten von \textit{ja} wiederum lizensiert (vgl. (\ref{284})).

\begin{exe}
	\ex\label{284} 
	Guten Morgen Perfektes Wetter...\\
	\textbf{Wenn das \underline{ja mal} immer so ginge!} Aber für unsere Leserinnen und Leser tun wir schließlich alles.
	\hfill\hbox{(RHZ05/OKT.11812 Rhein-Zeitung, 10.10.2005)}	
\end{exe}
Da \textit{doch} ebenfalls in dieser Umgebung stehen kann (vgl. (\ref{285})), steht auch dem gemeinsamen Auftreten von \textit{ja} und \textit{doch} nichts im Wege (vgl. (\ref{286}) und (\ref{287})).
		
\begin{exe}
	\ex\label{285} 
	Aufrechter sitzen! Mitschwingen! Schultern mitnehmen! Diese Signalwörter kennt wohl jeder aus dem Unterricht. Ach, \textbf{wenn das \underline{doch 		mal} so einfach wäre!} 
	\newline
	\hbox{}\hfill\hbox{(www.pferdemarkt.de/?page\_id=292999), (Google-Suche, eingesehen am 07.05.2013)}	
\end{exe}		

\begin{exe}
	\ex\label{286} 
	Wenn das \underline{\textbf{ja doch mal}} immer so ginge!
\end{exe}
\vspace{-0.4cm}
\begin{exe}
	\ex\label{287} 
	Ach, wenn das \underline{\textbf{ja doch mal}} so einfach wäre!
\end{exe}
Eine Erklärung muss an dieser Stelle offen bleiben.} Die Genera\-lisierung, dass \textit{ja} und \textit{doch} in Satzexklamativsätzen nicht kombiniert werden können, bliebe ebenfalls bestehen, wenn auch \textit{ja} (wie Thurmair argumentiert) in dieser Umgebung nicht auftreten könnte. 
									        
Aus dieser Verteilung ergibt sich, dass Exklamativsätze für \textit{ja} und \textit{doch} keine Umgebung darstellen, in der sie sich kombinieren können.\footnote{\label{Fn4}An dieser Stelle sei angemerkt, dass Thurmairs Unterscheidung zwischen w- und Satzexklamativsätzen auf formalen und nicht interpretatorischen Kriterien beruht. Aus semantischer Perspektive ist es eine gängige Annahme, Exklamativsätze, die ein Staunen über das Wie (d.h. zu welchem Grad/Ausmaß etwas der Fall ist) ausdrücken, zu unterscheiden von solchen, die ein Staunen über das Dass ausdrücken (d.h. über die Tatsache, dass ein Sachverhalt an sich gültig ist). w-Exklamative können prinzipiell nur Erstaunen über das Wie ausdrücken. Die angeführten Satzexklamative sind jedoch nicht (wie man vermuten könnte) der Lesart \glq staunen, dass\grq {} zuzuordnen. Auch sie drücken Erstaunen über Grad oder Ausmaß aus. Aufgrund der Distribution von \textit{ja} und \textit{doch} in Bezug auf diese zwei Interpretationen von Exklamativsätzen kann die Annahme aus \citet[161]{Hentschel1986}, dass sowohl \textit{ja} als auch \textit{doch} nicht auf das Wie, sondern nur das Dass Bezug nehmen können, nicht aufrechterhalten werden.}
	
Für \textit{dass}-Exklamativsätze \is{dass-Exklamativsatz} wie in (\ref{288}) gilt, dass \textit{doch}, aber nicht \textit{ja} auftreten kann (vgl. (\ref{289}) und (\ref{290})) (vgl. auch \citealt[109, Fn 49]{Doherty1985}, \citealt[152]{Zaefferer1988}, \citealt[135]{Meibauer1994}, \citealt[222]{Kwon2005}).

\begin{exe}
	\ex\label{288} 
		\begin{xlist}	
			\ex\label{288a} Daß ich dich hier treffen würde!
			\ex\label{288b} Daß ich das noch erleben muß!
		\end{xlist}
\end{exe}

\begin{exe}
	\ex\label{289} 
		\begin{xlist}	
			\ex\label{289a} Daß du dir das \textbf{doch} nie merken kannst!
			\hfill\hbox {\citet[56]{Thurmair1989}}
			\ex\label{289b} *Dass du dir das \textbf{ja} nie merken kannst!
		\end{xlist}
\end{exe}

\begin{exe}
	\ex\label{290} 
		\begin{xlist}	
			\ex\label{290a} Daß der mir \textbf{doch} die Vorfahrt nimmt!
			\ex\label{290b} *Dass der mir \textbf{ja} die Vorfahrt nimmt!
			\hfill\hbox {\citet[152]{Zaefferer1988}}
		\end{xlist}
\end{exe}
\citet[40-41]{Altmann1987} geht stattdessen davon aus, dass \textit{ja} und \textit{doch} in \textit{dass}-VL-Exklamativsätzen \is{dass-VL-Exklamativsatz} gar nicht (und somit auch nicht kombiniert) stehen können. In der Übersicht in \citet[59]{Karagjosova2004} findet man in dieser Domäne auch \textit{ja}. In ihrer Arbeit führt sie für diese Auftretensweise aber kein Beispiel an. Die Suche nach authentischen Belegen für \textit{ja} in dieser Umgebung im DeReKo und COW-Korpus (wobei es sich um die größten mir zugänglichen Korpora handelt), gestaltet sich erfolglos.

\textit{Dass-ja}-VL-Exklamativsätze sind anscheinend nicht aufzufinden und mit Ausnahme von \citet{Karagjosova2004} ist mir keine Arbeit bekannt, die von ihrer Existenz ausgeht. Allerdings ist an dieser Stelle anzumerken, dass auch \textit{doch}-VL-Exklamativsätze nur wenig frequent auftreten. Über das COW-Korpus lassen sich lediglich acht Treffer ausfindig machen, im DeReKo befindet sich gar kein Beleg. Und selbst da, wo Belege ausfindig zu machen sind, handelt es sich überwiegend um alte Texte (vgl. (\ref{291}) bis (\ref{293})).
 
\begin{exe}
	\ex\label{291} 
	\scriptsize
		An meiner Beschreibung dazu war ein Druckfehler, der die Säulen des Napoleon mit der troianischen in Rom verglich, das interessanteste. 					\textbf{Daß \underline{doch} immer die Setzer die witzigsten sind!}
			\newline
			\hbox{}\hfill\hbox{(http://www.hausen-im-wiesental.de/jphebel/briefe/brief\_hendel\_sch\%C3\%BCtz\_1809\_I.htm)}
			\newline
			\hbox{}\hfill\hbox{(1809: Brief von J.P. Hebel)}	
\end{exe}
\vspace{-0.65cm}
\begin{exe}
	\ex\label{292} 
	\scriptsize
		Daß mir \textbf{doch} dies alles so lebendig geblieben ist!
			\newline
			\hbox{}\hfill\hbox{(http://bfriends.brigitte.de/foren/allgemeines-forum/89544-gedicht-des-tages-17-a-18-print.html)}
			\newline
			\hbox{}\hfill\hbox{(19./erstes Drittel 20 Jhd.: Arno Holz)}		 
\end{exe}
			   
\begin{exe}
	\ex\label{293} 
	\scriptsize
		Täglich wird mir die Geschichte theurer. Ich habe diese Woche eine Geschichte des dreißigjährigen Kriegs gelesen, und mein Kopf ist mir noch ganz warm davon. \textbf{Daß \underline{doch} die Epoche des höchsten Nationen-Elends auch zugleich die glänzendste Epoche menschlicher Kraft ist!} Wie viele große Männer giengen aus dieser Nacht hervor! 		
			\newline
			\hbox{}\hfill\hbox{(http://www.kuehnle-online.de/literatur/schiller/briefe/1786/178604152.htm)} 
			\newline
			\hbox{}\hfill\hbox{(1786: Brief von Schiller)}			
\end{exe}
Die Struktur scheint folglich antiquiert, was vermutlich auch für den Satztyp \textit{dass}-VL-Exklamativsatz insgesamt angenommen werden kann. Auch \citet[140]{Hentschel1986} merkt schon an, dass \textit{dass-doch}-VL-Sätze außer in einigen wenigen konventionalisierten Verwendungen archaisch sind.

Ein weiterer Satztyp, in dem \textit{doch} auftritt und für den Thurmair annimmt, dass er zwar nicht unter die Exklamativsätze fällt, aber nah an diesen Typ heranreicht, ist in (\ref{294}) illustriert.

\begin{exe}
	\ex\label{294} 
	Stellt der \textbf{doch} glatt den Rotwein in den Kühlschrank!
			\hfill\hbox {\citet[115]{Thurmair1989}}
\end{exe}
Derartige Sätze weisen in der Regel V1-Stellung auf, V2-Stellung ist aber auch möglich (vgl. (\ref{295})).

\begin{exe}
	\ex\label{295} 
		\begin{xlist}	
			\ex\label{295a} (Da) LÄSST der sich \textbf{doch einfach} 'nen Bart wachsen!
			\ex\label{295b} BOHRT der sich \textbf{doch} in aller Öffentlichkeit in der Nase!
			\hfill\hbox {\citet[217]{Rinas2006}}
		\end{xlist}
\end{exe}
Die Partikel \textit{ja} kann in diesen Sätzen nicht auftreten wie (\ref{296}) zeigt.

\begin{exe}
	\ex\label{296} 
		\begin{xlist}	
			\ex\label{296a} *(Da) LÄSST der sich \textbf{ja einfach} 'nen Bart wachsen!
			\ex\label{296b} *BOHRT der sich \textbf{ja} in aller Öffentlichkeit in der Nase!
			\hfill\hbox {\citet[218]{Rinas2006}}
		\end{xlist}
\end{exe}
Wie von Thurmairs Schnittmengenbedingung vorhergesagt, ist aufgrund der Distribution von \textit{ja} und \textit{doch} das kombinierte Auftreten der beiden MPn ausgeschlossen (vgl. (\ref{297})).

\begin{exe}
	\ex\label{297} 
	*Bohrt der sich \textbf{ja doch} in aller Öffentlichkeit in der Nase!  
	\hfill\hbox {\citet[235]{Rinas2006}}
\end{exe}

\subsubsection{Emphatische Aussagen}
\label{sec:empha}
\citet{Thurmair1989} nimmt über die im letzten Abschnitt beschriebenen Satz- und w-Exklamativsätze hinaus keine weiteren Typen von Exklamativsätzen an. Doch verweist sie auf Sätze der Art \is{emphatische Aussage} in (\ref{298}) bis (\ref{302}).

\begin{exe}
	\ex\label{298} 
	Das Kind kommt vom Spielen heim. Die Mutter: \glqq Du blutest \textbf{ja}!\grqq{}
\end{exe}

\begin{exe}
	\ex\label{299} 
	Du hast \textbf{ja} grüne Augen!
\end{exe}	

\begin{exe}
	\ex\label{300} 
	Nelli: Feiert ihr vor Silvester noch ma?\\
	Bea: Ja!\\
	Nelli: Ach, du has \textbf{ja} Geburtstag!
\end{exe}	

\begin{exe}
	\ex\label{301} 
	Anna: Wir ham nach fünfeinhalb Jahren noch keins [Kind].\\
	Lisa: Mensch, ist das schon solange her?\\
	Anna: Ja!\\
	Lisa: Poh! Is \textbf{ja} Wahnsinn!
\end{exe}	
	
\begin{exe}
	\ex\label{302} 
	Da kommt \textbf{ja} der Heinz!
	\hfill\hbox {\citet[107-108/215]{Thurmair1989}}
\end{exe}
In den Klassifikationen mancher Autoren (vgl. z.B. \citealt[13]{Weydt1983b}, \citealt[157]{Hentschel1986}, \citealt[313]{Foolen1989}, \citealt[166-167]{Helbig1990}, \citealt[197]{Karagjosova2004}, \citealt[160-168]{Rinas2006}) zählen auch diese zu den Exklamativsätzen, die Staunen über das Dass ausdrücken.\footnote{Hier liegt ein weiterer konkreter Fall der anfänglich angeführten Problematik in der Angabe der Distribution von MPn und ihrer Kombinationen vor. Wenn die hier genannten Autoren Sätze der Art in (\ref{298}) bis (\ref{302}) als Exklamativsätze einstufen, ergeben sich für das Auftreten von \textit{ja} und \textit{doch} natürlich andere Klassifikationen. An dieser Stelle ist folglich Vorsicht geboten, wenn es zu entscheiden gilt, ob sich Ansätze tatsächlich widersprechen. Gerade im Bereich der Exklamativsätze gehen die Klassifikationen deutlich auseinander.} \citet[107-108]{Thurmair1989} zufolge (genauso auch \citealt[77-80]{Doherty1985}, \citealt[37]{Kwon2005}) liegen hier keine Exklamativsätze vor, sondern Aussagesätze – wenngleich sich die Funktion in Richtung Ausruf bewegt. Das gilt für alle Typen in (\ref{298}) bis (\ref{302}), der Typ in (\ref{298}) bis (\ref{300}) ist aufgrund der Unmöglichkeit der Kombination von \textit{ja} und \textit{doch} in dieser Umgebung für die weitere Untersuchung allerdings weniger interessant. Thurmair begründet ihre Entscheidung rein auf der Basis grammatischer Merkmale, d.h. Merkmale, die sie für Satzex\-klamativsätze als konstitutiv ansetzt, treffen nicht zu. Da sich die Formtypen, die Teil des Form-Funktionspaares des jeweiligen Satzmodus sind, ebenfalls auf der Basis bestimmter grammatischer Eigenschaften konstituieren, ist diese Argumentation plausibel und kohärent. So weisen Strukturen wie in (\ref{298}) bis (\ref{302}) keinen Exklamativakzent \is{Exklamativakzent} auf (wie etwa (\ref{303})).

\begin{exe}
	\ex\label{303} 
		\begin{xlist}	
			\ex\label{303a} Hast DU \textbf{aber} große Füße!
			\ex\label{303b} Hat DIE \textbf{vielleicht} einen kurzen Rock an!
			\ex\label{303c} Was hast DU \textbf{bloß} für komische Ansichten!
		\end{xlist}
\end{exe}	
V1-und V2-Stellung lässt sich nicht variieren (vgl. (\ref{304}) vs. (\ref{305})).

\begin{exe}
	\ex\label{304} 
		\begin{xlist}	
			\ex\label{304a} *Blutest du \textbf{ja}!	
			\ex\label{304b} *Hast du \textbf{ja} grüne Augen!
			\ex\label{304c} *Hast du \textbf{ja} Geburtstag!
			\ex\label{304d} *Ist das \textbf{ja} Wahnsinn!
			\ex\label{304e} *Kommt da \textbf{ja} der Heinz!
		\end{xlist}
\end{exe}
\begin{exe}
	\ex\label{305} 
		\begin{xlist}	
			\ex\label{305a} Hast DU eine schöne Wohnung!	
			\ex\label{305b} DU hast \textbf{(vielleicht)} eine schöne Wohnung!
		\end{xlist}
\end{exe}
Da sie für ihre beiden Typen von Exklamativsätzen \is{Exklamativsatz} ansetzt, dass sie die Sprecher\-einstellung kodieren, dass der Sprecher erstaunt ist, in welchem Maße etwas der Fall ist, handelt es sich auch aus dieser Perspektive nicht um Exklamativsätze. Sofern Staunen involviert ist, handelt es sich um das Staunen darüber, dass der beschriebene Sachverhalt zutrifft (aber vgl. Fußnote~\ref{Fn4}). Zuletzt kann \textit{ja} in den typi\-schen Exklamativsätzen (\textit{was für}/\textit{welch}/\textit{wie}-V2/VL) nicht auftreten (s.o. und (\ref{306})), was dann entsprechend merkwürdig erschiene, wenn es sich bei (\ref{298}) bis (\ref{302}) tatsächlich um Exklamativsätze handelte.

\begin{exe}
	\ex\label{306} 
		\begin{xlist}	
			\ex\label{306a} Was für eine Wohltat ist (*\textbf{ja}) dieses Buch!
			\ex\label{306b} Welch eine Simulation war (*\textbf{ja}) diese Wirklichkeit!	
			\hfill\hbox {\citet[37]{Kwon2005}}
		\end{xlist}
\end{exe}
Weitere Stützung für Thurmairs Entschluss, (\ref{298}) bis (\ref{302}) nicht zu den Exklamativsätzen zu zählen, liefert die Tatsache, dass das Kriterium des Staunens über das Dass zudem nicht einmal für alle diese Sätze gilt. Bei Äußerungen der Art in (\ref{301}) geht es nicht um den Ausdruck von Staunen, dass ein Sachverhalt besteht, sondern ein Sachverhalt wird bewertet. Der Sachverhalt, auf den sich die Be\-wertung bezieht, wird in dem Satz selbst nicht einmal ausgedrückt.
 
Für Thurmair sind die Sätze in (\ref{298}) bis (\ref{302}) \is{emphatische Aussage} \textit{emphatische Aussagen}. Z.T. finden sich hier auch zu (\ref{298}) bis (\ref{302}) parallele Fälle mit \textit{doch}. An dieser Stelle ist vor dem Hintergrund von Thurmairs satzmodaler Schnittmengenbedingung anzuführen, dass \textit{ja} und \textit{doch} in den Typen von emphatischen Aussagen, in denen sie prinzi\-piell in Isolation auftreten können, auch in Kombination zulässig sind. Dies trifft beispielsweise auf Bewertungen wie in (\ref{301}) zu. In derartigen Kontexten, in denen der Sprecher eine(n) erstaunliche(n) Vorgängerhandlung/Sachverhalt kommentiert/bewertet, kann \textit{doch} problemlos stehen (vgl. (\ref{307}) bis (\ref{309})).

\begin{exe}
	\ex\label{307} 
	Bin ich ja froh, dass das Kind nich mitgefahren war, stell dir das mal vor. Is doch Wahnsinn! (...) Das ist \textbf{doch} Wahnsinn!
\end{exe}
\vspace{-0.65cm}	
\begin{exe}
	\ex\label{308} 
	Das ist \textbf{doch} das Letzte!
\end{exe}	
\vspace{-0.65cm}
\begin{exe}
	\ex\label{309} 
	Jetzt nörgel nicht so rum! Das ist \textbf{doch} wirklich traumhaft hier!
	\newline
	\hbox{}\hfill\hbox{\citet[114-115]{Thurmair1989}}	
\end{exe}
Im Einklang mit Thurmairs Schnittmengenbedingung können sich \textit{ja} und \textit{doch} hier auch kombinieren. \citet[235]{Rinas2006} hält (\ref{310}) für fraglich. M.E. sind derartige Belege völlig akzeptabel.

\begin{exe}
	\ex\label{310} 
	?Das ist \textbf{ja doch} die Höhe!	
\end{exe}
Äußerungen der Art in (\ref{310}) lassen sich dazu recht einfach sowohl im heutigen Deutsch (vgl. (\ref{311}) und (\ref{312})) als auch in älteren Quellen finden (vgl. (\ref{313})).
\begin{exe}
	\ex\label{311} 
	\scriptsize
	\textbf{Oh man das ist \underline{ja doch} der Hammer.} Da sieht man mal wieder das Arzt nicht gleich Arzt ist. 
	\newline
	\hbox{}\hfill\hbox{(http://www.meerschwein-community.de/archive/index.php?thread-2607.html)}	
	\newline
	\hbox{}\hfill\hbox{(Google-Suche, eingesehen am 31.3.2013)}	
\end{exe}
	      
\begin{exe}
	\ex\label{312} 
	\scriptsize
	\textbf{Das ist \underline{ja doch} die Höhe}, meint eine großformatige, bunte deutsche Zeitung: \glqq Wie ein leichtes Mädchen\grqq{} habe sich Sarah 	Ferguson alias Fergie \glqq verkauft\grqq{}. 
	\hfill\hbox{(K96/NOV.24120 Kleine Zeitung, 03.11.1996)}	
\end{exe}
	
\begin{exe}
	\ex\label{313} 
	\scriptsize
	Es wurde ihm unbehaglich heiß.\\
	- \textbf{Aber das ist \underline{ja doch} niederträchtig!} Das ist ja Diebstahl! Pfui Teufel! ...
	- Und, wenn sie's beim Abrechnen merken? ...
	\newline
	\hbox{}\hfill\hbox{(Otto Julius Bierbaum: Stilpe – Kapitel 11, 1897) (eingesehen über: http://gutenberg.spiegel.de/)}	
\end{exe}								                       
Inwieweit sich die Funktion und Verwendung derartiger emphatischer Aussagen von denen von typischeren Aussagesätzen (vgl. Abschnitt~\ref{sec:synschnitt}) unterscheidet, wird in Abschnitt~\ref{sec:markiert} diskutiert. An dieser Stelle ist die Annahme entscheidend, dass diese Satztypen nicht zu den Exklamativsätzen zu rechnen sind, so dass an der Generalisierung festgehalten werden kann, dass sich \textit{ja} und \textit{doch} im Formtyp des Aussagesatzes kombinieren lassen.

\subsection{Semantische/pragmatische Schnittmengenbedingung}
Neben der im letzten Abschnitt erläuterten syntaktisch-distributionellen Schnitt\-mengenbedingung gibt es weitere Auftretensbeschränkungen, die Autoren (vgl. z.B. \citealt[218, 222-223]{Dahl1988}, \citealt[25-31]{Thurmair1991}) zu einer Verschärfung der Schnittmengenbedingung veranlasst haben. Es handelt sich hierbei um die Beobachtung, dass die Kompatibilität auch auf Ebene der Interpretation der Einzelpartikeln gegeben sein muss, damit die MP-Kombination zulässig ist. \citet[26-27]{Thurmair1991} führt beispielsweise die folgenden Sätze an.	 

\begin{exe}
	\ex\label{314} 
		\begin{xlist}	
			\ex\label{314a} Ist das Kleid \textbf{auch} durchsichtig?
			\ex\label{314b} Ist das Kleid \textbf{etwa} durchsichtig?
		\end{xlist}
\end{exe}

\begin{exe}
	\ex\label{315} 
		\begin{xlist}	
			\ex\label{315a} *Ist das Kleid \textbf{etwa auch} durchsichtig?
			\hfill\hbox {\citet[27]{Thurmair1991}}
			\ex\label{315b} *Ist das Kleid \textbf{auch etwa} durchsichtig?
		\end{xlist}
\end{exe}
Obwohl \textit{auch} und \textit{etwa} jeweils für sich in E-Fragesätzen auftreten können (vgl. (\ref{314})), ist ihre Kombination in diesem Satzmodus trotzdem ausgeschlossen (vgl. (\ref{315})). Verhältnisse wie in (\ref{314}) und (\ref{315}) legen deshalb nahe, dass eine geteilte satzmodale Umgebung als Kriterium nicht ausreichend ist. Thurmairs Anwendung der semantischen/pragmatischen Schnittmengenbedingung basiert auf den unterschiedlichen Antworterwartungen, die E-Fragen mit \textit{auch} bzw. \textit{etwa} mit sich bringen. Eine Frage wie (\ref{314a}) weist Thurmair zufolge eine positive Erwartung auf, d.h. sie legt die Reaktion der Zustimmung (bevorzugte Antwort  \textit{Ja.}) nahe. Die Frage in (\ref{314b}) hingegen bringt eine negative Erwartung mit sich, d.h. die bevorzugte Antwort ist \textit{Nein}. Die Unverträglichkeit der beiden MPn in E-Fragesät\-zen führt die Autorin nun darauf zurück, dass es keine sinnvolle Frage geben kann, die gleichzeitig eine positive und negative Antworterwartung aufweist. Auf dieser Ebene sind die Bedeutungen der beiden MPn inkompatibel und produzieren eine leere Schnittmenge.			        
								               
Auf ähnliche Art, wie das Kriterium der Kompatibilität der an einer Kombination beteiligten MPn in (\ref{315a}) und (\ref{315b}) über die rein satzmodale Verträglichkeit hinausgeht, lassen sich auch für das gemeinsame Auftreten von \textit{ja} und \textit{doch} restringiertere Auftretenskontexte formulieren als die (strukturelle) Domäne des Aussagesatzes. Im Folgenden wird dies aufgezeigt anhand der im letzten Abschnitt bereits angesprochenen \is{emphatische Aussage} emphatischen Aussagen. Diese Aussagen unterscheiden sich von typischeren Aussagen darin, dass sie auf funktionaler Ebene Eigenschaften mit Exklamativsätzen zu teilen scheinen, weil sie eine gewisse emotionale Färbung aufweisen, d.h. der Sprecher zu einem gewissen Grad Überraschung/Erstaunen über einen unmittelbar wahrgenommenen Sachverhalt zum Ausdruck bringt.
			   					         
In der Literatur werden kontroverse Annahmen zu der prinzipiellen Frage gemacht, ob \textit{doch} in Äußerungen, die sich durch eine gewisse Spontanität, den Bezug auf direkt Wahrgenommenes und den Bedeutungsas\-pekt der Überraschung charakterisieren, auftreten kann (vgl. für diese Annahme z.B. \citealt[26]{Weydt1969} (\textit{Kommst du \textbf{doch} spät!}), \citealt[37]{Helbig1977} (\textit{Kommst du \textbf{doch} unpünktlich!}, \textit{War das \textbf{doch} eine Überraschung!}), \citealt[116]{Helbig1990} (\textit{Du SCHNARCHST \textbf{doch}!}, \textit{Das war \textbf{doch} unsere ehemalige Studentin!})\footnote{Bei einigen der zitierten Strukturen handelt es sich um echte V1-Exklamativsätze, wenn man an den Kriterien zur Klassifikation von emphatischen Aussagen (s.o.) festhält. Dennoch eignen sich auch diese Daten dazu, zu diskutieren, ob man es bei der Interaktion von Staunen und dem MP-Beitrag von \textit{doch} mit einer semantischen/pragmatischen Inkompatibilität zu tun hat, die das kombinierte Auftreten von \textit{ja} und \textit{doch} in den emphatischen Aussagen deshalb unterbindet.} vs. dagegen \citealt[141]{Hentschel1986} (Beispiele von \citealt[26]{Weydt1969}, \citealt[37]{Helbig1977} [s.o.], \citealt[216]{Rinas2006} (*\textit{Du ISST \textbf{doch} gar nichts!})). \citet[141]{Hentschel1986} zufolge handelt es sich hierbei \glqq um einen veralteten Gebrauch von \textit{doch} [...], der zwar möglich, in der modernen Umgangssprache aber nicht mehr zu beobachten ist\grqq{}. Ob \textit{doch} in dieser Umgebung tatsächlich kate\-gorisch ausgeschlossen werden kann oder den Intuitionen aus \citet{Weydt1969} und \citet{Helbig1977, Helbig1990} zuzustimmen ist, bleibt noch zu klären (s.u.). Die Einschätzung der genannten Autoren, die sich gegen die Verträglichkeit von \textit{doch} und dem Aspekt des Erstaunens äußern, läuft konform mit der Annahme aus \citet[193]{Lindner1991}, dass \textit{doch} (anders als \textit{ja}) nicht auftreten kann, wenn die Äußerung das Bedeutungsmoment der Überraschung trägt. Die Beobachtung Lindners basiert auf der Adäquatheit der \textit{ja}-Äußerung und der Inadäquatheit der \textit{doch}-Äußerung im Kontext in (\ref{316}).  
	         
\begin{exe}
	\ex\label{316} 
	Im Kaufhaus: Ein kleines Mädchen betrachtet sich mit ihrem neuen Kleid im Spiegel. Ihre Mutter beobachtet sie.\\
	Mutter (überrascht):\\
	Das ist \textbf{ja}/*\textbf{doch}/*\textbf{ja doc}h ein hübsches Kleid. 
	\hfill\hbox {nach \citet[193]{Lindner1991}}
\end{exe}							
Angenommen, \textit{doch} kann in Kontexten, die die Komponente der Überraschung beinhalten, (anders als \textit{ja}) tatsächlich nicht auftreten, hätte man es hier wo\-möglich mit einem Fall zu tun, bei dem die beiden MPn nicht syntaktisch, sondern (ähnlich wie in (\ref{315a}) und (\ref{315b})) semantisch/pragmatisch inkompatibel sind, da nicht beide mit dem Bedeutungsas\-pekt von Überraschung verträglich sind. Im Einvernehmen mit dieser auf die Kompatibilität von Interpretation Bezug nehmenden (leeren) Schnittmengenbildung lassen sich \textit{ja} und \textit{doch} in dieser Umgebung auch nicht kombinieren.

Die Einschätzung Lindners hinsichtlich der Adäquatheit der \textit{ja}-, \textit{doch}- bzw. \textit{ja doch}-Äußerungen in (\ref{316}) ist zu teilen. Dennoch stellt sich die Frage, ob hier tatsächlich eine prinzipielle Unverträglichkeit der MP \textit{doch} mit dem Bedeutungsaspekt der Überraschung vorliegt. Dies scheint schon unplausibel vor dem Hintergrund, dass \textit{doch} in w-Exklamativsätzen (wie gesehen in Abschnitt~\ref{sec:exkl}) im Gegensatz zu \textit{ja} auftreten kann. Betrachtet man verschiedene Typen von emphatischen Aussagen (vgl. (\ref{298}) bis (\ref{302})), für die anzunehmen ist, dass ein Überraschungsmoment/Erstaunen/Spontanität beteiligt ist (s.o.), spricht die Datenlage für eine subtilere Verteilung von \textit{doch} in derartigen \glq Staunens\grq {}kontexten.  

In emphatischen Aussagen der Art in (\ref{317}) scheint \textit{doch} tatsächlich nicht auftre\-ten zu können. Für \textit{ja} lassen sich hier leicht Belege finden (vgl. z.B. (\ref{318}) und (\ref{319}) für Strukturen der Art \textit{Der hat \textbf{ja} x!}). 

\begin{exe}
	\ex\label{317}
	Du hast \textbf{ja} grüne Augen! 
	\hfill\hbox {\citet[108]{Thurmair1989}}
\end{exe}

\begin{exe}
	\ex\label{318}
	\scriptsize 
	Das war kein Gewitter! Das war Müller, und seine hellblauen Augen funkelten wie Laserblitze.
	\glqq Auweia, wie sieht der denn aus? \textbf{Der hat \underline{ja} weiße Wimpern!}\grqq{}, stellte Alfi fest.
	\newline
	\hbox{}\hfill\hbox{(RZ08/AUG.07411 Braunschweiger Zeitung, 16.08.2008)}	
\end{exe}

\begin{exe}
	\ex\label{319}
	\scriptsize 
	Bei der Kanzel müssen die Eltern ihre Kerzen hochhalten, damit alle die Mutter Gottes auf der Mondsichel sehen können. Beim Engel der Hoffnung fällt 		einem Kind auf: \glqq \textbf{Der hat \underline{ja} einen Anker in der Hand!}\grqq{}      
	\hfill\hbox{(BRZ09/APR.07349 Braunschweiger Zeitung, 18.04.2009)}	
\end{exe}							                                      							      
Für \textit{doch} liefern parallele Suchanfragen zwar durchaus Treffer (vgl. z.B. (\ref{320})). Allerdings deutet man in der Tat keine der zu findenden \textit{doch}-Äußerungen als spontane Reaktion des Erstaunens über das direkt Wahrgenommene.

\begin{exe}
	\ex\label{320}
	\scriptsize 
	Seinen großen Auftritt bei dieser WM hatte Thomas Hitzlsperger beim Achtelfinale gegen Schweden. Hä? Hitzlsperger?! \textbf{Der hat \underline{doch} 		gar nicht gespielt!}     
	\newline
	\hbox{}\hfill\hbox{(HMP06/JUN.02920 Hamburger Morgenpost, 27.06.2006)}	
\end{exe}						   
Die gleiche Beobachtung trifft auch auf strukturell anders, funktional aber sehr ähnliche Äußerungen zu. Strukturen des Musters \textit{Du bist ja x!} finden sich recht einfach (vgl. (\ref{321}) und (\ref{322})). Mit der gleichen Suchanfrage lassen sich jedoch keine vergleichbaren \textit{doch}-Äußerungen finden.

\begin{exe}
	\ex\label{321}
	\scriptsize 
	 \glqq Ist alles in Ordnung mit dir, Missy? \textbf{Du bist \underline{ja} ganz weiß im Gesicht!}\grqq{}    
	\newline
	\hbox{}\hfill\hbox{(BRZ08/JUL.06127 Braunschweiger Zeitung, 11.07.2008)}	
\end{exe}
\vspace{-0.65cm}
\begin{exe}
	\ex\label{322}
	\scriptsize 
	 \glqq \textbf{Du bist \underline{ja} völlig außer Atem!} Was ist passiert?\grqq{}    
	\newline
	\hbox{}\hfill\hbox{(DIV/APR.00001 Planert, Angela: Rubor Seleno. – Föritz, 2005 $[$S. 46$]$)}	
\end{exe}	 
Es ist nicht möglich, \textit{ja} in den belegten Beispielen durch \textit{doch} zu ersetzen unter Beibehaltung der Interpretation der Äußerung im Kontext (vgl. (\ref{323}) bis (\ref{326})). Und auch die Kombination aus \textit{ja} und \textit{doch} ist in diesen Kontexten auszuschließen.

\begin{exe}
	\ex\label{323}
	\glqq Auweia, wie sieht der denn aus? \#\textbf{Der hat \underline{doch}/\underline{ja doch} weiße Wimpern!}\grqq{}, stellte Alfi fest.
\end{exe}	

\begin{exe}
	\ex\label{324}
	Beim Engel der Hoffnung fällt einem Kind auf: \glqq \#\textbf{Der hat \underline{doch}/\underline{ja doch} einen Anker in der Hand!}\grqq{}
\end{exe}	

\begin{exe}
	\ex\label{325}
	\glqq Ist alles in Ordnung mit dir, Missy? \#\textbf{Du bist \underline{doch}/\underline{ja doch} ganz weiß im Gesicht!}\grqq{}
\end{exe}

\begin{exe}
	\ex\label{326}
	\glqq \#\textbf{Du bist \underline{doch}/\underline{ja doch} völlig außer Atem!} Was ist passiert?\grqq{}
\end{exe}
Hinsichtlich dieses Typus von emphatischer Aussage \is{emphatische Aussage} ist den Annahmen der oben angeführten Autoren hinsichtlich der Verträglichkeit von \textit{doch} und dem Aspekt der Überraschung somit zuzustimmen. 

Für den Typ der emphatischen Aussage in (\ref{327}) finden sich allerdings durchaus auch \textit{doch}-Belege.
	
\begin{exe}
	\ex\label{327} 
	Da kommt \textbf{ja} der Heinz!
	\hfill\hbox {\citet[215]{Thurmair1989}}
\end{exe}		            		
(\ref{328}) und (\ref{329}) sind Beispiele aus der Literatur.

\begin{exe}
	\ex\label{328} 
	Das ist \textbf{doch} mein ehemaliger Klassenlehrer!	
	\hfill\hbox {\citet[196]{Rinas2006}}
\end{exe}
\vspace{-0.65cm}
\begin{exe}
	\ex\label{329} 
	Das sind \textbf{doch} Klaus und Maria!	
	\hfill\hbox {\citet[86]{Dahl1988}}
\end{exe}											         
Funktional entsprechen die Sätze in (\ref{328}) und (\ref{329}) den emphatischen Aussagen aus (\ref{317}) bis (\ref{319}). Äußert ein Sprecher einen Satz der Art in (\ref{328}) oder (\ref{329}), bringt er in der Situation unmittelbar Erstaunen/Überraschung zum Ausdruck, die ge\-nannten Personen zu sehen/zu erkennen, genauso wie er mit Äußerungen wie in (\ref{317}) bis (\ref{319}) und (\ref{321}) und (\ref{322}) mit Erstaunen unmittelbar wahrnimmt und spontan die Erkenntnis/Beobachtung ausdrückt, dass die Person außer Atem ist, weiß im Gesicht ist, weiße Wimpern hat etc. Für diese Art der emphatischen Aussage finden sich auch authentische Belege (vgl. z.B. (\ref{330}) und (\ref{331})), die weder auf das Muster \textit{Das ist doch + Eigenname/definite Kennzeichnung!} noch auf die Struktur \textit{Das ist doch x!} beschränkt sind, wie (\ref{332}) und (\ref{333}) illustrieren.
	
\begin{exe}
	\ex\label{330}
	\scriptsize 
	Da kommt jemand hinter dem nächsten Pfeiler hervor und geht geradewegs auf mich zu. Es ist eine kleine rundliche Frau mit einem auffallend rosigen 			Gesicht. Sie trägt eine dunkelrote Wintermütze. \textbf{Das ist \underline{doch} Fräulein M.!}     
	\hfill\hbox{(NUZ05/FEB.02073 Nürnberger Zeitung, 18.02.2005)}	
\end{exe}	
	
\begin{exe}
	\ex\label{331}
	\scriptsize 
	Der Reisegruppe aus Thüringen ist sehr schnell klar, daß da politische Prominenz anrollt: Ein Troß Männer in dunklen Anzügen, umrandet von sportlichen 	Sicherheitsleuten [...].  \glqq \textbf{Das ist \underline{doch} der von der FDP!}\grqq{}, schallt es aus der Thüringer Touristengruppe, als der 			hessische Ministerpräsident und seine Begleitung vorbeihetzen.	     
	\hfill\hbox{(R99/JUL.54611 Frankfurter Rundschau, 09.07.1999)}	
\end{exe}				
							
\begin{exe}
	\ex\label{332}
	\scriptsize 
	Es ist die zweite Gemeinsamkeit, die Auma am Bruder entdeckt, den sie erst mit Ende 20 kennenlernt, weil er bei der Mutter auf Hawaii aufwächst, sie beim Vater in Kenia. Die erste flatterte mit der Post aus Amerika ins Haus. \glqq \textbf{Das ist \underline{doch} Vaters Schrift!}\grqq{}, durchfuhr es sie, als sie 1984 Baracks ersten Brief in der Hand hielt.        
	\hfill\hbox{(HMP10/SEP.02590 Hamburger Morgenpost, 26.09.2010)}	
\end{exe}
		
\begin{exe}
	\ex\label{333}
	\scriptsize 
	Mensch, \textbf{da geht \underline{doch} der Johnny Depp!} Kreisch, das ist er! – ruhig Blut, er ist es nicht.    
	\newline
	\hbox{}\hfill\hbox{(RHZ09/DEZ.02188 Rhein-Zeitung, 03.12.2009)}	
\end{exe}	                                                                
Da das \textit{doch} in (\ref{330}) bis (\ref{333}) jeweils problemlos durch \textit{ja} ersetzt werden kann, ist auch die Kombination wieder möglich. Beide MPn scheinen isoliert in diesem \glq Überraschungs\grq {}kontext auftreten zu können, so dass auch der Kombination nichts im Wege steht (vgl. (\ref{334}) bis (\ref{337}) sowie den Beleg in (\ref{338})).

\begin{exe}
	\ex\label{334}
	\scriptsize 
	Da kommt jemand hinter dem nächsten Pfeiler hervor und geht geradewegs auf mich zu. [...] \textbf{Das ist \underline{ja doch} Fräulein M.!}	
\end{exe}	

\begin{exe}
	\ex\label{335}
	\scriptsize 
	Der Reisegruppe aus Thüringen ist sehr schnell klar, daß da politische Prominenz anrollt: [...]  \glqq \textbf{Das ist \underline{ja doch} der von der FDP!}			\grqq{}, schallt es aus der Thüringer Touristengruppe [...].
\end{exe}

\begin{exe}
	\ex\label{336}
	\scriptsize 
	Es ist die zweite Gemeinsamkeit, die Auma am Bruder entdeckt [...]. Die erste flatterte mit der Post aus Amerika ins Haus. \glqq \textbf{Das ist 			\underline{ja doch} Vaters Schrift!}\grqq{}, durchfuhr es sie, als sie 1984 Baracks ersten Brief in der Hand hielt.
\end{exe}
	
\begin{exe}
	\ex\label{337}
	\scriptsize 
	Mensch, \textbf{da geht \underline{ja doch} der Johnny Depp!} Kreisch, das ist er! – ruhig Blut, er ist es nicht.
\end{exe}

\begin{exe}
	\ex\label{338}
	\scriptsize 
	Re: Kirmes in Wissel 2012\\
	\glqq Antwort \# 2 am: 24. Juli2012, 19:22:11\grqq{}\\
	\textbf{Das ist \underline{ja doch} der Star Light Petter, der Autoscooter.} Der sieht auch immer wieder klasse aus. Danke für die Bilder!	
	\hfill\hbox{(http://rummelforum.de/index.php?topic=8294.0)}
	\newline	
	\hbox{}\hfill\hbox{(Google-Suche, eingesehen am 30.05.2013)}	
\end{exe}									       
Der Beitrag in (\ref{338}) folgt unmittelbar als Reaktion auf eine Reihe von Fotos der Kirmes mit der einzigen Information desjenigen, der die Bilder eingestellt hat, dass und wann er diese Kirmes besucht hat, zzgl. einiger weniger Angaben zum Ort. D.h. andere Aspekte der Kirmes (insbesondere den Autoscooter betreffend) werden nicht diskutiert. Es lässt sich somit annehmen, dass die \textit{ja doch}-Äußerung in (\ref{338}) ebenso wie die Sätze in (\ref{334}) bis (\ref{337}) direkt Bezug nimmt auf das unmittelbar zuvor gesehene Bild und der Sprecher das Sehen des bestimmten Autoscooter spontan und unter Beteiligung eines gewissen Erstauntseins über das Wahrgenommene zum Ausdruck bringt.

Ein weiterer (möglicher) Fall, bei dem man es im Kontext der Kombination von \textit{ja} und \textit{doch} aufgrund der semantischen/pragmatischen Verteilung der beiden Partikeln mit einer leeren Schnittmenge zu tun hat, findet sich in (\ref{339}).

\begin{exe}
	\ex\label{339} 
		\begin{xlist}	
			\ex\label{339a} Da IST er \textbf{ja}!
			\ex\label{339b} \glqq Da bist du \textbf{ja}, Walter, ich dachte schon, du bist zu deinem Campari verschwunden!\grqq{}  	         
			\hfill\hbox {Frisch 1957: 136, zitiert nach \citet[167]{Rinas2006}}
		\end{xlist}
\end{exe}										      	     
Die Verwendung von \textit{ja} ist in diesem Kontext obligatorisch (vgl. (\ref{340})). 

\begin{exe}
	\ex\label{340} 
		\begin{xlist}	
			\ex\label{340a} Da IST er \textbf{ja}!/*Da IST er.
			\ex\label{340b} Weißt du, wo mein Stift ist? Ich kann ihn nirgendwo finden. – Ach, da IST er \textbf{ja}!/*Ach, da IST er!	         
			\hfill\hbox {\citet[167]{Rinas2006}}
		\end{xlist}
\end{exe}
Die MP kann \citet[168]{Rinas2006} zufolge nur ausgelassen werden, wenn der Haupt\-akzent verschoben wird (vgl. (\ref{341})). 

\begin{exe}
	\ex\label{341} 
	Weißt du, wo mein Stift ist? Ich kann ihn nirgendwo finden. – Ach, DA ist er!
	\hfill\hbox{\citet[168]{Rinas2006}}	
\end{exe}
Nach \citet[217]{Rinas2006} kann \textit{doch} in dieser Umgebung nicht auftreten, d.h. in den angeführten Verwendungen lässt sich \textit{ja} nicht durch \textit{doch} austauschen. Interessanterweise ist der interpretatorische Faktor, den Rinas als entscheidend ausmacht, wiederum das Staunen. Auch in Abgrenzung zum akzeptablen (\ref{342}) kann ihm zufolge \textit{doch} dann nicht auftreten, wenn der Satz allein Staunen ausdrücken soll. 

\begin{exe}
	\ex\label{342} 	
	Da IST er \textbf{ja}/*\textbf{doch}!	
	\hfill\hbox {\citet[217]{Rinas2006}}
\end{exe}											     
In Isolation oder mit wenig Kontext erscheint mir ein Urteil über eine \textit{da} $\plus$ Kopula $\plus$ Subjekt $\plus$ \textit{doch}-Äußerung schwierig, Recherchen bestätigen Rinas Annahme jedoch. Es lassen sich Belege für \textit{ja}-Äußerungen der Art in (\ref{340}) finden wie z.B. in (\ref{343}).

\begin{exe}
	\ex\label{343} 
	\scriptsize
	Ein elektronischer Poltergeist\\
	Es war genau 23 Uhr, als ein leises Piepsen mich aufschrecken ließ. Es kam vom Telefontisch her, steigerte sich in Lautstärke und Frequenz und 				verstummte erst nach einer Minute. Ich wunderte mich kurz und vergaß es über dem extrakniffligen Kreuzworträtsel. Bis zum nächsten Abend um 23 Uhr. Da 	piepste es wieder. Ich hob den Telefonhörer ab, drückte sämtliche Knöpfe auch an Feststation und Anrufbeantworter. [...]
	Vielleicht mal die Akkus tauschen, meinte er. Oder den Gerätehersteller fragen. Das mit den Akkus erwies sich als glatte Fehlinvestition. Es piepste 		weiter. Was Kundenservice, Konstruktionsabteilung, Verkaufsleitung und PR-Zentrale von Siemens glatt in Abrede stellten. – Aber es piepste. So langsam 	wollte ich schon an Poltergeister glauben. Bis gestern Abend, 23 Uhr. Da betrat meine Frau das Wohnzimmer, hörte das Piepsen, rollte das 					Telefontischchen beiseite, bückte sich und präsentierte ihn, mit einem strahlenden \glqq \textbf{Da ist er \underline{ja}!}\grqq{}: den seit Tagen 			vermissten Mini-Funkwecker... 
	\hfill\hbox{(RHZ01/OKT.07303 Rhein-Zeitung, 10.10.2001)}	
\end{exe}					
Die MP \textit{ja} scheint mir hier in der Tat nicht durch \textit{doch} ersetzt werden zu können. Im Einklang mit der Schnittmengenbedingung ist eine \textit{ja doch}-Äußerung in allen Kontexten, in denen \textit{doch} alleine nicht auftreten kann, ebenfalls nicht akzeptabel. Dies lässt sich durch Modifikation von (\ref{343}) nachweisen (vgl. (\ref{344})).

\begin{exe}
	\ex\label{344} 
	\scriptsize
	Da betrat meine Frau das Wohnzimmer, hörte das Piepsen, rollte das Telefontischchen beiseite, bückte sich und präsentierte ihn, mit einem strahlenden  	\glqq *\textbf{Da ist er \underline{ja doch}!}\grqq{}: den seit Tagen vermissten Mini-Funkwecker... 
\end{exe}	
Da auch hier der Bedeutungsaspekt \glq Staunen\grq {} als relevant ausgemacht wird, stellt sich die Frage, was die emphatischen Aussagen des Typs in (\ref{345}) und (\ref{346}) von den anderen angeführten Kontexten unterscheidet und was die Lizensierung des \textit{doch} – trotz der Bedeutungsaspekte von Erstaunen/Überraschung/spontaner Reaktion auf direkt Wahrgenommenes – lizensiert. Ich habe derzeit keine Antwort auf diese Frage.

\begin{exe}
	\ex\label{345} 
	Das ist \textbf{doch} mein ehemaliger Klassenlehrer!
	\hfill\hbox {\citet[196]{Rinas2006}}
\end{exe}
\vspace{-0.65cm}
\begin{exe}
	\ex\label{346} 
	Das sind \textbf{doch} Klaus und Maria!	
	\hfill\hbox {\citet[86]{Dahl1988}}
\end{exe}											         
Die in diesem Abschnitt betrachteten Daten zeigen deutlich, dass auch für die MPn \textit{ja} und \textit{doch} gilt, dass ihre prinzipielle Kombinierbarkeit nicht ausschließlich durch eine satzmodale Kompatibilität gesteuert wird, sondern die Forderung nach Kompatibilität auf interpretatorischer Ebene verschärft sein kann. Diese (zweite) semantische/pragmatische Schnittmengenbedingung greift, wenn die (erste) syn\-taktisch-distributionelle Voraussetzung bereits erfüllt ist. Die Verhältnisse bei der Kombination von \textit{ja} und \textit{doch}, wie hier beschrieben, entsprechen somit den Verhältnissen anderer Kombinationen, anhand derer \citet[26-27]{Thurmair1991} (vgl. auch \citealt[Kapitel 3]{Thurmair1989}) das gestaffelte Greifen dieser Schnittmengenbedingungen aufzeigt. 

Die Betrachtung in Abschnitt~\ref{sec:distributionjd} zeigt, dass die geteilte Umgebung von \textit{ja} und \textit{doch} (wenngleich es weitere interpretatorische Beschrän\-kungen gibt) Aussagesätze sind. Man hat es mit dem Formtyp zu tun, der sich durch ein $[\minus$w$]$ gefülltes Vorfeld, V2-Stellung, ein nicht-imperativisches finites Verb, keinen Exklamativakzent und einen fallenden Tonhöhenverlauf (vgl. \citealt[44]{Thurmair1989}, \citealt[176-177]{Oppenrieder1987}) auszeichnet. Der dem Aussagesatz zugeordnete Funktionstyp ist im Altmann'schen (\citeyear{Altmann1984, Altmann1987}) Satzmodusmodell die Assertion. Meine weitere Betrachtung untersucht die Funktion von \textit{ja}-, \textit{doch}- und \textit{ja doch}-Äußerun\-gen im Diskurs, d.h. ihren Effekt auf den Kontext sowie ihre kommunikativen Absichten. Die Generalisierung, auf die sich die folgende Argumentation im weiteren Verlauf stützt, ist deshalb, dass die relevante Domäne der Kombination dieser MPn aus Sicht des Funktionstyps die Assertion \is{Assertion} ist.

Die bisherigen Illustrationen zeigen allerdings auch, dass man es bei den Aussagesätzen, in denen sich die beiden Partikeln kombinieren lassen, nicht aus\-schließlich mit Standardassertionen (vgl. (\ref{347}), (\ref{348})) zu tun hat, sondern dass Äuße\-rungen auftreten, die funktional in den Bereich der Exklamativsätze (vgl. Abschnitt~\ref{sec:empha}) übergehen (vgl. (\ref{349}), (\ref{350})). Mit \citet[77-80]{Doherty1985}, \citet[107-108]{Thurmair1989} und \citet[37]{Kwon2005} (s.o.) gehe ich davon aus, dass es sich hierbei um Aussagen und somit ihrer Funktion nach ebenfalls Assertionen handelt.

\begin{exe}
	\ex\label{347} 
	Konrad ist \textbf{ja doch} verreist.	
\end{exe}
\vspace{-0.65cm}
\begin{exe}
	\ex\label{348} 
	\scriptsize
	A: Darfst das Bierglas nicht anfassen, weil du sonst eine Sehnenscheidenentzündung bekommst.\\
	B: \textbf{Die Sehnenscheidenentzündung war \underline{ja doch} von der Gitarre.}
	\newline
	\hbox{}\hfill\hbox{(FOLK\_E\_00039\_SE\_01\_T\_02) (Beleg bereinigt S.M.)}	
\end{exe}
\vspace{-0.65cm}
\begin{exe}
	\ex\label{349} 
	Das sind \textbf{ja doch} Paul und Maria! Was machen die denn hier?	
\end{exe}		
\vspace{-0.65cm}
\begin{exe}
	\ex\label{350} 
	Das ist \textbf{ja doch} die Höhe!
\end{exe}	
Inwiefern sich das assertive Potential in (\ref{347}) und (\ref{348}) von dem in (\ref{349}) und (\ref{350}) beteiligten unterscheidet, ist ein Aspekt, der in der weiteren Betrachtung aufgegriffen und in der Analyse der Abfolge von \textit{ja} und \textit{doch} entscheidend wird. 

Dass sich innerhalb der Assertionen weitere Untertypen ausmachen lassen, lässt sich auch beobachten, wenn man neben dem klassischen (assertiven) Formtyp auch andere assertive selbständige Sätze in die Betrachtung miteinbezieht, in denen die beiden MPn \textit{ja} und \textit{doch} jeweils einzeln und somit auch gemeinsam auftreten können. Auch diese Typen von Assertionen und ihre konkreten Funktionen spielen bei der Diskussion der (un)zulässigen Abfolgen der zwei Partikeln in Abschnitt~\ref{sec:unmarkiert} und \ref{sec:markiert} eine entscheidende Rolle.
				  
\subsection{Non-kanonische Deklarativsätze}
\label{sec:nonkan}
Eine Randerscheinung selbständiger Aussagesätze sind V1-Sätze des Typs in (\ref{351}) und (\ref{352}).

\begin{exe}
	\ex\label{351} 
	Zählt das Münster \textbf{doch} zu den wenigen romanischen Gotteshäusern, welche ...
	\hfill\hbox{\citet[294]{Oennerfors1997}}	
\end{exe}

\begin{exe}
	\ex\label{352} 
	\scriptsize
	Aufschlussreich sind Haiders Ausfälle allemal. \textbf{Bestätigt er \underline{doch} einmal mehr sein originelles Demo\-kratieverständnis:} Er glaubt 		immer noch, alles anordnen zu können, selbst Fußballsiege.  
	\newline
	\hbox{}\hfill\hbox{(A01/Nov. 40827, St. Galler Tagblatt, 06.11.2001) \citet[157]{Pittner2011}}	
\end{exe}
In Arbeiten zu diesen Strukturen wird angenommen, dass das Auftreten von \textit{doch} in diesem Satztyp obligatorisch ist (vgl. z.B. \citealt[295]{Oennerfors1997}, \citealt[172]{Pittner2011}). \citet[174]{Rinas2006} möchte für (satzintegrierte) Sätze dieses Typs aufgrund der Obligatorizität von \textit{doch} von einer idiomatisierten Konstruktion ausgehen (vgl. allerdings \citealt[162-169]{Pittner2011}, die aufzeigt, dass der Beitrag von \textit{doch} in dieser Umgebung kein anderer ist als in anderen Auftretenskontexten der Partikel, vgl. auch meine eigenen Ausführungen in Kapitel~\ref{chapter:dua}, Abschnitt~\ref{sec:Rand} sowie \citealt{MuellerimDruck})).

\citet[174]{Rinas2006} wie auch \citet[90]{Kwon2005} nehmen an, dass \textit{ja} in V1-Aussagesät\-zen \is{V1-Deklarativsatz} nicht möglich ist. \citet[158]{Oennerfors1997} geht davon aus, dass \textit{ja} in älteren Sprachstufen in dieser Umgebung auftreten konnte. Er verweist für Nachweise derartiger Beispiele auf alte Arbeiten wie \citet[74-75]{Sanders1883}, \citet[36-37]{Stenstad1917} und \citet[70-71]{Mattausch1965}. Entsprechende Daten sind aber auch durch\-aus im heutigen Deutschen zu finden wie die Belege in (\ref{353}) und (\ref{354}) aufzeigen.

\begin{exe}
	\ex\label{353} 
	\scriptsize
	\glqq Ich habe gute Kontakte zum Canisianum in Innsbruck\grqq{}, erzählt Kaplan Emil Bonetti. \textbf{Hat er dort \underline{ja} selbst während seinen 	Studienjahren gewohnt. }
	\newline
	\hbox{}\hfill\hbox{(V00/DEZ.60670 Vorarlberger Nachrichten, 04.12.2000)}	
\end{exe}

\begin{exe}
	\ex\label{354} 
	\scriptsize
	Dass die Züge voll sind, ist ein Problem, aber eigentlich ein gutes. \textbf{War es \underline{ja} das erklärte Ziel der Verkehrspolitik der letzten 		Jahre der Umwelt zuliebe die Bahn zu fördern.}
	\newline
	\hbox{}\hfill\hbox{(A11/APR.01450 St. Galler Tagblatt, 05.04.2011)}	
\end{exe}
Die Fälle sind eindeutig abzugrenzen von V2-Sätzen mit Vorfeld-Ellipse, die auf den ersten Blick für V1-Aussagesätze gehalten werden können und in denen das im Vorfeld ausgelassene Argument entweder später im Satz rechtsversetzt (vgl. (\ref{355})) oder gar nicht genannt wird (vgl. \ref{(356})).
	
\begin{exe}
	\ex\label{355} 
	\scriptsize
	Seit einigen Tagen weist ein brandneues Schild in den Signalfarben Rot und Weiß [...] auf die steilste Standseilbahn Deutschlands hin. \textbf{Ist 			\underline{ja} auch ein Schmuckstückchen, die Kurwaldbahn,} und noch dazu befördert sie die Fahrgäste in luftige Höhen über Bad Ems [...].
	\newline
	\hbox{}\hfill\hbox{(RHZ12/MAI.12144 Rhein-Zeitung, 11.05.2012)}	
\end{exe}

\begin{exe}
	\ex\label{356} 
	\scriptsize
	Bin gespannt, ob die jungen Alten vier Wochen durchhalten. Der Bundes-Berti ist davon überzeugt. \textbf{Hat \underline{ja} auch aus gutem Grund alle Positionen 		doppelt besetzt. }
	\newline
	\hbox{}\hfill\hbox{(RHZ98/JUN.14334 Rhein-Zeitung, 09.06.1998)}	
\end{exe}					                                   
Da auch \textit{ja} in diesem Randtyp von Aussagesatz auftreten kann, ist es nicht verwunderlich, dass in dieser Umgebung auch die Kombination von \textit{ja} und \textit{doch} zulässig ist. (\ref{357}) und (\ref{358}) sind völlig problemlos möglich und man muss auch hier nicht davon ausgehen, dass selbständige assertive V1-Sätze nur auf älteren Sprachstufen die \textit{ja doch}-Kombination erlaubten, wie das Beispiel in (\ref{359}) aus \citet{Oppenrieder1987} nahelegen könnte, auf das \citet[158, Fn 186]{Oennerfors1997} verweist.
 
\begin{exe}
	\ex\label{357} 
	\scriptsize
	Gründler geht davon aus, dass Peter S. die damals 27-jährige Arzthelferin am Tatmorgen in der Tiefgarage abfing, um sie in dieser Angelegenheit zur Rede zu stellen und letztlich mundtot zu machen. \textbf{\textit{Musste} S. \underline{ja doch} befürchten, dass Susanne M. tatsächlich Anzeige gegen ihn erstatten würde. }		   
	\newline
	\hbox{}\hfill\hbox{(NUZ09/DEZ.01570 Nürnberger Zeitung, 15.12.2009) (modifiziert durch S.M.)}	
\end{exe} 		
 		
\begin{exe}
	\ex\label{358} 
	\scriptsize
	Verliehen wird die Auszeichnung an Bürger, die sich besondere Verdienste erworben haben. Und da fand der Bürstädter Bürgermeister Alfons Haag viel 			Kurzweiliges. \textbf{\textit{Ist} Deckenbach \underline{ja doch} weit über die dörfliche Grenze hinaus als \glqq Kerwe-Opa\grqq{} bekannt}, der seit 		vielen Jahren dem \glqq Kerwevadder\grqq{} die Reden reimt.		   
	\hfill\hbox{(M09/DEZ.97728 Mannheimer Morgen, 09.12.2009) (modifiziert durch S.M.)}	
\end{exe} 	
							                               	
\begin{exe}
	\ex\label{359} 
	Stahl sie \textbf{ja doch}, dem Gatten zulieb, die Götzen des finsteren Vaters.	   
	\newline
	\hbox{}\hfill\hbox{\citet[187, Anm. 26]{Oppenrieder1987}}	
\end{exe}		
Wenn Önnerfors davon ausgeht, dass \textit{ja }nur zu früheren Zeiten hier auftreten konnte, würde sich auch das nur damalige kombinierte Auftreten plausibel in seine Argumentation einfügen. (\ref{353}) und (\ref{354}) sowie die akzeptablen (\ref{357}) und (\ref{358}) weisen allerdings darauf hin, dass die Beschränkung auf ältere Sprachstufen nicht angenommen werden muss. 

Ein anderer Randtyp eines Aussagesatzes ist \is{Wo-VL-Deklarativsatz} der \textit{wo}-VL-Satz, der hier weder lokal noch temporal interpretiert wird (vgl. auch zu diesem Satztyp ausführlich Kapitel~\ref{chapter:dua}, Abschnitt~\ref{sec:Rand} sowie \citealt{MuellerimDruck}). (\ref{360}) zeigt, dass \textit{doch} in dieser Umgebung aufzufinden ist, wenngleich es in diesem Satztyp nicht obligatorisch ist (vgl. z.B. \citealt[152]{Pasch1999} sowie \citealt[324]{Guenthner2002}).

\begin{exe}
	\ex\label{360} 
	\scriptsize
	Wobei sich die zuständige Stadträtin Karin Gutmann scherzhaft beschwerte: \glqq Es ist beinahe unfair, dass die Männer einen meist niedrigeren 				Cholesterinwert haben als die Frauen. \textbf{\textit{Wo} sie \underline{doch} sonst nicht so auf Gesundheit bedacht sind ...}\grqq{}    
	\newline
	\hbox{}\hfill\hbox{(NON12/APR.18615 Niederösterreichische Nachrichten, 26.04.2012)}	
\end{exe} 	
\textit{Ja} lässt sich ebenfalls belegen, auch wenn beispielsweise \citet[194]{Kwon2005} Gegenteiliges behauptet.

\begin{exe}
	\ex\label{361} 
	\scriptsize
	\glqq Wir sind schon sehr stolz darauf, dass wir das in dieser Saison erreicht haben. \textbf{\textit{Wo} wir \underline{ja} zu Beginn als Abstiegskandidat gehandelt wurden}\grqq{}, sagte Coach Mario Graf.    
	\newline
	\hbox{}\hfill\hbox{(NON07/MAI.03947 Niederösterreichische Nachrichten, 07.05.2007)}	
\end{exe}

\begin{exe}
	\ex\label{362} 
	\scriptsize
	Diese Heimspielrechnung wollte Bayer-Coach Klaus Toppmöller nicht mit aufmachen. \textbf{\textit{Wo} Bayer \underline{ja} auch nur noch ein Spiel in 		der BayArena hat} [...].   
	\hfill\hbox{(RHZ02/FEB.15088 Rhein-Zeitung, 21.02.2002)}	
\end{exe}
Das \textit{wo} wird in den Sätzen in (\ref{360}) bis (\ref{362}) jeweils im Sinne von \textit{denn}, \textit{weil}, \textit{da}, \textit{zumal} bzw. \textit{obwohl} verstanden, d.h. es wird kausal bzw. konzessiv interpretiert.

Im Einklang mit der syntaktischen Schnittmengenbedingung lassen sich \textit{ja} und \textit{doch} auch in diesem Typ von Aussagesatz kombinieren (vgl. (\ref{363})).

\begin{exe}
	\ex\label{363} 
	\scriptsize
	Es könnte ja komisch anmuten, sich ins Kino zu setzen und sich das jugendliche Leben reinzuziehen. \textbf{\textit{Wo} man es \underline{ja doch} 			täglich in Plattengeschäften, in Einkaufspassagen oder auf Pausenplätzen vorgeführt bekommt.}
	\hfill\hbox{(E97/SEP.22830 Zürcher Tagesanzeiger, 24.09.1997)}	
\end{exe}
Ausgehend von der Generalisierung, dass Assertionen die zulässige Domäne für Kombinationen aus \textit{ja} und \textit{doch} darstellen, zeigt der letzte Abschnitt, dass auch die Form der Assertionen von Standardassertionen in Gestalt von $[ \minus \textrm{w}]$, V2-Sätzen abweichen kann. Auch die Randtypen von V1- bzw. \textit{wo}-VL-Aussagesätzen sind in die Betrachtung einzubeziehen. In Abschnitt~\ref{sec:markiert} werden Besonderheiten in der Interpretation dieser Sätze genauer beleuchtet. 

Formuliert man die zulässige Domäne des Auftretens der MP-Kombination unter Bezug auf den Funktionstyp der Assertion, bringt dies den Vorteil mit sich, dass auch Nebensätze, in denen \textit{ja}, \textit{doch} und somit \textit{ja doch} auftreten können, in die Untersuchung einbezogen werden können. 

\subsection{Eingebettete Kontexte}
\label{sec:eingkon}
Die beiden Partikeln können z.B. in kausalen Nebensätzen\footnote{Ich erhebe an dieser Stelle keinen Anspruch auf Vollständigkeit der Auftretenskontexte. Zweck ist, das Auftreten in Nebensätzen soweit zu motivieren, wie es für die Ausführungen in Abschnitt~\ref{sec:markiert} relevant ist. Neben den verschiedenen Kausalsätzen erlauben auch Konzessiv-, non-restriktive Relativsätze sowie manche Komplementsätze das Auftreten von \textit{ja}, \textit{doch} und \textit{ja doch}. Ich klammere diese Typen aus der Darstellung aus, das sie für die weitere Betrachtung nur eine untergeordnete Rolle spielen. Gleiches gilt für unzulässige Kontexte wie restriktive Re\-lativsätze, Temporalsätze oder Konditionalsätze (vgl. z.B.: \citealt[11]{Borst1985}, \citealt[76]{Thurmair1989} \citealt[166]{Helbig1994}, \citealt[115]{Coniglio2007}, \citealt[139/152/147/152]{Coniglio2011}, \citealt[59]{Frey2011}).} auftreten, die durch verschiedene Konnektoren eingeleitet sein können.

\begin{exe}
	\ex\label{364} 
	Hans sollte Inge lieber nicht reizen, \textit{\textbf{da}} sie ihn \textbf{ja}/\textbf{doch} neulich geohrfeigt hat.
\end{exe}
\vspace{-0.65cm}	
\begin{exe}
	\ex\label{365} 
	Sie fürchtet sich, \textit{\textbf{weil}} ihr Vater sie \textbf{ja}/\textbf{doch} immer schlägt.
	\newline
	\hbox{}\hfill\hbox {nach \citet[77/81]{Borst1985}}
\end{exe}
\vspace{-0.65cm}
\begin{exe}
	\ex\label{366} 
	Der Kurzbesuch ist deinen Eltern hoch anzurechnen, \textit{\textbf{zumal}} sie \textbf{ja}/\textbf{doch} eine Anreise von 200km zurückzulegen haben.
\end{exe}
Im Einvernehmen mit der Schnittmengenbedingung ist dann jeweils auch das kombinierte Auftreten zulässig (vgl. (\ref{367}) bis (\ref{369})).

\begin{exe}
	\ex\label{367} 
	\scriptsize
	Ich würde eine schriftliche(e-mail reicht auch)Nachfrist setzen und dabei gleich mein Befremden zum Ausdruck bringen, über die Abwicklung und 				bezüglich der Ehrlichkeit gegenüber dem Kunden, \textbf{\textit{zumal}} du \textbf{ja doch} ein Stammkunde bist.
	\hfill\hbox {(DECOW 2012: 1188552685)}
\end{exe}
\vspace{-0.65cm}
\begin{exe}
	\ex\label{368} 
	da ist mir jedesmal zum heulen zumute wenn ich das sehe \textbf{\textit{da}} ich \textbf{ja doch} so leidenschaftlich koche.	
	\hfill\hbox {(DECOW 2012: 4194727)}
\end{exe}									                                 
\vspace{-0.65cm}
\begin{exe}
	\ex\label{369} 
	Das wäre schade, \textbf{\textit{weil}} das Trio \textbf{ja doch} genauso schön ist wie der Hauptteil.	
	\newline
	\hbox{}\hfill\hbox {(DECOW 2012: 199984396)}
\end{exe}		                                       
Gleiche Verhältnisse stellen sich bei kausalen \textit{wo}-Sätzen ein: \textit{doch} kann auftreten, genauso wie \textit{ja} (Gegenteiliges behaupten \citealt[63]{Thurmair1989} und \citealt[213]{Rinas2006}), weshalb auch das kombinierte Auftreten keine Überraschung darstellt.

\begin{exe}
	\ex\label{370} 
	Hans sollte Inge lieber nicht reizen, \textbf{\textit{wo}} sie ihn \textbf{doch} neulich geohrfeigt hat.	
	\hfill\hbox {\citet[77]{Borst1985}}
\end{exe}	

\begin{exe}
	\ex\label{371} 
	\scriptsize
	Es ist daher sehr unwahrscheinlich, dass Ihr Kleiner auf die Früchte in unseren Gläschen reagiert, \textit{\textbf{wo}} er sie \textbf{ja} sogar roh 		bereits vertragen hat.	
	\hfill\hbox {(http://www.hipp.de/forum/viewtopic.php?f=12\&t=4608)}
	\newline
	\hbox{}\hfill\hbox {(Google-Recherche 23.4.2012)}
\end{exe}

\begin{exe}
	\ex\label{372} 
	Können die mir denn eine Karte überhaupt verweigern \textbf{\textit{wo}} ich \textbf{ja doch} versichert bin?
	\hfill\hbox {(DECOW 2012: 1109601892)}
\end{exe}
Zu diesem Typ von Kausalsatz wurde in Abschnitt~\ref{sec:nonkan} bereits die selbständige Variante angeführt. Auch der dort im gleichen Zuge eingeführte selbständige V1-Satz weist ein unselbständiges Pendant auf. \textit{Doch} ist in diesen Sätzen möglich (nach Ansicht von \citealt[77-80]{Borst1985} und \citealt[13]{Rinas2006} sogar obligatorisch). Entgegen der Einschätzung von \citet[213]{Rinas2006} lässt sich auch \textit{ja} gut belegen, so dass auch dem kombinierten Auftreten nichts im Wege steht.
	
\begin{exe}
	\ex\label{373} 
	\scriptsize
	Auch die praktische Arbeit und Anschauung kamen nicht zu kurz, \textbf{\textit{verfügte}} Neustadt \textbf{doch} über eine günstige Lage [...].	
	\hfill\hbox {(DECOW-2012: 218534182)}
\end{exe}	
	
\begin{exe}
	\ex\label{374} 
	\scriptsize
	Eigentlich schade, den Bass nicht deutlicher in den Vordergrund zu stellen, \textbf{\textit{war}} Dixon \textbf{ja} nicht nur Komponist, sondern vor 		allem auch Bassist. 
	\hfill\hbox {(M11/OKT.08329 Mannheimer Morgen, 26.10.2011)}
\end{exe}

\begin{exe}
	\ex\label{375} 
	\scriptsize
	Öffentliche Gebäude gehören vermehrt genutzt, \textbf{\textit{werden}} sie \textbf{ja doch} vom Steuergeld unserer Bür\-gerInnen errichtet. 	
	\hfill\hbox {(NON08/MAI.05354 Niederösterreichische Nachrichten, 12.05.2008)}
\end{exe}
In der Literatur werden diejenigen Nebensätze, die MPn erlauben, zu den sogenannten \textit{peripheren Nebensätzen} \is{peripherer Nebensatz} gezählt.\footnote{Ich werde diese globale Zuordnung in Kapitel~\ref{chapter:hue} im Rahmen der Untersuchung des Auftretens von \textit{halt} und \textit{eben} in Relativsätzen in Frage stellen.} Diesen (adverbialen) Nebensätzen will man eine gewisse illokutive Selbständigkeit zuschreiben, während dem \glq Pendant\grq {} - den \textit{zentralen Adverbialsätzen} \is{zentraler Nebensatz} - diese Eigenschaft abgesprochen wird (vgl. \citealt{Haegeman2002, Haegeman2004, Haegeman2006}). Die Annahme, dass peripheren Nebensätzen eine gewisse illokutionäre Selbständigkeit zugeschrieben wird, geht auf die Beobachtung zurück, dass bestimmte Phänomene (die sogenannten \is{Wurzelphänomen} \textit{Wurzelphänomene}, die eigentlich auf Hauptsätze beschränkt sind, (je nach Sprache mehr oder weniger restringiert) auch in diesen Nebensätzen vorkommen können (vgl. \citealt{Emonds1969}, \citealt{Rutherford1970}, \citealt{Hooper1973}, für einen Überblick vgl. \citealt{Heycock2005}). \citet{Coniglio2011} und \citet{Abraham2012} vertreten, dass auch MPn zu den Wurzelphänomenen zu zählen sind. Coniglios Erklärung dafür ist, dass MPn illokutive/diskursive Funktion haben (Sprechereinstellungen kodieren, Illokuti\-onstypen modifizieren) und deshalb für die Domäne des Auftretens fordern, dass illokutive Selbständigkeit besteht. Gilt es nun zu entscheiden, welche Illokution in den die MPn lizensierenden Nebensätzen vorliegt, handelt es sich plausiblerweise um Assertivität. \citet[120]{Doherty1987} und \citet[93]{Kwon2005} behaupten auch schon vor der Diskussion um Wurzelphänomene und der Unterscheidung zwi\-schen peripheren und zentralen Nebensätzen -- ohne große Elaboration dieses Aspektes --, dass man es hier mit assertiven Kontexten zu tun hat.\\

\noindent
Es lässt sich also festhalten, dass mit \textit{ja}, \textit{doch} und \textit{ja doch} in selbständigen und eingebetteten assertiven Kontexten zu rechnen ist (vgl. auch schon \citealt[167-170]{Mueller2014a}; \citeyear[205-207]{Mueller2017b}). Da die Festlegung auf Assertionen möglich ist, wird auch meine eigene Ableitung der Abfolge von \textit{ja} und \textit{doch} auf Eigenschaften von Assertionen abheben. Der folgende Abschnitt skizziert auf diesem Weg zunächst bestehende Ansätze aus der Literatur zu dieser Frage. 

\section{Die Abfolge \textit{ja doch} und bestehende Erklärungsversuche}
\label{sec:abfolgejd}
In Arbeiten, die sich mit der konkreten Kombination aus \textit{ja} und \textit{doch} beschäftigen, fallen die Urteile so aus, dass das \textit{ja} dem \textit{doch} vorangeht (vgl. z.B. (\ref{376}) bis (\ref{378}), vgl. z.B. auch die Bewertungen in \citealt[101]{Meibauer1994}, \citealt[93]{Ormelius-Sandblom1997}, \citealt[431]{Rinas2007}).
\setcounter{equation}{0}
\begin{exe}
	\ex\label{376} 
	Konrad ist \textbf{ja doch}/*\textbf{doch ja} verreist.
\hfill\hbox {\citet[114]{Doherty1987}}
\end{exe}
\vspace{-0.5cm}
\begin{exe}
	\ex\label{377} 
	Er hat \textbf{ja doch}/??\textbf{doch ja} getanzt.
\hfill\hbox {\citet[20]{Struckmeier2014}}
\end{exe}
\vspace{-0.5cm}
\begin{exe}
	\ex\label{378} 
	Er hat sich \textbf{ja doch}/?\textbf{doch ja} sehr um sie bemüht.
\hfill\hbox {\citet[157]{Jacobs1991}}
\end{exe}
Wenngleich die Einschätzungen zwischen *, ?? und ? etwas schwanken, gilt für alle mir bekannten Betrachtungen, die Erklärungsvorschläge anbieten, dass sie beabsichtigen, die Abfolge \textit{doch ja} als ungrammatisch auszuschließen, und in der Regel im Rahmen der Modelle auch keine Möglichkeit besteht, die umgekehrte Reihung überhaupt zuzulassen.

Um diesen Aspekt der Ausgangslage meiner eigenen Betrachtung zu verdeutlichen, seien im Folgenden vor diesem Hintergrund einige Vorschläge (erneut) skizziert (zu ausführlichen Darstellungen s. Abschnitt~\ref{sec:forschung} in Kapitel~\ref{chapter:hintergrund}). \citet{Doherty1985, Doherty1987}, \citet{Ickler1994}, \citet{Ormelius-Sandblom1997} und \citet{Rinas2007}, die Erklärungsmodelle für die Ordnung von \textit{ja} und \textit{doch} vorschlagen, argumentieren dabei auf verschiedene Art unter Bezug auf die Interpretation der MP-Kombination, während \citet{Lindner1991} eine phonologische Beschränkung verantwortlich macht. Der entscheidende Punkt an dieser Stelle ist, dass eigentlich nur Lindner überhaupt die Möglichkeit für die Existenz der \textit{doch ja}-Abfolge zulässt (wenngleich sie sie auch nicht in Betracht zieht).

\citet{Doherty1985, Doherty1987} zufolge ordnen sich die Partikeln \textit{ja} und \textit{doch} entlang des Kriteriums der \is{assertive Stärke} abnehmenden assertiven Stärke, d.h. entlang unterschiedlicher Grade der Verbindlichkeit der Sprecherhaltung. \textit{Ja}-Äußerungen weisen einen höheren Grad an assertiver Stärke auf als \textit{doch}-Äußerungen, so dass die MP \textit{ja} der MP \textit{doch} stets vorangeht. Doherty nimmt diese Zuschreibung auf der Basis des jeweils angenommenen MP-Beitrags vor (vgl. (\ref{379})): \textit{ja} legt den Sprecher anders als \textit{doch} stets auf eine assertive Haltung zur Einstellung im Skopus der Partikel fest.

\begin{exe}
	\ex\label{379} 
		\textit{ja}: Ass$(\textrm{E}_{\textrm{S}}$(\textrm{p})$)$ und I\textrm{M}$(\textrm{E}_{\textrm{X}}$(\textrm{p})$)$\\
		\textit{doch}: $\textrm{Ass}^{\prime} (\textrm{E}_{\textrm{S}}(\textrm{p})) \ \textrm{und IM(neg}_{\textrm{X}}(\textrm{p}))$
		\hfill\hbox {\citet[80/71]{Doherty1985}}
\end{exe}
Da sie hier die inhärente, wörtliche Bedeutung der MPn modelliert, die durch nichts beeinflussbar sein sollte, ist in ihrem Modell nicht angelegt, dass sich an diesen Verhältnissen unterschiedlicher Grade von assertiver Stärke etwas ändert. Sobald von der MP-Ordnung \textit{ja doch} abgewichen wird, folgen die MPn auch nicht mehr der Stärkehierarchie von Assertivität und es sollte zu Akzeptabilitätsverlusten kommen.

Dieser Aspekt der Invarianz ihres Kriteriums bei der Ableitung der Abfolgen wird noch deutlicher, wenn man in ihre tiefere Ausdeutung dieses oberflächennahen Kriteriums schaut, die Bezug nimmt auf die Interaktion verschiedener Typen von Einstellungen (vgl. Abschnitt~\ref{subsec:input}). Der Beitrag von \textit{ja} ist es, in einer Äußerung die Einstellung in ihrem Skopus zu bestätigen Ass$(\textrm{E}_{\textrm{S}}$(\textrm{p})$)$. \textit{Ja} benötigt als Argument ein Objekt des Typs E. Da \textit{doch} keine Einstellung ausdrückt, sondern eine \underline{Haltung} zu einer Einstellung (in Assertionen ebenfalls \textit{Ass}), eignet sich der Ausdruck, der \textit{doch(p)} entspricht, nicht als Input für den Ausdruck von \textit{ja(p)}. Beide MPn assertieren, wenn sie gemeinsam auftreten, deshalb die Einstellung in ihrem Skopus. Sie nehmen den gleichen Skopus. Liegt die umgekehrte Abfolge vor, assertiert das \textit{ja} zunächst die Einstellung in seinem Skopus Ass$(\textrm{E}_{\textrm{S}}$(\textrm{p})$)$. Genau wie \textit{ja} benötigt \textit{doch} eine Einstellung als Argument, die es im Fall einer Assertion bestätigt. \textit{Ja(p)} kann an dieser Stelle der Berechnung jedoch nur den assertiven Einstellungs\underline{modus} beisteuern, der sich als Objekt im Skopus \is{Skopus} von \textit{doch} nicht eignet. Doherty zufolge scheidet diese MP-Abfolge somit aus, weil an diesem Punkt die Bedeutungszuschreibung scheitert (vgl. \citealt[84-85]{Doherty1985}). Wie in Abschnitt~\ref{subsec:input} ausgeführt, lässt sich Dohertys Ableitung als Erfüllung bzw. Verstoß von durch die beteiligten Objekte geforderten Inputbedingungen auffassen. Da es sich hierbei um semantische Anforderungen handelt, besteht keine Möglichkeit, den Verstoß im Falle der Abfolge \textit{doch ja} zu umgehen. Sie sollte folglich stets als ungrammatisch herausgefiltert werden.

Mit der Annahme, dass für \textit{doch} in der Abfolge \textit{doch ja} nach Applikation von \textit{ja} kein geeignetes semantisches Objekt vorliegt, auf das \textit{doch} Bezug nehmen kann, benennt Doherty einen Grund für den Ausschluss von \textit{doch ja}. Es bleibt aber ungeklärt, warum es hier nicht ebenfalls möglich ist, dass sich – wie im Falle von \textit{ja doch} (wo unter Skopus das gleiche Problem auftritt) – beide Partikeln auf die Einstellung E beziehen können.\footnote{Die einzige Erklärung, die mir plausibel scheint, ist, dass der Einstellungsmodus EM im Falle von \textit{doch ja} im Stadium der Verrechnung von \textit{doch} bereits festgelegt ist (\textit{Ass} durch \textit{ja}) und der Zugriff auf E in dessen Skopus nicht mehr möglich ist. Im Falle von \textit{ja doch} hingegen ist der EM nach der Integration von \textit{doch} noch nicht entschieden und \textit{ja} kann deshalb auf E zugreifen.}

Sowohl \citet{Ormelius-Sandblom1997} als auch \citet{Rinas2006, Rinas2007} schließen die Abfolge \textit{doch ja} aus, indem sie auf die vorliegenden syntaktischen/semantischen Skopusverhältnisse eingehen. Beide gehen davon aus, dass eine direkte Korrelation zwischen syntaktischer Struktur und semantischem Skopus besteht. Strukturell tiefer positionierte MPn stehen im Skopus von hierarchisch höher verankerten MPn. (\ref{380}) und (\ref{381}) zeigen erneut die Modellierung der Bedeutung der Einzelpartikeln nach \citet[420]{Rinas2007}.

\begin{exe}
	\ex\label{380} 
	\textit{ja}: JA(p) $»$ NICHT-GLAUBT(H, NICHT-p)
\end{exe}
\vspace{-0.65cm}
\begin{exe}
	\ex\label{381} 	
	\textit{doch}: DOCH(p) $»$ WIDERSPRICHT(p,q) \& (KENNT(H,p) $\lor$ KENNT(H,q))
	\newline
	\hbox{}\hfill\hbox{\citet[425/420]{Rinas2007}}	
\end{exe}
(\ref{382}) gibt nach Rinas die Bedeutung der MP-Kombination \textit{ja doch} adäquat wieder.
\begin{exe}
	\ex\label{382} 
		\begin{xlist}	
			\ex\label{382a} JA(DOCH(p) $»$ WIDERSPRICHT(p,q) \& KENNT(H,p) $\lor$ KENNT(H,q))
				$»$ NICHT GLAUBT(H, NICHT(DOCH(p) $»$ WIDERSPRICHT(p,q) \& \\ KENNT(H,p) $\lor$ KENNT(H,q)))
			\ex\label{382b} \glq Es ist unkontrovers (es ist nicht der Fall, dass der Hörer daran zweifelt), dass p im 							Widerspruch zu einer anderen Proposition q steht und dass dem Hörer p oder q bekannt ist.\grq {}
			\hfill\hbox {\citet[431]{Rinas2007}}
		\end{xlist}
\end{exe}
Aus dieser Bedeutungszuschreibung, der zufolge \textit{ja} Skopus über \textit{doch} nimmt, folgt s.E., dass \textit{ja} in der Struktur hierarchisch höher positioniert ist als \textit{doch}, was sich wiederum in der linearen Abfolge \textit{ja doch} niederschlage. Aufgrund dieses hierarchischen/asymmetrischen Verhältnisses zwischen den beiden MPn sei abzuleiten, dass die Umkehr der Reihung von \textit{ja} und \textit{doch} nicht möglich sei. Die Abfolge \textit{doch ja} existiert nach Rinas folglich nicht, weil diese Reihung nicht die Interpretation widerspiegelt, in der \textit{doch} im Skopus von \textit{ja} steht.

Wenngleich von Doherty und Rinas vertreten wird, dass \textit{doch ja} als ungrammatisch auszuschließen ist, unterscheiden sich die beiden Ansätze m.E. insofern, als dass es in Rinas' Modell nichts gibt, was die umgekehrte Abfolge ausschließt. Wenn man von einer Korrelation zwischen Skopus und linearer Abfolge ausgeht, spricht nichts dagegen (und eher einiges dafür), dass bei anderer Anordnung derselben MPn schlichtweg eine andere Interpretation vorliegt. In diesem Fall nimmt dann die Partikel \textit{doch} die Partikel \textit{ja} in ihren Skopus. Diese mögliche Interpretation wird von Rinas nicht in Betracht gezogen, obwohl sie – wie (\ref{383}) zeigt – unter Verwendung seiner Bedeutungszuschreibungen ohne Weiteres zu modellieren wäre.

\begin{exe}
	\ex\label{383} 
		\begin{xlist}	
			\ex\label{383a} DOCH(JA(p) $»$ NICHT-GLAUBT(H,NICHT-p))
				$»$ \\ WIDERSPRICHT((JA(p) $»$ NICHT-GLAUBT(H,NICHT-p)),q) \& \\
				(KENNT(H,JA(p) $»$ NICHT-GLAUBT(H,NICHT-p)) 
				\\ $\lor$ KENNT(H,q))
			\ex\label{383b} \glq Die Tatsache, dass p unkontrovers ist, steht im Widerspruch zu einer anderen Proposition q und 				dem Hörer ist bekannt, dass p unkontrovers ist oder q. \grq {}
			\hfill\hbox {\citet[431]{Rinas2007}}
		\end{xlist}
\end{exe}
Wollte Rinas die Kombination \textit{doch ja} im Rahmen seines Modells ausschließen, müsste er prinzipiellere Gründe gegen die Bedeutungszuschreibung in (\ref{383}) anführen. Auch wenn zwischen \textit{ja} und \textit{doch} in der Kombination \textit{ja doch} semantisch ein Skopusverhältnis besteht, mit dem ein hierarchisch-syntaktischer Strukturunterschied bzw. eine bestimmte Linearisierung einhergeht, berechtigt die Logik seiner Ableitung eigentlich nicht den Ausschluss von \textit{doch ja} bzw. fehlt die Angabe eines echten Grundes für diese Entscheidung. Parallele Annahmen können auch für die Ableitung von \citet{Ormelius-Sandblom1997} gemacht werden. Mit einer anderen Modellierung der Einzelpartikeln kommt sie zum gleichen Schluss wie Rinas: \textit{ja} geht doch voran, weil diese Abfolge dem Skopusverhältnis zwischen den beiden MPn entspricht (vgl. \citealt[92-93]{Ormelius-Sandblom1997}).

Ihre Überlegung hinsichtlich dieses Zusammenhangs schließt aber ebenso we\-nig wie bei Rinas aus, dass \textit{doch ja} in der Bedeutung \textit{doch(ja)} auftritt.

Auch \citet{Ickler1994} geht davon aus, dass allein \textit{ja doch} die akzeptable Kombination der zwei Partikeln darstellt. Er macht ebenfalls die Interpretation der Kombination für diese Annahme verantwortlich. Und es gilt ebenfalls, dass zwischen den MPn ein Skopusverhältnis vorliegt. Aus seinen Ausführungen ist allerdings ein Grund für den Ausschluss von \textit{doch ja} herauszulesen. Wie auch \citet{Vismans1994} geht \citet{Ickler1994} davon aus, dass verschiedene MPn auf unterschiedlichen (interpretatorischen) Ebenen wirken. Dadurch, dass die angenommenen Ebenen relativ zueinander hierarchisch angeordnet sind, ergeben sich für die diesen Ebenen zugeordneten MPn invariante Abfolgen. In diesem Sinne sind nur bestimmte MP-Abfolgen überhaupt interpretierbar.

In seiner Modellierung wirkt \textit{ja} auf der Ebene der Argumentation: Die Partikel weise einer Aussage in einer Argumentation ihren Stellenwert zu, indem diese bestätigt und bekräftigt werde (\citeyear[399]{Ickler1994}). \textit{Doch} zeige einen inhaltlichen Gegensatz an (\citeyear[401]{Ickler1994}), weshalb diese Partikel der Inhaltsebene zugeordnet ist. Die beiden beteiligten Ebenen sind derart geordnet, dass die Argumentationsebene \is{Argumentationsebene} der Inhaltsebene \is{Inhaltsebene} übergeordnet ist und erstere letztere somit in ihren Wirkungsbereich nimmt. In einer \textit{ja doch}-Äußerung werde durch \textit{doch} auf einen inhaltlichen Gegensatz verwiesen, \textit{ja} bekräftige diese Aussage anschließend und weise ihr damit ihren Stellenwert in der Argumentation zu. Interpretatorisch geht die Inhaltsebene der Argumentationsebene voran. Aus der Hierarchie \textit{Argumentation(Inhalt)} folgt dann (neben der (Skopus-)Interpretation \textit{ja(doch))} die Abfolge \textit{ja doch}. Der Ausschluss der umgekehrten Abfolge ist in Icklers Zugang auf die Art motiviert, dass er das umgekehrte Zusammenspiel zwischen den zwei Ebenen nicht für möglich hält (\citeyear[404]{Ickler1994}). Wenn zwischen den Ebenen das Ordnungsverhältnis \textit{Argumentation(Inhalt)} besteht, ist nicht denkbar, dass die mit der höheren Ebene assoziierte MP (\textit{ja}) in den Wirkungsbereich der der tieferen Ebene zugeschriebenen MP (\textit{doch}) fällt. Dies würde eine Umkehr des zwischen Inhalts- und Argumentationsebene bestehenden hierarchischen Verhältnisses bedeuten. Da diese Ordnung sowie die Zuordnung der beiden Partikeln zu den Ebenen invariant ist, ist die umgekehrte Abfolge \textit{doch ja} generell ausgeschlossen.

Anders als die vier angeführten Arbeiten, argumentiert \citet{Lindner1991} unter Bezug auf ein phonologisches Kriterium. Entscheidend für ihre konkrete Er\-klärung ist zum einen die \textit{Hierarchie konsonantischer Stärke} \is{konsonantische Stärke} und zum anderen drei \textit{Präferenzgesetze} \is{Präferenzgesetz} für den Silbenbau und die Silbenverkettung, die auf diesem Konzept basieren (vgl. \citealt[283/284]{Vennemann1982}). (\ref{384}) zeigt die Skala konsonantischer Stärke, die genau entgegen der üblicheren \textit{Sonoritätshierarchie} \is{Sonoritätshierarchie} verläuft.

\begin{exe}
	\ex\label{384} 
	Hierarchie konsonantischer Stärke\\
	Vokale < Liquide < Nasale < stimmhafte Frikative < stimmhafte Plosive/\\stimmlose Frikative < stimmlose Plosive
\end{exe}
Zu den sechs Präferenzgesetzen, die Vennemann formuliert, gehört z.B. \is{Silbenkontaktgesetz} das \textit{Silbenkontaktgesetz}. Es besagt, dass wenn zwei Silben aufeinander treffen, der letzte Laut der ersten Silbe und der erste Laut der zweiten Silbe eine große Differenz hinsichtlich ihrer konsonantischen Stärke aufweisen. Das \textit{Endrandgesetz} \is{Endrandgesetz} und das \textit{Anfangsrandgesetz} \is{Anfangsrandgesetz} beschreiben ferner, dass der letzte Laut der ersten Silbe wenig und der erste Laut der zweiten Silbe viel konsonantische Stärke aufweist.

Die Präferenz für die Abfolge \textit{ja doch} (gegenüber \textit{doch ja}) führt Lindner nun darauf zurück, dass der Silbenkontakt im ersten Fall besser ist als im zweiten: Sie klassifiziert /x/ als stimmlosen velaren Frikativ und /j/ als stimmhaften palatalen Frikativ. Nach der Hierarchie in (\ref{384}) folgen der stimmhafte und stimmlose Frikativ direkt aufeinander, während zwischen dem Vokal /a:/ und dem stimmhaften Plosiv /d/ drei Stufen liegen. Die Differenz konsonantischer Stärke zwischen dem Endrand der Vorsilbe und dem Anfangsrand der Folgesilbe ist bei \textit{ja doch} somit deutlich größer, so dass ein nahezu optimaler Silbenkontakt vorliegt. Nach \citet{Lindner1991} ist folglich der schlechtere Silbenkontakt bei \textit{doch ja} und der bessere Silbenkontakt im Falle von \textit{ja doch} dafür verantwortlich, dass die Abfolge \textit{ja doch} die präferierte ist.

An dieser Stelle erachte ich es für relevant, dass Lindners Ableitung des bevor\-zugten \textit{ja doch} (im Gegensatz zu den Modellen von Doherty, Ickler, Rinas und Ormelius-Sandblom) die umgekehr\-te Abfolge nicht kategorisch ausschließt. Sie selbst geht auf diese Möglichkeit zwar nicht ein, doch ihr Erklärungsmodell erlaubt sie prinzi\-piell, wenngleich es (korrekterweise) die Bevorzugung von \textit{ja doch} vorhersagt. Nicht-optimaler Silbenkontakt lässt sich schließlich generell \\ durchaus beobachten (vgl. z.B. \textit{himmlisch} $[/m/–/l/]$, \textit{Ampel} $[/m/–/p/]$ und \textit{Alpen} $[/l/–/p/]$). Und wie \citet[430]{Rinas2007}) bemerkt, ist auch das direkte Aufein\-andertreffen von /x/ und /j/ zu belegen: \textit{Ach ja!}, \textit{Ich mach ja gar nichts.}

Dieser erneute Blick auf Ansätze, die sich mit der MP-Kombination aus \textit{ja} und \textit{doch} beschäftigen, zeigt, dass die Arbeiten derart ausgerichtet sind, Gründe für die Abfolge \textit{ja doch} anzuführen. Die Sequenzierung \textit{doch ja} wird als inakzeptable Reihung angesehen, die es gilt herauszufiltern. Die Existenz der Reihung \textit{doch ja} wird in keiner der mir bekannten Arbeiten in Betracht gezogen und sie lassen in der Regel auch keine Option, die ihre E\-xistenz zulassen würde (vgl. auch schon \citealt[170-175]{Mueller2014a}).

\section{Die Distribution von \textit{doch ja}}
\label{sec:distributiondj}
Auf der Basis der Betrachtung großer Datenmengen in verschiedenen Korpora (DeReKo, DECOW, deWac, DWDS, Projekt Gutenberg) sowie \glq wilderen\grq {} Google-Suchen möchte ich neue Daten ins Feld führen, die die Annahme, dass die umge\-kehrte Abfolge \textit{doch ja} non-existent und kategorisch auszuschließen ist, herausfordern. Es gibt rund um diese Daten Aspekte, die als gesichert gelten können. An anderen Stellen sind auch sicherlich Fragen offen, die ich hier nicht alle beantworten kann und die auch die Grenzen der empirischen Möglichkeiten aufzeigen. Der Punkt des folgenden Abschnitts ist, dass sich deskriptiv drei Klassen ergeben, in die sich die Belege einordnen lassen. Basis dieser Klassenbildung sind ca. 150 Belege, die größtenteils, aber nicht ausschließlich, aus Webdaten stammen (DECOW). Da die Annahme der Belegbarkeit der umgekehrten Abfolge bisher nicht unternommen wurde, gebe ich viele Beispiele an. Ich bin nicht der Meinung, dass \textit{doch ja} in den angegebenen Kontexten akzeptabler ist als \textit{ja doch}. Diese Abfolge wird auch in diesen Kontexten deutlich bevorzugt. Auch ist die umgekehrte Abfolge in Korpora (im Vergleich zu den \textit{ja doch}-Treffern) unterrepräsentiert. Die Belege, die man finden kann, unterliegen aber entscheidenderweise einer Syste\-matik, was sie für mich zu einem interessanten Untersuchungsgegenstand macht. In Abschnitt~\ref{sec:status} diskutiere ich einige Aspekte und Fragen zum Status dieser Abfolge.
  
Ich gehe von den folgenden drei Klassen aus (vgl. auch schon \citealt[175-177]{Mueller2014a}; \citeyear[207-210]{Mueller2017b}): a) \is{Bewertung} Bewertungen, b) \is{epistemische Modalisierung} epistemisch modalisierte Sätze, c) \is{epistemischer Kausalsatz} \is{illokutionärer Kausalsatz} epistemisch bzw. illokutionär interpretierte Kausalsätze. In (\ref{385}) bis (\ref{387}) finden sich einige Beispiele für jede Klasse. Wie ich in Abschnitt~\ref{sec:markiert} erläutern werde, kann man verschiedene konkrete Realisierungen dieser Klassen ausmachen, d.h. das Auftreten verschiedener lexikalischer Mittel kann für die jeweilige Klassenzuordnung verantwortlich sein. 

In die erste Klasse der Bewertungen fallen Äußerungen, mit denen der Sprecher ein qualitatives Urteil über einen Sachverhalt abgibt, in (\ref{385}) bis (\ref{387}) anhand \is{evaluatives Adjektiv} eva\-luativer Adjektive.

\begin{exe}
	\ex\label{385}
	\scriptsize
	 \textbf{Das ist \underline{doch ja} wieder \textit{typisch}.} Ein \glqq Nerd\grqq{} läuft Amok wegen Frust auf Weib \&
 		Lehrer.
 	\newline
	\hbox{}\hfill\hbox{(http://webcache.googleusercontent.com/search?q=ca\_ch4Rz\\lhWolBoJ:identi.ca/notice/}
	\newline
	\hbox{}\hfill\hbox{66906092+\%22doc\_h+ja\%22\&cd=825\&hl=de\&ct=clnk\&gl=de\&source=www.google.-de)}
	\newline
	\hbox{}\hfill\hbox{(Beitrag vom 13.03.2011) (eingesehen über WebasCorpus am 21.09.2011)}						
	\newline
	\hbox{}\hfill\hbox{\citet[176]{Mueller2014a}}	 
\end{exe}

\begin{exe}
	\ex\label{386}
	\scriptsize
	Dann habe ich noch ein Problem. jetzt werden meine Beiträge die ich geschrieben habe auch als neu erkannt. \textbf{Das ist \underline{doch ja} 				\textit{falsch}}, für mich sind sie nicht neu. 
	\hfill\hbox{(DECOW2012-03X: 141373874)}	 
\end{exe}

\begin{exe}
	\ex\label{387}
	\scriptsize
	Ich denke, auch die meisten Frauen merken schon irgendwann rechtzeitig, dass sie selbst weiblich sind und wenn sie dann mal als Mann angesprochen 			werden, ist es wohl auch keine Kränkung. \glqq Sehr geehrte Frau Minister!\grqq{} \textbf{Ist \underline{doch ja} auch ganz \textit{hübsch}.} 
	\hfill\hbox{(DECOW2012-06: 697189512)}	 
	\newline
	\hbox{}\hfill\hbox{\citet[201]{Mueller2014a}}	
\end{exe}
Die zweite Klasse der epistemisch modalisierten Sätze wird durch (\ref{388}) bis (\ref{392}) illustriert. Hier tritt die \textit{doch ja}-Abfolge in Kontexten auf, die beispielsweise durch \is{epistemisches Modalverb} epistemisch interpretierte Modalverben (wie in (\ref{388}) \textit{sollten}, in (\ref{389}) \textit{müssten}), \is{epistemisches Adverb} modalisierende Adverbien (vgl. (\ref{390}), (\ref{391})) oder \is{Tag-Frage} auch Tag-Fragen (vgl. (\ref{392})) charakterisiert sind.
													         	     
\begin{exe}
	\ex\label{388}
	\scriptsize
	Nur zur Vollständigkeit: Was muss beim löschen des Computerkontos im AD denn noch beachtet werden? Einfach danach wieder in die Domäne bringen und 			fertig?\\
	\textbf{Benutzerrechte \textit{sollten} [sich] \underline{doch ja} nicht ändern} – gibt es noch Fallen? 
	\newline
	\hbox{}\hfill\hbox{(http://www.benutzer.de/Neuer\_Client\_unter\_gleichem\_}	
	\newline
	\hbox{}\hfill\hbox{Namen\_wieder\_in\_die\_Dom\%C3\%A4ne\_-\_geht\_das)}	
	\newline
	\hbox{}\hfill\hbox{(Google-Suche, eingesehen am 09.06.2012)}
\end{exe}																   	

\begin{exe}
	\ex\label{389}
	\scriptsize
	Wenn der Overmind schon Psi hat, wieso kann er nicht einen kollektiven Psi-Pool erstellen? \textbf{Die Eigenschaften \textit{müssten} \underline{doch 		ja} vererbbar sein...} Wieso können die Hellen Tepmler dann nicht kollektiv einen Zerebraten angreifen?                                               
	\hfill\hbox{(DECOW2012-05: 1062987308)}	
\end{exe}
	
\begin{exe}
	\ex\label{390}
	\scriptsize
	Wer aber sitzt, sieht kaum noch etwas von den stilvollen Leinwandfotos. Man sollte also einmal versuchen, seine Wandbilder etwas niedriger zu hängen. 		\textbf{Das trifft \underline{doch ja} \textit{eigentlich} auf jegliche Deko zu.} Schließlich stellt man um ein Beispiel zu nennen eine Blumenvase ja 		auch auf keinen Fall auf den Schrank, sondern drapiert sie auf einem schicken Sideboard. 
	\newline
	\hbox{}\hfill\hbox{(http://neueinrichtenjedentag.want2blog.net/2012/07/25/keine}	
	\newline
	\hbox{}\hfill\hbox{-behausung-ohne-wandbilder/, Beitrag vom 11.10.2005)}	
	\newline
	\hbox{}\hfill\hbox{(Google-Suche, eingesehen am 24.07.2012)}
\end{exe}	
	
\begin{exe}
	\ex\label{391}
	\scriptsize
	Oh je wer das wirklich denkt ist Blind...!!
	\textbf{Im VISA Werbespot das ist \underline{doch ja} \textit{nun wirklich} eindeutig ein S2.}
	Die Rahmenbreite stimmt, vordere Cam, Lautsprecher und Schrift stimmen, Powerknopf an der Seite stimmt und die Klinkenbuchse stimmt auch!
	\newline
	\hbox{}\hfill\hbox{(http://www.areamobile.de/b/1546-samsung-galaxy-}	
	\newline
	\hbox{}\hfill\hbox{s3-die-angeblichen-leaks-und-konzepte)}	
	\newline
	\hbox{}\hfill\hbox{(Google-Suche, eingesehen am 24.07.2012)}
\end{exe}						
							               	
\begin{exe}
	\ex\label{392}
	\scriptsize
	Meine alte Grafikkarte hatte ich so ausgebaut, ohne vorher auf irgendwelche Deinstallationen zu achten. Allerdings habe ich dann die Festplatte 			formatiert und ein frisches Windows installiert. \textbf{Demnach habe ich \underline{doch ja} beim Grafikkartenwechsel nichts falsch gemacht, 				\textit{oder??}}
	\newline
	\hbox{}\hfill\hbox{(http://www.drwindows.de/hardware-and-treiber/53 }	
	\newline
	\hbox{}\hfill\hbox{120-treiber-fuer-radeon-hd7850-mit-aero.html)}	
	\newline
	\hbox{}\hfill\hbox{(Google-Suche, eingesehen am 24.07.2012)}
	\newline
	\hbox{}\hfill\hbox{\citet[201]{Mueller2014a}}
\end{exe}		     
Es handelt sich hierbei um sprachliche Elemente, die ein eingeschränktes Sprecherbekenntnis kodieren. Oftmals treten derartige Markierungen auch in Kombination auf, wie in (\ref{393}) z.B. (hier: Adverb \textit{eigentlich} + epistemisches Modalverb \textit{sollte} + MP \textit{wohl} + \is{Diskursmarker} Diskursmarker \textit{denke ich}).
\begin{exe}
	\ex\label{393}
	\scriptsize
	Ich persönlich halte \glqq Alles für eine umfassendere und nicht ausschließende Einstellung. Aber bestätigen kann ich Dir das erst nach vielen Testfahrten ;-). \textbf{\textit{Eigentlich sollte} sich ein solches Gerät bei der Einstellung \glqq Sendersuche: automatisch\grqq{} \underline{doch ja} \textit{wohl} den besten, aber empfangbaren Sender nehmen, so \textit{denke ich}.}
	\hfill\hbox{(http://www.gopal-navigator.de/archive/index.php/t-5689.html)                                                                      }
	\newline
	\hbox{}\hfill\hbox{120-treiber-fuer-radeon-hd7850-mit-aero.html)}	
	\newline
	\hbox{}\hfill\hbox{(Google-Suche, eingesehen am 21.07.2014)}
	\newline
	\hbox{}\hfill\hbox{\citet[208]{Mueller2017b}}
\end{exe}
Die dritte Klasse konstituiert sich durch Kausalsätze, die keine Ursache-Wirkung-Relation ausdrücken, sondern erklären, warum der Sprecher eine bestimmte Annahme macht (bzw. allgemeiner eine bestimmte Einstellung vertritt) oder warum er einen bestimmten Sprechakt ausführt. 

In (\ref{394}) begründet der \textit{denn}-Satz die Einschätzung aus dem ersten Teil des Satzes, der zudem epistemisch modalisiert ist (\textit{dürfte}), wie es für Hauptsätze, auf die sich epistemische Kausalsätze beziehen, typisch ist. 
        
\begin{exe}
	\ex\label{394}
	\scriptsize
	Bei seiner letzen Sendung am 21.01.09 schalten wieder 10,63 Millionen TV  Zuschauer eine Sendung ein. Zur gleichen Zeit lief bei RTL \glqq Ich bin ein 	Star (Rentnerin \glqq Ingrid van Bergen\grqq{}) holt mich hier raus\grqq{}. \emph{Am 28.02.09 dürfte er dann noch einige junge Zuschauer weniger haben}, 		\textbf{\textit{denn} diese möchten \underline{doch ja} garantiert DSDS sehen}, da geht es ja um \glqq Alles oder Nichts\grqq{} bei den 15 DSDS 				Sternchen. 	
	\newline
	\hbox{}\hfill\hbox{(http://www.deutschlands-superstar.de/2009/02/27/                                                                                     	wettendass-gegen-dsds-die-entscheidung/)}	
	\newline
	\hbox{}\hfill\hbox{(eingesehen am 09.06.2011)}
	\newline
	\hbox{}\hfill\hbox{\citet[176]{Mueller2014a}}
\end{exe}        
Ähnlich begründet der Sprecher in (\ref{395}) seine Annahme, dass für die Liga kein Problem entstünde.

\begin{exe}
	\ex\label{395}
	\scriptsize
	Wenn Mannheim jetzt in die OL durchmüsste... \emph{wär das doch für die RL kein Problem...} \textbf{dann kann die \underline{doch ja} auch wieder mit 		18 Teams spielen}... je nachdem was mit den Abstiegern geregelt wär... ach warum nur alles so kompliziert!? 	
	\hfill\hbox{(DECOW2012-01: 956951068)}
\end{exe} 
(\ref{396}) und (\ref{397}) zeigen Beispiele für illokutionär \is{illokutionärer Kausalsatz} verwendete Kausalsätze.
 
\begin{exe}
	\ex\label{396}
	\scriptsize
	Für das Anfang Januar durch ein Feuer stark beschädigte CapaHaus in der Jahnallee 61 besteht immer noch akute Einsturzgefahr. [...] 
	Um das Capa-Haus schlussendlich zu retten, müsste es umfassend saniert werden, unter anderem durch die Abdichtung des Daches und die Instandsetzung 		der durch das Feuer beschädigten Teile. Die Veranlassung dieser Arbeiten obliegt aber dem Eigentümer, so das Amt weiter.\\
	Gepostet: 01.02.2012 12:31 anonym\\
	\emph{einfach abreißen!} 
	\textbf{\textit{denn} es fehlen \underline{doch ja} schon sämtliche zwischen decken} [...] 		
	\newline
	\hbox{}\hfill\hbox{(http://www.leipzigfernsehen.de/default.aspx?ID=5846\&showNews=}	
	\newline
	\hbox{}\hfill\hbox{1107962, Beitrag vom 01.02.2012) (eingesehen am 24.07.2012)}
	\newline
	\hbox{}\hfill\hbox{\citet[176]{Mueller2014a}}
\end{exe}        

\begin{exe}
	\ex\label{397}
	\scriptsize
	\emph{zu \glqq wie vielt\grqq{} müssen wir sein, damits als guild run zählt}, \textbf{\textit{weil} mc \underline{doch ja} solo machbar wär}^^ -– aber 	nichtsdesdotrotz bin ich dabei -– wird sicher n 	spaß!
	\newline
	\hbox{}\hfill\hbox{(http://webcache.googleusercontent.com/search?q=cache:ta5}	
	\newline
	\hbox{}\hfill\hbox{\%3Ff\%3D63\%26p\%3D29269+\%22doch+ja\%22\&cd=440\&}
	\newline
	\hbox{}\hfill\hbox{hl=de\&ct=clnk\&gl=de\&source=www.google.de)}
	\newline
	\hbox{}\hfill\hbox{(Google-Suche, eingesehen am 09.06.2012)}
\end{exe} 

In (\ref{396}) macht der Sprecher den Vorschlag, das Gebäude abzureißen und gibt dann an, warum er diesen Vorschlag macht: Die Zwischendecken fehlen schon. Hier wird folglich ein direktiver Sprechakt begründet. In (\ref{397}) handelt es sich bei dem zu begründenden Sprechakt um eine Frage: Die Aussage, dass das Beispiel allein machbar sei, ist das Motiv für die Frage, wie viele Spieler benötigt werden.\\

\noindent	
Ich nehme an, dass es nicht der Fall ist, dass \textit{ja doch} die grammatische Abfolge und \textit{doch ja} die ungrammatische Reihung ist, wie in den in Abschnitt~\ref{sec:abfolgejd} skizzier\-ten Ansätzen und überhaupt allen mir bekannten Arbeiten vertreten wird, sondern dass \textit{ja doch} die \underline{unmarkierte} \is{Markiertheit} und \textit{doch ja} die \underline{markierte} Abfolge ist. Letztere tritt deshalb seltener auf und ist vor allem auf bestimmte Kontexte beschränkt, die sich präzise benennen und auf der Basis von kookurrierendem sprachlichen Material charakterisieren lassen. Ich nehme an, dass \textit{ja doch} in jedem Kontext auftreten kann, in dem auch \textit{doch ja} auftritt. Allerdings kann \textit{doch ja} nicht in jedem Kontext auftreten wie \textit{ja doch}. Dies zeigen die Beispiele aus der Literatur in (\ref{376}) bis (\ref{378}) (hier wiederholt in (\ref{398}) bis (\ref{400})), deren Beurteilung ich teile.

\begin{exe}
	\ex\label{398} 
	Konrad ist \textbf{ja doch}/*\textbf{doch ja} verreist.
	\hfill\hbox {\citet[114]{Doherty1987}}
\end{exe}
\vspace{-0.5cm}	
\begin{exe}
	\ex\label{399} 
	Er hat \textbf{ja doch}/??\textbf{doch ja} getanzt.
	\hfill\hbox {\citet[20]{Struckmeier2014}}
\end{exe}
\vspace{-0.5cm}	
\begin{exe}
	\ex\label{400} 
	Er hat sich \textbf{ja doch}/?\textbf{doch ja} sehr um sie bemüht.
	\hfill\hbox {\citet[157]{Jacobs1991}}
\end{exe}
Die zwei Fragen, die ich im Folgenden auf der Basis dieser Ausgangslage beantworten möchte, sind: a) Warum ist \textit{ja doch} die unmarkierte Abfolge, die in allen assertiven Kontexten auftreten kann? b) Warum lässt sich die Abfolge in den drei charakterisierten Kontexten umkehren?
	
\section{Der interpretatorische Beitrag von \textit{ja}, \textit{doch} sowie ihrer Kombination}		
\label{sec:interpretationjd}			
Die im Folgenden beschriebene Analyse (vgl. auch \citealt[183-197]{Mueller2014a}; \citeyear[214-223]{Mueller2017b}) fußt zum einen auf dem Effekt, den MP-lose Assertionen auf den Diskurs\-kontext nehmen und zum anderen auf einer Modellierung des kontextuellen Effektes von \textit{ja} und \textit{doch} in Isolation, auf den ich die Interpretation der Kombination aufbauen werde. In Abschnitt~\ref{sec:diskursmodell}, Kapitel~\ref{chapter:hintergrund} wird das zugrunde gelegte formale Diskursmodell nach \citet{Farkas2010} ausführlich eingeführt. Der besseren Lesbarkeit halber seien an dieser Stelle die für die weitere Argumentation relevanten Aspekte des Modells bzw. der MP-Auffassung erneut angeführt (vgl. die Abschnitte~\ref{sec:mplass} und \ref{sec:mpn}), bevor sich Abschnitt~\ref{sec:inkdm} mit dem Bedeutungseffekt von \textit{ja}- und \textit{doch}-Äußerungen beschäftigt und in Abschnitt~\ref{sec:kombi} das kombinierte Auftreten betrachtet wird.
								
\subsection{Modalpartikellose Assertionen}
\label{sec:mplass}
Wird eine Assertion wie beispielsweise in (\ref{402}) im Diskurs geäußert, führt dies nach \citet{Farkas2010} vor dem Hintergrund des Ausgangszustands K$_{1}$ (vgl. (\ref{401})) die im Folgenden beschriebenen Kontextveränderungen mit sich.
     
\newcolumntype{C}[1]{>{\centering}p{#1}} 
\begin{exe}
\ex\label{401} K$_1$: initialer Kontextzustand\\[-0.6em]
\begin{tabular}[t]{| C{6em}|p{6em}|p{6em}|  C{6em}|}
\hline
 $\textrm{DC}_{\textrm{A}}$ & \multicolumn{2}{C{12em}|}{Tisch} &  $\textrm{DC}_{\textrm{B}}$ \tabularnewline
\hline
{} & \multicolumn{2}{p{12em}|}{} & {}  \tabularnewline
\hline
\multicolumn{2}{|p{12em}|}{cg $\textrm{s}_{1}$}&\multicolumn{2}{|p{12em}|}{$\textrm{ps}_{1} = \lbrace \textrm{s}_{1} \rbrace$} \tabularnewline
\hline
\end{tabular}
\end{exe}		                              
								  
\begin{exe}
	\ex\label{402} 
	A: Die Spieler der 2. Geige sind die wahren Geiger.
\end{exe}									         						   
Die ausgedrückte Proposition p wird dem Diskursbekenntnissystem \is{Diskursbekenntnissystem (discourse commitment set)} (\textit{discourse commitment set}) von Diskursteilnehmer A (DC$_{\textrm{A}}$) hinzugefügt. Die Proposition wird samt ihrer Negation auf den Tisch \is{Tisch} obenauf gelegt. Der cg-Zustand bleibt im Vergleich zu K$_{1}$ unverändert. Die projizierte Zukunft des cg, das \is{Projektionsmenge (projected set)} \textit{projected set} (ps), wird um p erweitert, da die kanonische Reaktion auf eine Assertion die Bestätigung dieser Assertion ist.

\newcolumntype{C}[1]{>{\centering}p{#1}} 
\begin{exe}
\ex\label{403} K$_{2}$: A hat assertiert: \textit{Die Spieler der 2. Geige sind die wahren Geiger.} relativ zu K$_{1}$\\[-0.6em]
\begin{tabular}[t]{| C{6em}|p{6em}|p{6em}|  C{6em}|}
\hline
 $\textrm{DC}_{\textrm{A}}$ & \multicolumn{2}{C{12em}|}{Tisch} &  $\textrm{DC}_{\textrm{B}}$ \tabularnewline
\hline
p & \multicolumn{2}{C{12em}|}{p $\vee$ $\neg$p} & {}  \tabularnewline
\hline
\multicolumn{2}{|p{12em}|}{cg $\textrm{s}_{2}$ = $\textrm{s}_{1}$}&\multicolumn{2}{|p{12em}|}{$\textrm{ps}_{2} = \lbrace \textrm{s}_{1} \cup \lbrace \textrm{p} \rbrace \rbrace$} \tabularnewline
\hline
\end{tabular}
\end{exe}		
Wenn Diskursteilnehmer B den Inhalt der Äußerung in (\ref{402}) annimmt/bestä\-tigt/ak\-zeptiert, wird p ebenfalls zu einem Diskursbekenntnis von Gesprächs\-teilnehmer B (vgl. (\ref{404a})). Als bewusst geteiltes öffentliches Diskursbekenntnis wird p zu einem cg-Inhalt und das Thema wird (da p $\vee$ $\neg$p nun nicht mehr zur Diskussion steht) vom Tisch geräumt (vgl. (\ref{404b})).


\newcolumntype{C}[1]{>{\centering}p{#1}} 
\begin{exe}
\ex\label{404} K$_{3}$: B bestätigt As Beitrag \\[-1em]
\begin{xlist}	
			\ex\label{404a} 
			Teil 1\\[-1em]
			\begin{tabular}[t]{| C{6em}|p{6em}|p{6em}|  C{6em}|}
			\hline
 			$\textrm{DC}_{\textrm{A}}$ & \multicolumn{2}{C{12em}|}{Tisch} &  $\textrm{DC}_{\textrm{B}}$ \tabularnewline
			\hline
			p & \multicolumn{2}{C{12em}|}{p $\vee$ $\neg$ p} & {p}  \tabularnewline
			\hline
			\multicolumn{2}{|p{12em}|}{cg $\textrm{s}_{3}$ = $\textrm{s}_{1}$}&\multicolumn{2}{|p{12em}|}{$\textrm{ps}_{3} = \textrm{ps}_{2}$} 							\tabularnewline
			\hline
			\end{tabular}
\end{xlist}

\begin{xlist}	
			\ex\label{404b} 
			Teil 2\\[-1em]
			\begin{tabular}[t]{| C{6em}|p{6em}|p{6em}|  C{6em}|}
			\hline
 			$\textrm{DC}_{\textrm{A}}$ & \multicolumn{2}{C{12em}|}{Tisch} &  $\textrm{DC}_{\textrm{B}}$ \tabularnewline
			\hline
			{} & \multicolumn{2}{C{12em}|}{} & {}  \tabularnewline
			\hline
			\multicolumn{2}{|p{12em}|}{cg $\textrm{s}_{4} = \lbrace \textrm{s}_{1} \cup \lbrace \textrm{p} \rbrace \rbrace$}&\multicolumn{2}{|p{12em}|}{$				\textrm{ps}_{4} = \lbrace \textrm{s}_{4} \rbrace$} \tabularnewline
			\hline
			\end{tabular}
\end{xlist}
\end{exe}
Entscheidend für meine Antwort auf die zwei Fragen vom Anfang ist Farkas \& Bruce' Überlegung zu kanonischen Verhaltensweisen in der Konversation (vgl. auch hierzu ausführlicher Abschnitt~\ref{sec:diskursmodell} in Kapitel~\ref{chapter:hintergrund}). Prinzipiell wird Konversation ihrer Ansicht nach durch die zwei Aspekte in (\ref{405}) getrieben (vgl. \citealt [87]{Farkas2010}).

\begin{exe}
	\ex\label{405} 
		Zwei fundamentale Antriebe für Gespräche
		\begin{xlist}	
			\ex\label{405a} Erweiterung des cg
			\ex\label{405b} Herstellung eines stabilen Kontextzustands
		\end{xlist}
\end{exe}
Gesprächsteilnehmer folgen dem Bedürfnis, den cg zu erweitern. Aus diesem Grund legen sie Elemente auf dem Tisch ab. Ferner streben sie danach, einen Kontextzustand zu schaffen, in dem sich kein Element auf dem Tisch befindet.

Neben diesen allgemeinen Diskursbestreben formulieren die Autoren zudem kanoni\-sche Reaktionen, die mit einzelnen Sprechakten verbunden sind. Die kanonische Reaktion auf Assertionen ist die Bestätigung der Assertion. In dem Sinne, dass der Sprecher der Assertion ein öffentliches Diskursbekenntnis zu ihrem Inhalt macht und die enthaltene Proposition mit ihrer Alternative auf den Tisch gelegt wird, machen Assertionen Vorschläge. Durch die Annahme dieses Vor\-schlags, d.h. die Akzeptanz/Zustimmung durch den Gesprächsteilnehmer, wird diese Proposition auf direktem Wege zu einem cg-Inhalt. Diese Überlegung lässt sich auffassen als Vorliegen einer Art von konversationellem Druck, den cg anzurei\-chern, indem veröffentlichte Bekenntnisse zu geteilten Bekenntnissen gemacht werden. Um dies zu erfüllen, muss der Gesprächspartner den Inhalt annehmen. In diesem Sinne hat eine Assertion eine Voreingenommenheit zugunsten der ausgedrückten Proposition. Vor dem Hintergrund dieser Überlegungen formulieren die Autoren Kriterien, die von prototypischen und weniger prototypischen Assertionen \is{prototypische Assertion} in unterschiedlichem Maße erfüllt werden. (\ref{406}) zeigt die formalisierte Version der Autoren, (\ref{407}) eine Paraphrasierung dieser Darstellung.

\begin{exe}
	\ex\label{406} 
		S($[\textrm{D}], \textrm{a}, \textrm{K}_{\textrm{i}}) = \textrm{K}_{\textrm{o}}$ so that
		\begin{xlist}	
			\ex\label{406a} $\textrm{DC}_{\textrm{a,o}} = \textrm{DC}_{\textrm{a, i}} \cup \lbrace\textrm{p}\rbrace$
			\ex\label{406b} $\textrm{T}_{\textrm{o}} = \textrm{push}(\langle \textrm{S}[\textrm{D}]; \lbrace \textrm{p} \rbrace \rangle, \textrm{T}_{\textrm{i}})$
			\ex\label{406c} $\textrm{ps}_{\textrm{o}} =  \textrm{ps}_{\textrm{i}} \ \overline\cup \ \lbrace \textrm{p} \rbrace$
			\hfill\hbox {\citet[92]{Farkas2010}}
		\end{xlist}
\end{exe}

\begin{exe}
	\ex\label{407} 
	Nachdem ein Sprecher eine Assertion mit Proposition p geäußert hat, gilt:
		\begin{xlist}	
			\ex\label{407a} Die neue Diskursbekenntnismenge des Sprechers beinhaltet p.
			\ex\label{407b} Die Proposition p (vs. $\neg$p) wird oben auf den Stapel des alten Tisches gelegt.
			\ex\label{407c} Die projizierte Zukunft des alten cg beinhaltet p (unter Bewahrung der Konsistenz des cg).
		\end{xlist}
\end{exe}
Wichtig für meine Ableitung der Markiertheit von \textit{doch ja} und Unmarkiertheit von \textit{ja doch} ist, dass der Effekt in (\ref{407a}) auf alle Assertionen zutrifft, d.h. alle Äußerungen, die man als assertiv einstuft, haben den Effekt im Kontext, dass die Diskursbekenntnismenge des Autors der Assertion mit der ausgedrückten Proposition angereichert wird. Dieser Effekt trifft auf prototypische Assertionen (s.u.) genauso zu wie auf weniger prototypische und auf selbständige Sätze genauso wie auf assertive Nebensätze. Eine als Standardassertion oder mit anderen Worten \textit{prototypische Assertion} eingestufte Äußerung muss hingegen alle drei Kriterien erfüllen: Der Sprecher der Assertion bekennt sich zu ihrem propositionalen Inhalt, die Assertion eröffnet ein Thema, indem sie ein Element auf den Tisch legt und sie lenkt die Konversation in die Richtung einer Auflösung des Themas durch die Bestätigung der ausgedrückten Proposition. Eine kano\-nische Assertion führt zu einem Kontext, der kategorisch voreingenommen ist hinsichtlich einer Bestätigung der assertierten Proposition. 

Ein Aspekt, den ich in der Beschreibung des Diskursmodells in Abschnitt~\ref{sec:diskursmodell} ausgelassen habe, betrifft die Art von Information, die auf dem Tisch liegt und somit im Diskurs zur Diskussion stehen kann. In den bisher angeführten Beispielen handelte es sich um wörtliche Bedeutungsanteile, die mitgeteilt und zum Thema des Gesprächs werden. Assertionen und andere Sprechakte sind bekannt\-lich nicht nur mit wörtlichem Inhalt assoziiert, sondern beispielsweise auch mit implikatiertem. \citet[94]{Farkas2010} nehmen an, dass auch dieser auf dem Tisch abgelegt werden kann. Weitere Diskursbeiträge können sowohl auf wörtlichen als auch implikatierten Inhalt Bezug nehmen.\footnote{\citet[94]{Farkas2010} machen für die Zwecke des von ihnen untersuchten Phänomens einen Unterschied hinsichtlich der Art, mit der derartige Inhalte auf dem Tisch liegen. Im\-plikatierter Inhalt sei zwar verbunden mit, aber getrennt von wörtlichem Inhalt. Damit meinen die Autoren, dass nur der wörtliche Inhalt mit syntaktischem Material auf dem Tisch verbunden ist. Da die syntaktische Information für die von mir untersuchten Strukturen nicht relevant ist, ist es für meine Argumentation unerheblich, ob implikatierte Inhalte auf die gleiche oder eine andere Art auf dem Tisch verfügbar sind wie wörtliche Inhalte.} Dass implikatierte Inhalte auf dem Tisch liegen können, lässt sich leicht dadurch zeigen, dass Implikaturen \is{Implikatur} thematisiert und in Frage gestellt werden können (vgl. (\ref{408})).

\begin{exe}
	\ex\label{408} 
	A: Peter hat Maria mit einem Mann in der Stadt gesehen.\\
	B: Aha. Interessant. Vor allem, dass sie nicht mit ihrem Ehemann in die Stadt geht.\\
	A: Oh nein nein, es war schon ihr Ehemann.
\end{exe}
In (\ref{408}) hat die Äußerung von Sprecher A (mit p = dass Peter Maria mit einem Mann in der Stadt gesehen hat) die Implikatur $\neq$q (= dass Peter Maria nicht mit ihrem Ehemann in der Stadt gesehen hat). Genau diesen Inhalt kann B thematisieren und somit gelangt über p auch $\neg$q auf den Tisch und q $\vee$ $\neg$q kann ein potenzielles Thema sein.\footnote{\label{Fn16}Eine offene Frage ist, ob man annehmen möchte, dass Sprecher sich auch stets zu den Implikaturen ihrer Äußerungen bekennen, d.h. ob A in diesem Fall auch ein Bekenntnis zu $\neg$q hat. Ich halte ein Bekenntnis zu implikatierten Inhalten nicht für plausibel und zwar insbesondere aus dem Grund, dass sich die Frage zuspitzt, je pragmatischer (d.h. auch kontextgebundener) derartige Inferenzen werden. Die gleiche Frage kann man stellen zu Präsuppositionen und konventionellen Implikaturen. Wenn ein öffentliches Diskursbekenntnis auch beinhaltet, dass ein Sprecher sich dieser gegebenen Information bewusst ist, möchte ich doch in Frage stellen, dass ein Sprecher sich aller Schlüsse seiner Äußerungen bewusst ist. Die Antwort auf diese Frage ist an dieser Stelle aber auch nicht entscheidend, anders als die unkontroverse Annahme, dass eine derartige implikatierte (oder auch präsupponierte) Information ebenfalls auf dem Tisch liegen kann und die Gesprächsteilnehmer auf diese reagieren können.}
	
\begin{exe}
	\ex\label{409}
     \begin{tabular}[t]{ll}
     		A: & Peter hat Maria mit einem Mann in der Stadt gesehen.\\
            DC$_{\textrm{A}}$: & p\\
            Tisch: & p $\vee$ $\neg$p\\
 			 & q $\vee$ $\neg$q\\
 			DC$_{\textrm{B}}$: & --
      \end{tabular}
\end{exe}	
Denn Sprecher B kann auch auf diese implikatierte \is{Implikatur} Information reagieren. Nach Bs Äußerung in (\ref{408}) liegt ein Bekenntnis von B zu p vor sowie zu $\neg$q. Da beide Diskursteilnehmer ein Bekenntnis zu p haben, wird diese Proposition zu einem cg-Inhalt. Sprecher A streitet $\neg$q ab, weshalb das Thema q vs. $\neg$q unentschieden auf dem Tisch zurückbleibt. 

\begin{exe}
	\ex\label{410}
     \begin{tabular}[t]{lp{20em}}
     		B: & Aha. Interessant. Vor allem, dass sie nicht mit ihrem Ehe-\\
     		& mann in die Stadt geht.\\
            DC$_{\textrm{A}}$: & p\\
            Tisch: & q $\vee$ $\neg$q\\
 			DC$_{\textrm{B}}$: & p, $\neg$q\\
 			cg = & $\lbrace \textrm{p} \rbrace$
      \end{tabular}
\end{exe}	
Um den Effekt von MP-Äußerungen auf den Kontext auf die gleiche Art wie den Diskursbeitrag von MP-losen Assertionen zu be\-schreiben, werde ich im Folgenden die MP-Auffassung und -Modellierung nach Arbeiten von Diewald in das Diskursmodell nach Farkas \& Bruce integrieren.

\subsection{Die rückverweisende Funktion von Modalpartikeln}
\label{sec:mpn}
Wie in Abschnitt~\ref{sec:zugang} in Kapitel~\ref{chapter:hintergrund} ausführlich beschrieben, folgt Diewald in einer Reihe von Arbeiten der generellen Auffassung, dass MPn wie andere grammatische Ka\-tegorien eine relationale Komponente aufweisen: Sie nehmen Bezug auf pragmatisch präsupponierte Einheiten im Vorgängerkontext der tatsächlichen MP-Äußerung (vgl. \citealt[414-415]{Diewald2006}). Die genau vorliegende Relation unterscheidet sich je nach MP. Als Operationalisierung dieser grundsätzlichen Funktion verwendet Diewald ein dreistufiges Beschreibungsmodell. 

Für \textit{ja} und \textit{doch} nehmen \citet{Diewald1998} die folgenden konkreten Relationen an. Als Beispiel dient die \textit{doch}-Äußerung in (\ref{411}).

\begin{exe}
	\ex\label{411} 
	Das war \textbf{doch} richtig.
\end{exe}
\vspace{-0.65cm}	
\begin{exe}
	\ex\label{412} Bedeutungsschema der MP \textit{doch}\\[-0.6em]
     \begin{tabular}[t]{|l|p{7cm}|}
     	\hline
      	pragmatischer Prätext & im Raum steht: ob das richtig war\\
        \hline
        relevante Situation & ich denke: das war richtig\\
        \hline
        $\rightarrow$ Äußerung & Das war \textbf{doch} richtig.\\
        \hline
     \end{tabular}\\
     \hbox{}\hfill\hbox{\citet[92]{Diewald1998}}
\end{exe}
Einer \textit{doch}-Äußerung geht eine Situation voran, in der die Frage offen ist, ob p gilt, d.h. hier ist möglich \glq es war richtig\grq {} oder \glq es war nicht richtig\grq {}. Vor dem Hintergrund dieser zwei im Kontext bestehenden Alternativen vertritt der Sprecher eine der beiden Möglichkeiten. Die Partikel \textit{doch} indiziert hier \\ letztlich eine konzessive Relation: Der Sprecher entscheidet sich für die in seiner Äußerung ausgedrückte Proposition, obwohl die gegenteilige Annahme ebenfalls kontextuell aktiviert ist.

 \begin{exe}
	\ex\label{413} 	
	Es soll \textbf{ja} fliegen.
	\hfill\hbox {\citet[93-94]{Diewald1998}}
\end{exe}
\vspace{-0.65cm}	
\begin{exe}
	\ex\label{414} Bedeutungsschema der MP \textit{ja}\\[-0.6em]
     \begin{tabular}[t]{|l|p{7cm}|}
     	\hline
      	pragmatischer Prätext & im Raum steht: jmd. denkt, dass es fliegen soll\\
        \hline
        relevante Situation & ich denke: es soll fliegen\\
        \hline
        $\rightarrow$ Äußerung & Es soll \textbf{ja} fliegen.\\
        \hline
     \end{tabular}\\
     \hbox{}\hfill\hbox{\citet[84]{Diewald1998}}
\end{exe}
Für eine \textit{ja}-Äußerung wie in (\ref{413}) setzen die Autoren an, dass im Kontext vor der MP-Äußerung jemand anders (in der Regel der Angesprochene) genau die Proposition vertritt, die in der (kommenden) \textit{ja}-Assertion enthalten ist. Der Sprecher schließt sich somit einer bereits bestehenden Annahme an.

\subsection{Das Einzelauftreten von \textit{ja} und \textit{doch}}
\label{sec:inkdm}
Um im Rahmen des Modells von Farkas \& Bruce auch MP-Äußerungen zu beschrei\-ben, gebe ich im Folgenden zwei Kontextzustände an: die Beschaffenheit des Kontextes vor der MP-Äußerung und nach der MP-Äußerung.\footnote{Der Übersichtlichkeit halber verzichte ich auf die Füllung des \textit{projected set}.}

\subsubsection{\textit{doch}}
\label{sec:doch1}
Für eine \textit{doch}-Äußerung nehme ich an, dass im Kontext vor der MP-Äußerung die Bipartition p $\vee$ $\neg$p auf dem Tisch liegt (entspricht bei Diewald: es steht im Raum, ob p) (vgl. (\ref{415})). Der cg ist unverändert im Vergleich zum Vorgängerkontext.

\begin{exe}
	\ex\label{415} Kontext vor der \textit{doch}-Äußerung\\[-1em]	
 \begin{tabular}[t]{|C{6em}|C{6em}|C{6em}|} 
 \hline 	
   $\textrm{DC}_{\textrm{A}}$ & {Tisch} & $\textrm{DC}_{\textrm{B}}$ \tabularnewline
  \hline
    & p $\vee$ $\neg$p & \tabularnewline
  \hline      
   \multicolumn{3}{|l|}{cg s$_{1}$} \tabularnewline 
   \hline
 \end{tabular}
\end{exe}
Im Kontext nach der Äußerung einer \textit{doch}-Assertion liegt (durch die Assertion) ein Sprecherbekenntnis zu p vor, das der Sprecher vor dem Hintergrund macht, dass zwei Möglichkeiten offen sind und somit auch das gegenteilige Bekenntnis denkbar wäre (entspricht bei Diewald: \textit{ich denke p, obwohl die beiden Alternativen offen sind}). Der cg bleibt unverändert.

\begin{exe}
	\ex\label{416} Kontext nach der \textit{doch}-Äußerung\\[-1em]	
 \begin{tabular}[t]{|C{6em}|C{6em}|C{6em}|} 
 \hline 	
   $\textrm{DC}_{\textrm{A}}$ & {Tisch} & $\textrm{DC}_{\textrm{B}}$ \tabularnewline
  \hline
    p & p $\vee$ $\neg$p & \tabularnewline
  \hline      
   \multicolumn{3}{|l|}{cg s$_{2}$ = s$_{1}$} \tabularnewline 
   \hline
 \end{tabular}
\end{exe}
Ein wichtiger Punkt ist an dieser Stelle, dass die Modellierungen in (\ref{415}) und (\ref{416}) Minimalanforderungen beschreiben. In konkreten Dialogen kann es durchaus so sein, dass ggf. auch die anderen Systeme \underline{zusätzlich} gefüllt sind. Die Charak\-terisierungen beabsichtigen nicht, jegliche denkbaren Szenarien abzudecken. Auch können mit der Füllung der Komponenten im konkreten Kontext weitere pragmatische Effekte verbunden sein (z.B. direkter Widerspruch, Erinnerung). Meine Hypothese hinsichtlich der Modellierung des Diskursbeitrags der beiden betrachte\-ten MPn ist allein, dass die Komponenten mit diesen Inhalten minimal beteiligt sind. Man hat es hier folglich (wie auch bei Diewalds Beschreibung) mit einer \textit{bedeutungsminimalistischen} (vs. -\textit{maximalistischen}) Auffassung \is{Bedeutungsminimalismus/-maximalismus} zu tun.

Eine abstrakte Bedeutungszuschreibung, wie ich sie vorschlage, bleibt natürlich den Nachweis schuldig, dass sie den Beitrag der MP-Äußerungen in konkreten Fällen erfassen kann.

Typische Beispiele für eine \textit{doch}-Assertion finden sich in (\ref{417}) bis (\ref{419}).

\begin{exe}
	\ex\label{417} 
	A: Patrick ist nicht zu Hause.\\
	B: Aber sein Auto ist \textbf{doch} da.	
	\hfill\hbox {\citet[83]{Ormelius-Sandblom1997}}
\end{exe}

\begin{exe}
	\ex\label{418} 
	A: Peter wird auch mitkommen.\\
	B: Er ist \textbf{doch} krank.
	\hfill\hbox {\citet[126]{Egg2013}}
\end{exe}

\begin{exe}
	\ex\label{419} 
	Ich habe wieder Schnupfen. Dabei lebe ich \textbf{doch} ganz vernünftig.
	\newline
	\hbox{}\hfill\hbox{\citet[84]{Dahl1988}}	
\end{exe}
Die \textit{doch}-Äußerung bezieht sich hier in gewissem Sinne auf Konsequenzen aus der ersten Äußerung. 

\begin{exe}
	\ex\label{420} 
	Patrick ist nicht zu Hause. ($\neg$p) $>$ Patricks Auto ist nicht da. ($\neg$q)\\
	Wenn Patrick nicht zu Hause ist, ist sein Auto normalerweise nicht da. 	
\end{exe}

\begin{exe}
	\ex\label{421} 
	Peter wird auch mitkommen. $>$ Peter ist nicht krank.\\
	Wenn eine Person mitkommt, ist sie normalerweise nicht krank. 
\end{exe}		
		 
\begin{exe}
	\ex\label{422} 
	Ich habe wieder Schnupfen. $>$ Ich lebe nicht vernünftig.\\
	Wenn eine Person wiederholt erkältet ist, lebt sie normalerweise nicht vernünftig.  	
\end{exe}
Die beiden beteiligten Propositionen stehen sicherlich nicht in der Relation einer logischen Folgerung. Ich schließe mich Ormelius-Sandbloms Annahme an, dass ein pragmatischer Schluss beteiligt ist, der die zwei Propositionen hier in einen plausiblen Zusammenhang bringt und auf Hintergrund- oder Weltwissen basiert. Da der abgeleitete Inhalt in diesem Fall leicht tilgbar oder verstärkbar ist und Wissen um Patricks Fortbewegungsvorlieben vonnöten sind, handelt es sich vermutlich um eine \is{konversationelle Implikatur} konversationelle Implikatur. Der Bezug der \textit{doch}-Äußerung auf die Implikatur der ersten Äußerung lässt sich mit der in Abschnitt~\ref{sec:mplass} vorgeschlagenen Modellierung gut erfassen, da - wie wir gesehen haben - auch implikatierte Inhalte auf dem Tisch liegen können.

\begin{exe}
	\ex\label{423} Kontext nach As Äußerung\\
	A: Patrick ist nicht zu Hause. (= $\neg$p) $>$ Patricks Auto ist nicht da. (= $\neg$q)\\[-1em]	
 \begin{tabular}[t]{|C{6em}|C{6em}|C{6em}|} 
 \hline 	
   $\textrm{DC}_{\textrm{A}}$ & {Tisch} & $\textrm{DC}_{\textrm{B}}$ \tabularnewline
  \hline
    $\neg$p & p $\vee$ $\neg$p & \tabularnewline
    {} & q $\vee$ $\neg$q & \tabularnewline
  \hline      
   \multicolumn{3}{|l|}{cg s$_{1}$} \tabularnewline   
   \hline
 \end{tabular}
\end{exe}
In (\ref{423}) hat Sprecher A ein Bekenntnis zu $\neg$p, so dass auf dem Tisch die Frage eröffnet wird, ob p $\vee$ $\neg$p. Da aus $\neg$p $\neg$q ableitbar ist, öffnet sich auf dem Tisch auch die Frage q $\vee$ $\neg$q. Der cg ist nicht beeinflusst. B reagiert nun mit seinem öffentlichen Bekenntnis zu q auf die offene Frage q $\vee$ $\neg$q auf dem Tisch  (vgl. (\ref{424})).

\begin{exe}
	\ex\label{424} Kontext nach Bs Äußerung\\
	B: Aber sein Auto ist \textbf{doch} da. (= q)\\[-1em]	
 \begin{tabular}[t]{|C{6em}|C{6em}|C{6em}|} 
 \hline 	
   $\textrm{DC}_{\textrm{A}}$ & {Tisch} & $\textrm{DC}_{\textrm{B}}$ \tabularnewline
  \hline
    $\neg$p & p $\vee$ $\neg$p & \tabularnewline
    {} & q $\vee$ $\neg$q & q\tabularnewline
  \hline      
   \multicolumn{3}{|l|}{cg s$_{2}$ = s$_{1}$} \tabularnewline   
   \hline
 \end{tabular}
\end{exe}
Eine \textit{doch}-Äußerung kann sich auch auf Glückensbedingungen \is{Glückensbedingung} eines vorangehenden Sprechaktes \is{Sprechakt} beziehen.
\begin{exe}
	\ex\label{425} 
	A: Übersetze mir bitte diesen Brief.\\
	B: Ich kann \textbf{doch} kein Englisch. ($\neg$q)
\end{exe}

\begin{exe}
	\ex\label{426} 
	A: Bestell dir die Schweinshaxe.\\
	B: Ich bin \textbf{doch} Vegetarier.(q)
	\hfill\hbox {\citet[133]{Egg2013}}
\end{exe}
Die \textit{doch}-Äußerung thematisiert hier die 1. Einleitungsregel für Aufforderungen bzw. Raten.

\begin{exe}
	\ex\label{427} 
	1. Einleitungsregel Aufforderung \is{Aufforderung}\\
	\glqq H ist in der Lage, A zu tun. S glaubt, daß H in der Lage ist, A zu tun.\grqq{}
\end{exe}

\begin{exe}
	\ex\label{428} 
	1. Einleitungsregel Raten \is{Rat}\\
	\glqq S hat Grund zu glauben, daß A H nützen wird.\grqq{}
	\hfill\hbox {\citet[100/104]{Searle1971}}
\end{exe}		  
Nimmt man an, dass nach As Aufforderung bzw. Rat q $\vee$ $\neg$q auf dem Tisch liegt, bezieht sich B mit seiner \textit{doch}-Äußerung auf eine dieser beiden Propositionen. Dass es sich hierbei um die Disjunktionen \textit{Du kannst Englisch.} vs. \textit{Du kannst kein Englisch.} und \textit{Du bist Vegetarier.} vs. \textit{Du bist kein Vegetarier.} handelt, ergibt sich über weitere Schlussprozesse: Wenn der Brief auf Englisch ist und B den Brief übersetzen soll, beherrscht B in As Augen Englisch. Wenn die Empfehlung an B lautet, Schweinshaxe zu bestellen, nimmt S an, dass B kein Vegetarier ist.

Neben Implikaturen und Sprechaktbedingungen kann sich eine \textit{doch}-Assertion auch auf eine Implikation \is{Implikation} beziehen, wie das Beispiel in (\ref{429}) nahelegt.

\begin{exe}
	\ex\label{429} 
	A: Ich bin oft krank.(= p)\\
	B: Du bist \textbf{doch} immer gesund.(= q)
	\hfill\hbox {\citet[132]{Egg2013}}
\end{exe}
$\neg$q ist eine logische Folgerung aus p. Durch die Assertion von p gelangt neben p $\vee$ $\neg$p auch q $\vee$ $\neg$q auf den Tisch. Die \textit{doch}-Assertion von B nimmt in der Reaktion auf As Beitrag Bezug auf q.

\textit{Doch}-Äußerungen können nicht nur reaktiv auftreten, sondern es gibt auch Verwendungen wie in (\ref{430}) und (\ref{431}).

\begin{exe}
	\ex\label{430} 
	Dein Bruder ist \textbf{doch} Arzt. Ich habe nämlich da dieses Ziehen im Arm... 
	\newline
	\hbox{}\hfill\hbox {\citet[133-134]{Hentschel1986}}
\end{exe}	
\vspace{-0.65cm}
\begin{exe}
	\ex\label{431} 
	Sie sind \textbf{doch} Paul Meier.
	\hfill\hbox {\citet[126]{Egg2013}}
\end{exe}
Die \textit{doch}-Äußerungen in (\ref{430}) und (\ref{431}) treten diskursinitial auf. Zumindest explizit (s.u.) fehlt ihnen das reaktive Moment. Es handelt sich hierbei um ein Auftreten von \textit{doch}, das in jeder mir bekannten \textit{doch}-Beschreibung zu Problemen führt und oftmals auf eher fragwürdige Art in ansonsten plausible Analysen integriert wird.\footnote{Ein Beispiel für einen solchen Fall ist die Bedeutungsmodellierung von \textit{doch} bei \citet[68]{Koenig1997}. Prinzipiell nimmt er an, dass die MP \textit{doch} auf einen Widerspruch verweist. Zu Verwendungen der Art in (\ref{430}) und (\ref{431}) schreibt er nach der Diskussion klassischer Beispiele:  \glqq Da es sich $[$...$]$ hier nicht um einen reaktiven Gebrauch von \textit{doch} handelt, kann es noch keine Widersprüche geben, die durch das Verhaltens $[$sic!$]$ des Hörers ausgelöst wurden. Was in diesen Fällen geschieht, ist eine Ausformulierung des Kontextes, die für spätere Züge des Hö\-rers relevant sind. $[$...$]$ Im Unterschied zu den Standardfällen der Verwendung von \textit{doch} erfolgt $[$...$]$ kein Rückverweis auf Inkonsistenzen zwischen neuer Information und bestehenden Annahmen, sondern ein Vorverweis auf mögliche Inkonsistenzen zwischen Zustimmung zu einer Äußerung und späteren Interaktionsschritten.\grqq{}}

Wenngleich eine Vorgängeräußerung, auf die sich die \textit{doch}-Äußerung bezieht, hier sicherlich nicht explizit vorliegt, gilt es als ein bekanntes Phänomen, dass MPn auch verwendet werden können, um das Vorhandensein anderer Information im Kontext vorzugeben, d.h. zu suggerieren. Genau dies trifft auf die Fälle in (\ref{430}) und (\ref{431}) zu. \citet[Fn 14]{Repp2013} beschreibt den Eindruck, dass derartige Äußerungen ohne \textit{doch} sehr unhöflich wirken würden. \citet[138]{Egg2013} schreibt, der Sprecher teile dem Diskurspartner diesem bereits Bekanntes mit. Diese Eindrücke lassen sich auffangen, wenn man annimmt, dass der Sprecher die Offenheit des Themas (der Bruder ist Arzt vs. der Bruder ist nicht Arzt bzw. es ist Paul Meier vs. es ist nicht Paul Meier) suggeriert und sich mit der \textit{doch}-Äußerung zu p bekennt. Mein Eindruck ist, dass derartige Assertionen defensiver wirken. Und diese Wirkungsweise folgt aus der suggerierten reaktiven Verwendung. Der Sprecher sagt etwas, das der Gesprächspartner weiß, präsentiert es aber als Reaktion auf ein bereits eröffnetes Thema, zu dem seine Reaktion gewünscht ist. Er hat dieses Thema im Diskurs folglich nicht selbst eröffnet. Mit einer MP-losen Assertion würde er sich bei diskursinitialer Verwendung der Assertion zu p bekennen, damit selbst ein neues Thema eröffnen und den Hörer dazu auffordern, eine Antwort zu geben bzw. zu drängen, die Proposition zu bestätigen. Es ist recht plausibel, dass es unhöflich wirkt, wenn ein Sprecher Information, von der er weiß, dass sie bekannt ist, von sich aus präsentiert, als stünde sie zur Diskussion. Der Eindruck der Unhöflichkeit und der Bekanntheitsstatus des Asser\-tionsgehaltes hängen meiner Beschreibung der Situation nach folglich zusammen. Darüber hinaus ist jede Gesprächseröffnung potenziell gesichtsbedrohend, da der Angesprochene das Vorhaben des Sprechers auch abblocken kann. Ein suggeriertes offenes Thema weicht dieser Gefahr aus, da der Gesprächsrahmen zwi\-schen Sprecher und Hörer als bereits bestehend ausgegeben wird und nicht erst etabliert werden muss.

Eine Unklarheit/Kontroverse, auf die ich an dieser Stelle hinweisen möchte, die ich allerdings nicht beseitigen kann, ist, ob davon auszugehen ist, dass die Disjunktion, für die ich annehme, dass sie im Kontext vor der \textit{doch}-Äußerung auf dem Tisch liegt, zum offenen Thema wird, weil der Sprecher sich zu p bzw. $\neg$p bekennt (vgl. auch \ref{Fn16}). Ich halte diese Frage nicht für leicht entscheidbar: Für die diskursinitialen \textit{doch}-Assertionen scheint mir dies äußerst unplausibel. Sicherlich möchte man hier nicht annehmen, dass der Sprecher beim Gesprächs\-partner ein Bekenntnis suggeriert, dass sein Bruder nicht Arzt ist oder er nicht Paul Meier ist. Für die anderen angeführten Beispiele wäre dies eher denkbar. In (\ref{425}) und (\ref{429}) spricht erstmal nichts dagegen, dass Sprecher A im Zuge seiner Äußerung auch ein Diskursbekenntnis zu q (\textit{Du kannst Englisch.}) und $\neg$q (\textit{Der Sprecher ist nicht immer gesund.}) macht. Fraglicher scheint dies schon wieder in (\ref{419}) und (\ref{417}). In (\ref{419}) müsste der Sprecher sich dann zuerst dazu bekennen, dass er nicht vernünftig lebt, um im direkten Anschluss das Gegenteil mitzuteilen. In (\ref{417}) ist es auch durchaus denkbar, dass A gar nicht um den Zusammenhang weiß, aufgrund dessen B seine Äußerung macht. 

Die generellere Frage, die sich hier auftut, ist, ob der Sprecher einer Äußerung sich auch zu mit dieser eingeführten \is{Implikation} Implikationen, Implikaturen \is{Implikatur} und Sprechaktbedingungen \is{Sprech\-aktbedingung} bekennt bzw. ob es sich hierbei um Bekenntnisse des Gesprächspartners oder sogar cg-Inhalt handelt. Diese steuern in den hier angeführten Beispielen die Proposition bei, auf deren entgegengesetzte Polarität sich die \textit{doch}-Äuße\-rung bezieht. Wenn es sich bei Diskursbekenntnissen \is{Diskursbekenntnis} um Inhalte handelt, derer der Sprecher sich bewusst ist, halte ich es (zumindest bei Implikaturen) nicht für plausibel, anzunehmen, dass ein Sprecher stets um diese vermittelten Inhalte weiß. Jingyang Xue (p.c.) hat mich darauf hingewiesen, dass es ebenfalls möglich ist, dass der Gesprächspartner die Implikatur erst durch die \textit{doch}-Äußerung etabliert. Für (\ref{417}) wäre dies denkbar. Dann könnte nicht davon ausgegangen werden, dass A mit seiner Äußerung auch ein Diskursbekenntnis zur implikatierten Proposition macht, aber B von diesem Zusammenhang ausgeht und sich aus seiner Sicht das Thema eröffnet.

Folglich ist im Einzelfall zu entscheiden, in welchem der Systeme die Proposition, die an der Disjunktion auf dem Tisch beteiligt ist, zu verankern ist. Die Beobachtung, dass je nach Szenario aber DC$_{\textrm{A}}$, DC$_\textrm{B}$ und/oder der cg in Frage kommen, spricht deutlich dafür, diese Verankerung der der \textit{doch}-Proposition entgegengesetzten Proposition nicht zur Bedeutungsbeschreibung von \textit{doch} zu machen. Viele Arbeiten (z.B. \citealt[71]{Doherty1985}, \citealt[83]{Ormelius-Sandblom1997}) gehen davon aus, dass aus der Vorgängeräußerung die Negation der \textit{doch}-Asser\-tion abzuleiten ist und demnach ein adversatives Moment beteiligt ist. Die obigen Überlegungen zeigen, dass davon nicht generell auszugehen ist und diese Formulierung deshalb zu stark ist. Entscheidend und klar scheint mir aber zu sein, dass die Disjunktion auf dem Tisch zu liegen kommt, d.h. die Frage steht zur Diskussion und kann durch die \textit{doch}-Äußerung thematisiert werden. Das Vorhandensein dieser beiden offenen Alternativen im Kontext – unabhängig von ihrem Zustandekommen – ist die Anforderung, die meiner Modellierung des Diskursbeitrags der \textit{doch}-Assertion nach erfüllt sein muss. In Abschnitt~\ref{sec:direktive} in Kapitel~\ref{chapter:dua} wird sich zeigen, dass sich auch weitere Auftretensweisen von \textit{doch} mit dieser Modellierung erfassen lassen.

\subsubsection{\textit{ja}}
\label{sec:ja}
Um auch den Diskursbeitrag einer \textit{ja}-Äußerung im Rahmen des Modells von Farkas \& Bruce unter Rückbezug auf Diewalds MP-Charakterisierung aufzufangen, übersetze ich Diewalds \textit{jemand denkt, dass p} im Prätext derart, dass der Diskuspartner im Kontext vor der \textit{ja}-Äußerung ein öffentliches Bekenntnis zu p hat.
\begin{exe}
	\ex\label{432} Kontext vor der \textit{ja}-Äußerung \\[-1em]	
 		\begin{tabular}[t]{|C{6em}|C{6em}|C{6em}|} 
 		\hline 	
   		$\textrm{DC}_{\textrm{A}}$ & {Tisch} & $\textrm{DC}_{\textrm{B}}$ \tabularnewline
  		\hline
   		{} & {} & p \tabularnewline
  		\hline      
   		\multicolumn{3}{|l|}{cg s$_{1}$} \tabularnewline   
  		 \hline
 		\end{tabular}
\end{exe}
Wie bei \textit{doch} verstehe ich dies als die minimale Anforderung an den Kontext und schließe damit aber nicht aus, dass in konkreten Dialogen weitere Komponenten beteiligt sind. Mehr Anforderungen müssen allerdings nicht erfüllt sein. Im Kontext nach der \textit{ja}-Äußerung entspricht Diewalds \textit{ich denke p} in meiner Mo\-dellierung dem öffentlichen Diskursbekenntnis des Sprechers zu p. Durch dieses Bekenntnis öffnet sich auf dem Tisch das Thema p $\vee$ $\neg$p (Teil 1) (vgl. (\ref{433a})) und wird aber sofort vom Tisch entfernt, da B bereits das Bekenntnis zu p hat, das benötigt wird, um diese Proposition zum Teil des cg zu machen. Die Frage auf dem Tisch wird somit sofort zugunsten von p entschieden (Teil 2) (vgl. (\ref{433b})).

\begin{exe}
	\ex\label{433} Kontext nach der \textit{ja}-Äußerung\\[-1em]
	\begin{xlist}
		\ex\label{433a} Teil 1\\[-1em]
 			\begin{tabular}[t]{|C{6em}|C{6em}|C{6em}|} 
 			\hline 	
   			$\textrm{DC}_{\textrm{A}}$ & {Tisch} & $\textrm{DC}_{\textrm{B}}$ \tabularnewline
  			\hline
   			p & p $\vee$ $\neg$p & p \tabularnewline
  			\hline      
   			\multicolumn{3}{|l|}{cg s$_{2}$ = s$_{1}$} \tabularnewline   
  			 \hline
 			\end{tabular}
 		\ex\label{433b}	Teil 2\\[-1em]
 			\begin{tabular}[t]{|C{6em}|C{6em}|C{6em}|} 
 			\hline 	
   			$\textrm{DC}_{\textrm{A}}$ & {Tisch} & $\textrm{DC}_{\textrm{B}}$ \tabularnewline
  			\hline
   			{} & {} & {} \tabularnewline
  			\hline      
   			\multicolumn{3}{|l|}{cg $\textrm{s}_{2} = \lbrace \textrm{s}_{1} \cup \lbrace \textrm{p} \rbrace \rbrace$} \tabularnewline   
  			 \hline
 			\end{tabular} 			
 	\end{xlist}	
\end{exe}
Bei der Auftretensweise von \textit{ja} lassen sich drei prinzipielle Fälle unterscheiden.

In der ersten Situation hat der Diskurspartner tatsächlich ein Bekenntnis zu p und p wird nach der Äußerung in den cg aufgenommen. Es handelt sich hierbei um Sequenzen, in denen es Gründe dafür gibt, dem Gesprächspartner ein Diskursbekenntnis zuschreiben zu können. Dies kann beispielsweise daran liegen, dass Sprecher und Hörer die Wahrnehmungsebene teilen, wie im folgenden Beispiel.

\begin{exe}
	\ex\label{433} 
	Oh dann mußt du es \textbf{ja} nochmal abmachen.
	\hfill\hbox {\citet[93]{Diewald1998}}
\end{exe}																		        
Die Äußerung erfolgt in einem Kontext, in dem ein Sprecher nach Anweisung eines anderen ein Flugzeugmodell zusammenbaut. Beide Diskursteilnehmer sehen dabei die gleichen Dinge. Der Sprecher der Assertion in (\ref{433}) weiß folglich, dass der Angesprochene (ebenfalls) sehen kann, dass das angesprochene (oder ein späteres) Teil nicht passt und das bezeichnete Teil deshalb abgemacht werden muss. Nur weil die Gesprächsteilnehmer die gleiche Wahrnehmungssituation haben, kann der Sprecher dem Adressaten das öffentliche Bekenntnis zur in (\ref{433}) ausgedrückten Proposition zuschreiben. Hierbei handelt es sich nach meiner Modellierung in (\ref{432}) und (\ref{433}) um die Voraussetzung für die Verwendung von \textit{ja}.

Unter diese Verwendungsweise von \textit{ja} fasse ich auch Beispiele wie in (\ref{434}).

\begin{exe}
	\ex\label{434} 
	\begin{xlist}
		\ex\label{434a} Du bist \textbf{ja} ganz nass.	
 		\ex\label{434b}	Du hast \textbf{ja} einen Fleck auf dem Hemd.
 		\ex\label{434c}	Du blutest \textbf{ja}.				
 	\end{xlist}	
\end{exe}	
Derartige Äußerungen beziehen sich auf Sachverhalte, die der Sprecher in der Situation feststellt und bei denen er davon ausgehen kann, dass diese für die Wahrnehmung des Hörers ebenfalls aktuell zugänglich sind. An diesen Beispielen sieht man auch, dass das vorweggenommene Hörerbekenntnis nicht nur durch Äußerungen zu kommunizieren ist. Wenn der Angesprochene z.B. in (\ref{434a}) nass in einen Raum kommt, kann ihm auch das Bekenntnis zugeschrieben werden, nass zu sein; genauso wie ihm in (\ref{434b}) transparent das Bekenntnis einen Fleck auf dem Hemd zu haben oder in (\ref{434c}) zu bluten zuzuschreiben ist. Möchte der Sprecher diese Inhalte im aktuellen Gespräch offiziell zu geteiltem Wissen machen, würde es auch recht merkwürdig anmuten, wenn er diese Sachverhalte als für den Gesprächspartner neue Information in den Diskurs einführen würde und als offenes Thema zur Diskussion stellen würde, ohne dem Angesprochenen eine Haltung zu diesem Sachverhalt zuzuschreiben.
		
In einer weiteren Verwendung von \textit{ja} \underline{unterstellt} der Sprecher dem Adressaten ein Bekenntnis zu p und p wird ebenfalls zu einem cg-Inhalt. Es handelt sich hierbei um Situationen, in denen der Sprecher sich darauf beruft, dass der Angesprochene die gleiche Annahme teilt, d.h. ein öffentliches Bekenntnis hat, obwohl der Kontext nahelegt, dass er dieses Diskursbekenntnis gerade nicht vorweist, sondern es ihm vom Sprecher unterstellt wird. (\ref{435}) zeigt ein Beispiel, in dem diese Unterstellung einen harmlosen Effekt mit sich führt. Es handelt sich um den Ausschnitt aus einem Interview. Die Gesprächspartner kennen sich \\ folglich nicht. Da die Proposition \textit{dass ich als Halbwaise aufgewachsen bin} im Kontext nicht vorerwähnt ist, ist davon auszugehen, dass der Reporter p nicht wissen kann.

\begin{exe}
	\ex\label{435} 
	\scriptsize
	[...] Meine Frau hat sowas mit den Kindern geklärt. Sie war für die Erziehung zuständig, ich habe mich nur eingemischt, wenn es wichtig war. \textbf{Ich selbst bin \underline{ja} als Halbwaise aufgewachsen.} Mein Vater starb, als ich zwölf war.   
	\hfill\hbox {\citet[204]{Rinas2006}}
\end{exe}
Obwohl klar ist, dass der Interviewer nicht schon wissen kann, dass der Sprecher Halbwaise ist, beabsichtigt der Sprecher hier nicht, den Inhalt zur Diskussion zu stellen. Er verfolgt das Ziel, ihn auf direktem Weg in den cg einzufügen, was er dadurch erreicht, dass er das Bekenntnis des Hörers zu p voraussetzt. Ich bezeichne diesen Gebrauch von \textit{ja} oben als \glqq harmlos\grqq{}, da es auch perfide Verwendungen (vgl. \citealt{Reiter1980}) gibt wie in (\ref{436}) und (\ref{437}).

\begin{exe}
	\ex\label{436} 
	\begin{xlist}
		\ex\label{436a} Du bist \textbf{ja} ein Versager.	
		\hfill\hbox {\citet[134]{Dahl1988}}	
 		\ex\label{436b}	Bist \textbf{ja} doof.		
 		\hfill\hbox {\citet[345]{Reiter1980}}		
 	\end{xlist}	
\end{exe}	

\begin{exe}
	\ex\label{437} 
(gesungen + Tanz) Ätschibätsch, wir gehn \textbf{ja} heut ins Kino.	
	\newline
 	\hbox{}\hfill\hbox {nach \citet[346]{Reiter1980}}		
\end{exe}	
Der Effekt des unterstellten Bekenntnisses wird hier in dem Sinne \glq ausgebeutet\grq {}, dass ein Hörerbekenntnis angenommen wird, obwohl der Adressat sich in (\ref{436}) mit Sicherheit nicht dazu bekannt hat oder freiwillig bekennen wird, dass er ein Versager oder doof ist. Ähnlich erlangt die Äußerung in (\ref{437}) ihre abwertende kommunikative Funktion erst dadurch, dass der Sprecher auf Seiten des Hörers ein Wissen ansetzt, das dieser sicherlich nicht hat, weil er nicht wissen kann, was der Sprecher vorhat. Der Sprecher suggeriert dem Adressaten absichtlich, dass er etwas weiß, was er de facto nicht weiß, um ihn als ausgeschlossen hinzustellen.

In der dritten Gebrauchweise von \textit{ja}, die ich unterscheiden möchte und die oft als Standardfall behandelt wird, ist die ausgedrückte Proposition bereits Teil des cg und der Sprecher verweist auf diese, um sie zu bestimmten rhetorischen Zwecken in der aktuellen Gesprächssituation zu aktivieren. In diesem Fall befindet sich p im Kontext vor der \textit{ja}-Äußerung unter den Diskursbekenntnissen beider Gesprächsteilnehmer. Der Sprecher macht p somit zum Thema im Gespräch, indem er auf eine Proposition verweist, für die im Diskurs bereits Einigkeit unter den Beteiligten hergestellt worden ist. Die Proposition wird auf den Tisch gelegt, die offene Frage wird aber im gleichen Zug reduziert, da p in diesem Kontext nicht mehr zur Diskussion gestellt werden muss.

Es scheint mir schwierig, Fälle anzuführen, die eindeutig und ausschließlich diese Verwendung widerspiegeln. Ein plausibler Gesprächszug sind Begründungen der Art in (\ref{438}).

\begin{exe}
	\ex\label{438} 
	Ich gehe nicht schwimmen. Das Wasser ist \textbf{ja} noch zu kalt.  
 		\hfill\hbox {\citet[132]{Dahl1988}}		
\end{exe}																						
Sicherlich muss dieses Beispiel nicht so interpretiert werden (Kontext 1 und 2 sind auch denkbar), doch handelt es sich um eine plausible Interpretation, dass Sprecher und Adressat sich hinsichtlich des Sachverhalts, dass das Wasser kalt ist, einig sind und dies zu einem späteren Zeitpunkt des Gesprächs als Begründung für das Nicht-Schwimmen wieder aktivieren. Die Bekanntheit des Inhalts einer Äußerung, die als Begründung beabsichtigt ist, scheint mir für den intendierten Argumentationsverlauf durchaus förderlich. 

Unter diese Gebrauchsweise von \textit{ja} können auch Äußerungen gefasst werden, die der Fremderinnerung dienen. 

\begin{exe}
	\ex\label{439} 
	\begin{xlist}
		\ex\label{439a} Du warst \textbf{ja} damals nicht dabei.	
 		\ex\label{439b}	Ich hab's dir \textbf{ja} gesagt.		
 		\hfill\hbox {\citet[344/345]{Reiter1980}}		
 	\end{xlist}	
\end{exe}
Hier kann davon ausgegangen werden, dass der Sprecher auf Sachverhalte verweist, die dem Gegenüber tatsächlich bekannt sein sollten, weil sie Teil der gemeinsamen Wissensbasis sind. 

Plausibel wäre die Annahme, dass p tatsächlich eine cg-Information ist, auch in (\ref{440}).

\begin{exe}
	\ex\label{440} 
	A: Soll ich dir beim Tragen helfen?\\
	B: Das ist \textbf{ja} viel zu schwer für dich.
	\hfill\hbox {\citet[141]{Dahl1988}}
\end{exe}
Hier wird von B eine Voraussetzung zurückgewiesen, die der Partner macht, indem er seinen Sprechakt äußert. Es ist durchaus denkbar, dass die Gesprächspartner sich tatsächlich schon geeinigt haben, welche Last A tragen kann und B diese Assertion an dieser Stelle im Gespräch anführt, um eine Voraussetzung des Angebots von A und damit das Angebot selbst zurückzuweisen. Zu den Voraussetzungen der Annahme eines Angebots kann plausiblerweise gezählt werden, dass der Inhalt des Angebots für den Sprecher machbar ist und keine Nachteile mit sich bringt. Ein kooperativer bzw. anständiger Sprecher wird den Diskurspartner auf die Nicht-Erfülltheit dieses Aspektes hinweisen. Auch in diesem Fall nutzt der Sprecher den cg-Inhalt für die Zwecke seiner Argumentation: Unter Ausdruck der Einigung in Bezug auf den Sachverhalt, dass die Sache für den Hörer zu schwer ist, kann das Angebot sofort als zurückgewiesen gelten.\footnote{Diese Modellierung baut auf der gleichen Bedeutungszuschreibung auf wie sie z.B: auch von \citet[101]{Doherty1987}, \citet[104]{Thurmair1989} oder \citet[425]{Rinas2007} vertreten wird. Andere Autoren setzen eine leicht andere \textit{ja}-Bedeutung an, die etwas schwächer nicht von der Bekannt\-heit der Proposition auf Seiten des Hörers ausgeht. \citet[146]{Jacobs1991} und \citet[178]{Lindner1991} formulieren den Beitrag der Partikel z.B. derart, dass der Hörer nicht Gegenteiliges glaubt und in der aktuellen Situation auch nicht die Falschheit von p überhaupt in Betracht zieht. Ich glaube nicht, dass meine Modellierung zu stark ist und diesen Charakterisierungen nicht nachkommen kann: Das Diskursmodell erlaubt vielmehr gerade die Abbildung dieser Situation, in der die Zustimmung auf Seiten des Hörers durch den Sprecher präsupponiert wird, und p somit akkommodiert  wird, weil der Sprecher nicht mit der Inanspruchnahme des Hö\-rers rechnet. Problematischer scheint es mir, bedeutungsminimalistisch nur den Aspekt der Kontroverse zu modellieren, der dann in vielen Fällen zum stärkeren Bedeutungsmoment der Bekanntheit erweitert werden muss.}

Mit dem Diskursmodell nach \citet{Farkas2010}, das es vermag, den Einfluss von Assertionen auf den Diskurskontext zu erfassen, sowie einer MP-Beschrei\-bung, die ich in dieses Modell integriert habe, liegen die zwei Komponenten vor, auf deren Basis ich im Folgenden meinen Vorschlag zur Beantwortung der zwei Fragen aus Abschnitt~\ref{sec:distributiondj} erläutern werde.

Meine Absicht ist es, ein Erklärungsmodell für die Beschränktheit der MP-Abfolge vorzuschlagen. Da sich diese Ableitung auf die Interpretation der Kombination gründen soll, setzt dies eine Klärung der Frage voraus, wie die MP-Kombination(en) aus \textit{ja} und \textit{doch} interpretiert werden. 	

\subsection{Das kombinierte Auftreten von \textit{ja} und \textit{doch}}
\label{sec:kombi}
Wie bereits in Abschnitt~\ref{sec:transparenz} in Kapitel~\ref{chapter:hintergrund} ausgeführt, ist ein in der Literatur kontrovers diskutierter Aspekt in diesem Zusammenhang, welche Skopusrelationen \is{Skopus} in MP-Kombi\-nationen vorliegen. Unzweifelhaft ist, dass beide MPn Skopus nehmen, was sie bei isoliertem Auftreten schließlich auch tun (vgl. (\ref{441})).

\begin{exe}
	\ex\label{441} 
	\begin{xlist}
		\ex\label{441a} Fritz wurde von einem Golfball getroffen. (p)	
 		\ex\label{441b}	Fritz wurde \textbf{ja} von einem Golfball getroffen. (ja(p))		
		\ex\label{441c}	Fritz wurde \textbf{doch} von einem Golfball getroffen. (doch(p))	
 	\end{xlist}	
\end{exe}
Die offene Frage ist allerdings, ob bei ihrem kombinierten Auftreten (vgl. (\ref{442})) zwischen ihnen eine hierarchische Relation besteht (wie in (\ref{443})) oder ob die beiden MPn den gleichen Skopus nehmen, wie in (\ref{444})\footnote{In Abschnitt~\ref{sec:transparenz} habe ich im Falle der additiven Interpretation auch die Möglichkeit erwogen, dass sich die Partikeln gleichzeitig auf p beziehen. Da ich beabsichtige, aus der Applikation der Partikeln die Abfolge abzuleiten, führe ich diese Möglichkeit nicht weiter an. Die grund\-sätzliche Interpretation, von der ich ausgehe, ist die Interpretation in (\ref{444}): Die Bedeutungen der beiden MPn addieren sich. Erneut möchte ich darauf hinweisen, dass ich nicht beabsichtige, anzunehmen, dass p selbst mehrfach in die Bedeutungskonstitution eingeht \textit{(ja \& doch)(p)} suggeriert m.E. die Annahme eines MP-Clusters, von dem ich nicht ausgehe.}.

\begin{exe}
	\ex\label{442} 
	Fritz wurde \textbf{ja doch} von einem Golfball getroffen.
\end{exe}
\vspace{-0.65cm}	
\begin{exe}
	\ex\label{443} Skopusrelation zwischen \textit{ja} und \textit{doch}\\[-1em]
	\begin{xlist}
		\ex\label{443a} ja(doch(p))	
 		\ex\label{443b}	doch(ja(p))		
 	\end{xlist}	
\end{exe}

\begin{exe}
	\ex\label{444} Keine Skopusrelation zwischen \textit{ja} und \textit{doch}\\[-1em]
	\begin{xlist}
		\ex\label{443a} 1. ja(p), 2. doch(p)	
 		\ex\label{443b}	1. doch(p), 2. ja(p)		
 	\end{xlist}	
\end{exe}
Bei den Alternativen in (\ref{443}) und (\ref{444}) handelt es sich um vier prinzipiell denkbare Möglichkeiten der Relation zwischen zwei Partikeln, vorausgesetzt sie nehmen beide Skopus in einer Äußerung, in der sie kombiniert auftreten. Jede dieser Möglichkeiten entspricht dabei einer eigenen Interpretation. Umso verwunderlicher ist es, dass verschiedene dieser Bedeutungen für die konkrete Kombination \textit{ja doch} vorgeschlagen wurden (vgl. die Zuordnungen in (\ref{445})).

\begin{exe}
	\ex\label{445} Keine Skopusrelation zwischen \textit{ja} und \textit{doch}\\[-1em]
	\begin{xlist}
		\ex\label{445a} ja(doch(p))
		\hfill\hbox {\citet{Ormelius-Sandblom1997}, \citet{Rinas2007}}	
 		\ex\label{445b}	doch(ja(p))	
 		\hfill\hbox {\citet{Lindner1991}}	
 		\ex\label{445c} 1. ja(p), 2. doch(p)
 		\hfill\hbox {\citet{Thurmair1989}}		
 		\ex\label{445d}	1. doch(p), 2. ja(p) 	
 		\hfill\hbox {\citet{Doherty1985}}		
 	\end{xlist}	
\end{exe}									
Selbst wenn die in (\ref{445}) zugeordneten Arbeiten sich natürlich in ihren Bedeutungszuschreibungen an die Einzelpartikeln unterscheiden, halte ich es für äußerst unplausibel, dass alle vier Interpretationen zutreffen können. Eine Analyse kann erst dann eine Entscheidung zwischen den Möglichkeiten in (\ref{445}) treffen, wenn sie in der Lage ist, alle diese Fälle abzubilden und wenn sie sie nach ihrer Gene\-rierung an konkreten Beispielen für das kombinierte Auftreten von \textit{ja} und \textit{doch} gegeneinander abwägt, um zu entscheiden, welcher Skopusverlauf den Diskursbeitrag am passendsten erfasst. Für bestehende Ansätze, die sich hinsichtlich dieser Frage äußern, gilt, dass sie teilweise aufgrund ihres Verfahrens der Model\-lierung der MP-Bedeutung nicht alle vier Möglichkeiten abbilden können, d.h. I\-diosynkrasien der Erklärungsmodelle schließen Lesarten aus, gegen die es prinzi\-piell keine Einwände gäbe. Bzw. es werden manche Lesarten von den Autoren nicht in Betracht gezogen, obwohl diese Bedeutungen sich mit dem verwendeten Instrumentarium durchaus formulieren ließen. 

Ich möchte im Folgenden versuchen, mit meiner eigenen Analyse diesen Forde\-rungen nachzukommen (vgl. auch schon \citealt[192-197]{Mueller2014a}; \citeyear[221-223]{Mueller2017b}). Ich halte es für möglich, mit meinem Modell alle vier Varianten aus (\ref{445}) abzubilden.
	
(\ref{446}) bis (\ref{449}) bilden die vier denkbaren Interaktionen zwischen den Skopoi von \textit{ja} und \textit{doch} im Kontext vor der \textit{ja doch}-Äußerung ab.

Dass \textit{ja} über \textit{doch} Skopus \is{Skopus} nimmt (vgl. (\ref{446})), würde dann bedeuten, dass der Kontextzustand vor der MP-Äußerung derart beschaffen ist, dass Gesprächsteilnehmer B ein Bekenntnis dazu hat, dass p $\vee$ $\neg$p auf dem Tisch liegt. Zunächst appliziert \textit{doch} auf p (p $\vee$ $\neg$p liegt auf dem Tisch), anschließend dient diese komplexe Struktur als Input für \textit{ja} (B hat ein Bekenntnis zu p bzw. hier dazu, dass p $\vee$ $\neg$p auf dem Tisch liegt).
 
\begin{exe}
	\ex\label{446} Kontext vor der MP-Äußerung: ja(doch(p))\\[-1em]	
 		\begin{tabular}[t]{|C{6em}|C{6em}|C{6em}|} 
 		\hline 	
   		$\textrm{DC}_{\textrm{A}}$ & {Tisch} & $\textrm{DC}_{\textrm{B}}$ \tabularnewline
  		\hline
   		{} & {} & (p $\vee$ $\neg$p) $\in$ T \tabularnewline
  		\hline      
   		\multicolumn{3}{|l|}{cg s$_{1}$} \tabularnewline   
  		 \hline
 		\end{tabular}
\end{exe}
Das umgekehrte Skopusverhältnis führt zu der Lesart, dass auf dem Tisch liegt, dass B ein Bekenntnis zu p hat oder dass B kein Bekenntnis zu p hat (vgl. (\ref{447})). Das zuerst zur Anwendung kommende \textit{ja} schreibt B ein Bekenntnis zu p zu, wes\-halb dieser Ausdruck im Skopus von \textit{doch} dazu führt, dass er zum Teil der Disjunktion auf dem Tisch wird.

\begin{exe}
	\ex\label{447} Kontext vor der MP-Äußerung: doch(ja(p))\\[-1em]	
 		\begin{tabular}[t]{|C{6em}|C{6em}|C{6em}|} 
 		\hline 	
   		$\textrm{DC}_{\textrm{A}}$ & {Tisch} & $\textrm{DC}_{\textrm{B}}$ \tabularnewline
  		\hline
   		{} & {} & (p $\in$ $\textrm{DC}_{\textrm{B}}$) $\vee$ $\neg$(p $\in$ $\textrm{DC}_{\textrm{B}}$) \tabularnewline
  		\hline      
   		\multicolumn{3}{|l|}{cg s$_{1}$} \tabularnewline   
  		 \hline
 		\end{tabular}
\end{exe}
Haben \textit{ja} und \textit{doch} den gleichen Wirkungsbereich und applizieren in dieser Reihenfolge, führt dies zu einem Kontextzustand, in dem Diskursteilnehmer B ein Bekenntnis zu p hat (Beitrag von \textit{ja}) und zusätzlich auf dem Tisch liegt p $\vee$ $\neg$p (Beitrag von \textit{doch}) (vgl. (\ref{448})).

\begin{exe}
	\ex\label{448} Kontext vor der MP-Äußerung: 1. ja(p), 2. doch(p)\\[-1em]	
 		\begin{tabular}[t]{|C{6em}|C{6em}|C{6em}|} 
 		\hline 	
   		$\textrm{DC}_{\textrm{A}}$ & {Tisch} & $\textrm{DC}_{\textrm{B}}$ \tabularnewline
  		\hline
   		{} & p $\vee$ $\neg$p & p \tabularnewline
  		\hline      
   		\multicolumn{3}{|l|}{cg s$_{1}$} \tabularnewline   
  		 \hline
 		\end{tabular}
\end{exe}
Wirkt erst \textit{doch} und dann \textit{ja} (vgl. (\ref{449})), ist die Situation, was den hier jeweils beschriebenen Vorzustand angeht, die gleiche wie in (\ref{448}), da bei Zuständen als statische Objekte die Reihenfolge erst einmal keine Rolle spielt. Auch in diesem Fall hat Teilnehmer B ein Bekenntnis zu p und auf dem Tisch liegt p $\vee$ $\neg$p. 

\begin{exe}
	\ex\label{449} Kontext vor der MP-Äußerung: 1. doch(p), 2. ja(p)\\[-1em]	
 		\begin{tabular}[t]{|C{6em}|C{6em}|C{6em}|} 
 		\hline 	
   		$\textrm{DC}_{\textrm{A}}$ & {Tisch} & $\textrm{DC}_{\textrm{B}}$ \tabularnewline
  		\hline
   		{} & p $\vee$ $\neg$p & p \tabularnewline
  		\hline      
   		\multicolumn{3}{|l|}{cg s$_{1}$} \tabularnewline   
  		 \hline
 		\end{tabular}
\end{exe}
(\ref{446}) bis (\ref{449}) sowie meine Erläuterungen zeigen folglich, dass meine Beschreibung der Einzelpartikeln es vermag, die denkbaren Skopusverläufe und -interaktionen prinzipiell abzubilden.

Im Folgenden möchte ich dafür argumentieren, dass die Interpretation unter gleichem Skopus zu der angemessenen Interpretation führt, wenn man die Verwendung von \textit{ja doch}-Äußerungen im Kontext betrachtet (s.u.). Als prinzipielles Argument gegen die Skopusinterpretation ist auch die Tatsache anzusehen, dass die umgekehr\-te Abfolge der MPn belegt ist. Autoren, die (\ref{445a}) als die adäquate Interpretation ansehen, argumentieren, dass die (angeblich) feste Abfolge ein Reflex der Skopusrelation \is{Skopus} ist bzw. dass deshalb nur die Interpretation unter Skopus in Frage kommt, weil die MPn in dieser Abfolge auftreten. Wenn es so ist, dass in der Abfolge \textit{ja doch} das \textit{ja} über \textit{doch} Skopus nimmt, müssten die Vertreter dieser Annahme plausiblerweise auch annehmen, dass in der Abfolge \textit{doch ja} das \textit{doch} über \textit{ja} Skopus nimmt. Da es sich meiner Meinung nach so verhält, dass sich \textit{doch ja} stets durch \textit{ja doch} austauschen lässt, der umgekehrte Ersatz aber nicht gegeben ist, halte ich es nicht für haltbar, dass für die beiden Abfolgen im Sinne der beiden Skopusmöglichkeiten eine andere Bedeutung anzunehmen ist. Wie ich in Abschnitt~\ref{sec:unmarkiert} und \ref{sec:markiert} ausführen werde, möchte ich die verschiedenen Abfolgen auf verschiedene kommunikative Absichten zurückführen, die sich in der Reihenfolge der Applikation der beiden MPn widerspiegeln. Die Bedeutung im Sinne des reinen Diskursbeitrags ist aber in beiden Fällen die gleiche. Ich gehe dabei konkret von der Interpretation in (\ref{448}) bzw. (\ref{449}) aus. Warum ich diese Interpretation für passend halte, zeige ich im Folgenden anhand zweier Auftretensweisen der MP-Kombination. 

Der erste Fall, der auch schon in \citet{Lindner1991} und \citet{Rinas2007} diskutiert worden ist, ist ein Ausschnitt aus Hofmannsthals \textit{Der Schwierige}.

\begin{exe}
\scriptsize
	\ex\label{450} Teil eines Gesprächs zwischen Graf Hans Karl Bühl und der Magd der Dame Agathe.\\
			(Komödie \textit{Der Schwierige} von Hugo von Hofmannsthal, Akt 1, Szene 6)\\
			\begin{tabular}[t]{ll} 
 				Hans Karl: & Guten Abend, Agathe. \tabularnewline
				Agathe: & Daß ich Sie sehe, Eure Gnaden Erlaucht! Ich zittre ja. \tabularnewline
				Hans Karl: & Wollen Sie sich nicht setzen? \tabularnewline
				Agathe (stehend): & Oh, Euer Gnaden, seien nur nicht ungehalten darüber,  \tabularnewline
				& daß ich gekommen bin statt dem Brandstätter. \tabularnewline
				Hans Karl: & Aber liebe Agathe, \textbf{wir sind \underline{ja doch} alte Bekannte}.  \tabularnewline
				& Was bringt Sie denn zu mir? \tabularnewline
 				Agathe: & Mein Gott, das wissen doch Erlaucht. Ich komm \tabularnewline 	
 				& wegen der Briefe.  		
 			\end{tabular}		
\end{exe}
Ich sehe die zwei Propositionen in (\ref{451}) an der Analyse als beteiligt an.

\begin{exe}
	\ex\label{451} 
	\begin{xlist}
		\ex\label{451a} p = dass Agathe sich ergeben zeigen muss
 		\ex\label{451b}	q = dass wir alte Bekannte sind	
 	\end{xlist}	
\end{exe}		
Die \textit{doch}-Äußerung nimmt Bezug auf q. Agathe macht in dieser Szene durch ihr Verhalten (sie zittert, sie sagt, dass Hans Karl nicht ungehalten sein soll, weil sie gekommen ist und nicht der Brandstätter) ein Diskursbekenntnis zu p, d.h. sie muss sich ergeben zeigen. Dadurch öffnet sich auf dem Tisch p $\vee$ $\neg$p. Ihre devote Haltung bringt nun die plausible Voraussetzung mit sich, dass Grund zu einer solchen Haltung besteht. Eine derartige Voraussetzung für devotes Verhalten ist z.B., dass die Beteiligten nicht auf einer Ebene stehen. Dies ist hier in der vorliegenden sozialen Hierarchie tatsächlich der Fall. Ein solcher Unterschied kann sich nun aus anderen Gründen auflösen. Hierzu kann beispielsweise gehören, dass man sich gut kennt und deshalb den offiziellen Gepflogenheiten nicht nachkommen muss. Für die hier betrachtete Szene heißt dies, dass wenn Hans Karl und Agathe alte Bekannte sind, sie ihm gegenüber nicht ergeben sein muss. Da sie diesen offiziellen Gepflogenheiten aber nachkommt und sich devot gibt, lässt sich aus der auf dem Tisch liegenden Proposition p ableiten, dass sie nicht alte Bekannte sind ($\neg$q). Diese Proposition beschreibt den Sachverhalt, der eine pragmatische Folgerung von p ist (wenn sich jemand devot verhält, kennt er den Interaktionspartner nicht gut) (p $>$ $\neg$q) bzw. handelt es sich hier (schwächer) um einen Sachverhalt, der in Frage gestellt wird (Kennen sie sich gut?). Sie zeigt definitiv kein Verhalten, aus dem q pragmatisch folgen würde. Das in dieser Situation offene Thema q $\vee$ $\neg$q ist auch problemlos angreifbar (vgl. (\ref{452})), wodurch Agathes untergebenes Verhalten als unangebracht zurückgewiesen werden kann.

\begin{exe}
	\ex\label{452} 
	Benimm dich mal nicht so. Wir kennen uns schon so lange. Warum machst du dich hier so klein?
\end{exe}
Ich gehe auf der Basis der obigen Erläuterungen sowie (\ref{452}) deshalb davon aus, dass auch q $\vee$ $\neg$q auf den Tisch gelangt. Die Folgerung aus p ($\neg$q) ist diskutierbar. Für diesen konkreten Dialog ist auch anzunehmen, dass der Schluss p $>$ $\neg$q Teil des cg ist. Agathe macht mit dem Bekenntnis zu p somit ebenfalls ein Diskursbekenntnis zu $\neg$q. Vor diesem Hintergrund äußert Hans Karl (\ref{453}).

\begin{exe}
	\ex\label{453} 
	Aber liebe Agathe, wir sind \textbf{ja doch} alte Bekannte.
\end{exe}
Es stellt sich der Eindruck ein, dass er damit Agathes ergebenes Verhalten als unnötig zurückweist. Die MP-Äußerung bezieht sich folglich auf die Folgerung aus der Devotheit, die mit q $\vee$ $\neg$q offenes Thema ist und die Hans Karl dadurch angreift, dass er aus dieser Disjunktion q auswählt. Ich habe den Eindruck, dass eine \textit{ja doch}-Äußerung stärker wirkt als die gleiche Assertion ohne \textit{ja}. Ich führe dies darauf zurück bzw. erfasse dies dadurch, dass q im Kontext vor der \textit{ja doch}-Äußerung in Agathes Bekenntnissystem verankert ist. Dadurch macht Hans Karl q im Zuge der \textit{ja doch}-Assertion zu einem cg-Inhalt bzw. verweist er in diesem Dialog sogar wiederholt darauf, dass sie eigentlich weiß, dass q der Fall ist. Die Proposition q scheint einen Sachverhalt zu beschreiben, über den sie sich eigentlich einig sind. Dies lässt sich schließen aus vorangehenden und folgenden Szenen: Beispielsweise sagt Agathe später, dass sie den Sekretär von früher kennt. Hans Karl nennt sie immer wieder \glqq liebe Agathe\grqq{}. Agathe weiß um die Briefe, die sie von der Gräfin überbringt. Und wenn sie sagt, die Gräfin habe ihr eingeschärft, sie solle nichts verderben, ist klar, dass sie mit der Gräfin in Kontakt steht. Aus den Anschlussszenen ist abzulesen, dass Agathe und Hans Karl sich bereits lange kennen und sich vertrauter sind als dies zwischen Leuten dieser Standesunterschiede der Fall sein müsste. In diesem Sinne kann q hier auch plausibel bereits im cg sein und zu Zwecken der Argumentation hervorgeholt werden.

\begin{exe}
	\ex\label{454} Kontext vor der \textit{ja doch}-Äußerung\\[-1em]	
 		\begin{tabular}[t]{|C{6em}|C{6em}|C{6em}|} 
 		\hline 	
   		$\textrm{DC}_{\textrm{Hans Karl}}$ & {Tisch} & $\textrm{DC}_{\textrm{Agathe}}$ \tabularnewline
  		\hline
   		{} & p $\vee$ $\neg$p & p \tabularnewline
   		{} & q $\vee$ $\neg$q & $\neg$q \tabularnewline
  		\hline      
   		\multicolumn{3}{|l|}{cg s$_{1}$ = $\lbrace$ p $<$ $\neg$q, q $\rbrace$} \tabularnewline   
  		 \hline
 		\end{tabular}
\end{exe}
\pagebreak
\begin{exe}
	\ex\label{455} Kontext nach der \textit{ja doch}-Äußerung\\[-1em]
		\begin{xlist}	
			\ex\label{455a} Teil 1\\[-1em]
 				\begin{tabular}[t]{|C{6em}|C{6em}|C{6em}|} 
 				\hline 	
   				$\textrm{DC}_{\textrm{Hans Karl}}$ & {Tisch} & $\textrm{DC}_{\textrm{Agathe}}$ \tabularnewline
  				\hline
   				{} & p $\vee$ $\neg$p & p \tabularnewline
   				q & q $\vee$ $\neg$q & $\neg$q \tabularnewline
  				\hline      
   				\multicolumn{3}{|l|}{cg s$_{2}$ = s$_{1}$} \tabularnewline   
  				 \hline
 				\end{tabular}
 			\ex\label{455b} Teil 2\\[-1em]	
 				\begin{tabular}[t]{|C{6em}|C{6em}|C{6em}|} 
 				\hline 	
   				$\textrm{DC}_{\textrm{Hans Karl}}$ & {Tisch} & $\textrm{DC}_{\textrm{Agathe}}$ \tabularnewline
  				\hline
   				{} & p $\vee$ $\neg$p & p \tabularnewline
  				\hline      
   				\multicolumn{3}{|l|}{cg s$_{3}$ = s$_{1}$} \tabularnewline   
  				 \hline
 				\end{tabular}
 		\end{xlist}		
\end{exe}
Ein in (\ref{454}) und (\ref{455}) unberücksichtigter Effekt, der für die Interpretation der MP-Äußerung aber nicht direkt relevant ist, ist, dass wenn q im cg enthalten ist bzw. zu einem cg-Inhalt wird, und dazu der Schluss p $>$ $\neg$q im cg ist, auch $\neg$q im cg sein müsste: Hans Karl \glq überschreibt\grq {} dann sowohl Agathes Bekenntnis zu $\neg$q als auch ihr Bekenntnis zu p. In diesem Sinne findet sich der Eindruck wieder, dass Hans Karl Agathes von ihr vermittelte Ergebenheit als unnötig zurückweist.

Ich möchte folglich vertreten, dass eine \textit{ja doch}-Äußerung eine sinnvolle Interpretation erfährt, wenn man davon ausgeht, dass die beteiligten Partikeln sich beide gleichermaßen auf die gleiche Proposition beziehen, d.h. den gleichen Skopus nehmen. Dies entspricht der Bedeutungszuschreibung aus (\ref{445c}/\ref{445d}) bzw. den diskursstrukturellen Modellierungen in (\ref{448}) und (\ref{449}).

In diesem Sinne soll die Verwendung der beiden MPn in dem Kontext in (\ref{450}) nicht darauf verweisen, dass zur Diskussion steht, ob Agathe ein Bekenntnis zu q oder $\neg$q hat. Dies entspricht der Lesart, in der \textit{doch} Skopus über \textit{ja} nimmt. Für genauso unpassend halte ich die Interpretation, unter der Hans Karl sich mit seiner \textit{ja doch}-Äußerung auf ein Bekenntnis von Agathe bezieht, das beinhaltet, dass q vs. $\neg$q auf dem Tisch liegt. Diese Lesart entsteht, wenn \textit{doch} im Skopus von \textit{ja} steht. In beiden Fällen müsste das Thema (Sind sie alte Bekannte?) nicht einmal wirklich auf dem Tisch liegen. Es würde sich allein um die von ihr vertretene Annahme handeln bzw. es stünde zur Diskussion, was sie vertritt. Mir scheinen diese Lesarten zu schwach. Das Thema q $\vee$ $\neg$q steht hier tatsächlich zur Diskussion, auf der Basis ihres Verhaltens. Auch unter der von mir zugeschriebenen Bedeutung kann Agathe durchaus ebenfalls annehmen, dass auf dem Tisch liegt, ob q gilt bzw. dass (aufgrund ihres Bekenntnisses zu $\neg$q) zur Diskussion steht, ob sie q annimmt oder nicht. Beides beschreibt aber nicht den erforderlichen Kontextzustand für die MP-Äußerung von Hans Karl. Setzt man die additive Lesart an, können diese Verhältnisse ebenfalls eintreten. Sie können aber nicht als Minimalanforderung an den Vorkontext einer \textit{ja doch}-Assertion angesehen werden. Ich halte hier die non-Skopus-Lesart für korrekt. Und dies ist genau die Frage, die es zu entscheiden gilt: Wie sieht der Kontext aus, auf den ein Sprecher mit einer \textit{ja doch}-Äußerung reagiert? Mit anderen Worten, was muss vorliegen, damit Hans Karl sich zu dieser Äußerung veranlasst sieht? Da die Kontextanforderungen, wie ich sie formuliert habe, Minimalanforderungen abbilden, gilt dies gleichermaßen für die Modellierungsmöglichkeiten der Kombinationen.

(\ref{456}) zeigt ein authentisches Beispiel einer \textit{ja doch}-Assertion aus dem FOLK-Korpus der DGD2. EL und NO, die dem Gespräch nach zu urteilen beide Friseure sind, unterhalten sich über Sehnenscheidenentzündungen und das Risiko für ihre berufliche Tätigkeit.

EL hat bisher nur mit leichten Entzündungen zu tun gehabt, NO berichtet, dass er sich jeden Abend den Arm mit Voltaren eincremt. Schließlich kommt das Thema auf eine Sehnenscheidenentzündung, wobei nicht klar ersichtlich ist, wen von beiden sie betraf.\footnote{Für meine Analyse ist dies allerdings auch unerheblich. Ich gehe davon aus, dass NO betroffen war.}

\begin{exe}
	\ex\label{456} 
		\begin{tabular}[t]{lll} 
 		1172 & NO & oder darfst\_s (.) bierglas nich falsch anfassen kriegst ooch \tabularnewline
 		& & $[$ne sehnenschei$]$denent$[$zündung (.) ne$]$ \tabularnewline
 		1173 & EL & $[$hm\_hm$]$ \tabularnewline
 		1174 & EL & $[$((lacht))$]$ \tabularnewline
 		1175 &	& (0.31) \tabularnewline
		1176 & EL &	\textbf{na det war \underline{ja doch} (.) von der gitarre} \tabularnewline
		1177 &	& (0.31) \tabularnewline
		1178 & NO & ((lacht)) \tabularnewline
 		1179 &	& (1.84) \tabularnewline
		1180 & EL & die sehnenscheidengitarre \tabularnewline
		1181 & & (0.2) \tabularnewline
		1182 & NO & hm\_h$[$m \tabularnewline
		1183 & EL & $[$((lacht))
		\hfill\hbox {(FOLK\_E\_00039\_SE\_01\_T\_02)}			
  		\end{tabular} 						
\end{exe}	
NOs Äußerung suggeriert, dass besagte Entzündung vom Heben eines Bierglases stammt (= p). Damit macht er ein Bekenntnis zu p und legt p mit seiner Alternative auf den Tisch. Die Proposition p \is{Implikation} impliziert, dass die Entzündung nicht auf einen anderen Umstand zurückzuführen ist, wie z.B. auf eine Gitarre ($\neg$q) (wenn man davon ausgeht, dass es nur einen Grund gibt). p $\rightarrow$ $\neg$q kann als cg-Wissen angesehen werden. Durch das Bekenntnis zu p bekennt NO sich auch zu $\neg$q, wodurch sich auch das Thema q $\vee$ $\neg$q eröffnet. Aus dem Folgekontext ist zudem klar, dass beide wissen, dass die Entzündung durch die Gitarre bzw. ein als solches benutztes Objekt zustandegekommen ist, d.h. q ist im cg.

\begin{exe}
	\ex\label{457} Kontext vor der \textit{ja doch}-Äußerung\\[-1em]	
 		\begin{tabular}[t]{|C{6em}|C{6em}|C{6em}|} 
 		\hline 	
   		$\textrm{DC}_{\textrm{EL}}$ & {Tisch} & $\textrm{DC}_{\textrm{NO}}$ \tabularnewline
  		\hline
   		{} & p $\vee$ $\neg$p & p \tabularnewline
   		{} & q $\vee$ $\neg$q & $\neg$q \tabularnewline
  		\hline      
   		\multicolumn{3}{|l|}{cg s$_{1}$ = $\lbrace$ p $\rightarrow$ $\neg$q, q $\rbrace$} \tabularnewline   
  		 \hline
 		\end{tabular}
\end{exe}
EL äußert die \textit{ja doch}-Assertion und reagiert damit auf das offene Thema, ob es die Gitarre war. q kann transparent als bekannt ausgegeben werden, wodurch NOs Bekenntnis zu p, das überhaupt Anlass für die \textit{doch}-Verwendung ist, \glq überschrieben\grq {} und zurückgewiesen wird. Wenn Einigkeit hinsichtlich der Implikation \is{Implikation} besteht und hinsichtlich q, kann nur $\neg$p gelten. Dies entspricht auch der Realität, weil NO eigentlich sehr genau weiß, dass die Gitarre der Grund für die Entzündung ist, d.h. er meint seinen Beitrag sicherlich nicht ganz ernst.

Die Non-Skopus-Lesart führt erneut zu einer adäquaten Interpretation der Szene. EL reagiert nicht derart, weil NO annimmt, dass zur Diskussion steht, ob die Sehnenscheidenentzündung von der Gitarre herrührt oder nicht (ja(doch(q))) und auch nicht, weil auf dem Tisch liegt, ob NO davon ausgeht, dass die Gitarre der Grund ist oder dass er nicht davon ausgeht (doch(ja(q))). Wie in (\ref{450}) ist mein Punkt nicht, zu vertreten, dass diese Bedeutungsaspekte undenkbar und unmöglich beteiligt sein können. Aber ich bin der Meinung, dass ja(q) \& doch(q) die Absichten der MP-Äußerung am besten auffängt. Es geht in erster Linie darum, im Kontext q selbst zu klären.\\

\noindent
Anhand dieser zwei \textit{ja doch}-Assertionen im Kontext möchte ich folglich dafür argumentieren, dass auch – neben der Tatsache, dass die umgekehrte Abfolge in nicht abweichender Skopuslesart belegbar ist – die Betrachtung derartiger Äußerungen im Kontext die Entscheidung zugunsten der additiven Lesart nahelegt.

Angenommen die additive Lesart bildet die Interpretation der \textit{ja doch}-Äußerung korrekt ab, bleibt dennoch die Frage bestehen, wie die Präferenz der Reihung \textit{ja doch} zustande kommt. Unabhängig davon, ob zuerst \textit{ja} und anschließend \textit{doch} seinen Beitrag leistet oder \textit{doch} vor \textit{ja} appliziert, bleibt die Interpretation hinsichtlich der Bezugsbereiche schließlich gleich.

\section{Erklärung der unmarkierten Abfolge}
\label{sec:unmarkiert} 
\subsection{Ikonizität}
Den Markiertheitsunterschied zwischen \textit{ja doch} und \textit{doch ja} möchte ich im Folgenden über die Annahme einer Form von \textit{Ikonizität} \is{Ikonizität} ableiten (vgl. \citealt[197-200]{Mueller2014a}; \citeyear[223-226]{Mueller2017b}). Ich gehe davon aus, dass die syntaktische Oberflächenabfolge die Verwendung der Elemente widerspiegelt. Diese prinzipielle Überlegung wird auch in anderen funktionalen Erklärungen zur Wortstellung vertreten. \citet[399]{Dik1997} formuliert beispielsweise das Prinzip in (\ref{458}).
\setcounter{equation}{0}
\begin{exe}
	\ex\label{458} Generelles Prinzip 1\\
 		The Principle of Iconic Ordering\\
		Constituents \is{The Principle of Iconic Ordering} conform to (GP1) when their ordering in one way or another iconically reflects the semantic 				content of the expression in which they occur. 
	\hfill\hbox {\citet[399]{Dik1997}}
\end{exe}
Ich habe in Abschnitt~\ref{sec:ikonizität} in Kapitel~\ref{chapter:hintergrund} diese Form von Ikonizität als \textit{diagrammatische ikonische Motivation} (\citealt[516]{Haiman1980}) \is{diagrammatische ikonische Motivation} als einen Typus von Ikonizität eingeordnet (zu einer ausführlicheren Darstellung des Konzeptes vgl. diesen Abschnitt).

Dieses Prinzip äußert sich nach \citet[399]{Dik1997} z.B. darin, dass die Ordnung von Sätzen in einem Text im unmarkierten Fall die Reihenfolge der Ereignisse widerspiegelt, die sie beschreiben. Je nach temporaler Konjunktion \is{Konjunktion} ergeben sich z.B. Unterschiede hinsichtlich der Markiertheit der Abfolge von Haupt- und Nebensatz. Im unmarkierten Fall gehen Nebensätze mit der Bedeutung \glq nachdem p\grq {} dem Hauptsatz deshalb voran (vgl. (\ref{459})), während Nebensätze mit der Bedeutung \glq bevor p\grq {} dem Hauptsatz folgen (vgl. (\ref{460}), vgl. auch \citealt{Diessel2008}).

\begin{exe}
	\ex\label{459} 
		\begin{xlist}
			\ex\label{459a} After John had arrived, the meeting started. (unmarkiert)
 			\ex\label{459b}	The meeting started after John had arrived. (markiert)
 		\end{xlist}		
\end{exe}

\begin{exe}
	\ex\label{460}
 		\begin{xlist}
			\ex\label{460a} The meeting started before John arrived. (unmarkiert)
 			\ex\label{460b}	Before John arrived, the meeting started. (markiert)	
 		\hfill\hbox {\citet[400]{Dik1997}}	
 		\end{xlist}				
\end{exe}
\citet{Dik1997} argumentiert ähnlich für markierte und unmarkierte Abfolgen von Haupt- und Nebensatz bei Konditional- und \is{Konditionalsatz} \is{Finalsatz} Finalsätzen (vgl. (\ref{461}), (\ref{462})).

\begin{exe}
	\ex\label{461}
 		\begin{xlist}
			\ex\label{461a} If you are hungry, you must eat. (unmarkiert)
 			\ex\label{461b}	You must eat if you are hungry. (markiert)
 		\end{xlist}			
\end{exe}

\begin{exe}
	\ex\label{462}
 		\begin{xlist}
			\ex\label{462a} John went to the forest in order to catch a deer. (unmarkiert)
 			\ex\label{462b}	In order to catch a deer, John went to the forest. (markiert)	
 		\end{xlist}			
	\hfill\hbox {\citet[400]{Dik1997}}
\end{exe}
Die Überlegung ist, dass die Bedingung der Konsequenz in gewissem Sinne konzep\-tuell überlegen ist, ähnlich wie die Ausführung der Handlung ihrem finalen Ziel.

In Abschnitt~\ref{sec:ikonizität} habe ich auch gezeigt, dass ikonische Erklärungen nicht nur für prinzipiell akzeptable, lediglich weniger frequente Strukturen herangezogen worden sind, sondern durchaus auch stärkere Akzeptabilitätsabfälle zu beobachten sind (vgl. (\ref{464}) und (\ref{465})).

\begin{exe}
	\ex\label{464}
 		\begin{xlist}
			\ex\label{464a} *He killed and shot her.
 			\ex\label{464b}	He shot and killed her.	
 		\hfill\hbox {\citet[92]{Givon1991}}	
 		\end{xlist}				
\end{exe}

\begin{exe}
	\ex\label{465}
 		\begin{xlist}
			\ex\label{465a} Herz und Nieren
 			\ex\label{465b}	*Nieren und Herz
 			\hfill\hbox {\citet[140]{Plank1979}}	
 		\end{xlist}			
\end{exe}
Im Folgenden möchte ich der Idee nachgehen, zu sagen, dass die Abfolge der MPn in einem Sinne motiviert ist, in dem motiviert ist, warum z.B. temporale Abfolgen Einfluss auf markierte und unmarkierte Anordnungen von Haupt- und Nebensatz nehmen. In dem von mir untersuchten Fall sind nicht temporale Relationen beteiligt oder Konzeptualisierungen von oben nach unten, die sich in \is{irreversible Binomiale} Binomialen spiegeln, sondern es geht um die direkteste Abbildung des Diskursverlaufs, d.h. die Reihenfolge der Kontextaktualisierungen mit den beiden MPn.

\subsection{Stabile und instabile Kontextzustände}
In Abschnitt~\ref{sec:mplass} habe ich die zwei generellen Antriebe für Gespräche nach \citet{Farkas2010} angeführt (vgl. (\ref{466})).

\begin{exe}
	\ex\label{466} Zwei fundamentale Antriebe für Gespräche
 		\begin{xlist}
			\ex\label{466a} Erweiterung des cg
 			\ex\label{466b}	Herstellen eines stabilen Kontextzustands
 		\end{xlist}			
\end{exe}
Zum einen folgen Teilnehmer dem Bedürfnis, den cg zu erweitern. Zum anderen streben sie danach, einen stabilen Kontextzustand zu erreichen, d.h. einen Zustand, in dem kein offenes Thema zur Diskussion auf dem Tisch liegt. Die Gesprächsteilnehmer beabsichtigen somit, die Elemente auf die Art vom Tisch zu entfernen, dass der cg erweitert wird. 

Betrachtet man die Diskursbeiträge, die ich in Abschnitt~\ref{sec:inkdm} für \textit{ja}- und \textit{doch}-Äußerungen formuliert habe, vor dem Hintergrund der Stabilität von Kontextzuständen, bezieht sich \textit{doch} stets auf einen instabilen Kontextzu\-stand: Die Disjunktion p $\vee$ $\neg$p liegt vor der MP-Äußerung auf dem Tisch. Der Sprecher bekennt sich im Zuge der \textit{doch}-Äußerung zu einer der beiden Propositionen. Mit einer \textit{doch}-Äußerung kann aber nie Einigung hergestellt werden, so dass das Thema vor und nach der MP-Äußerung offen ist. Der Kontext bleibt instabil. Die Verwendung von \textit{ja} resultiert hingegen immer in einem stabilen Kontextzustand. Im Kontext vor der MP-Äußerung ist im Diskursbekenntnissystem des Gesprächspartners bereits genau die Annahme enthalten, die die Assertion im nächsten Kontextzustand einführen wird. Als Resultat haben Sprecher und Hörer das gleiche öffentliche Bekenntnis und die Proposition gelangt in den cg. Die Partikel \textit{ja} involviert in diesem Sinne stets einen stabilen Kontextzustand, \textit{ja} fordert nie, dass p zur Debatte steht. 

Wenn \textit{ja} und \textit{doch} zusammen auftreten, ist somit immer ein Element beteiligt, das einen stabilen Kontextzustand herstellt (\textit{ja}) und ein Element, das auf einen instabilen Zu\-stand Bezug nimmt (\textit{doch}), der auch bestehen bleiben würde, wenn es allein aufträte (d.h. ohne \textit{ja}).

\subsection{Diskursstrukturelle Ikonizität}
Wenn es nun die oberste Absicht eines Gespräches ist, den cg zu erweitern und einen stabilen Kontextzustand zu erreichen, kommt ein Sprecher diesem obersten kommunikativen Ziel am direktesten nach, wenn er das Element, das die Stabilität des Kontextes herbeiführen kann und die Proposition zu cg-Material machen kann, sofort einführt und zur Wirkung bringen lässt.

Führt er erst das \textit{ja} ein, wird das, was er wünscht, nämlich Stabilität, direkt hergestellt, da er dadurch ausdrückt, dass die Diskursteilnehmer sich hinsichtlich der zur Diskussion stehenden Proposition einig sind. Führt der Sprecher zuerst das \textit{doch} ein, bringt er zunächst nur die konzessive Relation zum Ausdruck (trotz der beiden zur Diskussion stehenden Optionen p $\vee$ $\neg$p vertritt der Sprecher p). Und erst im nächsten Schritt vermittelt er, dass es sich bei diesem Inhalt um eine Annahme handelt, die auch der Gesprächspartner vertritt, weshalb sie sich hinsichtlich p einig sind und p Teil des cgs ist. 

Vor diesem Hintergrund halte ich es für unmarkiert, weil ikonisch, das \textit{ja} vor dem \textit{doch} einzuführen, da das \textit{ja} den von \textit{doch} vorausgesetzten instabilen Zustand sofort auflöst. Der eigentliche Diskursbeitrag ist zwar unter beiden Abfolgen der gleiche, am direktesten, d.h. \is{Isomorphie} isomorph, kommt aber die Reihenfolge \textit{ja doch} dem kommunikativen Ziel nach. Diese Reihenfolge ist folglich motiviert in dem Sinne, dass sie unter Auftreten dieser beiden MPn die direkteste Möglichkeit darstellt, den gewünschten Kontextzustand herbeizuführen.

Die Anordnung der MPn leite ich hier aus der Annahme ab, dass die Anrei\-cherung des cg sowie die Herstellung eines stabilen Kontextzustandes von den Gesprächsteilnehmern beabsichtigt ist. Hierbei handelt es sich um ein übergeordnetes Prinzip, das im Diskurs wirkt. Es macht eine sehr allgemeine Annahme über Diskursabsichten, die zunächst nicht an bestimmte Sprechakte oder Konstruktionstypen gebunden ist. Deshalb verwundert es nicht, dass dieses Prinzip über alle assertiven Kontexte hinweg greift, wann immer \textit{ja} und \textit{doch} gemeinsam auftreten. Die Reihung \textit{ja doch} ist folglich stets die bevorzugte Abfolge. Dies gilt sowohl für die Beispiele, die ich zu Beginn von Abschnitt~\ref{sec:abfolgejd} aus der Literatur angeführt habe, als auch für die sprachlichen Kontexte, in denen sich die umgekehrte Abfolge \textit{doch ja} finden lässt (vgl. Abschnitt~\ref{sec:distributiondj}). 

\subsection{Prototypische Assertionen}
Wenngleich es sich bei (\ref{466}) um ein übergeordnetes Prinzip kommunikativer Absichten handelt, wird es dennoch durch konkrete Konstruktions- bzw. Sprechakttypen \is{Sprechakt} realisiert.

Diese Typen sind zwar in dem Sinne gleich, dass sie assertiv \is{Assertivität} sind, sie unterscheiden sich aber auch auf die Art, dass sie auch eigene Absichten mitbringen (s.u.). Genau diese Anforderungen oder kommunikativen Absichten von Äußerungen sind der Aspekt, über den ich ableiten möchte, warum die Abfolge von \textit{ja} und \textit{doch} vornehmlich in ganz bestimmten Fällen umkehrbar zu sein scheint. Die Idee ist, dass es Sprechakttypen gibt, deren Eigenschaften sowieso – d.h. unabhängig des Auftretens jeglicher MPn – den Diskurseigenschaften entsprechen, die die \textit{ja doch}-Abfolge dem allgemeinen Prinzip in (\ref{466}) nach widerspiegelt. Hierbei handelt es sich \is{prototypische Assertion} um prototypische Assertionen, d.h. Assertionen, die alle drei Kriterien aus (\ref{467}) erfüllen.

\begin{exe}
	\ex\label{467} Prototypische Assertion
 		\begin{xlist}
			\ex\label{467a} Bekenntnis des Autors zu p.
 			\ex\label{467b}	p (vs. non-p) wird auf dem Tisch oben auf gelegt.
 			\ex\label{467c}	Projektion eines zukünftigen cg, der p beinhaltet.
 		\end{xlist}			
 		\hfill\hbox{\citet[92]{Farkas2010}}
\end{exe}
Dieser Typ von Assertion hält sich, wenn \textit{ja} und \textit{doch} zusammen auftreten, sehr einfach an das übergeordnete Diskursprinzip, weil es dem Charakter dieses assertiven Typus entspricht: Es ist die Absicht einer solchen MP-losen Assertion, p zum Inhalt des cg zu machen. Da dieser Typ Assertion p sowieso in den cg einfügen möchte, ist es nur natürlich, dass – wenn zwei Lexeme auftreten, von denen eines diese Forderung erfüllen kann (\textit{ja}) und das andere nicht (\textit{doch}) – der Sprecher ersteres (das \textit{ja}) durch seine unmittelbare Einführung sofort zur Wirkung bringt, um dem Ziel der cg-Herstellung auf direktestem Wege nach\-zukommen.

\section{Erklärung der markierten Abfolge}
\label{sec:markiert} 
Nun ist die prototypische Assertion nur \underline{ein} Typ von Assertion im Diskurs. Meiner Meinung nach lassen sich manchen der als assertiv eingestuften Konstruktionen eigene Absichten zuschreiben, die von der prototypischen Assertion abweichen. In diesen Fällen ist es nicht die oberste Absicht der assertiven Äußerung, die ausgedrückte Proposition zu einem cg-Inhalt zu machen. Aus diesem Grund ist es auch nicht ihre oberste Absicht, den cg-Marker \textit{ja} möglichst früh applizieren zu lassen. Die eigenen Eigenschaften dieses Typs von Assertion entsprechen folg\-lich nicht sowieso der Diskursveränderung, die die \textit{ja doch}-Abfolge unmarkiert abbildet – dem möglichst unmittelbaren und direkten Schaffen eines stabilen Kontextes. Liegt ein solcher assertiver Typ vor, der nicht primär dieses Ziel verfolgt, lässt sich die Abfolge am leichtesten umkehren. Meine Vorhersage ist somit, dass die Abfolge \textit{doch ja} nicht in prototypischen Assertionen auftritt, deren Absicht es ist, die ausgedrückte Proposition zum Inhalt des cg zu machen, d.h. zu bewusst geteiltem Inhalt zwischen den Diskurspartnern.

Die Frage, die sich an dieser Stelle auftut, ist nun, inwiefern die drei Kontexte, für die ich in Abschnitt~\ref{sec:distributiondj} argumentiere, dass \textit{doch ja} dort aufzufinden ist, diese Überlegung bestätigen. Es gilt nachzuweisen, inwiefern es nicht das oberste kommunikative Ziel dieser Assertionen ist, die enthaltene Proposition zu geteiltem Inhalt zu machen. Ich werde im Folgenden die drei Kontexte nacheinander beschreiben, um anschließend ihren gemeinsamen Nenner im Sinne der obigen Überlegung zu formulieren (vgl. auch schon \citealt[200-204]{Mueller2014a}; \citealt[226-231]{Mueller2017b}). 

Der erste Kontext sind Bewertungen. (\ref{468}) bis (\ref{475}) zeigen erneut Beispiele aus Abschnitt~\ref{sec:distributiondj} bzw. weitere Belege.

\begin{exe}
	\ex\label{468} 
	\textbf{Das ist \underline{doch ja} wieder \textit{typisch}.} Ein \glqq Nerd\grqq{} läuft Amok wegen Frust auf Weib \& Lehrer. 
\end{exe}

\begin{exe}
	\ex\label{469} 
	\scriptsize
	Ich denke, auch die meisten Frauen merken schon irgendwann rechtzeitig, dass sie selbst weiblich sind und wenn sie dann mal als Mann angesprochen 			werden, ist es wohl auch keine Kränkung. \glqq Sehr geehrte Frau Minister!\grqq{} \textbf{Ist \underline{doch ja} auch ganz \textit{hübsch}.}		
\end{exe}	

\begin{exe}
	\ex\label{470} 
	\scriptsize
	die FR hat als bollwerkspresse in diesem fall mal wieder genau im richtigen moment die gelegenheit ergriffen zu initiieren! bravo, danke! dass bronski 	jetzt auch noch so genial ist das mit dem thema kultur überhaupt zu verflechetn, \textbf{ist \underline{doch ja} schon \textit{die speerspitze der europäischen bewegung}} (*grins)
	\newline		
	\hbox{}\hfill\hbox{(DECOW2012-00:B00: 331498526)}		
\end{exe}

\begin{exe}
	\ex\label{471} 
	\scriptsize
	Davon ab: für uns KLingonen seit IHR die Aliens! ( hrhr ) eben und ihr von der Förderation seit ja sooo bööööseeeeeeeeeee : D \textbf{Das ist 				\underline{doch ja} auch \textit{einer der interessanten Aspekte an Star Trek}}, dass es so eine klare Trennung zwischen Gut und Böse nicht gibt.		
	\hfill\hbox{(DECOW2012-03:B03: 254862193)}	
	\newline		
	\hbox{}\hfill\hbox{\citet[227]{Mueller2017b}}	
\end{exe}

\begin{exe}
	\ex\label{472} 
	\scriptsize
	Wir Verbraucher sind doch so leicht zu manipulieren, würden uns auch in der Wüste eine Heizung aufschwatzen lassen (\textbf{klar 8 h ist \underline{doch ja} auch \textit{mal recht kalt}}). HiHi wie doof der Otto-Normal Verbraucher doch zu weilen ist.		
	\hfill\hbox{(http://www.motor-talk.de/forum/ab-heute-}	
	\newline		
	\hbox{}\hfill\hbox{13-08-in-der--dacia-duster-gegen-lada-niva-}	
	\newline		
	\hbox{}\hfill\hbox{t2847278.html?page=2, Beitrag vom 20.08.2010)}	
	\newline		
	\hbox{}\hfill\hbox{(Google-Recherche vom 24.07.2012)}
\end{exe} 
							                    
\begin{exe}
	\ex\label{473} 
	warum sollte EA schwache Screenshots vom PC veröffentlichen – \textbf{das passt \underline{doch ja} \textit{irgendwie} nicht}?!
	\hbox{}\hfill\hbox{(DECOW2012-00:B00: 693856613)}
\end{exe} 
		               		 		        
\begin{exe}
	\ex\label{474} 
	Mittlerweile bin ich gar nicht mehr so abgeneigt. Frischer Wind kann ja nur gut tun und \textbf{viel schlimmer als Rosi letzte Saison \textit{wird} er \underline{doch ja} nich sein}.
	\hbox{}\hfill\hbox{(DECOW2012-00:B00: 972050785)}
\end{exe} 	
	
\begin{exe}
	\ex\label{475} 
	US Termin ist noch unbekannt, wird aber vor dem UK/ EU release sein. \textbf{aber das war \underline{doch ja} \textit{klar}} das es in eu kommen 			wuerde.    
	\newline		                                              
	\hbox{}\hfill\hbox{(DECOW2012-02:B02: 346714563)}
\end{exe} 
Mit allen diesen Äußerungen gibt der Sprecher eine Bewertung eines Sachverhalts ab. In fast allen Fällen dieser Art, die ich gefunden habe, handelt es sich um Kopula-Konstruktionen mit der Struktur \textit{das ist + X}. Typischerweise wird das Prädikat durch ein eva\-luatives Adjektiv realisiert, wie in den Beispielen vom Anfang (vgl. (\ref{468}) $[$\textit{typisch}$]$, (\ref{469}) $[$\textit{hübsch}$]$) oder wie in (\ref{475}) $[$\textit{klar}$]$). Es können hier aber auch Nominalphrasen stehen, die entweder selbst bewertende Bedeutungsanteile haben (vgl. (\ref{470}) $[$\textit{Speerspitze}$]$) oder die – da selbst semantisch relativ bedeutungsarm – in Verbindung mit einem bewertenden Adjektiv (vgl. (\ref{471}) mit dem semantisch relativ leeren Nomen \textit{Aspekte}) auftreten. Der (be)wertende Aspekt geht dann entweder auf das Nomen selbst zurück (vgl. (\ref{470})) oder auf ein bewertendes attribuierendes Adjektiv (vgl. (\ref{471})). Gelegentlich treten in Äußerungen dieser Art auch Abschwächer oder Relativierer auf wie in (\ref{472}) und (\ref{473}) \textit{mal}, \textit{recht} oder \textit{irgendwie}, die zu den \textit{Heckenausdrücken} (engl. \textit{hedge}) \is{Heckenausdruck (hedge)} gezählt werden können. Der Terminus geht ursprünglich auf \citet{Lakoff1973} zurück und be\-zeichnet Ausdrücke, die anzeigen, in welchem Maß ein Sprecher eine Sache einer Kategorie zuordnet. In auf Lakoffs Arbeit folgende Forschung ist die Klasse erweitert worden und – wie so häufig – hat man es hier mittlerweile mit den verschiedensten Klassifikationen und damit verbunden Abgrenzungs- und Terminologieproblemen zu tun (zu einem Überblick vgl. \citealt[Kapitel 3]{Clemen1998}). Für die in (\ref{472}) und (\ref{473}) auftretenden Ausdrücke scheint es mir allerdings unproblematisch, zu behaupten, dass sie, da sie eine \glqq gewisse Reserve gegenüber einer eindeutigen Einordnung\grqq{} (\citealt[10]{Clemen1998}) anzeigen, zur subjektiven Färbung \is{Subjektivierung} der Äußerung beitragen. 

In Beleg (\ref{474}) tritt auch \textit{werden} in epistemischem Gebrauch \is{epistemisches Modalverb} auf, für das angenommen wird, dass es eine \glqq subjektive Prognose\grqq{}  (\citealt[39]{Clemen1998}) oder eine \glqq Inferenz aus subjektiven Annahmen oder Überzeugungen\grqq{} (\citealt[1901]{Zifonun1997}) anzeigt. 

Bewertet ein Sprecher einen Sachverhalt, wie ich für derartige Beispiele annehmen möchte, dann verfolgt er mit dem Ausdruck solch eines Urteils nicht in erster Linie, den Hörer dazu zu bewegen, diese Bewertung zu teilen. Und in diesem Sinne ist es nicht das oberste Ziel dieses Typs von Äußerung, den cg um diese Information zu erweitern. Es besteht deshalb nicht die Notwendigkeit, die MP, die den Zustand der cg-Erweiterung direkt herbeiführen kann, frühst möglich zur Anwendung zu bringen. Der mit einer Bewertung \is{Bewertung} ausgedrückte Inhalt wird natürlich dennoch geteilte Information zwischen den Diskursteilnehmern (andernfalls müsste \textit{ja} schließlich überhaupt nicht verwendet werden), dies ist jedoch nicht das oberste Ziel solch einer Äußerung.

Ein weiterer Kontext, in dem sich die markierte Abfolge \textit{doch ja} belegen lässt, sind epistemisch \is{epistemische Modalisierung} modalisierte Sätze. Hier findet sich eine große Bandbreite der konkreten Realisierungen. In Abschnitt~\ref{sec:distributiondj} habe ich Beispiele angeführt, in denen \is{epistemisches Modalverb} epistemische Modalverben, modalisierende Adverbien \is{epistemisches Adverb} und Tag-Fragen \is{Tag-Frage} die epistemische Modalisierung bedingen. (\ref{476}) bis (\ref{485}) zeigt einige andere Beispiele, in denen diese sprachlichen Mittel auftreten.

\begin{exe}
	\ex\label{476} 
	\scriptsize
	Dazu kam also ein schlechtes Gewissen – ändern konnte ich ja nichts mehr daran, was ich den Ohren der Menschen \glqq angetan\grqq{} habe. Die Einzige, 	die mir da Mut gemacht hat, war meine Logopädin, die mir sagte, ich hätte auch ohne CI mein Sprechen gut kontrollierte, \textbf{es \textit{konnte} 			\underline{doch ja} also so schlimm nicht gewesen sein}. 	
	\hfill\hbox{http://www.kestner.de/n/elternhilfe/berichte/nf2.htm)}	
	\newline		
	\hbox{}\hfill\hbox{(eingesehen am 9.6.2012, Google-Recherche)}	
	\newline		
	\hbox{}\hfill\hbox{\citet[228]{Mueller2017b}}
\end{exe} 

\begin{exe}
	\ex\label{477} 
	\scriptsize
	Ist das der Luke der in der aktuellen Reptilia einen Bericht veröffentlicht hat? höchstwahrscheinlich wenn man die Nachnamen vergleicht war der schon 		out of order als er gefunden worden war? ich mein wenn \textbf{der \textit{musste} \underline{doch ja} noch zeit gehabt haben}, telefoniert oder 			jemanden bescheid gegeben haben das er gebissen worden ist. 	
	\hfill\hbox{(DECOW2012-04: 445498466)}	
\end{exe}

\begin{exe}
	\ex\label{478} 
	\scriptsize
	Exklusivinterview mit Josh Bazell\\
	Was danach kommt ... mhhh ... im Grunde finde ich den Gedanken ausgesprochen reizvoll, diese Figur bis zum Gehtnichtmehr auszuquetschen (lacht) und 		wenn ich es recht überlege, werde ich wahrscheinlich noch über Pietro schreiben, wenn ich achtzig bin (lacht noch mehr). Warum nicht? \textbf{\textit{Könnte} \underline{doch ja} gut sein}, dass die Leute dann immer noch seine Geschichten hören wollen ...			
	\newline		
	\hbox{}\hfill\hbox{(http://www.krimi-forum.de/Datenbank/Interviews/fi002279.html)}	
	\newline		
	\hbox{}\hfill\hbox{(Google-Recherche vom 24.07.2012)}	
\end{exe} 

\begin{exe}
	\ex\label{479} 
	\scriptsize
	Bei einem Abenteuer in z.B. Haelgard sollte ich ja jetzt nicht zuviel Gefahr haben, daß hier eine Orkarmee alles kurz und klein schlägt. \textbf{Wir 		haben \underline{doch ja} \textit{nun wirklich} bei Gott genügend weiße Flecken auf der Karte wo was plaziert werden kann und auch sollte.} 		
	\hfill\hbox{(DECOW2012-00: 618521882)}	
	\newline		
	\hbox{}\hfill\hbox{\citet[228]{Mueller2017b}}	
\end{exe} 		      

\begin{exe}
	\ex\label{480} 
	\scriptsize
	Aber damit hat man noch keinen Unterschied definiert. \textbf{Vorallem ist es \underline{doch ja} \textit{eigentlich} abhängig vom Hörenden ob jetzt ne Glocke schellt oder glocknet.}
	\newline
	\hbox{}\hfill\hbox{(http://kumanomori.wordpress.com/2008/08/20/glocke-triichle-schelle-und-balolzeli/)}	
	\newline		
	\hbox{}\hfill\hbox{(Google-Recherche vom 24.07.2012)}	
	\newline		
	\hbox{}\hfill\hbox{\citet[201]{Mueller2014a}}	
\end{exe} 	
\vspace{-0.65cm}
\begin{exe}
\ex\label{481}
\scriptsize
\begin{tabular}[t]{ll}
	SPIEGEL: & Aber die Grenze selbst war noch nicht erreicht? \tabularnewline
	SCHILLER: & Ich sage ja. entlang der Grenze. \tabularnewline 
	SPIEGEL: &	Können Sie uns ein paar Grenzsteine nennen. \textbf{Sie haben \underline{doch ja} auch \textit{gewiß} Vor-} \tabularnewline
	& stellungen, wo die stehen. \tabularnewline
	SCHILLER: & Sicherlich, aber die leuchtet man nicht an. 
\end{tabular}
	\newline		
	\hbox{}\hfill\hbox{(http://www.spiegel.de/spiegel/print/d-42928443.html)}
	\newline		
	\hbox{}\hfill\hbox{(eingesehen am 05.10.2015)}
\end{exe}

\begin{exe}
	\ex\label{482} 
	\scriptsize
	Während der Installation wurde dann mein Benutzername gefragt (\textbf{\textit{eigentlich} hatte ich \underline{doch ja} schon einen!?}).
	\hfill\hbox{(http://www.computerhilfen.de/hilfen-5-86354-0.html, Beitrag vom 11.10.2005)}	
	\newline		
	\hbox{}\hfill\hbox{(eingesehen am 31.7.2014)}	
\end{exe} 		
	
\begin{exe}
	\ex\label{483} 
	\scriptsize
	Wir sollen was machen? Aufgabe? Pflicht? Wirkung? Jetzt? Ähm, liebe Öffentlichkeit ... ... ... ... tut doch bitte einfach so, als wären wir nicht da . 	: ) \textbf{Wir sind \underline{doch ja} auch gar nicht da, \textit{oder}?}
	\newline		
	\hbox{}\hfill\hbox{(DECOW2012-00: 384726199)}	
\end{exe} 		
				        		                        
\begin{exe}
	\ex\label{484} 
	\scriptsize
	sagt mal es gibt ja eine rassen kunde für hunde und katzen und nager, \textbf{für schweinchen gibt es das \underline{doch ja} auch \textit{oder}?}
	\hfill\hbox{(DECOW2012-03:B03: 64250787)}	
\end{exe} 									                        
 
\begin{exe}
	\ex\label{485} 
	\scriptsize
	Hoffe das sich hier in geraumer Zeit etwas ändert!. Gegen wenn soll sich die Bluray durchsetzen? \textbf{Sie hat \underline{doch ja} schon gegen HDDVD 	gewonnen \textit{oder}?}    
	\hfill\hbox{(DECOW2012-02: 568050573)}	
	\newline		
	\hbox{}\hfill\hbox{(Google-Recherche vom 24.07.2012)}	
	\newline		
	\hbox{}\hfill\hbox{\citet[228]{Mueller2017b}}	
\end{exe} 									   		                 
Der Effekt von derartigen epistemischen Modalausdrücken ist, anzuzeigen, dass der Sprecher der Realisierung des beschriebenen Sachverhalts eine größere oder kleinere Wahrscheinlichkeit zuschreibt. Im Zentrum solcher Sätze steht für den Sprecher nicht, p zu geteiltem Wissen zu machen, sondern seine Einschätzung hinsichtlich der Proposition kund zu tun. In diesem Sinne ähneln die in (\ref{476}) bis (\ref{485}) auftretenden Mittel den expliziten Bewertungen des oben angeführten ersten \textit{doch ja}-Kontextes, in dem sich das Bedeutungsmoment der Bewertung aus dem Inhalt ergibt. Diese Verhältnisse, die ich im Folgenden für die einzelnen sprachlichen Mittel konkreter ausführen werde, liefern die Begründung, anzunehmen, dass das Eigenbedürfnis des Äußerungstyps nicht derart beschaffen ist, unabhängig die Voranstellung von \textit{ja} und dessen direkte Anwendung vor der anderen MP (\textit{doch}) zu präferieren, weil es die oberste Absicht des Sprechers ist, cg herzustellen.

In (\ref{476}) bis (\ref{478}) treten epistemische Modalverben \is{epistemisches Modalverb} auf. \citet[26]{Mache2009} zählt zu diesen Verben: \textit{kann}, \textit{könnte}, \textit{muss}, \textit{müsste}, \textit{dürfte}, \textit{sollte}, \textit{mag}, (\textit{will}) und (\textit{möchte}). (\ref{486}) zeigt einige Beispiele für die Verwendung dieser Verben.
	
\begin{exe}
	\ex\label{486} 
		\begin{xlist}	
			\ex\label{486a} Sie \textit{\textbf{dürfte}} inzwischen fertig sein.
			\ex\label{486b} Sie \textbf{\textit{kann}} mit dem Auto gefahren sein.
			\ex\label{486c} Sie \textit{\textbf{mag}} recht haben.
			\ex\label{486d} Sie \textit{\textbf{muß}} in der Stadt sein.	
			\hfill\hbox {\citet[220]{Diewald1999b}}
			\ex\label{486e} Morgen \textit{\textbf{dürfte}}/\textit{\textbf{sollte}} das Wetter besser sein.			
		\end{xlist}
\end{exe}	
Trotz ansonsten deutlich abweichender Beschreibung und Analyse der Modalverben in der Literatur, scheint in Bezug auf die Klasse der epistemischen Modalverben \is{Modalverb} Einigkeit zu bestehen, dass sie vermitteln, dass der Sprecher dem Sachverhalt eine mehr oder weniger große Wahrscheinlichkeit zuschreibt. \citet[25]{Diewald1997} beispielsweise spricht von einer \glqq sprecherabhängige$[$n$]$ Einschätzung der Rea\-lität des dargestellten Sachverhalts\grqq{}. Ähnlich heißt es bei \citet[28]{Oehlschlaeger1989} \glqq $[$...$]$ daß die Modalverben hier eine Einstellung des Sprechers hinsichtlich des Bestehens eines Sachverhalts ausdrücken, Grade der Gewissheit des Sprechers, daß ein bestimmter Sachverhalt besteht\grqq{}. \citet[350]{Loetscher1991} fasst den Bedeutungsbeitrag als \glqq durch epistemische Inferenzen gewonnene relative Sicherheitsgrade bezüglich einer Behauptung\grqq{}. Die verschiedenen Modalverben werden dann gerne entlang einer Skala der Gewissheit, wie z.B. in (\ref{487}) geordnet. Hier nimmt der Gewissheitsgrad von links nach rechts ab. 

\begin{exe}
	\ex\label{487} 
	müssen $>$ werden $>$ dürfen $>$ mögen $>$ können
	\hfill\hbox {\citet[206]{Oehlschlaeger1989}}
\end{exe}
\citet[108]{Liedke2000} spricht hier mit \citet[21]{Buscha1981[1971]} von der \glqq Vermutungsbedeutung\grqq{}. Dazu wird manchmal davon ausgegangen, dass man es generell mit einer \glqq Abschwächung des Wahrheitsanspruchs\grqq{} (\citealt[109]{Liedke2000}) zu tun hat (vgl. auch \citealt[222]{Diewald1993}, \citealt[205]{Diewald1999b}).

In den \textit{doch ja}-Beispielen (s.o.) treten mit \textit{können} und \textit{müssen} Verben auf, deren epistemische Verwendung unter obiger Interpretation niemand anzweifeln würde. Es finden sich aber auch Belege mit \textit{sollen} in einem Gebrauch wie in (\ref{488}) und (\ref{489}) (vgl. schon Abschnitt~\ref{sec:distributiondj}).

\begin{exe}
	\ex\label{488} 
	\scriptsize
	Nur zur Vollständigkeit: Was muss beim löschen des Computerkontos im AD denn noch beachtet werden? Einfach danach wieder in die Domäne bringen und fertig? \textbf{Benutzerrechte \textit{sollten} [sich] \underline{doch ja} nicht ändern} – gibt es noch Fallen?
\end{exe}

\begin{exe}
	\ex\label{489} 
	\scriptsize
	Ich persönlich halte \glqq Alles für eine umfassendere und nicht ausschließende Einstellung. Aber bestätigen kann ich Dir das erst nach vielen Testfahrten ;-). \textbf{\textit{Eigentlich} \textit{sollte} sich ein solches Gerät bei der Einstellung \glqq Sendersuche: automatisch\grqq{} \underline{doch ja} \textit{wohl} den besten, aber empfangbaren Sender nehmen, so \textit{denke ich}.}
\end{exe}
Mit \citet[33]{Heine1995} schreibe ich auch \textit{sollen} in dieser Verwendung eine epistemische Interpretation zu (vgl. auch die Beispiele in \citealt[350]{Loetscher1991}; vgl. auch \citealt[27, Fn 4]{Mache2009}). Heine führt hier das Beispiel in (\ref{490}) an, dessen epistemische Lesart er paraphrasiert als \glq Ich habe Grund zur Annahme, dass das vor mir stehende Bier kalt ist.\grq {} (im Gegensatz zur non-epistemischen Interpretation \glq Ich möchte, dass das Bier kalt ist. Deshalb solltest du es besser wieder in den Kühlschrank tun.\grq {}).

\begin{exe}
	\ex\label{490} 
	Das Bier \textbf{\textit{sollte}} kalt sein.
\end{exe}
\citet[202, Fn 32-34]{Diewald1999b} möchte dieses epistemische \textit{sollen} anders nicht anerkennen.

Modalverben stellen ein \is{epistemisches Modalverb} sprachliches Mittel dar, um eine epistemische Modalisierung zu bewirken. Eine andere Möglichkeit ist die Verwendung von \is{epistemisches Adverb} Adverbien (vgl. (\ref{491})).

\begin{exe}
	\ex\label{491} 
		\begin{xlist}	
			\ex\label{491a} Sie ist \textit{\textbf{wahrscheinlich}}/\textit{\textbf{vermutlich}} inzwischen zu hause.
			\newline
			\hbox{}\hfill\hbox {\citet[29]{Diewald1997}}
			\ex\label{491b} \textit{\textbf{Vielleicht}} habe ich mich getäuscht.	
			\hfill\hbox {\citet[278]{Diewald1999b}}
			\ex\label{491c} Hier hat es \textit{\textbf{sicher}} mal Wasser gegeben.
			\hfill\hbox {\citet[67]{Dietrich1992}}			
		\end{xlist}
\end{exe}
Die \glqq sprecherbasierte Faktizitätsbewertung\grqq{} (\citealt[14]{Diewald1999b}), die ich oben mit verschiedenen Autoren den epistemischen Modalverben zugeschrieben habe, ist hier noch offensichtlicher auf die lexikalische Bedeutung der Ausdrücke zurückzuführen (bei den Modalverben beeinflussen auch weitere grammatische und kontextuelle Faktoren $[$vgl. z.B. \citealt[223-229]{Diewald1993}, \citealt[23-33]{Heine1995}$]$). Die Adverbien dienen dem Zweck, die Proposition als mehr oder weniger (un)gewiss auszuzeichnen. Für die Exemplare in (\ref{491}) ist wohl die Skala in (\ref{492}) anzunehmen.

\begin{exe}
	\ex\label{492} 
	sicher $>$ wahrscheinlich/vermutlich $>$ vielleicht
\end{exe}
Ohne behaupten zu wollen, dass zwischen den (a)-, (b)- und (c)-Beispielen in (\ref{491}) und (\ref{493}) jeweils Bedeutungsidentität besteht, werden diese Adverbien gerade herangezogen, um den epistemischen Gebrauch der Modalverben wiederzugeben.

\begin{exe}
	\ex\label{493} 
		\begin{xlist}	
			\ex\label{493a} Sie \textit{\textbf{dürfte}} inzwischen zu hause sein.
			\hfill\hbox {\citet[29]{Diewald1997}}
			\ex\label{493b} Ich \textit{\textbf{kann}} mich getäuscht haben.	
			\hfill\hbox {\citet[278]{Diewald1999b}}
			\ex\label{493c} Hier \textit{\textbf{muß}} es mal Wasser gegeben haben.
			\hfill\hbox {\citet[67]{Dietrich1992}}			
		\end{xlist}
\end{exe}
Ein Adverb, das sehr auffällig vertreten ist unter den \textit{doch ja}-Belegen, ist \textit{eigentlich}. Die Charakterisierungen seiner Verwendung als Satzadverb \is{Satzadverb} aus der Literatur fügen sich gut in meine Überlegung, warum die Abfolge der Partikeln bei einem Kovorkommen hier in umgekehrter Abfolge auftritt. \textit{Eigentlich} wird in dieser Verwendung die Funktion zugeschrieben, den Gültigkeitsanspruch des dargestellten Sachverhalts einzuschränken oder zu relativieren (vgl. z.B. \citealt[26]{Albrecht1977}, \citealt[77]{Koenig1990}). Es trete eine \glqq Abschwächung der Assertion\grqq{} ein (\citealt[26]{Albrecht1977}) und es würden Behauptungen eingeleitet, \glqq die dem Sprecher selbst $[$...$]$ zweifelhaft vorkommen\grqq{} (\citealt[340]{Reiners1943}. Der Sprecher bekennt sich folglich nicht vorbehaltlos zum ausgedrückten Sachverhalt, von dem er den Adressaten zu überzeugen beabsichtigt. Vielmehr nimmt er eine reserviertere Haltung ein und präsentiert somit – wie im Falle der obigen klassischen epistemischen Modalisierungen – seine subjektive \is{Subjektivierung} Sicht auf die Dinge. Die Nähe zu Modalisierungen zeigt sich auch in der von \citet[26]{Albrecht1977} vorgeschlagenen Paraphrasierung von \textit{eigentlich p} durch \textit{p sollte gelten}. Da es deshalb auch in diesem Fall nicht das oberste Ziel der Äußerungen ist, p zu einem cg-Inhalt zu machen, erklärt sich, weshalb \textit{ja} in der Kombination auch erst später zur Wirkung gebracht werden \underline{kann}.
 
In (\ref{483}) bis (\ref{485}) treten Tag-Fragen \is{Tag-Frage} auf, die in deutschen Arbeiten als \glqq Rückversicherungssignale\grqq{} (\citealt{Schwitalla2002}), \glqq Vergewisserungssignale\grqq{} (\citealt{Weinrich2005[1993]}) oder \glqq Vergewisserungsfragen\grqq{} (\citealt{Willkop1988}) bezeichnet werden. Willkop sieht derartige \glq Fragen\grq {} (worunter sie keinen formalen Fragetyp verstanden wissen will) durch die \glqq frageähnlich verwendete$[$n$]$ Partikeln\grqq{} bzw. \glqq ste\-reotype$[$n$]$ Wendungen\grqq{} (\citeyear[70]{Willkop1988}) \textit{ne}, \textit{nich}, \textit{nicht}, \textit{nicht wahr}, \textit{gell}, \textit{ge}, \textit{oder}, \textit{ja}, \textit{hm}, \textit{nein} realisiert (\citeyear[71]{Willkop1988}). Es lassen sich auch regionale (\textit{gell?}, \textit{odr?}, \textit{wa?}, \textit{woll?}) oder jugendsprachliche (\textit{ey}) Varianten ausmachen (vgl. \citealt[265]{Schwitalla2002}, \citealt[128]{Imo2011}, \citealt{Frey2010}).
 
Als die zwei Kernfunktionen derartiger Ausdrücke gelten a) das Einleiten eines Sprecherwechsels und b) die Anregung zu einem Hörer-Feedback (ohne Wechsel der Sprecherrolle). Während b) das Heischen um Aufmerksamkeit oder die Verständnissicherung zum Zweck hat, ist mit a) die Erwartung verbunden, dass dem Sprecher fehlendes Wissen geliefert wird oder der Angesprochene die Ansicht des Sprechers teilt (nach \citealt[146]{Hagemann2009}, vgl. auch die metakommunikativen Fragen, die \citealt[73]{Willkop1988} den Vergewisserungssignalen zuordnet $[$z.B. \textit{Stimmt es, daß...}; \textit{Hörst du mir zu?}$]$). 

Insbesondere aufgrund der beim Adressaten eingeforderten Bestätigung der Sicht/Einstellung des Sprechers werden Frage-Tags \is{Tag-Frage} mitunter (vorschnell) kategorisch mit epistemischer Unsicherheit in Verbindung gebracht, wie z.B. im folgenden Zitat aus \citet[146]{Imo2011} 146): \glqq Das eigene Gesicht wird dadurch gewahrt, dass man durch ein Vergewisserungssignal die eigene Aussage zur Diskussion stellt und ihren Wahrheitsanspruch abschwächt.\grqq{} Es ist allerdings vielmehr so, dass hier zwischen verschiedenen Tags unterschieden werden muss, da sie Gebrauchsunterschiede aufweisen (z.B. verschiedenen Äußerungstypen angehängt werden können) und es je nach Ausdruck gerade nicht der Fall ist, dass der Sprecher die Absicht verfolgt, sich nicht zum Gesagten bekennen zu wollen (vgl. die Einzelanalysen in \citealt[125-134]{Bublitz1978}, \citealt[253-261, 262-270, 271-275]{Willkop1988}). \citet{Hagemann2009} macht auch darauf aufmerksam, dass die Position eines Tags (redezugintern vs. -final) Einfluss auf seine interaktive Funktion nimmt. Es gibt durchaus Ausdrücke, die gerade Nachdrücklichkeit (vs. ein abgeschwächtes Be\-kenntnis) indizieren und die die Sprecherrolle sichern (vs. sie übergeben) wollen. 

Die alleinige Tatsache, dass die Abfolge \textit{doch ja} in meinen Belegen auch gerne von Frage-Tags begleitet wird, fügt sich deshalb noch nicht in das Bild, dass epistemische Modalisierungen \is{epistemische Modalisierung} die Umkehr der unmarkierten \textit{ja doch}-Abfolge begünstigen. Entscheidend ist, dass es sich um das Vergewisserungssignal \textit{oder?} handelt (teilweise auch begleitet von (übermäßig) vielen Fragezeichen oder Frage- und Ausrufezeichen).

\textit{Oder?} wird in seiner Funktion als Frage-Tag nämlich genau mit der Ver\-mittlung epistemischer Unsicherheit auf Seiten des Sprechers in Verbindung gebracht. \citet[273]{Willkop1988} beschreibt den Effekt dieses Vergewisserungssignals derart, dass die vorweggehende Behauptung teilweise zurückgenommen werde. Der Sachverhalt werde \glqq nachträglich als lediglich wahrscheinlich gekennzeichnet\grqq{} (\citeyear[271]{Willkop1988}) und der Sprecher markiere seine Behauptung als Vermutung (\citeyear[272]{Willkop1988}). In diese Charakterisierung fügt sich die Einschätzung aus \citet[131]{Bublitz1978}, dass \textit{oder?} nicht auftreten kann, wenn der Sprecher von seiner Aussage sehr überzeugt ist. Es deute eine \glqq Alternative zum Vordersatz an\grqq{} (\citeyear[126]{Bublitz1978}) (wobei \citealt[276]{Willkop1988} auch betont, dass – wie im Falle anderer Vergewisserungsfragen – eine affirmative Reaktion erwartet werde). Aus diesem Grund könne \textit{oder?} ebenfalls nicht gut stehen, wenn die Alternative unmittelbar zuvor ausgeschlossen worden ist.

\begin{exe}
	\ex\label{494} 
	So, das freut mich; ihr habt jetzt also doch endlich geheiratet, \textit{\textbf{ja}}?/*\textit{\textbf{oder}}?
	\newline
	\hbox{}\hfill\hbox {\citet[127]{Bublitz1978}}			
\end{exe}
Der Sprecher ist sich der Gültigkeit seiner Assertion nicht sicher, es handelt sich um eine \is{Assertion} abgeschwächte Assertion. Mit dem sprecherseitig vertretenen einge\-schränkten Gültigkeitsanspruch gegenüber p und der durch den Adressaten benö\-tigten Bestätigung geht Willkops weitere Beschreibung einher, dass vom Adressaten eine Stellungnahme konkret erwartet werde (\citeyear[272]{Willkop1988}), in dem Sinne, dass sein Gebrauch i.d.R. der Übergabe der Sprecherrolle diene. Der Sprecher wünsche eine Auflösung durch den Hörer. Sie sieht hier sogar die Nähe zu Informationsfragen (auch im Unterschied zu \textit{ne}, \textit{gell}).

Wie bei den Bewertungen und epistemischen Modalisierungen durch Modalverben oder Adverbien sowie beim Auftreten von \textit{eigentlich} liegt folglich wiederum eine gewisse Distanzierung des Sprechers vom ausgedrückten Inhalt vor, was meiner Ansicht nach die Umkehr der unmarkierten MP-Abfolge ermöglicht. Es handelt sich zwar um die Einschätzung des Sprechers, dass p gilt, er beabsichtigt aber kein cg-Update mit der beteiligten Proposition. Er vertritt eine abwartendere Haltung als mit einer unmodalisierten Assertion, in dem Sinne, dass die Entscheidung, ob p Gültigkeit hat, von der Hörerreaktion abhängig ist. Prinzipiell erlaubt der Sprecher auch noch die Alternative, was bei einer Standardassertion ausgeschlossen ist. Es steht somit nicht im Mittelpunkt, p zu geteiltem Wissen zu machen, was durch das nachgestellte \textit{ja} (und nicht seine frühest mögliche Ap\-plikation) gespiegelt wird.

Neben diesen drei sprachlichen Mitteln, für die ich in Abschnitt~\ref{sec:distributiondj} bereits Beispie\-le angeführt habe, finden sich auch funktional verwandte Erscheinungen, die durch (\ref{495}) bis (\ref{500}) illustriert werden.

\begin{exe}
	\ex\label{495} 
	\scriptsize
	Das gibt ja n Problem, wenn ich evtl. noch gar kein Zimmer habe, wenn das Studium schon losgehtb:crazy: \textbf{V.a. müssen wir \underline{doch ja} 		nicht nur zur Uni, sondern auch mal ins Oberwiesenfeld oder nach Oberschleißheim, \textit{seh ich das richtig?}}
	\hfill\hbox{(DECOW2012-07: 515043714)}	
\end{exe}

\begin{exe}
	\ex\label{496} 
	\scriptsize
	Sowie Gast 5 meine ich das. Man sollte ja versuchen die angegebene Auflösung zu nutzen. \textbf{Das trifft \underline{doch ja} wohl auch für Spiele 		zu, \textit{\textbf{oder irre ich da?}}}   
	\hfill\hbox{(DECOW2012-07: 333693059)}	
\end{exe}
In (\ref{495}) und (\ref{496}) treten andere Rückfragen \is{Rückfrage} auf, die durch ganze Phrasen realisiert sind, die gerade lexikalisch die fragende Interpretation nahelegen, die anzeigt, dass der Sprecher vom dargestellten Sachverhalt nicht vollends überzeugt ist und ihn nicht – ohne ein Hörerstatement abzuwarten – zu geteiltem Wissen machen möchte. 

Ein weiteres Realisierungsmittel epistemischer Modalisierung, das in Ab\-schnitt~\ref{sec:distributiondj} bei einem kombinierten Vorkommen erwähnt wurde, sind \is{Diskursmarker} Diskursmarker. In (\ref{497}) tritt ein solcher Ausdruck in Isolation auf und markiert die vorweggehende Äußerung offensichtlich einerseits überhaupt als subjektive Einschätzung des Sprechers und stuft sie andererseits in der Mitte einer Commitment-Skala ein (vgl. \citealt[18]{Aijmer1997}, vgl. auch \citeyear[24]{Aijmer1997} zu \textit{I think} in finaler Position).	

\begin{exe}
	\ex\label{497} 
	\scriptsize
	So rein spekulativ nur, aber von der Theorie her, damit ich weiß ob ich es richtig verstehe. Aber, so ganz sinnvoll würde mir das ja nicht erscheinen, 	\textbf{würde \underline{doch ja} auch die Notebooks betreffen, \textit{denke ich}}, ich kenn mich da nicht aus $[$...$]$.  
	\hfill\hbox{(DECOW201200: 304543630)}	
\end{exe}
Ebenfalls zu den Mitteln der Realisierung epistemischer Modalität möchte ich Distanzierungen zählen, mit Hilfe derer ein Sprecher anzeigt, dass die Quelle der Information ein anderer Sprecher ist. Ausgedrückt werden kann diese \is{quotative Lesart} quotative/reportative \is{reportative Lesart}Interpretation beispielsweise durch eine Lesart des Modalverbs \textit{sollen} (wie in (\ref{498}) und (\ref{500})) sowie durch Adverbien wie \textit{angeblich} (vgl. (\ref{499}), (\ref{500})).

\begin{exe}
	\ex\label{498} 
	\scriptsize
	Das Tempo in dem Du diese Quallität ablieferst ist einfach atemberaubend. Vielen Dank für die tolle Beschreibung. \textbf{\textit{Soll} 					\underline{doch ja} eigentlich ganz einfach sein}, wenn man Deinen SBS folgt.  
	\newline
	\hbox{}\hfill\hbox{(DECOW2012-01: 931561866)}	
\end{exe}

\begin{exe}
	\ex\label{499} 
	\scriptsize
	Mit welchen Recht darf nur die Frau allein entscheiden, ob sie sich einem Kind mit Herztönen erledigen möchte? Und die Babyklappe? Warum wird in allen 	Punkten der Vater nicht in die Entscheidung mit einbezogen? \textbf{Er hat \underline{doch ja} \textit{angeblich} auch das \glqq Sorgerrecht\grqq{}}, 		zumindest wenn man verheiratet ist.
	\hfill\hbox{(DECOW2012-01: 253859941)}	
\end{exe}
												                    
\begin{exe}
	\ex\label{500} 
	\scriptsize
	Im Horizontalmodus war ich bisher nicht unterwegs da er mir trotz sauber eingestellter TS immer sehr stark nach links weg driftet. Eigenartigerweise 		macht er das im Pos Mod oberhalb 15 meter nicht so stark, \textbf{wo er \underline{doch ja} \textit{angeblich} dann in den Horizontalmod automatisch 		umschalten \textit{soll}}. 	
	\newline
	\hbox{}\hfill\hbox{(DECOW2012-00: 536949567)}	
\end{exe}		      								   				  
Neben \textit{sollen} hat auch \textit{wollen} eine ähnliche quotative Verwendung (vgl. (\ref{501})).

\begin{exe}
	\ex\label{501} 
	Sie \textbf{\textit{will}} den Betrag vor einer Woche überwiesen haben.
\end{exe}	
Je nach Autor werden diese reportativen Gebrauchsweisen auch zu den epistemischen Lesarten \is{epistemisches Modalverb} gezählt (vgl. z.B. \citealt[235]{Oehlschlaeger1989}, \citealt[218, 119-220]{Diewald1993}, \citealt[20]{Heine1995}). In dem Sinne, dass epistemische Modalverben \glqq besagen, daß es dem Sprecher nicht möglich ist, den dargestellten Sachverhalt als faktisch einzuschätzen\grqq{} (\citealt[221]{Diewald1993}), ist dies sicherlich zutreffend, wenn\-gleich hier natürlich keine Vermutungseinschätzung kund getan wird (was in anderen Arbeiten wiederum der Grund ist, diese Verwendungen nicht als i.e.S. epistemisch einzustufen $[$vgl. z.B. \citealt[41]{Mache2009}$]$), sondern die Verantwortung für die Faktizität des Inhalts einer anderen Person zugeschrieben wird. Diese ausdrückliche Entbindung von der Verpflichtung, für die Gültigkeit des Sachverhalts selbst einzutreten, kann man als Distanzierung (vgl. \citealt[49]{Bruenner1983}) bzw. \glqq Abschwächung der Assertion\grqq{} (\citealt[103]{Glas1984}) verstehen. Diese Charakterisierung von \textit{sollen} und \textit{wollen} in der epistemischen Lesart lässt sich auf \textit{angeblich} übertragen, dessen Bedeutung durch \glq wie behauptet/gesagt wird\grq {}, \glq scheinbar\grq {}, \glq wohl\grq {} charakterisiert wird (vgl. den Eintrag auf www.duden.de zu \textit{angeblich}). 

Auch durch diese Ausdrücke wird folglich ein eingeschränktes Sprecherbekennt\-nis zur Proposition bewirkt, was zu dem Eindruck einer abgeschwächten Assertion führt, deren Inhalt der Sprecher nicht vorbehaltlos zu geteiltem Wissen zwischen den Diskurspartnern zu machen beabsichtigt.

Und wie auch eingangs bereits erwähnt, finden sich zudem zahlreiche Beispiele, in denen diese sprachlichen Mittel kombiniert werden. Teilweise betrifft dies schon die obigen Belege. Weitere Fälle kombinierten Auftretens zeigen die folgenden Beispiele.

\begin{exe}
	\ex\label{502} 
	\scriptsize
	das fragst ausgerechnet ! a u s g e r e c h n e t ! du ? ! \textbf{\textit{müsstest} du \underline{doch ja} \textit{eigentlich} am besten wissen} 			WARUM man provoziert und stänkert.		 		 
	\hfill\hbox{(DECOW2012-02: 359345024)}	
\end{exe}

\begin{exe}
	\ex\label{503} 
	\scriptsize
	Nur habe ich jetzt was vergessen worauf ich auch noch achten muss? \textbf{Und eine logo/sps darf \underline{doch ja} \textit{eig.} nicht die Stanze 		direkt ansteuern \textit{oder?}} Bzw. das Ventil und das ventil die Stanze.		 		 
	\newline
	\hbox{}\hfill\hbox{(DECOW2012-05: 986319955)}	
\end{exe}
			              
\begin{exe}
	\ex\label{504} 
	\scriptsize
	Habe mal günstig Hering in Sahnesoße ergattert....nun läuft er in 4 Tagen ab...wollte fragen ob man diesen einfrieren kann?
	hm, \textbf{es kann \underline{doch ja} \textit{eigentlich} nichts passieren \textit{oder?????}} 
	\newline
	\hbox{}\hfill\hbox {(http://www.chefkoch.de/forum/2,56,281434/Hering-in-Sahnesosse-einfrieren.html)}
	\newline
	\hbox{}\hfill\hbox {(Google-Recherche vom 24.07.2012)}
\end{exe}	

\begin{exe}
	\ex\label{505} 
	\scriptsize
	Ihr fragt euch wahrscheinlich, weshalb ich in der letzten Zeit so wenige News über JC schreibe. \textbf{Der \textit{sollte} \underline{doch ja} 			\textit{eigentlich} fleissig im Training sein und riesen Fortschritte machen, \textit{\textbf{oder}?}} Leider nein.	
	\hfill\hbox {(http://www.myreininghorse.ch/, Beitrag vom 11.02.2012)}
	\newline
	\hbox{}\hfill\hbox {(eingesehen am 24.07.2012)}
	\newline
	\hbox{}\hfill\hbox {\citet[176]{Mueller2014a}}
\end{exe}        	        
Generell sehe ich die fördernde Wirkung epistemischer Modalisierungen (unter die ich hier verschiedene sprachliche Mittel gefasst habe) auf das Zulassen der umgekehrten Abfolge von \textit{ja} und \textit{doch} folglich – wie bei den expliziten Bewertungen – darin, dass der Sprecher kein uneingeschränktes Bekenntnis zu p abgibt, das er in den cg hinzufügen möchte, sondern, dass seine subjektive Einschätzung \is{Bewertung} im Mittelpunkt steht. Es besteht deshalb kein erhöhter Bedarf, die Partikel ja, die ein cg-Update bewirkt, unmittelbar einzuführen.		
	
Der dritte Kontext, in dem die umgekehrte Abfolge zu finden ist, sind \is{modaler Kausalsatz} modal interpretierte Kausalsätze (vgl. (\ref{506}) bis (\ref{510})), d.h. Kausalsätze, die keine Sachverhalte, sondern Annahmen, Einstellungen oder Sprechakte begründen.

\begin{exe}
	\ex\label{506} 
	\scriptsize
	Terror, Afghanistan-Krieg, Koalitionskrise: \emph{Ärgert Sie}, dass das Jubiläum unter diesen Vorzeichen steht? \emph{Nein}\textbf{, \textit{denn} das ist \underline{doch ja} gerade das Spannende am Bundespresseball.} Er ist aufregend, weil es immer ein Überraschungsmoment gibt, da man nie weiß, unter welchen Vorzeichen der Abend stattfinden wird.	 		 
	\hfill\hbox{(unbekannt, in: Der Tagespiegel 2001-11-14, S. -1)}	
\end{exe}

\begin{exe}
	\ex\label{507} 
	\scriptsize
	Trotzdem nehm ich mich vom Vorwurf aus, verhätschelt zu sein, gewiss nicht :D \emph{Mir schlägt dieses Thema sauer auf}\textbf{, \textit{da} viele \underline{doch ja} heutzutage bis 25 nur Party, Spaß und Blödsinn im Kopf haben}, im Alter von 16,17,18 ganz zu schweigen.  		 
	\hfill\hbox{(DECOW2012-02: 355741630)}	
\end{exe}			 

\begin{exe}
	\ex\label{508} 
	\scriptsize
	\textbf{\emph{warum}} hat keiner mal vorher die Gerüchte über Krell Morat überprüft und untersucht, \textbf{\textit{wo} \underline{doch ja} 				anscheinend alle Daten vorhanden waren? }
	\newline
	\hbox{}\hfill\hbox{(http://www.scifi-forum.de/archive/index.php/t-227.html)}
	\newline
	\hbox{}\hfill\hbox{(eingesehen am 09.06.2012)}	
	\newline
	\hbox{}\hfill\hbox{\citet[203]{Mueller2014a}}	
\end{exe}	                 

\begin{exe}
	\ex\label{509} 
	\scriptsize
	Ich habe mir ja auch sagen lassen, dass man auch mit MS-Frontpage gute Homepages machen kann. Habt ihr schon mit dem Erfahrungen gemacht? 					\emph{Ist MS-Frontpage für ne ganz einfache Homepage vor Dreamweaver zu bevorzugen.} \textbf{\textit{Denn} MS-Frontpage verwendet 					\underline{doch ja} auch ne HTML-Oberfläche.}
	\newline
	\hbox{}\hfill\hbox{(http://www.informatik-forum.at/showthread.php?43549-Typo3-vs-Dreamweaver)}
	\newline
	\hbox{}\hfill\hbox{(eingesehen am 09.06.2012)}	
\end{exe}	                 

\begin{exe}
	\ex\label{510} 
	\scriptsize
	Wenn ich bloß ne Erklärung hätte für die Ursache, aber ich weiß es nicht:mauer: PS: Auf gute Freundschaft Pietbear und Danke Als Anhang noch den Kleinschaden: @ Heli-Player, mal so ne Frage und an alle natürlich auch. \emph{Muss da echt das Heckrohr gewechselt werden?} \emph{Wenn der Riemen noch schön läuft?} \textbf{Die Heckabspannung ist \underline{doch ja} auch nur für die Optik.}
	\hfill\hbox{(DECOW2012-02: 318143690)}
	\newline
	\hbox{}\hfill\hbox{\citet[230]{Mueller2017b}}	
\end{exe}	                 
In (\ref{506}) ist dies die Einstellung des Sprechers, sich nicht zu ärgern, in (\ref{507}) die Haltung, warum ihn das Thema sauer macht. In (\ref{508}) motiviert die Angabe, dass alle Daten vorhanden zu sein schienen, die Frage, warum die Überprüfung ausblieb. In (\ref{509}) ist die Information, dass MS-Frontpage auch eine HTML-Oberfläche verwendet, das Motiv für die vorweggehende Frage, ob MS-Frontpage zu bevorzugen sei. In beiden Beispielen fehlen die Fragezeichen, die Äußerungen sind im Kontext aber als Fragen zu erkennen. Und auch in (\ref{510}) begründet die letzte Aussage die Frage nach der Notwendigkeit des Austausches des Heckrohrs.

Der Unterschied der oben angesprochenen drei Typen von Kausalsätzen lässt sich anhand von (\ref{507a}) erklären.

\begin{exe}
	\ex\label{507a} 
	Peter bleibt zu /HAU\textbackslash se, // weil es so stark /REG\textbackslash net. 		 
	\hfill\hbox{\citet[265]{Bluehdorn2006}}	
\end{exe}
Nach Blühdorn kann man diesen Satz auf drei Arten lesen, die er \is{dipositioneller Kausalsatz} als \textit{dispositionell} (eine geläufigere Bezeichnung ist \is{propositionaler Kausalsatz} \textit{propositional}), \textit{epistemisch} \is{epistemischer Kausalsatz} und \textit{deontisch-illokutionär} \is{deontisch-illokutionärer Kausalsatz} bezeichnet (m.E. ist \textit{illokutionär} passender, weil die deontische Lesart sicherlich nicht stets vorliegt). Die jeweiligen Interpretationen werden durch die Paraphrasen in (\ref{508}) erfasst.	
 		                   							   
\begin{exe}
	\ex\label{508} 
		\begin{xlist}	
			\ex\label{508a} dispositionell/propositional\\
			Peter bleibt zu Hause, und der Grund dafür ist die Tatsache, dass es so stark  regnet.
			\ex\label{508b} epistemisch\\
			Ich bin überzeugt davon, dass Peter zu Hause bleibt, und der Grund für diese Überzeugung ist mein Wissen um die Tatsache, dass es so stark 					regnet.	
			\ex\label{508c} deontisch-illokutionär/illokutionär\\
			Ich ordne an, dass Peter zu Hause bleibt, und der Grund für diese Anordnung ist meine Bewertung der Tatsache, dass es so stark regnet.
			\hfill\hbox {\citet[265]{Bluehdorn2006}}			
		\end{xlist}
\end{exe}	
Die epistemische und illokutionäre Lesart werden als \textit{modale} Lesarten \is{modaler Kausalsatz} zusammengefasst, im Gegensatz zur \textit{nicht-modalen} propositionalen Lesart (vgl. \citealt[265-266]{Bluehdorn2006}).

Es ist also die prominente Funktion der modalen Kausalsätze, ihren Bezugssatz zu begründen und zu motivieren. Oberstes Ziel ihres Auftretens ist aber nicht, zu bewirken, dass Einigung zwischen den Beteiligten in Bezug auf ihren eigenen Inhalt hergestellt wird. Wenngleich es ihre Funktion ist, zu stützen (womit einhergeht, dass keine neue Kontroverse geschaffen werden soll), steht die Vorgänger\-äußerung im Zentrum. In den Fällen, in denen im Vorgängersatz tatsächlich eine Proposition vorliegt, hinsichtlich derer eine gemeinsame Übereinkunft beabsichtigt ist, wäre der Sprecher vermutlich auch zufrieden, wenn der Hörer diese annähme – auch wenn er die Evidenz oder das Motiv zurückweisen würde. Dies sind andere Umstände als sie bei den propositional interpretierten Kausalsätzen mit Ursache-Wirkung-Relation vorliegen. In diesem Fall ist das kommunikative Ziel nämlich, tatsächlich p und q, d.h. den begründeten und begründenden Sachverhalt, zu geteiltem Wissen zu machen. Der Sprecher erhebt hier einen Wahrheitsanspruch auf p und q (vgl. auch \citealt[140]{Pasch1999}). Die modalen Kausalsätze tun das, was Äußerungen tun, denen man zuschreibt, \textit{illokutiv subsidiär} \is{illokutiv subsidiär} zu sein (vgl. \citealt[58]{Motsch1987}, \citealt[21]{Brandt1992a}, \citealt[54]{Pittner2007}, \citealt[178]{Pittner2011} zu diesem Konzept). Sie dienen der Erfolgssicherung eines anderen \is{Illokutionstyp} Illokutionstyps, womit genau meine Einschätzung aufgefangen wird, dass der Fokus auf der Äußerung liegt, für die die Kausalsätze Begründungen/Motive anführen und nicht auf ihnen selbst.
	
Für den Eigenbedarf dieser Äußerungen ist es – wie auch in den Kontexten 1 (Bewertungen) und 2 (epistemische Modalisierungen) – wieder nicht so entscheidend, dass \textit{ja} früh in der Kombination eingeführt wird: Die Einigung auf den Inhalt des Kausalsatzes ist nicht das prominente Diskursziel. Dies ermöglicht die Umkehr der Abfolge. 
	
In allen drei angeführten Fällen steht gerade nicht die Sachverhaltsbeschreibung im Mittelpunkt, sondern das Gewicht liegt auf der Sprecherhaltung. Es handelt sich in allen drei Fällen letztlich um epistemische Kontexte: Es liegt eine als solche markierte Bewertung/Einschätzung des Sprechers vor. Diese kann sich inhaltlich ergeben, die Sätze können epistemisch modalisiert sein oder es können modal interpretierte Kausalsätze auftreten. Die Kontexte weisen folglich alle einen gewissen Grad an Subjektivierung \is{Subjektivierung} auf.

Vorausgesetzt, für prototypische Assertionen gilt, dass der Sprecher sich zur Wahrheit des ausgedrückten Sachverhalts bekennt und die Absicht verfolgt, ihren Inhalt zu geteiltem Wissen zu machen (vgl. (\ref{509}) und (\ref{510})), teilen Äußerungen, in denen die Abfolge \textit{doch ja} zu finden ist, die Eigenschaft, keine prototypischen Assertionen zu sein.

\begin{exe}
	\ex\label{509} 
		S($[\textrm{D}], \textrm{a}, \textrm{K}_{\textrm{i}}) = \textrm{K}_{\textrm{o}}$ so that
		\begin{xlist}	
			\ex\label{509a} $\textrm{DC}_{\textrm{a,o}} = \textrm{DC}_{\textrm{a, i}} \cup \lbrace\textrm{p}\rbrace$
			\ex\label{509b} $\textrm{T}_{\textrm{o}} = \textrm{push}(\langle \textrm{S}[\textrm{D}]; \lbrace \textrm{p} \rbrace \rangle, \textrm{T}_{\textrm{i}})$
			\ex\label{509c} $\textrm{ps}_{\textrm{o}} = \textrm{ps}_{\textrm{i}} \ \overline\cup \ \lbrace \textrm{p} \rbrace$
			\hfill\hbox {\citet[92]{Farkas2010}}
		\end{xlist}
\end{exe}

\begin{exe}
	\ex\label{510} 
	Nachdem ein Sprecher eine Assertion mit Proposition p geäußert hat, gilt:
		\begin{xlist}	
			\ex\label{510a} Die neue Diskursbekenntnismenge des Sprechers beinhaltet p.
			\ex\label{510b} Die Proposition p (vs. $\neg$p) wird oben auf den Stapel des alten Tisches gelegt.
			\ex\label{510c} Die projizierte Zukunft des alten cg beinhaltet p (unter Bewahrung der Konsistenz des cg).
		\end{xlist}
\end{exe}
Es geht folglich nicht prominent darum, den Gesprächspartner von dieser Einschätzung zu überzeugen. Und wenn Einigung nicht das oberste Ziel ist (was \textit{ja} unmittelbar bewirkt), ist auch motiviert, warum es in der Kombination nicht vorn stehen muss. In allen drei Fällen ist aber nicht ausgeschlossen, dass der Sprecher dem übergeordneten Diskursprinzip (vgl. (\ref{511})) folgt und einen stabilen Kontextzustand herstellt, sobald es ihm möglich ist. 

\begin{exe}
	\ex\label{511} Zwei fundamentale Antriebe für Gespräche
		\begin{xlist}	
			\ex\label{511a} Erweiterung des cg
			\ex\label{511b} Herstellung eines stabilen Kontextzustands
			\hfill\hbox {\citet[87]{Farkas2010}}
		\end{xlist}
\end{exe}
In diesem Fall wählt er die unmarkierte Abfolge \textit{ja doch}.
								         
\section{Der Status der Abfolge \textit{doch ja}}
\label{sec:status}
Da sich meines Wissens nach noch kein Autor für die Existenz der umgekehrten Abfolge \textit{doch ja} ausgesprochen hat und ohnehin äußerst selten darauf hingewie\-sen wird, dass die Sequenzierungen nicht absolut sind (vgl. z.B. \citealt[289]{Thurmair1989}), bin ich über Kritik nicht überrascht. Ich möchte in diesem Kapitel deshalb einige Aspekte zum Status der Abfolge \textit{doch ja} ausführen. In Abschnitt~\ref{sec:ort} geht es insbesondere um den Fundort der Daten, d.h. die Frage, welche Bedeutung (positiv oder negativ) der Tatsache zuzuschreiben ist, dass viele Belege aus Webdaten stammen.\footnote{Ich danke den Gutachtern zu \citet{Mueller2017b} für einige dieser potenziellen Einwände.} Wenngleich ich von der Existenz der Abfolge \textit{doch ja} überzeugt bin und besagte Einwände nur für bedingt berechtigt halte, gibt es darüber hi\-naus auch Lücken, auf die ich verweisen möchte, weil ich sie mit meinen bisherigen empirischen Bemühungen noch nicht vollends schließen kann und deshalb wei\-tere Untersuchungen nötig sind. Abschnitt~\ref{sec:akz} präsentiert anschließend die Ergebnisse einer Akzeptabilitätsstudie, die zu einer der vorher aufgeworfenen Fragen einen Beitrag leistet.

\subsection{Der Fundort}
\label{sec:ort}
Mein Hauptargument für die Behauptung der Existenz der umgekehrten Abfolge von \textit{ja} und \textit{doch} ist entscheidenderweise, dass sich bei der Durchsicht der aufgefundenen Belege eine Systematik in Form der drei angeführten Kontexte offenbart. Aus diesem Grund greift meiner Ansicht nach der Einwand nicht, dass die Daten im Wesentlichen ‚nur‘ aus Webdaten stammen und deshalb als Unachtsamkeiten, Schlampigkeit oder Performanzfehler abgetan werden können, da sich im Internet sowieso nahezu alles finden lässt, wenn man nur danach sucht. Egal, um welches Phänomen es sich handelt, eine Systematik sollte aufhorchen lassen.

Man sieht, dass die Daten mitunter abweichende Orthografie und fehlende Interpunktion aufweisen sowie umgangssprachlichen Stil haben. Ich halte diesen Umstand nicht für ein wirkliches Problem, sondern eher einen Aspekt, den es in Kauf zu nehmen gilt, wenn man ein seltenes Phänomen der (konzeptionellen) Mündlichkeit untersuchen möchte. Scheinbar \glq seriösere\grq {} Alternativen wie Zei\-tungssprache oder Protokolle von politischen Reden/Debatten, die diesen \glq Makel\grq {} nicht aufweisen, eignen sich für meine Fragestellung nicht besser. MPn sind ein Phäno\-men der konzeptionellen Mündlichkeit und treten deshalb in der gesprochenen Umgangs\-sprache auf bzw. da, wo diese simuliert werden soll. Die einzig sinn\-volle Alternative wären gesprochene Daten. Hierbei hat man es aber zum einen ebenso mit \glq unsaubererer\grq {} Sprache zu tun, und zum anderen kann man auch nicht auf Orthografie und Interpunktion bauen. Ein zusätzliches Problem, das sich beim Arbeiten mit gesprochenen Daten einstellt, ist, dass diese im Ver\-gleich zu zugängli\-chen geschriebenen Daten in viel geringerer Anzahl vorliegen. Da MP-Kombina\-tionen an sich schon seltener sind als man meinen mag, ist es umso schwieriger, für eine markierte Struktur bei einem ohnehin markierten Phänomen genügend Belege zu finden. Die Datenmenge muss zunächst einmal genug \textit{ja doch}s zu Tage fördern, bevor mit einem \textit{doch ja} zu rechnen ist. Im \textit{FOLK}-Korpus der DGD2 (Gespräche von 34,5 Stunden zwischen 2007–2011) gibt es zwei relevante \textit{ja doch}-Treffer, im \textit{Wendekorpus} (Gespräche zwischen 1993–1996) finden sich 22 Belege und im \textit{Freiburger Korpus} (Gespräche 68 Stunden zwischen 1960–1974) sind 37 \textit{ja doch}-Belege enthalten. Auch im \textit{Dortmunder Chat-Korpus}, das mit medial schriftlichen Daten bei konzeptioneller Mündlichkeit den Webdaten mehr entspricht als viele andere Korpora, findet man nur eine \textit{ja doch}-Äußerung.\footnote{Der Angabe dieser Zahlen geht natürlich die Betrachtung der Strukturen im Kontext bzw. eine Überprüfung der Hörbelege voraus, um gleichlautende Formen (insbesondere akzentuiertes \textit{doch}) auszuschließen.} Es liegt nahe, auf dieser Schiene gegen die Daten argumentieren zu wollen. Wenngleich natürlich jedes Vorgehen auch Mankos mit sich bringt (derer man sich bewusst sein sollte), bin ich der Meinung, dass man gerade das Potenzial dieser noch wenig genutzten Daten sehen sollte. Webdaten geben ein realistisches Bild von konzeptioneller Mündlichkeit \is{konzeptionelle Mündlichkeit} im geschriebenen Medium ab. Möchte man herausfinden, wie Spre\-cher Strukturen in authentischen Kontexten verwenden, muss man die obigen \glq Makel\grq {} der Daten wohl in Kauf nehmen. Dazu kommt, dass es auch sein kann, dass man es hier mit sprachlichen Veränderungen zu tun hat (s.u.). Unter dieser Prämisse liegt dann auch ein Genre vor, in dem man die Struktur beobachten sollte, bevor man sie in Zeitungen oder Protokollen findet. Nicht zuletzt erlauben Webdaten das Durchsuchen sehr großer Datenmengen, was in diesem Fall unbedingt vonnöten ist. 

Dass man die umgekehrten Abfolgen hier findet, ist meiner Ansicht nach folg\-lich nicht auf einen umgangssprachlichen Stil oder eine schlampige Sprache zurückzuführen, die man im Internet vermutet. Wer die Umgangssprache so sehr als negativen Aspekt wertet, hat möglicherweise auch eine falsche Vorstellung davon, wie im Internet geschrieben wird. Einerseits kann man von den umgangssprachli\-chen, orthografisch abweichenden und interpunktionsarmen Belegen nicht behaupten, dass sie schwer ungrammatisch und unverständlich sind, weshalb sie nicht als Beispiele dienen können. Nur weil man Webdaten verwendet, verliert man nicht seinen Sachverstand, völlig abgebrochene Sätze etc. nicht zum Gegenstand der Betrachtung zu machen. Dies gilt analog für gesprochene Sprache. Andererseits werden mit Webkorpora oder auch schlichten Google-Suchen auch nicht nur Foren oder Wikipedia-Diskussionen durchsucht (woran man zunächst denken mag), sondern genauso sonstige Erzähltexte, Berichte, Artikel und Interviews, in denen MPn eben auftreten. Es gibt folglich auch innerhalb dieses Teilgenres Unterschiede zwischen verschiedenen Graden an \is{Umgangssprache} Umgangssprache, in der MPn vorkommen. Kritik allein auf der Ebene der Qualität der Daten möchte ich folglich zurückweisen. 

Es ist aber sicherlich so, dass das Phänomen der MPn/MP-Kombinationen stärker als andere Untersuchungsgegenstände für falsche Einschätzungen anfällig ist. Dieser Aspekt gilt allerdings relativ unabhängig vom verwendeten Datentyp.

Ein Klassiker, mit dem man sich bei der Untersuchung von MPn stets konfrontiert sieht, ist, dass man die Bestandteile der Kombination als MPn behandle, obwohl gleichlautende Formen anderer Wortarten vorlägen. Die einfachste Variante, gegen die Existenz der \textit{doch ja}-Abfolge zu argumentieren, ist aus dieser Perspektive, zu sagen, dass das auftretende \textit{doch} das betonte Adverb \is{Adverb} ist. \citet[209]{Thurmair1989} schreibt über \textit{ja doch}, dass schwer zu entscheiden ist, ob zwei MPn vorliegen. Sie meint sogar, dass das betonte \textit{doch} häufiger auftritt und die Kombination zweier Partikeln hier selten sei. Ich glaube, dass es generell fast unmöglich ist, das Vorkommen von \glq Dubletten\grq {} in schriftlich vorliegenden Daten auszuschließen. Es ist deshalb ebenso nahezu unmöglich, zu behaupten, dass man die Adverb-Verwendung von \textit{doch} ausschließen kann, und zwar in der Abfolge \textit{ja doch} wie in der Reihung \textit{doch ja}. Je nach MP-Kombination ist dieser Ausschluss unterschiedlich schwierig. Einfacher wird es, wenn sich die Bedeutungen von MP und Non-MP ferner sind als im Falle des betonten und unbetonten \textit{doch}. Trotz dieser Problematik ist für mich – genauso wie im Falle von \textit{ja doch} – relevant, ob die MP-Lesart plausibel verfügbar ist. Wenn man weiß, dass die Formen in den unmarkierten MP-Abfolgen prinzipiell ambig sein können, kann man nicht für die umgekehrte Abfolge fordern, dass dies nicht so ist. Die Argumentation kann nicht so verlaufen, dass man sich in der umgekehrten Abfolge sicher ist, dass die Form \underline{nicht} die MP ist, während man der unmarkierten Ordnung potenzielle Ambiguität zuschreibt. Ich habe viele Belege angeführt, so dass es dem Leser frei steht, diese dahingehend durchzusehen, ob er sich sicher ist, dass es sich stets eindeutig um das betonte \textit{doch} handelt. Ich glaube dies nicht und halte die Lesart in manchen Fällen sogar für unmöglich. Ich gehe folglich nicht davon aus, dass dieser Aspekt, der die Beschäftigung mit MPn generell begleitet, die \textit{doch ja}-Belege erklärt. Das Problem lässt sich verkleinern mit gesprochenen Daten oder konstruierten Beispielen. Im ersten Fall findet man aus Frequenzgründen nicht genügend Treffer. Letztere eignen sich nicht gut, um für eine Struktur zu argumentieren, deren Existenz bisher abgesprochen wurde. Mein Punkt ist es, zu zeigen, dass beide Abfolgen verwendet werden (s.u. zu Akzeptabilitäts\-urteilen), weshalb auch auf Korpusdaten zurückgegriffen werden muss. In Kapitel~\ref{chapter:dua} wird sich bei der Beschäftigung mit Kombinationen aus \textit{doch} und \textit{auch} zeigen, dass die gleichlautenden Adverbien unbedingt im Blick behalten werden müssen. Im vorliegenden Fall halte ich das Intervenieren des Adverbs \is{Adverb} allerdings für weniger wahrscheinlich. Über \textit{doch halt}/\textit{eben} und \textit{halt}/\textit{eben doch} schreibt \citet[216]{Thurmair1989}, dass die Voranstellung von \textit{doch} mit seiner Unakzentuiertheit einhergeht, die Nachstellung mit der betonten Form.\footnote{Ich kann dies bestätigen für die Kombination aus \textit{doch} und \textit{auch}, deren Untersuchung dadurch erschwert wird, dass beide Partikeln eine Verwendung als Adverb haben. Das weiter rechts stehende Element wird i.d.R. als das Adverb interpretiert.} Die Frage wäre also, warum sich \textit{doch} in Verbindung mit \textit{ja} in dieser Hinsicht anders verhalten sollte. \citet[215-216]{Thurmair1989} nimmt zudem an, dass sowohl \textit{doch halt}/\textit{eben} als auch \textit{halt}/\textit{eben doch} zwei Partikeln enthalten können. Die MP-Lesart sei klarer beim vorangestellten \textit{doch}. 

Die Argumentation eines Kritikers kann nun natürlich weiterlaufen unter Berufung auf Autoren, die vertreten, dass Adverbien scrambeln können. Dann hätte man es hier in allen \textit{doch ja}-Belegen mit einem gescrambelten \is{Scrambling} Adverb zu tun. Diese Annahme scheint grundsätzlich auch nicht unsinnig aus der Perspektive, dass Autoren den Akzent auf \textit{doch} mit Kontrast \is{Kontrast} in Verbindung gebracht haben (wie z.B. \citealt{Meibauer1994}) und kontrastiertes Material im Mittelfeld umgestellt werden kann. Es ergibt sich dann aber die Frage, warum \textit{doch} gerade in diesen drei Kontexten scrambelt und nicht auch in Standardassertionen. Bringt man diesen Aspekt mit der obigen Annahme zu schlampiger Sprache im Internet zusammen, würde man hier auch einen Zusammenhang zu gescrambelten Strukturen aufmachen, den man vermutlich nicht eröffnen möchte. Da man auch hier sicherlich eher davon ausgehen möchte, dass diese Umstellung durch einen gewissen Informationsstatus/Prosodie etc. bedingt ist und nicht Performanzfehler vorliegen, sollte die Annahme um vorliegendes Scrambling den obigen Einwand zur Qua\-lität der Daten auch auflösen. Abgesehen davon, dass ich nicht glaube, dass stets das betonte \textit{doch} auftritt, scheint mir auch nicht genug über die Interaktion von Adverbien und MPn bekannt zu sein, um die Möglichkeit gescrambelter Adverbien als sicheres Gegenargument vertreten zu können. Ich halte die Datenlage hier eher für unklar (vgl. z.B. \citealt{Coniglio2007}). Auch gibt es durchaus Ansätze, die nicht von gescrambelten Adverbien, sondern umgestellten MPn ausgehen (vgl. \citealt{Coniglio2007}). Unter diesem Blickwinkel müsste man dann zunächst um die Basisposition von betontem \textit{doch} wissen, um entscheiden zu können, wie \textit{ja} relativ zu diesem positioniert werden kann.

Ein anderer Einwand zu \citet{Mueller2017b}) war, dass es sich nicht um die Sequenz \textit{doch ja} handelt, sondern \is{Konstruktion} um \textit{Konstruktionen}, in denen \textit{ja} mit dem folgenden Material eine Einheit bildet: \textit{xxx doch + ja xxx}. Ich möchte nicht gänz\-lich zurückweisen, dass Umstände dieser Art eine Rolle spielen können (s. auch meine Ausführungen zu \textit{auch doch} in Abschnitt~\ref{sec:distributionad} in Kapitel~\ref{chapter:dua}). Man müsste im Fall von \textit{ja} und \textit{doch} aber dann natürlich eine handhabbare Definition dazu geben können, dass/wann/ob/warum eine Konstrution (nicht) vorliegt. Schaut man auf die Daten, die ich angeführt habe, müssten ziemlich viele Abfolgen als Konstruktionen klassifiziert werden (\textit{ist-\textbf{doch}} + \textit{\textbf{ja}-wieder-typisch}, \textit{müssen-\textbf{doch}} + \textit{\textbf{ja}-vererbbar}, \textit{trifft-\textbf{doch}} + \textit{\textbf{ja}-eigentlich}, \textit{habe-ich-\textbf{doch}} + \textit{\textbf{ja}-beim-Grafikkartenwech\-sel}, \textit{möchte-\textbf{doch}} + \textit{\textbf{ja}-garantiert}, \textit{kann-die-\textbf{doch}} + \textit{\textbf{ja}-auch}, \textit{weil-mc-\textbf{doch}} + \textit{\textbf{ja}-solo} usw.). Ich halte das sprachliche Material, das ich aufgedeckt habe, für zu vielfältig, um hier von Konstruktionen zu sprechen. Aber natürlich ist es auch schwierig, zu sagen, wo ein/e Muster/Systematik endet und eine eingefrorene Konstruktion anfängt. Ich gehe gerade vom Kovorkommen bestimmten lexikali\-schen Materials aus, und diese sprachlichen Mittel sind z.T. auch an feste Positionen gebunden, so dass nicht weiter verwunderlich ist, dass \textit{ja} oftmals einem Adjektiv oder Adverb vorangeht. Dazu kommt, dass \textit{ja} dann auch nur über seine Konstruktion Skopus nehmen würde. Ich denke aber, dass beide Partikeln weiten Skopus \is{Skopus} nehmen; enger Skopus wäre für MPn generell sehr unüblich.

Aus der Diskussion ist mitzunehmen, dass aufgrund der manchmal unklaren Wortartzugehörigkeit/Funktionen der Elemente bei der Untersuchung von MPn sicherlich Vorsicht geboten ist. Dies spricht aber nicht gegen die Arbeit mit Webdaten. Genauso wenig ist die umgekehrte Abfolge eine Illusion, deren Nachweis sich anderweitig erklären lässt.\footnote{Vgl. auch \citet[232]{Mueller2017b} zu Gegenargumenten zum Vorschlag, es handle sich bei \textit{doch ja} um eine Selbst-Korrektur des Sprechers/Schreibers.}

In der Forschung wird von vielen Autoren (z.B. \citealt{Abraham1991b}, \citealt{Diewald1997}, \citealt{Wegener2002}) vertreten, dass MPn das Ergebnis eines Grammatikalisierungspro\-zesses sind. Mir ist keine Untersuchung zur Diachronie von MP-Kombinationen im Deutschen bekannt (zum Niederländischen vgl. \citealt{Hoeksema2008}). Ich sehe aber keinen Grund, warum hier nicht ebenfalls mit Veränderungen zu rechnen sein sollte. Bevor man die Untersuchung nicht vorgenommen hat, ist nicht auszu\-schließen, dass die umgekehrte Abfolge deshalb in den Webdaten so präsent ist, weil es sich um eine neue Entwicklung handelt. Potenzielle Ambiguität \is{Ambiguität} gilt als wichtiger Zwischenschritt in \is{Grammatikalisierung} Grammatikalisierungsprozessen (vgl. z.B. \citealt[137-138, 141, 144]{Diewald2008})  und ist demzufolge auch in den Kombinationen zu erwarten. Es ist aber auch möglich, die umgekehrte Abfolge in älteren Daten zu belegen. Die ältesten Belege, die ich gefunden habe, sind von 1545, vermehrte Tref\-fer habe ich ab dem 19. Jahrhundert finden können (vornehmlich über \textit{Projekt Gutenberg}, \textit{zeno.org} bzw. fanden sich Belege unter den DECOW-Treffern). (\ref{512}) bis (\ref{522}) zeigen einige Beispiele aus verschiedenen Jahrhunderten.

\begin{exe}
	\ex\label{512} 
	\scriptsize
	Aber vmb deines Namens willen / \emph{las vns nicht geschendet werden} / \emph{Las den Thron deiner Herrligkeit nicht verspottet 			werden} / \emph{Gedenck doch} / \emph{vnd las deinen Bund mit vns} / \emph{nicht auffhören}. 22 \textbf{Es ist \underline{doch ja} vnter der Heiden Götzen keiner} / der Regen künd geben / 
	\hfill\hbox{Jeremia 14.22}
\end{exe}	

\begin{exe}
	\ex\label{513} 
	\scriptsize
	\emph{Warumb stellestu dich} / \emph{als ein Helt der verzagt ist} / \emph{vnd als ein Rise} / \emph{der nicht 						helffen kan?} \textbf{Du bist \underline{doch ja} vnter vns HERR} / vnd wir heissen nach deinem Namen / verlas vns nicht. 
	\hfill\hbox{Jeremia 14.9}
\end{exe}	

\begin{exe}
	\ex\label{514} 
	\scriptsize
	Sie gebirt vnd hat doch keine wehe / als were sie nicht schwanger? \emph{Kan auch} / \emph{ehe denn ein Land die wehe kriegt} / 			\emph{ein Volck zu gleich geborn werden?} \textbf{Nu hat \underline{doch ja} Zion jre Kinder on die wehe geboren.}Jesaja 66.8			
	\hfill\hbox{(1545, Martin Luther: Luther-Bibel)}
	\newline
	\hbox{}\hfill\hbox{(eingesehen über http://www.bibel-online.net/)}
\end{exe}
\vspace{-0.65cm}	
\begin{exe}
	\ex\label{515} 
	\scriptsize
	O Jesu! der du hoch am Creutz stehst aufgericht:\\
	Mein Heyland! \emph{ach neig  ab dein blutig Angesicht!}\\
	\textbf{Ich bin \underline{doch ja} der Preiß}/ \textbf{um welchen du gerungen}/\\
	Als du durch deinen Tod hast meinen Tod verschlungen.
	\newline
	\hbox{}\hfill\hbox{(Literatur im Volltext: Andreas Gryphius: Gesamtausgabe der}
	\newline
	\hbox{}\hfill\hbox{deutschsprachigen Werke. Band 3, Tübingen 1963, S. 93-95.:}
	\newline
	\hbox{}\hfill\hbox{So walt es Gott) (eingesehen über zeno.org)}
\end{exe}           
	
\begin{exe}
	\ex\label{516} 
	\scriptsize
	Sie betrachteten diesen Mann, dem ein so großer Ruf vorangegangen war, \emph{vielleicht} nicht mit geringerem Interesse als wir, wenn wir die kaiserlichen oder königlichen Söhne des Mars die Dienste eines Feldherrn verrichten sahen. \textbf{\textit{Knüpft} sich \underline{doch ja} gerade an die Person eines ausgezeichneten Führers das Interesse, das dem ganzen Heer gilt}, ja, wir meinen oft, die Schlachten, von denen uns die Sage oder die öffentlichen Blätter erzählen, um so deutlicher zu verstehen, wenn wir uns die Gestalt des Heerführers vor das Auge zurückrufen können.
	\newline
	\hbox{}\hfill\hbox{(1826 $[$1981$]$ Wilhelm Hauff: Lichtenstein – Kapitel 10)}
	\newline
	\hbox{}\hfill\hbox{(eingesehen über http://gutenberg.spiegel.de/)}
\end{exe}		
							  
\begin{exe}
	\ex\label{517} 
	\scriptsize
	Seine Sorgen hatte er zurückgelassen, sie folgten ihm nicht durch das Tor der Träume; nur liebliche Erinnerungen verschmolzen und mischten sich zu 			neuen reizenden Bildern; das Mädchen aus der St.-Séverin-Straße mit ihrer schmelzenden Stimme schwebte zu ihm her, und erzählte ihm von ihrer Mutter; 		\emph{er schalt sie, daß sie so lange auf sich habe warten lassen}, \textbf{\textit{da} er \underline{doch ja} den Ersten und Fünf\-zehnten gekommen sei}; er wollte sie küssen zur Strafe, sie sträubte sich, er hob den Schleier auf, er hob das schöne Gesichtchen am Kinn empor, und siehe – 		es war Don Pedro, der sich in des Mädchens Gewänder gesteckt hatte, und Diego sein Diener wollte sich totlachen über den herrlichen Spaß. 
	\newline
	\hbox{}\hfill\hbox{(1828 $[$1970$]$ Wilhelm Hauff: Novellen – Kapitel 31)}
	\newline
	\hbox{}\hfill\hbox{(eingesehen über http://gutenberg.spiegel.de/)}
\end{exe}	
						    	        
\begin{exe}
	\ex\label{518} 
	\scriptsize
	11$]$ \emph{Daher also geht mit Mir}, damit ihr zu eurem großen Troste solches alles ehedem erfahrt denn alle die andern in der Hütte der 			Purista; \textbf{\textit{denn} ihr habt für die Errettung der Tiefe vor dem Untergange Meines Wissens \underline{doch ja} auch in dieser Zeit am 			meisten und am lebendigsten zu Gott Tag und Nacht gefleht!}	
	\hfill\hbox{(1842 $[$k.A.$]$ Jakob Lorber: Die Haushaltung Gottes Bd.)}
	\newline
	\hbox{}\hfill\hbox{(DECOW2012-02: 263034439)}
\end{exe}								                        
											  
\begin{exe}
	\ex\label{519} 
	\scriptsize
	Ob eine solche Empfänglichkeit sich in Tat oder auch nur in Begierde äußert, hängt von Nebenumständen ab, Umständen, die von der Tugend unabhängig 			sind. \textbf{So viel \textit{wird} \underline{doch ja} Herr Dr. TAUBERT uns zugeben}, daß das Wesen der Tugend eine Gesinnung ist.	
	\newline
	\hbox{}\hfill\hbox{(1874 $[$k.A.$]$ Frederickn Anthony Hartsen: Die Moral des Pessimismus)}
	\newline
	\hbox{}\hfill\hbox{(DE2012-02: 858568178)}
\end{exe}						                        
		      
\begin{exe}
	\ex\label{520} 
	\scriptsize
	Er sollte gleich am ersten Tage herausfühlen, daß ich nicht so dumm sei, seine Natur, seine Gaben, seine Vorzüge zu knechten und zu knebeln. 				\textbf{\textit{Das war} \underline{doch ja} \textit{das beste Mittel}, diese Gaben und Vorzüge kennen zu lernen!}			
	\hfill\hbox{(1887 $[$k.A.$]$ Karl May: Deutsche Herzen, deutsche Helden)}
	\newline
	\hbox{}\hfill\hbox{(DECOW2012-02: 248350430)}
\end{exe}									 

\begin{exe}
	\ex\label{521} 
	\scriptsize
	Der erstaunte Graf tröstete sie freundlich, zeigte ihr den Grafenring und sagte, sie solle gleich das Hennenmädel in den Saal kommen lassen. – \glqq 		Aber, mein lieber Himmel! \textbf{\textit{die ist} \underline{doch ja} \textit{so garstig und schmutzig}!}\grqq{} meinte die Köchin.
	\newline
	\hbox{}\hfill\hbox{(1911 Ignaz und Josef Zingerle: Kinder- und Hausmärchen aus Tirol)}
	\newline
	\hbox{}\hfill\hbox{(DECOW2012-07: 750624)}
\end{exe}				             								            
  
\begin{exe}
	\ex\label{522} 
	\scriptsize
	\glqq Konstantin Dmitritsch,\grqq{} begann sie zu Lewin, \glqq sagt mir doch, ich bitte recht schön – was hat das zu 				bedeuten – \textbf{Ihr wißt 			\underline{doch ja} alles.} Bei uns in Kaluga haben alle die Bauern und alle die 			Weiber alles vertrunken, was sie hatten, und zahlen uns jetzt keine Steuern mehr. Was hat das zu bedeuten? Ihr lobt doch die 	Bauern sonst stets!\grqq{} – 
	\hfill\hbox{(1920 [1920] Lew Tolstoi: Anna Karenina – 1 – Kapitel 15)}
	\newline
	\hbox{}\hfill\hbox{ (eingesehen über http://gutenberg.spiegel.de/)}
\end{exe}					 
In (\ref{512}) bis (\ref{515}) und (\ref{522}) tritt \textit{doch ja} in nicht weiter sprachlich markierten V2-Deklarativsätzen auf, die plausibel als Begründungen des vorweggehenden Sprech\-aktes interpretiert werden können. In (\ref{512}), (\ref{515}) und (\ref{522}) ist dies ein Direktiv, in (\ref{513}) und (\ref{514}) gehen jeweils Fragen vorweg. Als Sprechaktbegründungen \is{illokutionärer Kausalsatz} lassen sich auch der \textit{da}- und \textit{denn}-Satz in (\ref{517}) und (\ref{518}) einstufen. In (\ref{517}) wird dann der wiedergegebene \is{Vorwurf} Vorwurf, in (\ref{518}) ebenfalls ein Direktiv \is{Direktiv} motiviert. In (\ref{516}) tritt ein V1-Satz auf, der die vorangehende Annahme (markiert durch \textit{vielleicht}) begründet. In (\ref{519}) liegt mit dem Modalverb \textit{werden} eine epistemische Modalisierung \is{epistemische Modalisierung} vor, während die \textit{doch ja}-Äußerungen in (\ref{520}) und (\ref{521}) als lexikalisch als solche markierten Bewertungen gelesen werden können. 

Ich habe nicht den Eindruck, dass hier jeweils klar das betonte \textit{doch} vorliegt. In (\ref{523}) habe ich einige weitere Werke gelistet, in denen ich \textit{doch ja}-Belege gefunden habe.
								                        
\begin{exe}
	\ex\label{523}
	\tiny
     \begin{tabular}[t]{|p{5em}|p{18em}|p{5em}|}
     		\hline
     		Jahr & Quelle & Kontext\\
            \hline
            1545 $[$1888$]$ & Martin Luther: Wider das Papsttum zu Rom, vom Teufel gestiftet & Bewertung\\
            \hline
            1582 $[$1845$]$ & Anonym: Das Ambraser Liederbuch vom Jahre 1582. $[$Der mond der scheint so helle$]$  & 							illokutionär kausal\\
            \hline
            1658 $[$1996$]$ & Andreas Gryphius: Absurda Comica Oder Herr Peter Squentz – Kapitel 6  & Bewertung\\
            \hline
            1789 (eigentlich: 300–700 n. Chr.) & Hermes Trismegistos (dt.): Das andere Buch, Hermetis: Das Gemüt am Hermes & 					illokutionär kausal\\
            \hline
            1817 $[$1984$]$ & E.T.A. Hoffmann: Das Gelübde – Kapitel 1 & illokutionär kausal\\
            \hline
            1838 $[$1898$]$ & Jeremias Gotthelf: Leiden und Freuden eines Schulmeisters & Bewertung\\
            \hline
            1839 $[$k.A.$]$ & Ludwig I. von Bayern: Gedichte – Kapitel 125 & illokutionär kausal\\
            \hline
            1847 $[$1991$]$ & Franz Grillparzer: Libussa – Kapitel 4  & epistemisch kausal\\
            \hline
            19. Jhd. (1810–1976) $[$k.A.$]$ & Ferdinand Freiligrath: Ein Glaubensbekenntnis & illokutionär kausal\\
            \hline
            1909 $[$1985$]$ & Robert Walser: Jakob von Gunten & illokutionär kausal\\             
            \hline
      \end{tabular}\\
\end{exe}
Auch hier ist Vorsicht geboten, da ich nicht die Erstausgaben eingesehen habe, sondern die in den genannten Korpora eingespeisten Ausgaben (soweit bekannt in (\ref{512}) bis (\ref{522}) und (\ref{523}) in Klammern hinzugefügt). Inwiefern hier ggf. Text geändert wurde, kann ich nicht kontrollieren. Für diese wenigen alten Belege gilt, dass selten tatsächlich explizit sprachliches Material beteiligt ist, wie ich es in Abschnitt~\ref{sec:markiert} identifiziert habe. Es verhält sich eher so, dass sich die Interpretation anbietet – aus der Perspektive der Kontexte, in denen ich die Abfolge in aktuelleren Daten finde. Sprachliches Material stellt man in diesen Beispielen erstmalig 1826 fest. 

Ob die umgekehrte Abfolge neu ist, kann ich im Rahmen dieser Arbeit nicht klären. Belegbar ist sie bereits in älteren Quellen. Möglicherweise hat man es auch mit einer Frequenzzunahme zu tun (vgl. auch \citealt[2]{Imo2010} zum Aufdecken \glqq neuer\grqq{} Daten im Sinne von \glqq neu für die traditionellen Grammatiken\grqq{}; vgl. auch \citealt[279]{Freywald2008} zur \textit{dass}-V2-Struktur). Diese Aspekte muss man gesondert untersuchen. 

Neben der historischen Dimension, die ein Forschungsdesiderat darstellt, gibt es auch aus synchroner Perspektive offene empirische Fragen. Insbesondere halte ich für untersuchenswert, inwiefern die Kontexte, in denen ich die umgekehr\-te Abfolge finde, wirklich genuin mit dieser Abfolge verbunden sind. Zu sagen, man findet \textit{doch ja} in diesen Kontexten, ist eine Annahme, die ich für richtig halte. Die sich anschließende schwierige Frage ist, wie man nachweist, dass sie sich im gegenpoligen Kontext, d.h. in Strukturen, in denen der Sprecher es darauf anlegt, den Inhalt zu geteiltem Wissen zu machen, nicht auftreten kann – anders als \textit{ja doch}, das eine weitere Verwendung vorweist (vgl. Abschnitt~\ref{sec:akz} zu einer Untersuchung, die diese Absicht über Akzeptabilitätsurteile verfolgt). Für diesen Äußerungskontext kommt vor allem die V2-Standardassertion in Frage. Da ich nicht glaube (wie vielfach angenommen wird), dass MPn auf periphere Nebensätze beschränkt sind (vgl. hierzu detaillierter Abschnitt~\ref{sec:rs} in Kapitel~\ref{chapter:hue}), stellen auch propositionale \textit{weil}-Sätze einen solchen Testkontext dar.

Man findet \textit{doch ja} nicht in Standardassertionen und man würde generell erwarten, dass diese häufig vorkommen – zumal modalisierte Strukturen im Grunde auch bereits markierte Assertionen sind. Meine bisherigen Untersuchungen können noch nicht ausschließen, dass das auffällige Auftreten in diesen drei Kontexten nicht durch den Datentyp bzw. die Kombination dieser beiden MPn bedingt ist. Betrachtet man Beispiele aus der Literatur, sucht Belege oder konstruiert neue Beispiele, kann \textit{ja doch} natürlich prinzipiell in mehr Kontexten als den subjektivierten auftreten. In diesen erscheint \textit{doch ja} in konstruierten Beispielen merkwürdig, was entsprechend zur Annahme seines Ausschlusses in der Lite\-ratur geführt hat. Ich habe nun allerdings Verwendungsweisen untersucht. Und aus der Annahme, dass \textit{ja doch} prinzipiell auch in non-modalisierten Kontexten gebraucht werden \underline{kann}, ist nicht zu folgern, dass es auch so verwendet \underline{wird}. Wahrscheinlich tritt auch diese Abfolge gerne in modalisierten Kontexten auf. Es ist verschiedentlich beobachtet worden, dass sich Modalisierungen auch anderweitig modale Umgebungen suchen (vgl. z.B. \citealt[26]{Albrecht1977}, \citealt[26]{Aijmer1997}, \citealt[278-279]{Bluehdorn2006}). Die modalisierten Kontexte sind wiederum ggf. auch an die Textsorte gebunden, in denen man überhaupt MPn findet oder möglicherweise auch genau diese beiden. Ich glaube, dies ist ein Aspekt, der oft nicht berücksichtigt wird, wenn Behauptungen über das Auftreten bestimmter Phänomene in bestimmten Domänen gemacht werden (je nach Phänomen/Domä\-ne und Datensatz ist dieser Aspekt auch mit überschaubarem Aufwand zu kontrollieren, vgl. Abschnitt~\ref{sec:gebrauchheeh}). Die Kombinationen aus \textit{ja} und \textit{doch} betreffend, halte ich es aber für sehr schwierig, diesen Aspekt aufzufangen. Um festzustellen, ob \textit{ja doch} nicht auch bevorzugt (vielleicht gar ausschließlich) in den drei Kontexten auftritt, müsste man eine sehr große Menge von \textit{ja doch}s durchsehen, weil diese Abfolge sehr viel häufiger in den Korpora vertreten ist als \textit{doch ja}. Stellt man hier einen Unterschied fest, ist immer noch nicht geklärt, ob das Auftreten in diesen Kontexten nicht dadurch bedingt ist, dass sie sehr häufig in diesen Textsorten zu finden sind. Dies herauszufinden halte ich für fast nicht praktizierbar. Stellt man keinen Unterschied fest, ist die Nicht-Existenz der \textit{doch ja}-Abfolge dennoch nicht nachgewiesen. Auf der Basis meines Vorgehens, alle auffindbaren \textit{doch ja}-Treffer in verschiedenen Korpora zu sammeln, ist es zudem im Grunde nicht möglich, aus diesen Korpora auch alle \textit{ja doch}s zu untersuchen. Stichproben zu untersuchen gestaltet sich auch schwierig, wenn mehr als ein Korpus die Belege für \textit{doch ja} liefert. Man bräuchte eine finite Menge von Daten, in denen man nach \textit{ja doch} und \textit{doch ja} suchen kann, und in denen ei\-nerseits genug \textit{doch ja}s auftreten und man es andererseits mit einer durchsehbaren Menge von \textit{ja doch}s zu tun hat. Diese Menge ist aber schwierig zu bekommen, da man die relevante Menge von \textit{doch ja}s erst erreicht, wenn man sich sehr große Datenmengen anschaut.

Man könnte die Untersuchung zu den beiden MP-Abfolgen folglich definitiv noch ausbauen. Es gilt bei diesem Vorhaben dann, die schwierige Frage zu beantworten, ob die Prominenz der drei Kontexte von a) der Textsorte (Sind die drei Kontexte nicht sowieso typisch für \is{konzeptionelle Mündlichkeit} konzeptionelle Mündlichkeit?) und b) dem Auftreten von \textit{ja} und \textit{doch} (Sind dies nicht auch beliebte Kontexte für \textit{ja doch}?) gelöst werden kann.

Weitere Evidenz könnte auch die Untersuchung der Partikel-Kombinationen mit Distanzstellung erbringen (vgl. z.B. (\ref{524}) und (\ref{525})).
	
\begin{exe}
	\ex\label{524} 
	Aber das \textit{\textbf{wird}} \underline{\textbf{doch}} beim Internisten \underline{\textbf{ja}} gar nicht lange dauern 			\textbf{\textit{oder?}}
	\newline
	\hbox{}\hfill\hbox{(DECOW2012-06X: 685575138)}
\end{exe}		

\begin{exe}
	\ex\label{525} 
	wenn das klappt \textbf{\textit{müsste}} sich das \underline{\textbf{doch}} im prinzip \underline{\textbf{ja}} auf jedes 			model von u1 bis mw ausweiten lassen, \textbf{\textit{oder?}}                                                 
	\hfill\hbox{(DECOW2012-06X: 999332683)}
\end{exe}		
Es wird manchmal angeführt, dass bei Distanzstellung \is{Distanzstellung} Kombinationen akzep\-tabler werden, die es bei Kontaktstellung \is{Kontaktstellung} nicht sind (vgl. \citealt[219]{Dahl1988}). Es wäre eine Untersuchung wert, zu schauen, ob sich die drei Kontexte auch dann als Domäne herauskristallisieren, wenn die Abfolge auf Distanz ohnehin weniger Restriktionen unterliegt. Die beiden oben aufgeworfenen Fragen stehen dann aber immer noch im Raum.

Mit der bisher geleisteten Empirie kann ich noch nicht alle diese Fragen beantworten. Die Ausführungen stellen den derzeitigen Stand meiner Untersuchungen dar.

Der nächste Abschnitt stellt eine Akzeptabilitätsstudie\footnote{Ich danke Thomas Weskott für seinen Rat bei der Planung des Experiments sowie Sandra Pappert für ihre Hilfe bei der statistischen Auswertung.} vor, die mit der Absicht durchgeführt wurde, der obigen Frage nach dem ausgeschlossenen Kontext der \textit{doch ja}-Abfolge nachzugehen.

\subsection{Akzeptabilitätsurteile}
\label{sec:akz}
Getestet wurden ein \glq besserer\grq {} und ein \glq schlechterer\grq {} Kontext, d.h. in einem Kontext sollte sich die Abfolge schlechter umkehren lassen als im anderen. 70 deutsche Muttersprachler haben Urteile zu \textit{ja doch}- und \textit{doch ja}-Sätzen auf einer 5er Skala abgegeben. Der \glq schlechte\grq {} Kontext wurde durch eine Standardassertion \is{Standardassertion} repräsentiert, d.h. völlig unmarkierte $[$-w$]$, V2-Sätze. Insbesondere liegt keine Modalisierung vor und der Kontext schließt auch kausale Interpretationen aus. Epistemische oder illokutionäre Begründungen \is{epistemischer Kausalsatz} \is{illokutionärer Kausalsatz} bieten sich nicht an. Die Standardassertion zeichnet sich gerade dadurch aus, dass keine besonderen Markierungen vorliegen. Die Testitems sind so konstruiert, dass die MP-Äußerung auf eine Implikatur \is{Implikatur} aus dem Vorgängerkontext reagiert, d.h. es liegt ein typischer Kontext für eine \textit{doch}-Äußerung vor. Die Partikel \textit{ja} tritt hier in der Akkommoda\-tions-Verwendung \is{Akkommodation} auf. (\ref{528}) und (\ref{529}) zeigen zwei Testitems (mit den Implikaturen in (\ref{530}) und (\ref{531})).

\begin{exe}
	\ex\label{528} 
	\begin{tabular}[t]{p{2em} p{25em}}
	\multicolumn{2}{l}{D1 Ein wichtiges Spiel}\\
    Olli: & Am Samstag steigt das wichtigste Spiel der Saison. Der Kader ist zum Glück komplett. Alexander spielt im Sturm.\\
	Stefan: & \underline{Er ist ja doch noch verletzt}.\\
	& \underline{Er ist doch ja noch verletzt.} \\		
    \end{tabular}
\end{exe}

\begin{exe}
	\ex\label{529} 
	\begin{tabular}[t]{p{3em} p{25em}}
	\multicolumn{2}{l}{D8 Kochen}\\
    Marina: & Die Tomate dort auf dem Tisch kannst du auch noch waschen, schneiden und dann in den Salat geben.\\
	Sophie: & \underline{Sie ist ja doch ganz verfault.}\\
	& \underline{Sie ist doch ja ganz verfault.}\\	
    \end{tabular}
\end{exe}
	
\begin{exe}
	\ex\label{530}
	Alexander spielt. $>$ Er ist nicht verletzt.\\
	\glq Wenn man spielt, ist man normalerweise nicht verletzt.\grq {}
\end{exe}		
		
\begin{exe}
	\ex\label{531}
	Tomate zum Essen geben. $>$ Die Tomate ist nicht verfault.\\
	\glq Wenn eine Tomate mit in den Salat soll, ist sie normalerweise nicht verfault.\grq {}
\end{exe}
Das Material bestand aus acht Lexikalisierungen, wobei die Testsätze dem Muster in (\ref{532}) entsprechen.

\begin{exe}
	\ex\label{532}
	Pronomen $\plus$ ist $\plus$ \textit{ja doch}/\textit{doch ja} $\plus$ schwaches Element (\textit{noch}/\textit{ganz}) + 			Partizip Perfekt (zweite Silbe akzentuiert)
\end{exe}
Hierbei handelt es sich meiner Argumentation nach um einen \glq schlechten\grq {} Kontext, da eine normale Assertion vorliegt. Vor dem Hintergrund der Charakterisie\-rung nach \citet{Farkas2010} stellt sich die Situation ein, dass der Sprecher p mitzuteilen beabsichtigt und den Diskurspartner von dieser Information über\-zeugen möchte. In diesem Kontext sollte die Abfolge der beiden Partikeln folglich nicht (bzw. schlechter als an anderer Stelle) umkehrbar sein. Der \glq gute\grq {} Kontext wird durch illokutionär interpretierte Kausalsätze \is{illokutionärer Kausalsatz} dargestellt, d.h. durch Kausalsätze, die einen Sprechakt \is{Sprechakt} begründen: Der Sprecher nennt das Motiv, aufgrund dessen er den Sprechakt äußert. Für die möglichst eindeutige Modellierung derart interpretierter Kausalsätze habe ich mir die Erkenntnis aus der Literatur zu Nutze gemacht, dass mit manchen Konnektoren bestimmte kausale Lesarten einhergehen. Mit ihnen können nur Interpretationen auf manchen der drei Ebenen, auf denen kausale Zusammenhänge prinzipiell möglich sind, kodiert werden. Entscheidend ist hier, dass angenommen wurde, dass \textit{denn} keine propositionale Verwendung hat, d.h. es wirkt nicht auf der Sachverhaltsebene (vgl. \citealt[320]{Volodina2010}), bzw. – selbst wenn der Sachverhaltsbezug nicht generell auszuschließen ist (vgl. \citealt[141]{Schmidhauser1995}) – bevorzugt es zumindest die \is{epistemischer Kausalsatz} \is{illokutionärer Kausalsatz} epistemische/illokutionäre Lesart (vgl. \citealt[270]{Bluehdorn2006}; \citeyear[29]{Bluehdorn2008}). Wenn zusätzlich ein non-assertiver Sprechakt vorangeht, liegt sehr deutlich die illokutionäre Lesart vor.

(\ref{533}) und (\ref{534}) zeigen auch für diesen Kontext je ein Item.

\begin{exe}
	\ex\label{533} 
	\begin{tabular}[t]{p{3em} p{22em}}
	\multicolumn{2}{l}{C1 Go-Go-Tänzerin}\\
    Janina:	& Mein Chef hat herausbekommen, dass ich im Club frei\-tags abends als Go-Go-Tänzerin arbeite. Er hält das für unseriös und droht mit der 			Kündigung, wenn ich diese Tätigkeit nicht aufgebe. Jetzt habe ich aber schon zugesagt, am Wochenende auf der Bühne zu stehen. Was soll ich jetzt nur 		machen?\\
	Kristina: & Na, hör mal. Sag den Auftritt ab!\\
	& \underline{Denn du riskierst ja doch sonst deinen Job.}\\
	& \underline{Denn du riskierst doch ja sonst deinen Job.}
    \end{tabular}
\end{exe}

\begin{exe}
	\ex\label{534} 
	\begin{tabular}[t]{p{2em} p{25em}}
	\multicolumn{2}{l}{C5 Ansehen}\\
    Nina: & Erst gestern habe ich erfahren, dass Erik Gruber im Ort als jemand gilt, der jede Frau abschleppt und anschließend 			sofort wieder fallen lässt. Von dem sollte man sich also fernhalten, wenn man nicht ins Gerede kommen möchte. Nun habe ich 			aber nichts ahnend zugestimmt, als er mich für Freitag zum Cocktailtrinken eingeladen hat. Was soll ich jetzt nur machen?\\
	Ida: & Na, hör mal. Sag das Date ab!\\
	& \underline{Denn du ruinierst ja doch sonst deinen Ruf.}\\
	& \underline{Denn du ruinierst doch ja sonst deinen Ruf.}
    \end{tabular}
\end{exe}
Inhaltlich folgen die Lexikalisierungen für diesen Kontext dem Muster, dass der erste Sprecher ein Problem mitteilt und den Gesprächspartner fragt, was er tun soll. Der zweite Sprecher rät von der geplanten problematischen Handlung ab und begründet, warum er dies rät. Strukturell weisen die zu bewertenden Sätze (bzw. ihr unmittelbarer Vorkontext) die Form in (\ref{535}) auf.

\begin{exe}
	\ex\label{535} 
	(Was soll ich jetzt nur machen? – Na, hör mal. Sag X ab!\\
	\textit{Denn} $\plus$ \textit{du} $\plus$ finites Verb auf $\lbrace$-ierst$\rbrace$ (letzte Silbe akzentuiert) \textit{ja doch}/\textit{doch ja} $\plus$ \textit{sonst} $\plus$ \textit{deinen} $\plus$ Nomen (einsilbig)
\end{exe}
Da die beiden getesteten Kontexte sich gerade interpretatorisch und sprachlich unterscheiden müssen, ist die Bildung von ,echten\grq {} lexikalischen Sets unmöglich. Die Testzusammensetzung einer jeden Version ist beschaffen wie in (\ref{536}).

\begin{exe}
	\ex\label{536} Items pro Testant\\[-0.6em]
     \begin{tabular}[t]{|l|l|l|}
     		\hline
     		 \diagbox{Satzkontext:}{Partikelfolge:} &
     		 \textit{ja doch} & \textit{doch ja}\\
            \hline
            Standardassertion & 4 & 4\\
            \hline
            Sprechaktbegründung & 4 & 4\\
            \hline
      \end{tabular}\\
\end{exe}
Jeder Testant hat vier verschiedene Sätze in jeder der vier Bedingungen gesehen, d.h. insgesamt 16 Testitems. Dazu kamen 32 andere Dialoge. Unter den Fillern fanden sich ganz verschiedene Phänomene (aus den Bereichen Wortstellung, Nebensätze, Korrelate, Extraktionsstrukturen, Satzmodus, Flexion), für die zudem Bewertungen von \glq hart schlecht\grq {}, \glq schlecht\grq {} , \glq leicht schlecht\grq {}  und \glq gut\grq {} anzunehmen sind. Auf ähnliche Art (an die Darbietung der Testitems und Aufgabenstellung/Art der Bewertung angepasst) waren sie Teil der in \citet{Mueller2012} beschriebenen Studie zu w-Fragen mit imperativischer Verbmorphologie. Einige Sätze sind auch unter Bezug auf ein sechsstufiges Ranking, das in \citet{Featherston2009} getestet wurde, konstruiert. Ich gehe deshalb davon aus, dass die Filler nicht nur bestimmte Stufen der Skala abdecken.

(\ref{537}) zeigt die arithmetischen Mittel für alle Bedingungen über alle 70 Teilnehmer der Studie. Die Urteile erfolgten auf einer 5er Skala, wobei \glq 1\grq {} das untere Ende der Skala darstellte (\glq völlig inakzeptabel\grq {} ) und \glq 5\grq {}  das obere Ende (\glq völlig akzeptabel\grq {}).

\begin{exe}
	\ex\label{537} Arithmetisches Mittel\\[-0.6em]
     \begin{tabular}[t]{|l|l|l|l|}
     		\hline
     		70 Teilnehmer & \textit{ja doch} & \textit{doch ja} & Unterschied\\
            \hline
            Standardassertion & 2,44 & 1,86 & $-$0,58\\
            \hline
            Sprechaktbegründung & 2,51 & 2,28 & $-$0,23\\
            \hline
            Unterschied & $+$0,07 & $+$0,42 & \\
            \hline
      \end{tabular}\\
\end{exe}
Im schlechten Kontext beträgt der Unterschied zwischen \textit{ja doch} und \textit{doch ja} 0,58, im guten Kontext liegt er bei 0,23.

Wenngleich die Sätze insgesamt recht niedrige Urteile erhalten haben (dazu s.u.), zeigen die Bewertungen dennoch, dass der Unterschied zwischen der Abfolge \textit{ja doch} und \textit{doch ja} in der Sprechaktbegründung \is{illokutionärer Kausalsatz} kleiner ist als in der \is{Standardassertion} Standardassertion. Sprecher können die Abfolge im gutem Kontext scheinbar etwas besser umdrehen als im schlechten. Darüber hinaus liegt ein sehr kleiner Unterschied von 0,07 zwischen den \textit{ja doch}-Auftreten in den zwei Kontexten vor und ein Unterschied von 0,42 zwischen den \textit{doch ja}-Äußerungen in den beiden Kontexten. Sprecher akzeptieren \textit{doch ja} also scheinbar etwas leichter im guten als im schlechten Kontext.

Eine zweifaktorielle Varianzanalyse mit Messwiederholung ergab sowohl für die by-subject- als auch die by-item-Analyse einen Effekt von \glqq Kontext\grqq{} (F$_{1}$(1,69) = 10,24, p $<$ 0,01, $\eta_{\textrm{p}}^{2}$ = 0,13; F$_{2}$(1,14) = 21,28, p $<$ 0,001, $\eta_{\textrm{p}}^{2}$ = 0,60), \glqq Abfolge\grqq{} (F$_{1}$(1,69) = 44,04, p $<$ 0,001, $\eta_{\textrm{p}}^{2}$ = 0,39; F$_{2}$(1,14) = 64,33, p $<$ 0,001, $\eta_{\textrm{p}}^{2}$ = 0,82) und der Interaktion von \glqq Kontext\grqq{} und \glqq Abfolge\grqq{} (F$_{1}$(1,69) = 7,71, p $<$ 0,01, $\eta_{\textrm{p}}^{2}$ = 0,10; F$_{2}$(1,14) = 11,39, p $<$ 0,01, 
$\eta_{\textrm{p}}^{2}$ = 0,45).

Interessant für meine Argumentation ist insbesondere die Feststellung, dass die Interaktion der beiden Faktoren einen Effekt auf die Bewertung nimmt. Aufgrund dieser signifikanten Interaktion wurde ein Scheff\'{e}-Test (p $<$ 0,05) als Post-hoc-Test kalkuliert. 

Der Test gibt aus, dass \textit{ja doch} in Standardassertionen signifikant besser be\-wertet wird als \textit{doch ja} in Standardassertionen und dass \textit{ja doch} in illokutionär interpretierten Kausalsätzen höhere Bewertungen erhält als \textit{doch ja} in illokutionären Kausalsätzen. Dies sind die Verhältnisse, wie sie zu erwarten sind, wenn zwi\-schen \textit{ja doch} und \textit{doch ja} ein Markiertheitsunterschied \is{Markiertheit} besteht. Der für meine Hypothese wichtige Aspekt ist, dass ebenfalls \textit{doch ja} in Sprechaktbegründungen \is{illokutionärer Kausalsatz} besser bewertet wird als in \is{Standardassertion} Standardassertionen, während sich der Unterschied zwischen \textit{ja doch} in den beiden Kontexten nicht als signifikant herausstellt. Die Ergebnisse der Studie zeigen auf, dass die Unterschiede subtil sind\footnote{M.E. lehren die feinen Unterschiede auch, dass sich zwischen der unmarkierten und markierten Partikelabfolge keine sehr stark voneinander abweichenden Urteile einstellen, die zu der Annahme veranlassen müssten, dass die umgekehrte Abfolge im Gegensatz zur üblicheren Ordnung gar als ungrammatisch angesehen werden muss. Wie der Blick auf andere Ansätze in Abschnitt~\ref{sec:abfolgejd} gezeigt hat, ist dies aber eine Ansicht, die den Arbeiten zu Partikelabfolgen in der Regel zugrunde liegt. Letztlich scheint mir aber auch hier Vorsicht geboten, weil Unterschiede auch je nach Methode ggf. unterschiedlich deutlich erscheinen können. D.h. ein Unterschied, der auf der 5-Punkt-Skala klein aussieht, kann in Paarvergleichen ggf. nach einem deutlicheren Kontrast aussehen.} und \glq Abfolge\grq {}  klar ein dominanter Faktor ist. Doch da die umgekehrte Abfolge nur gefunden werden kann, wenn sehr große Datenmengen betrachtet werden, ist sicherlich auch nicht mit großen Effekten zu rechnen. Dies gilt insbesondere dann, wenn die Urteile zusätzlich Interpretationen im Kontext involvieren, die eine prinzi\-pielle Unsicherheit beisteuern. Nichtsdestotrotz weisen die statistischen Auswertungen darauf hin, dass man nicht annehmen kann, dass \glq Abfolge\grq {}  der einzige relevante Faktor ist. \glq Abfolge abhängig von Kontext\grq {} ist ebenfalls relevant für den Unterschied zwischen den Mittelwerten. Die Ergebnisse sind folglich konform mit meiner Hypothese: Es macht einen Unterschied, ob/wie gut sich die Abfolge umkehren lässt, abhängig vom Kontext. Der hier betrachtete Kontext sind Sprechaktbegründungen. Dies ist einer der Kontexte, in dem ich die Abfolge \textit{doch ja} (vornehmlich) in den Webdaten gefunden habe. Ich sehe das Ergebnis dieser Akzeptabilitätsstudie deshalb als Stützung meiner Annahme, dass es sich hierbei um einen Kontext handelt, in dem die Abfolge einfacher umgekehrt werden kann als in \is{prototypische Assertion} prototypischen Assertionen, die in den \textit{doch ja}-Treffern aus den Korpusdaten nicht belegt sind. Beide Untersuchungen weisen folglich in die gleiche Richtung. Die Frage nach der Qualität der Webdaten möchte ich in Bezug auf dieses konkrete Phänomen deshalb so beantworten, dass die Webdaten eine genauso gute oder schlechte Quelle für linguistische Analysen sind wie die Akzeptabilitätsbewertungen. 

Als Grund für die prinzipiell recht niedrigen Bewertungen kommen m.E. verschiedene Aspekte in Frage, die nicht alle mit dem Phänomen an sich zusammenhängen. Sprecher nutzen z.B. selten die gesamte Spanne der Skala aus. Wenige Sprecher verwenden 1 und wenige 5. Die Skala wird in der Realität folglich bei vielen Sprechern zu einer 3er Skala. Ein weiterer Aspekt ist, dass unter den Filler-Sätzen auch absolut wohlgeformte deutsche Sätze waren. Wenn Sprecher diesen lediglich eine Bewertung von 2 oder 3 zuordnen, erhalten MP-Äußerungen natürlich niedrigere Urteile. Was man auch nicht unterschätzen sollte, ist, dass MP-Kombinationen gar nicht so häufig auftreten, wie man denken mag. Es würde aus dieser Sicht nicht verwundern, wenn derartige Äußerungen für Sprecher \\ deshalb schon generell einen gewissen Markiertheitsgrad \is{Markiertheit} aufwiesen. Und selbst wenn die Bewertungen insgesamt etwas höher ausfielen, würde die Subtilität der Unterschiede dennoch bestehen bleiben. An dem grundsätzlich sicherlich kleinen Unterschied würde sich auch nichts ändern, wenn die Ergebnisse im 3er- oder 4er-Bereich der Skala lägen. Wie ich schon angeführt habe, ist mit größeren Unterschieden nicht zu rechnen. Vielleicht ist es deshalb auch nur die impressionistische Sicht, die sich über die recht niedrigen Bewertungen wundert. Statistisch ergeben sich schließlich nur an manchen (und zwar an durch meine Analyse vorhergesehenen) Stellen Effekte.\\

\noindent
Mit diesen Ausführungen zum Status der Abfolge \textit{doch ja} schließe ich die Betrachtung von \textit{ja}, \textit{doch} sowie ihren Kombinationen. Der nächs\-te Teil der Arbeit beschäftigt sich mit zwei anderen MPn: \textit{halt} und \textit{eben}. Zum einen stellen sich für ihr kombiniertes Auftreten gleiche Fragen, wie sie in Bezug auf \textit{ja doch} und \textit{doch ja} aufgeworfen wurden, zum anderen gibt es aber auch spezielle Themen, die der Betrachtung dieser beiden Partikeln eigen sind.

