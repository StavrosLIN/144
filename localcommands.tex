%add all your local new commands to this file

\newcommand{\smiley}{:)}

\renewbibmacro*{index:name}[5]{%
  \usebibmacro{index:entry}{#1}
    {\iffieldundef{usera}{}{\thefield{usera}\actualoperator}\mkbibindexname{#2}{#3}{#4}{#5}}}


\renewcommand{\partname}{Unit}
	
\newcommand{\appref}[1]{Appendix \ref{#1}}
\newcommand{\fnref}[1]{Footnote \ref{#1}}
\newcommand{\vernacular}[1]{\emph{#1}}
\newcommand{\gloss}[1]{#1}

\newenvironment{stylepoints}{\ea}{\z}
\newcommand{\furtherreading}[1]{\tblssy{book}{Further reading}{#1}}

\newcommand{\discussionexercises}[1]{\setcounter{equation}{0}
\tblssy{people}{Discussion exercises}{\vspace*{-7mm}#1}}

\newcommand{\homeworkexercises}[1]{\setcounter{equation}{0}
\tblssy{pencil}{Homework exercises}{\vspace*{-5mm}#1}}

\newcommand{\modelanswer}[2]{\fbox{\parbox{.95\textwidth}{\small\sffamily #1
\vspace*{2mm}

\parbox{.88\textwidth}{#2}

\rule{0mm}{0pt} \vspace*{3mm}~}}
}

\newcommand{\tablehead}[1]{#1}
\newcommand{\textstylest}[1]{\textsc{#1}}
\newcommand{\biberror}[1]{{\color{red}#1}}

\renewcommand{\emptyset}{⌀}

% H-Rule of Inference ("Schlussstrich")
\newcommand*{\WernersHRule}{% Slightly Modified
	\par\kern\dimexpr.7\itemsep-\parskip-.6\baselineskip\relax%
	\hrulefill%
	\par\kern\dimexpr.3\itemsep-.3\parskip-.6\baselineskip\relax%
}%

\newcommand*{\FelixHRule}{% Slightly Modified
	\par\kern\dimexpr.7\itemsep-\parskip-.6\baselineskip\relax%
	\hrulefill%
	\par\kern\dimexpr.3\itemsep-.2\baselineskip\relax%
}%


\definecolor{RED}{cmyk}{0.05,1,0.8,0}

\newfontfamily\cn[Mapping=tex-text,Ligatures=Common,Scale=MatchUppercase]{AR PL UMing CN} %Chinese
\newcommand{\zh}[1]{{\cn #1}}
\XeTeXlinebreaklocale 'zh'  
\XeTeXlinebreakskip = 0pt plus 1pt


\newcommand{\kroegertriple}[3]{\noindent%
#1 %
\parbox{6cm}{#2}%
\parbox{6cm}{#3}\\\medskip
}