\addchap{Preface}
\begin{refsection}

This book provides an introduction to the study of meaning in human language, from a linguistic perspective. It covers a fairly broad range of topics, including lexical semantics, compositional semantics, and pragmatics. The approach is largely descriptive and non-formal, although some basic logical notation is introduced.


The book is written at level which should be appropriate for advanced undergraduate or beginning graduate students. It presupposes some previous coursework in linguistics, including at least a full semester of morpho-syntax and some familiarity with phonological concepts and terminology. It does not presuppose any previous background in formal logic or set theory.



Semantics and pragmatics are both enormous fields, and an introduction to either can easily fill an entire semester (and typically does); so it is no easy matter to give a reasonable introduction to both fields in a single course. However, I believe there are good reasons to teach them together.



In order to cover such a broad range of topics in relatively little space, I have not been able to provide as much depth as I would have liked in any of them. As the title indicates, this book is truly an \textsc{introduction}: it attempts to provide students with a solid foundation which will prepare them to take more advanced and specialized courses in semantics and/or pragmatics. It is also intended as a reference for fieldworkers doing primary research on under-documented languages, to help them write grammatical descriptions that deal carefully and clearly with semantic issues. (This has been a weak point in many descriptive grammars.) At several points I have also pointed out the relevance of the material being discussed to practical applications such as translation and lexicography, but due to limitations of space this is not a major focus of attention.



The book is organized into six Units: 2: (1) Foundational concepts; (2) Word meanings; (3) Implicature (including indirect speech acts); (4) Compositional semantics; (5) Modals, conditionals, and causation; (6) Tense \& aspect. The sequence of chapters is important; in general, each chapter draws fairly heavily on preceding chapters. The book is intended to be teachable in a typical one-semester course module. However, if the instructor needs to reduce the amount of material to be covered, it would be possible to skip chapters 6 (Lexical sense relations), 15 (Intensional contexts), 17 (Evidentiality), and/or 22 (Varieties of the Perfect) without seriously affecting the students’ comprehension of the other chapters. Alternatively, one might skip the entire last section, on tense \& aspect.



Most of the chapters (after the first) include exercises which are labeled as being for “Discussion” or “Homework”, depending on how I have used them in my own teaching. (Of course other instructors are free to use them in any way that seems best to them.) A few chapters have only “Discussion exercises”, and two (ch. 15 and 17) have no exercises at all in the current version of the book. Additional exercises for many of the topics covered here can be found in \citet{Saeed2009} and \citet{Kearns2000}.



Suggestions which will help to improve any aspect of the book will be most welcome. \textit{Soli Deo Gloria}.

\printbibliography[heading=subbibliography]
\end{refsection}

