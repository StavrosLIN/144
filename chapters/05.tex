\chapter{Word senses}\label{sec:5}

\section{Introduction}\label{sec:5.1}

In \chapref{sec:2} we introduced the important distinction between sense and denotation. We noted that a single word may have more than one sense, a situation referred to as \textsc{lexical ambiguity}. We also noted that two expressions which have different senses may have the same denotation in some particular context, but two expressions which have the same sense must have the same denotation in every imaginable context. So what if a single word can be used to refer to several different kinds of things? Does that mean it has several different senses? The answer is, sometimes yes and sometimes no. This chapter is designed to help you answer this kind of question for specific cases.



We begin in \sectref{sec:5.2} with the observation that a speaker often has a variety of ways to refer to a particular thing. The speaker’s choice of words reflect different \textsc{construals}, or ways of thinking about the thing. In \sectref{sec:5.3} we discuss several diagnostic tests that can be used to distinguish true lexical ambiguity from other similar patterns, such as vagueness and underspecification. We then distinguish two different types of lexical ambiguity, \textsc{polysemy} vs. \textsc{homonymy}, recognizing that making this distinction is not always easy; and we discuss the role of context in enabling hearers to choose the intended sense of ambiguous word forms.



In \sectref{sec:5.4} we discuss some ways in which new senses of words can be created, including \textsc{coercion} and figures of speech. In \sectref{sec:5.5} we apply the principles developed in \sectref{sec:5.3} to a certain pattern of variable denotation, illustrated by words like \textit{book} which can be used to name either a physical object or the text or discourse that it contains.


\section{Word meanings as construals of external reality}\label{sec:5.2}

Words give us a way to describe the world. However, our linguistic descriptions are never complete. In choosing a word to describe a particular thing or event, we choose to express certain bits of information and leave many others unexpressed. For example, suppose that I am holding a rag in my right hand and moving it back and forth across the surface of a table. If you ask me what I am doing, I might reply with either (\ref{ex:5.1}a) or (\ref{ex:5.1}b).


\ea \label{ex:5.1}
\ea I am wiping the table.\\
\ex I am cleaning the table.\\
\ex I wiped/??cleaned the table but it is no cleaner than before.\\
\ex I cleaned/\#wiped the table without touching it.
                       \z
\z


In this situation, both (\ref{ex:5.1}a) and (\ref{ex:5.1}b) would be true descriptions of the event, but they do not mean the same thing. By choosing the word \textit{clean}, I would be specifying a certain change in the state of the table, but leaving the manner unspecified. By choosing the word \textit{wipe}, I would be specifying a certain manner, but not asserting anything about a change of state. The different entailments associated with these two verbs can be demonstrated using examples like (\ref{ex:5.1}c--d).



To take a second example, suppose that you have a large quartz crystal on your desk, which you use as a paperweight. If I want to look more closely at this object, I could ask for it by saying: \textit{May I look at your paperweight?}; or by saying: \textit{May I look at that quartz crystal?} Clearly the words \textit{paperweight} and \textit{quartz crystal} do not mean the same thing; but in this context, they can have the same referent. The lexical meaning of each word includes features which are true of this referent, but neither word encodes all of the properties of the referent. The choice of which word to use reflects the speaker’s \textsc{construal} of (or way of thinking about) the object, and commits the speaker to certain beliefs but not others concerning the nature of the object.



In analyzing word meanings, we are trying to account for linguistically coded information, rather than all the encyclopedic knowledge (or knowledge about the world) which may be associated with a particular word. For example, the fact that a quartz crystal sinks in water is a fact about the world, but probably not a linguistic property of the word \textit{quartz}. But we need to be aware that this distinction between linguistic knowledge vs. knowledge about the world is often difficult to make.


\section{Lexical ambiguity}\label{sec:5.3}
\subsection{Ambiguity, vagueness, and indeterminacy}\label{sec:5.3.1}

In \chapref{sec:2} we discussed cases of lexical ambiguity like those in \REF{ex:5.2}. These sentences are ambiguous because they contain a word-form which has more than one sense, and as a result can be used to refer to very different kinds of things. For example, we can use the word \textit{case} to refer to a kind of container or to a legal proceeding; \textit{lies} can be a noun referring to false statements or a verb specifying the posture or location of something. These words have a variety of referents because they have multiple senses, i.e., they are ambiguous. And as we noted in \chapref{sec:2}, the truth value of each of these sentences in a particular context will depend on which sense of the ambiguous word is chosen.


\ea \label{ex:5.2}
\ea The farmer allows walkers to cross the field for free, but the bull \textit{charges}.\\
\ex Headline: Drunk gets nine months in violin \textit{case}.\\
\ex Headline: Reagan wins on budget, but more \textit{lies} ahead.
                       \z
\z


However, there are other kinds of variable reference as well, ways in which a word can be used to refer to different sorts of things even though it may have only a single sense. For example, I can use the word \textit{cousin} to refer to a child of my parent’s sibling, but the person referred to may be either male or female. Similarly, the word \textit{kick} means to hit something with one’s foot, but does not specify whether the left or right foot is used.\footnote{\citet{Lakoff1970}.} We will say that the word \textit{cousin} is \textsc{indeterminate} with respect to gender, and that the word \textit{kick} is indeterminate with respect to which foot is used.\footnote{We follow \citet{Kennedy2011} in using the term \textsc{indeterminacy}; as he points out, some other authors have used the term \textsc{generality} instead. \citet{Gillon1990} makes a distinction between the two terms, using \textsc{generality} for superordinate terms.} We will argue that such examples are not instances of lexical ambiguity: neither of these cases requires us to posit two distinct senses for a single word form. Our basis for making this claim will be discussed in \sectref{sec:5.3.2} below.



Another kind of variable reference is observed with words like \textit{tall} or \textit{bald}. How tall does a person have to be to be called “tall”? How much hair can a person lose without being considered “bald”? Context is a factor; a young man who is considered tall among the members of his gymnastics club might not be considered tall if he tries out for a professional basketball team. But even if we restrict our discussion to professional basketball players, there is no specific height (e.g. two meters) above which a player is considered tall and below which he is not considered tall. We say that such words are \textsc{vague}, meaning that the limits of their possible denotations cannot be precisely defined.\footnote{A number of authors (\citealt{Kempson1977}, \citealt{Lakoff1970}, \citealt{Tuggy1993}) have used the term \textsc{vagueness} as a cover term which includes generality or indeterminacy as a sub-type.}



\citet{Kennedy2011} mentions three distinguishing characteristics of vagueness. First, context-dependent truth conditions: we have already seen that a single individual may be truly said to be tall in one context (a gymnastics club) but not tall in another (a professional basketball team). This is not the case with indeterminacy; if a certain person is my cousin in one context, he or she will normally be my cousin in other contexts as well.



Second, vague predicates have borderline cases. Most people would probably agree that a bottle of wine costing two dollars is cheap, while one that costs five hundred dollars is expensive. But what about a bottle that costs fifty dollars? Most people would probably agree that Einstein was a genius, and that certain other individuals are clearly not. But there are extremely bright people about whom we might disagree when asked whether the term \textit{genius} can be applied to them; or we might simply say “I’m not sure”. Such borderline cases do not typically arise with indeterminacy; we do not usually disagree about whether a certain person is or is not our cousin.



\citet{Gillon1990} provides another example:


\begin{quote}
Vagueness is well exemplified by such words as \textit{city}. Though a definite answer does exist as to whether or not it applies to Montreal [1991 population: 1,016,376 within the city limits] or to Kingsville (Ontario) [1991 population: 5,716]; nonetheless, no definite answer exists as to whether or not it applies to Red Deer (Alberta) [1991 population: 58,145] or Moose Jaw (Saskatchewan) [1991 population: 33,593]. Nor is the lack of an answer here due to ignorance (at least if one is familiar with the geography of Western Canada): no amount of knowledge about Red Deer or Moose Jaw will settle whether or not \textit{city} applies. Any case in which further knowledge will settle whether or not the expression applies is simply not a case evincing the expression’s vagueness; rather it evinces the ignorance of its user… Vagueness is not alleviated by the growth of knowledge, ignorance is.
\end{quote}


Third, vague predicates give rise to “little-by-little” paradoxes.\footnote{The technical term is the \textit{sorites} paradox, also known as the paradox of the heap, the fallacy of the beard, the continuum fallacy, etc.} For example, Ringo Starr was clearly not bald in 1964; in fact, the Beatles’ famous haircut was an important part of their image during that era. Now if in 1964 Ringo had allowed you to pluck out one of his hairs as a souvenir, he would still not have been bald. It seems reasonable to assume that a man who is not bald can always lose one hair without becoming bald. But if Ringo had given permission for every person in Europe to pluck out one of his hairs, he would have become bald long before every fan was satisfied. But it would be impossible to say which specific hair it was whose loss caused him to become bald, because \textit{bald} is a vague predicate.



Another property which may distinguish vagueness from indeterminacy is the degree to which these properties are preserved in translation. Indeterminacy tends to be language-specific. There are many interesting and well-known cases where pairs of translation equivalents differ with respect to their degree of specificity. For example, Malay has no exact equivalent for the English words \textit{brother} and \textit{sister}. The language uses three terms for siblings: \textit{abang} ‘older brother’, \textit{kakak} ‘older sister’, and \textit{adek} ‘younger sibling’. The term \textit{adek} is indeterminate with respect to gender, while the English words \textit{brother} and \textit{sister} are indeterminate with respect to relative age.



Mandarin has several different and more specific words corresponding to the English word \textit{uncle}: 伯伯 (bóbo) ‘father’s elder brother’; 叔叔 (sh\=ushu) ‘father’s younger brother’; 姑丈 (g\=uzhàng) ‘father’s sister’s husband’; 舅舅 (jiùjiu) ‘mother’s brother’; 姨丈 (yízhàng) ‘mother’s sister’s husband’.\footnote{\url{http://www.omniglot.com/language/kinship/chinese.htm}}  So the English word \textit{uncle} is indeterminate with respect to various factors that are lexically distinguished in Mandarin.



The English word \textit{carry} is indeterminate with respect to manner, but many other languages use different words for specific ways of carrying. Tzeltal, a Mayan language spoken in the State of Chiapas (Mexico), is reported to have twenty-five words for ‘carry’:\footnote{\url{http://www-01.sil.org/mexico/museo/3di-Carry.htm}} 


\ea
1. \textit{cuch} ‘carry on one’s back’\\
2. \textit{q'uech} ‘carry on one’s shoulder’ \\
3. \textit{pach} ‘carry on one’s head’ \\
4. \textit{cajnuc'tay} ‘carry over one’s shoulder’\\
5. \textit{lats'} ‘carry under one’s arm’\\
6. \textit{chup} ‘carry in one’s pocket’\\
7. \textit{tom} ‘carry in a bundle’\\
8. \textit{pet} ‘carry in one’s arms’\\
9. \textit{nol} ‘carry in one’s palm’\\
10. \textit{jelup'in} ‘carry across one’s shoulder’\\
11. \textit{nop'} ‘carry in one’s fist’\\
12. \textit{lat'} ‘carry on a plate’\\
13. \textit{lip'} ‘carry by the corner’\\
14. \textit{chuy} ‘carry in a bag’\\
15. \textit{lup} ‘carry in a spoon’\\
16. \textit{cats'} ‘carry between one’s teeth’\\
17. \textit{tuch} ‘carry upright’\\
18. \textit{toy} ‘carry holding up high’\\
19. \textit{lic} ‘carry dangling from the hand’\\
20. \textit{bal} ‘carry rolled up (like a map)’\\
21. \textit{ch'et} ‘carry coiled up (like a rope)’\\
22. \textit{chech} ‘carry by both sides’\\
23. \textit{lut'} ‘carry with tongs’\\
24. \textit{yom} ‘carry several things together’\\
25. \textit{pich'} ‘carry by the neck’
\z


In contrast, words which are vague in English tend to have translation equivalents in other languages which are also vague. This is because vagueness is associated with certain semantic classes of words, notably with scalar adjectives like \textit{big}, \textit{tall}, \textit{expensive}, etc. Vagueness is a particularly interesting and challenging problem for semantic analysis, and we will discuss it again in later chapters.


\subsection{Distinguishing ambiguity from vagueness and indeterminacy}\label{sec:5.3.2}

The Spanish word \textit{llave} can be used to refer to things which would be called \textit{key}, \textit{faucet} or \textit{wrench}/\textit{spanner} in English.\footnote{Jonatan Cordova (p.c.) informs me that the word can also be used to mean ‘lock’ in wrestling.} How do we figure out whether \textit{llave} has multiple senses (i.e. is ambiguous), or whether it has a single sense that is vague or indeterminate? A number of linguistic tests have been proposed which can help us to make this decision. 



The most common tests are based on the principle that distinct senses of an ambiguous word are \textsc{antagonistic}.\footnote{\citet[61]{Cruse1986}.} This means that two senses of the word cannot both apply simultaneously. Sentences which seem to require two senses for a single use of a particular word, like those in \REF{ex:5.4}, are called \textsc{puns}. A clash or incompatibility of senses for a single word in sentences containing a co-ordinate structure, like those in \REF{ex:5.5}, is often referred to using the Greek term \textsc{zeugma} (pronounced ['zugmə]). The odd or humorous nature of these kinds of sentences provides evidence that two distinct senses are involved; that is, evidence for a real lexical ambiguity.


\ea \label{ex:5.4}
\ea The hunter went home with five bucks in his pocket.\\
\ex The batteries were given out free of charge.\\
\ex I didn’t like my beard at first. Then it grew on me.\\
\ex When she saw her first strands of gray hair, she thought she’d dye.\\
\ex When the chair in the Philosophy Department became vacant,\\
  the Appointment Committee sat on it for six months.\footnote{\citet[108]{Cruse2000}.}
                       \z
\z

\ea \label{ex:5.5}
\ea Mary and her visa expired on the same day.\footnote{Adapted from \citet[61]{Cruse1986}.}\\
\ex He carried a strobe light and the responsibility for the lives of his men.\footnote{Tim O’Brien, \textit{The Things They Carried}, via grammar.about.com.}\\
\ex On his fishing trip, he caught three trout and a cold.\footnote{\url{http://dictionary.reference.com/browse/zeugma}}
                       \z
\z


Sentence (\ref{ex:5.4}d) illustrates a problem with English spelling, namely that words which are pronounced the same can be spelled differently (\textit{dye} vs. \textit{die}). Because linguistic analysis normally focuses on spoken rather than written language, we consider such word-forms to be ambiguous; we will discuss this issue further in the following section.



Another widely used test for antagonism between two senses is the \textsc{identity test}.\footnote{\citet{Lakoff1970}; \citet{ZwickySadock1975}.} This test makes use of the fact that certain kinds of ellipsis require parallel interpretations for the deleted material and its antecedent. We will illustrate the test first with an instance of structural ambiguity:\footnote{Examples adapted from \citet[512]{Kennedy2011}.}


\ea \label{ex:5.6}
\ea The fish is ready to eat.\\
\ex The fish is ready to eat, and so is the chicken.\\
\ex The fish is ready to eat, but the chicken is not.\\
\ex \#The potatoes are ready to eat, but the children are not.
                       \z
\z


Sentence (\ref{ex:5.6}a) is structurally ambiguous: the fish can be interpreted as either the agent or the patient of \textit{eat}. Both of the clauses in example (\ref{ex:5.6}b) are ambiguous in the same way. This predicts that there should be four logically possible interpretations of this sentence; but in fact only two are acceptable to most English speakers. If the fish is interpreted as an agent, then the chicken must be interpreted as an agent; if the fish is interpreted as a patient, then the chicken must be interpreted as a patient. The parallelism constraint rules out readings where the fish is the eater while the chicken is eaten, or vice versa. The same holds true for example (\ref{ex:5.6}c). Sentence (\ref{ex:5.6}d) is odd because the nouns used strongly favor different interpretations for the two clauses: potatoes must be the patient, while children must be the agent, violating the parallelism constraint.



Example \REF{ex:5.7} illustrates the use of the identity test with an apparent case of lexical ambiguity: \textit{duck} can refer to an action (lowering the head or upper body) or to a water fowl. Sentence (\ref{ex:5.7}a) is ambiguous, because the two senses of \textit{duck} generate two different readings, and one of these readings could be true while the other was false in a particular situation. The same potential ambiguity applies to both of the clauses in (\ref{ex:5.7}b), so again we would predict that four interpretations should be logically possible; but in fact only two are acceptable. Sentence (\ref{ex:5.7}b) can mean either that John and Bill both saw her perform a certain action or that they both saw a water fowl belonging to her. The fact that the parallelism constraint blocks the “crossed” readings provides evidence that these two different interpretations of \textit{duck} are truly distinct senses, i.e. that \textit{duck} is in fact lexically ambiguous.


\ea \label{ex:5.7}
\ea John saw her duck.\\
\ex John saw her duck, and so did Bill.
\z
                       \z


Contrast this with the examples in \REF{ex:5.8}. The word \textit{cousin} in the first clause of (\ref{ex:5.8}a) refers to a male person, while the implicit reference to \textit{cousin} in the second clause of (\ref{ex:5.8}a) refers to a female person. This difference of reference does not violate the parallelism constraint, because the two uses of \textit{cousin} are not distinct senses, even though they would be translated by different words in a language like Italian. The identity test indicates that \textit{cousin} is not lexically ambiguous, but merely unspecified for gender.


\ea \label{ex:5.8}
\ea John is my cousin, and so is Mary.\\
\ex John carried a briefcase, and Bill a backpack.\\
\ex That 3-year old is quite tall, but then so is his father.
                       \z
\z


Similarly, the word \textit{carry} in the first clause of (\ref{ex:5.8}b) probably describes a different action from the implicit reference to \textit{carry} in the second clause. The sentence allows an interpretation under which John carried the briefcase by holding it at his side with one hand, while Bill carried the backpack on his back; in fact, this would be the most likely interpretation in most contexts. The fact that this interpretation is not blocked by the parallelism constraint indicates that \textit{carry} is not lexically ambiguous, but merely unspecified (i.e., indeterminate) for manner. The two uses of \textit{carry} would be translated by different words in a language like Tzeltal, but they are not distinct senses.



The actual height described by the word \textit{tall} in the first clause of (\ref{ex:5.8}c) is presumably much less than the height described by the implicit reference to \textit{tall} in the second clause. The fact that this interpretation is acceptable indicates that \textit{tall} is not lexically ambiguous, but merely vague.



Example \REF{ex:5.9} shows how we might use the identity test to investigate the ambiguity of the Spanish word \textit{llave} mentioned above. These sentences could appropriately be used if both Pedro and Juan bought, broke or found the same kind of thing, whether keys, faucets, or wrenches. But the sentences cannot naturally describe a situation where different objects are involved, e.g. if Pedro bought a key but Juan bought a wrench, etc.\footnote{Jonatan Cordova, Steve and Monica Parker (p.c.).} This fact provides evidence that \textit{llave} is truly ambiguous and not merely indeterminate or vague.


\ea \label{ex:5.9}
\ea  \gll Pedro  compró/rompió  una  llave  y  también  Juan.\\
Pedro  bought/broke  a  key/etc.  and  also  Juan\\
\glt ‘Pedro bought/broke a key/faucet/wrench, and so did Juan.’
\ex \gll  Pedro  encontró  una  llave  al  igual  que  Juan.\\
Pedro  found  a  key/etc.  to.the  same  that  Juan\\
\glt ‘Pedro found a key/faucet/wrench, just like Juan did.’
\z \z


Another test which is sometimes used is the \textsc{sense relations test}: distinct senses will have different sets of synonyms, antonyms, etc. (see discussion of sense relations in \chapref{sec:6}). For example, the word \textit{light} has two distinct senses; one is the opposite of \textit{heavy}, the other is the opposite of \textit{dark}. However, \citet[56-57]{Cruse1986} warns that this test is not always reliable, because contextual features may restrict the range of possible synonyms or antonyms for a particular use of a word which is merely vague or indeterminate.



Another kind of evidence for lexical ambiguity is provided by the \textsc{test of contradiction}.\footnote{\citet{Quine1960}; \citet{ZwickySadock1975}; \citet{Kennedy2011}.} If a sentence of the form \textit{X but not X} can be true (i.e., not a contradiction), then expression X must be ambiguous. For example, the fact that the statement in \REF{ex:5.10} is not felt to be a contradiction provides good evidence for the claim that the two uses of \textit{child} represented here (‘offspring’ vs. ‘pre-adolescent human’) are truly distinct senses.


\ea \label{ex:5.10}
(Aged mother discussing her grown sons and daughters)\\
\textit{They are not children any more, but they are still my children}.
\z


This is an excellent test in some ways, because the essential property of ambiguity is that the two senses must have different truth conditions, and this test involves asserting one reading while simultaneously denying the other. In many cases, however, it can be difficult to find contexts in which such sentences sound truly natural. A few attempts at creating such examples are presented in \REF{ex:5.11}. The fact that such sentences are even possible provides strong evidence for two distinct senses of the relevant word.


\ea \label{ex:5.11}
\ea  Criminal mastermind planning to stage a traffic accident in order to cheat the insurance company: \textit{After the crash, you lie down behind the bus and tell the police you were thrown out of the bus through a window}.\\
Unwilling accomplice: \textit{I’ll lie there, but I won’t lie}.
\ex   Foreman: \textit{I told you to collect a sample of uranium ore from the pit and row it across the river to be tested}.\\
Miner: \textit{I have the ore but I don’t have the oar}.
\ex   Rancher (speaking on the telephone): \textit{I’ve lost my expensive fountain pen; I think I may have dropped it while we were inspecting the sheep. Can you check the sheep pen to see if it is there?}\\
Hired hand: \textit{I am looking at the pen, but I don’t see a pen}.
\z \z


An equivalent way of describing this test is to say that if there exists some state of affairs or context in which a sentence can be both truly affirmed and truly denied, then the sentence must be ambiguous.\footnote{Adapted from \citet[407]{Gillon1990}.} An example showing how this test might be applied to two uses of the word \textit{drink} (alcoholic beverage vs. any beverage) is quoted in \REF{ex:5.12}:


\ea \label{ex:5.12}
\ea \textit{Ferrell has a drink each night before going to bed}.
\ex  “Imagine… this state of affairs: Ferrell has a medical problem which requires that he consume no alcoholic beverages but that he have a glass of water each night before going to bed. One person knows only that he does not consume alcoholic beverages; another knows only that he has a glass of water each night at bedtime. The latter person can truly affirm the sentence in (12a)… But the former person can truly deny it.” (\citealt{Gillon1990}:407)
\z \z


Gillon points out that this is a very useful test because “generality and indeterminacy do not permit a sentence to be both truly affirmed and truly denied” (1990: 410). Sentences like those in \REF{ex:5.13} can only be interpreted as contradictions; they require some kind of pragmatic inference in order to make sense.\footnote{The word \textit{vertebrate} is more “general”, in Gillon’s terms, than words like \textit{fish} or \textit{dog}. We will discuss this kind of sense relation in the following chapter.}


\ea \label{ex:5.13}
\ea[\#]{She is my cousin and she is not my cousin.\\}
\ex[\#]{I am carrying the bag and I am not carrying the bag.\\}
\ex[\#]{This creature is a vertebrate and it is not a vertebrate.}
                       \z
\z

\subsection{Polysemy vs. homonymy}\label{sec:5.3.3}

It is traditional to distinguish between two types of lexical ambiguity, \textsc{polysemy} (one word with multiple senses) vs. \textsc{homonymy} (different words that happen to sound the same). Both cases involve an ambiguous word form; the difference lies in how the information is organized in the speaker’s mental lexicon.


Of course, it is not easy to determine how information is stored in the mental lexicon. This is not something that native speakers are consciously aware of, so asking them directly whether two senses are “the same word” or not is generally not a reliable procedure. The basic criterion for making this distinction is that in cases of polysemy, the two senses are felt to be “related” in some way; there is “an intelligible connection of some sort” between the two senses.\footnote{\citet[109]{Cruse2000}.} In cases of homonymy, the two senses are unrelated; that is, the semantic relationship between the two senses is similar to that between any two words selected at random.



It is difficult to draw a clear boundary between these two types of ambiguity, and some authors reject the distinction entirely. However, many ambiguous words clearly belong to one type or the other, and the distinction is a useful one. We will adopt a prototype approach, suggesting some properties that are prototypical of polysemy vs. homonymy while recognizing there will be cases which are very difficult to classify.


Some general guidelines for distinguishing polysemy vs. homonymy:

\begin{enumerate}[label=\alph*.]
\item Two senses of a polysemous word generally share at least one salient feature or component of meaning, whereas this need not be true for homonyms.\footnote{\citet{BeekmanCallow1974} suggest that \textit{all} the senses of a polysemous word will share at least one component of meaning, but this claim is certainly too strong.} For example, the sense of \textit{foot} that denotes a unit of length (‘12 inches’) shares with the body-part sense the same approximate size. The sense of \textit{foot} that means ‘base’ (as in \textit{foot of} \textit{a tree/mountain}) shares with the body-part sense the same position or location relative to the object of which it is a part. These common features suggest that \textit{foot} is polysemous. In contrast, the two senses of \textit{row} (pull the oars vs. things arranged in a line) seem to have nothing in common, suggesting that \textit{row} is homonymous.
\item If one sense seems to be a figurative extension of the other (see discussion of figurative senses below), the word is probably polysemous. For example, the sense of \textit{run} in \textit{This road runs from Rangoon to Mandalay} is arguably based on a metonymy between the act of running and the path traversed by the runner, suggesting that this is a case of polysemy.
\item \citet{BeekmanCallow1974} suggest that, for polysemous words, one sense can often be identified as the \textsc{primary sense}, with other senses being classified as secondary or figurative. The primary sense will typically be the one most likely to be chosen if you ask a native speaker to illustrate how the word X is used in a sentence, or if you ask a bilingual speaker what the word X means (i.e., ask for a translation equivalent). For homonymous words, neither sense is likely to be “primary” in this way.\footnote{A similar point is made by    \citet[100]{FillmoreAtkins2000}.}
\item Etymology (historical source) is used as a criterion in most dictionaries, but for synchronic linguistic analysis it is not a reliable basis for analysis. (Speakers may or may not know where certain words come from historically, and their ideas about such questions are often mistaken.) However, there is often a correlation between etymology and the criteria listed above, because figurative extension is a common factor in semantic change over time, as discussed in \sectref{sec:5.4}. English spelling may give a clue about etymology, but again is not directly relevant to synchronic linguistic analysis, which normally focuses on spoken language.
\end{enumerate}

Point (d) is a specific application of a more general principle in the study of lexical meaning: word meanings may change over time, and the historical meaning of a word may be quite different from its modern meaning. It is important to base our analysis of the current meanings of words on \textsc{synchronic} (i.e., contemporaneous) evidence, unless we are specifically studying the \textsc{diachronic} (historical) developments. \citet[244]{Lyons1977} expresses this principle as follows:


\begin{quote}
A particular manifestation of the failure to respect the distinction of the diachronic and the synchronic in semantics … is what might be called the \textsc{etymological fallacy}: the common belief that the meaning of words can be determined by investigating their origins. The etymology of a lexeme is, in principle, synchronically irrelevant.
\end{quote}


As an example, Lyons points out that it would be silly to claim that the “real” meaning of the word \textit{curious} in Modern English is ‘careful’, even though that was the meaning of the Latin word from which it is derived.



A number of authors have distinguished between \textsc{regular} or \textsc{systematic} polysemy vs. non-systematic polysemy. Systematic polysemy involves senses which are related in recurring or predictable ways. For example, many verbs naming a change of state (\textit{break, melt, split}, etc.) have two senses, one transitive (V\textsc{\textsubscript{tr}}) and the other intransitive (V\textsc{\textsubscript{intr}}), with V\textsc{\textsubscript{tr}} meaning roughly ‘cause to V\textsc{\textsubscript{intr}}’. Similarly, many nouns that refer to things used as instruments (\textit{hammer, saw, paddle, whip, brush, comb, rake, shovel, plow, sandpaper, anchor, tape, chain, telephone}, etc.) can also be used as verbs meaning roughly to use the instrument to act on an appropriate object. (A single sense can have only a single part of speech, so the verbal and nominal uses of such words must represent distinct senses.)



The kinds of regularities involved in systematic polysemy can be stated in the form of rules. Some authors have suggested that only the base or core meaning needs to be included in the lexicon, because the secondary senses can be derived by rule.\footnote{For example, \citet{Pustejovsky1995}.} But even in the case of systematic polysemy, secondary senses need to be listed because not every extended sense which the rules would license actually occurs in the language. For example, there are no verbal uses for some instrumental nouns, e.g. \textit{scalpel, yardstick, hatchet, pliers, tweezers}, etc. For others, verbal uses are possible only for non-standard uses of the instrument or non-literal senses:


\ea \label{ex:5.14}
\ea Australian Prime Minister Kevin Rudd has \textit{axed} the carbon tax.\\
\ex Alaska Airlines \textit{axed} the flights as a precaution.\\
\ex ?*John \textit{axed} the tree.
\z \z


Traditionally it has been assumed that all the senses of a polysemous word will be listed within a single lexical entry, while homonyms will occur in separate lexical entries. Most dictionaries adopt a format that reflects this organization of the lexicon. The format is illustrated in the partial dictionary listing for the word form \textit{lean} presented in \REF{ex:5.15}.\footnote{Adapted from the Merriam-Webster Online Dictionary (\url{http://www.merriam-webster.com/dictionary/lean} ).} The verbal and adjectival uses of \textit{lean} are treated as homonyms, each with its own lexical entry. Each of the homonyms is analyzed as being polysemous, with the various senses listed inside the appropriate entry.


\ea \label{ex:5.15}
\textit{lean}\textsubscript{1} (V): 1. to incline, deviate, or bend from a vertical position; 2. to cast one’s weight to one side for support ; 3. to rely on for support or inspiration; 4. to incline in opinion, taste, or desire (e.g., \textit{leaning toward a career in chemistry}).\\[1.25\baselineskip]

\textit{lean}\textsubscript{2} (Adj): 1. lacking or deficient in flesh; 2. containing little or no fat (\textit{lean meat}); 3. lacking richness, sufficiency, or productiveness (\textit{lean profits}, \textit{the lean years}); 4. deficient in an essential or important quality or ingredient, e.g. (a) of ore: containing little valuable mineral; (b) of fuel mixtures: low in combustible component.
\z


This is not the only way in which a lexicon could be organized, but we will not explore the various alternatives here. The crucial point is that polysemous senses are “related” while homonymous senses are not.


\subsection{One sense at a time}\label{sec:5.3.4}

When a lexically ambiguous word is used, the context normally makes it clear which of the senses is intended. As \citet[53]{Cruse1986} points out, a speaker generally intends the hearer to be able to identify the single intended sense based on context:

\begin{quote}
“[A] context normally also acts in such a way as to cause a single sense, from among those associated with any ambiguous word form, to become operative. When a sentence is uttered, it is rarely the utterer’s intention that it should be interpreted in two (or more) different ways simultaneously… This means that, for the vast majority of utterances, hearers are expected to identify specific intended senses for every ambiguous word form that they contain.”
\end{quote}


\citet[54]{Cruse1986} cites the sentence in \REF{ex:5.16}, which contains five lexically ambiguous words. (Note that the intended sense of \textit{burn} in this sentence, ‘a small stream’, is characteristic of Scottish English.)


\ea \label{ex:5.16}
Several rare ferns grow on the steep banks of the burn where it runs into the lake.
\z

Cruse writes,

\begin{quote}
In such cases, there will occur a kind of mutual negotiation between the various options [so as to determine which sense for each word produces a coherent meaning for the sentence as a whole]… It is highly unlikely that any reader of this sentence will interpret \textit{rare} in the sense of ‘undercooked’ (as in \textit{rare steak}), or \textit{steep} in the sense of ‘unjustifiably high’ (as in \textit{steep charges})… or \textit{run} in the sense of ‘progress by advancing each foot alternately never having both feet on the ground simultaneously’, etc.
\end{quote}


A very interesting use of this principle occurs in the short story “Xingu”, by Edith Wharton (1916). In the following passage, Mrs. Roby is describing something to the members of her ladies’ club, which they believe (and which she allows them to believe) to be a deep, philosophical book. After the discussion is over, however, the other members discover that she was actually describing a river in Brazil. The words which are underlined below are ambiguous; all of them must be interpreted with one sense in a discussion of a philosophical work, but another sense in a discussion of a river.

\todo[inline]{underlining emphasis gone missing here}
\ea \label{ex:5.17}
“Of course,” Mrs. Roby admitted, “the difficulty is that one must give up so much time to it. It’s very long.”\\
“I can’t imagine,” said Miss Van Vluyck tartly, “grudging the time given to such a subject.”\\
“And deep in places,” Mrs. Roby pursued; (so then it was a book!) “And it isn’t easy to skip.”\\
“I never skip,” said Mrs. Plinth dogmatically.\\
“Ah, it’s dangerous to, in Xingu. Even at the start there are places where one can’t. One must just wade through.”\\
“I should hardly call it wading ,” said Mrs. Ballinger sarcastically.\\
Mrs. Roby sent her a look of interest. “Ah — you always found it went swimmingly?”\\
Mrs. Ballinger hesitated. “Of course there are difficult passages,” she conceded modestly.\\
“Yes; some are not at all clear — even,” Mrs. Roby added, “if one is familiar with the original.\footnote{Apparently a play upon an archaic sense of \textit{original} meaning ‘source’ or ‘origin’.}”\\
“As I suppose you are?” Osric Dane interposed, suddenly fixing her with a look of challenge.\\
Mrs. Roby met it by a deprecating smile. “Oh, it’s really not difficult up to a certain point; though some of the branches are very little known, and it’s almost impossible to get at the source.”
\z

Mrs. Roby’s motives seem to be noble — she is rescuing the ladies of the club from further humiliation by an arrogant visiting celebrity, Mrs. Osric Dane (a popular author). But when the other members discover the deception, they are so provoked that they demand Mrs. Roby’s resignation.

\citet[175]{CotterellTurner1989} point out the implications of the “one sense at a time” principle for exegetical work:

\begin{quote}
The context of the utterance usually singles out … the \textit{one} sense, which is intended, from amongst the various senses of which the word is potentially capable…  When an interpreter tells us his author could be using such-and-such a word with sense \textit{a}, or he could be using it with sense \textit{b}, and then sits on the fence claiming perhaps the author means \textit{both}, we should not too easily be discouraged from the suspicion that the interpreter is simply fudging the exegesis.
\end{quote}


Sometimes, of course, the speaker does intend both senses to be available to the hearer; but this is normally intended as some kind of play on words, e.g. a pun. The humor in a pun (for those people who enjoy them) lies precisely in the fact that this is not the way language is normally used.


\subsection{Disambiguation in context}\label{sec:5.3.5}

Word meanings are clarified or restricted by their context of use in several different ways. If a word is indeterminate with respect to a certain feature, the feature can be specified by linguistic or pragmatic context. For example, the word \textit{nurse} is indeterminate with respect to gender; but if I say \textit{The nurse who checked my blood pressure was pregnant}, the context makes it clear that the nurse I am referring to is female.



We noted in the preceding section that the context of use generally makes it clear which sense of a lexically ambiguous word is intended. This is not to say that misunderstandings never arise, but in a large majority of cases hearers filter out unintended senses automatically and unconsciously. It is important to recognize that knowledge about the world plays an important role in making this disambiguation possible. For example, a slogan on the package of Wasa crispbread proudly announces, \textit{Baked since 1919}. There is a potential ambiguity in the aspect of the past participle here. It is our knowledge about the world (and specifically about how long breads and crackers can safely be left in the oven), rather than any feature of the linguistic context, which enables us to correctly select the habitual, rather than the durative, reading. The process is automatic; most people who see the slogan are probably not even aware of the ambiguity.



Because knowledge about the world plays such an important role, disambiguation will be more difficult with translated material, or in other situations where the content is culturally unfamiliar to the reader/hearer. But in most monocultural settings, \citegen{RavinLeacock2000} assessment seems fair:


\begin{quote}
Polysemy is rarely a problem for communication among people. We are so adept at using contextual cues that we select the appropriate senses of words effortlessly and unconsciously… Although rarely a problem in language use, except as a source of humour and puns, polysemy poses a problem for semantic theory and in semantic applications, such as translation or lexicography.
\end{quote}


If lexical ambiguity is not (usually) a problem for human speakers, it is a significant problem for computers. Much of the recent work on polysemy has been carried out within the field of computational linguistics. Because computational work typically deals with written language, more attention has been paid to \textsc{homographs} (words which are spelled the same) than to \textsc{homophones} (words which are pronounced the same), in contrast to traditional linguistics which has been more concerned with spoken language. Because of English spelling inconsistencies, the two cases do not always coincide; Ravin \& Leacock cite the example of \textit{bass} [bæs] ‘fish species’ vs. \textit{bass} [be\textsuperscript{j}s] ‘voice or instrument with lowest range’, homographs which are not homophones.



As Ravin \& Leacock note, lexical ambiguity poses a problem for translation. The problem arises because distinct senses of a given word-form are unlikely to have the same translation equivalent in another language. Lexical ambiguity can cause problems for translation in at least two ways: either the wrong sense may be chosen for a word which is ambiguous in the source language, or the nearest translation equivalent for some word in the source language may be ambiguous in the target language. In the latter case, the translated version may be ambiguous in a way that the original version was not.



A striking example of the former type occurred on the menu of a Chinese restaurant which offered ‘fried enema’ rather than ‘fried sausage’. The Chinese name of the dish is \textit{zhá guànchang} (\texttt{炸灌腸}). The last two characters in the name refer to a kind of sausage made of wheat flour stuffed into hog casings; but they also have another sense, namely ‘enema’.\footnote{\url{http://languagelog.ldc.upenn.edu/nll/?p=2236}} 



Much medieval and renaissance art, most famously the sculptural masterpiece by Michelangelo, depicts Moses with horns coming out of his forehead. This practice was based on the Latin Vulgate translation of a passage in Exodus which describes Moses’ appearance when he came down from Mt. Sinai.\footnote{Exodus 34:29-35.} The Hebrew text uses the verb \textit{qaran} to describe his face. This verb is derived from the noun \textit{qeren} meaning ‘horn’, and in some contexts it can mean ‘having horns’;\footnote{Psalm 69:31.} but most translators, both ancient and modern, have agreed that in this context it has another sense, namely ‘shining, radiant’ or ‘emitting rays’. St. Jerome, however, translated \textit{qaran} with the Latin adjective \textit{cornuta} ‘horned’.\footnote{There is some disagreement as to whether St. Jerome simply made a mistake, or whether he viewed the reference to horns as a live metaphor and chose to preserve the image in his translation. The first artistic depiction of a horned Moses appeared roughly 700 years after Jerome’s translation, which might be taken as an indication that the metaphorical sense was in fact understood by readers of the Vulgate at first, but was lost over time. (see Ruth Mellinkoff. 1970. \textit{The Horned Moses in Medieval Art and Thought} (California Studies in the History of Art, 14). University of California Press.)}



As noted above, a translation equivalent which is ambiguous in the target language can create ambiguity in the translated version that is not present in the original. For example, the French word \textit{apprivoiser} ‘to tame’ plays a major role in the book \textit{Le Petit Prince} ‘The Little Prince’ by Antoine de Saint-Exupéry. In most (if not all) Portuguese versions this word is translated as \textit{cativar}, which can mean ‘tame’ but can also mean ‘catch’, ‘capture’, ‘enslave’, ‘captivate’, ‘enthrall’, ‘charm’, etc. This means that the translation is potentially ambiguous in a way that the original is not. The first occurrence of the word is spoken by a fox, who explains to the little prince what the word means; so in that context the intended sense is clear. However, the word occurs frequently in the book, and many of the later occurrences might be difficult for readers to disambiguate on the basis of the immediate context alone.



It is not surprising that homonymy should pose a problem for translation, because homonymy is an accidental similarity of form; there is no reason to expect the two senses to be associated with a single form in another language. If we do happen to find a pair of homonyms in some other language which are good translation equivalents for a pair of English homonyms, we regard it as a remarkable coincidence. But even with polysemy, where the senses are related in some way, we cannot in general expect that the different senses can be translated using the same word in the target language.   \citet[103]{BeekmanCallow1974} state:


\begin{quote}
Whether multiple senses of a word arise from a shared [component] of meaning or from relations which associate the senses [i.e. figurative extensions—PK], the cluster of senses symbolized by a single word is always specific to the language under study.
\end{quote}


Perhaps Beekman \& Callow overstate the unlikelihood that a single word in the target language can carry some or all of the senses of a polysemous word in the source language. Since there is an intelligible relationship between polysemous senses, it is certainly possible for the same relationship to be found in more than one language; but often this turns out not to be the case, and that is why polysemy is often a source of problems.


\section{Context-dependent extensions of meaning}\label{sec:5.4}

\citet{Cruse1986,Cruse2000} distinguishes between \textsc{established} vs. \textsc{non-established} senses. An established sense is one that is permanently stored in the speaker’s mental lexicon, one which is always available; these are the senses that would normally be listed in a dictionary. A lexically ambiguous word is one that has two or more established senses.



We have seen how context determines a choice between existing (i.e., established) senses of lexically ambiguous words. But context can also force the hearer to “invent” a new, non-established sense for a word. When Mark Twain described a certain person as “a good man in the worst sense of the word,” his hearers were forced to interpret the word \textit{good} with something close to the opposite of its normal meaning (e.g., puritanical, self-righteous, or judgmental). Clearly this “sense” of the word \textit{good} is not permanently stored in the hearer’s mental lexicon, and we would not expect to see it listed in a dictionary entry for \textit{good}. It exists only on the occasion of its use in this specific context.



A general term for the process by which context creates non-established senses is \textsc{coercion}.\footnote{This term was coined by \citet{MoensSteedman1988}.} Coercion provides a mechanism for extending the range of meanings of a given word. It is motivated by the assumption that the speaker intends to communicate something intelligible, relevant to current purposes, etc. If none of the established senses of a word allow for a coherent or intelligible sentence meaning, the hearer tries to create an extended meaning for one or more words that makes sense in the current speech context.



Coerced meanings are not stored in the lexicon, but are calculated as needed from the established or default meaning of the word plus contextual factors; so there is generally some identifiable relationship between the basic and extended senses. Several common patterns of extended meaning were identified and named by ancient Greek philosophers; these are often referred to as \textsc{tropes}, or “figures of speech”.


\subsection{Figurative senses}\label{sec:5.4.1}

Some of the best-known figures of speech are listed in \REF{ex:5.18}:

\ea \label{ex:5.18}
\textbf{Some well-known tropes}\\
\begin{description}
\item[Metaphor:] traditionally defined as a figure of speech in which an implied comparison is made between two unlike things; but see comments below.
\item[Hyperbole:] A figure of speech in which exaggeration is used for emphasis or effect; an extravagant statement. (e.g., \textit{I have eaten more salt than you have eaten rice}. — Chinese saying implying seniority in age and wisdom)
\item[Euphemism:] Substitution of an inoffensive term (such as \textit{passed away}) for one considered offensively explicit (\textit{died}).
\item[Metonymy:] A figure of speech in which one word or phrase is substituted for another with which it is closely associated (such as \textit{crown} for \textit{monarch}).
\item[Synecdoche (/sɪˈnɛk də ki/):] A figure of speech in which a part is used to represent the whole, the whole for a part, the specific for the general, the general for the specific, or the material for the thing made from it. Considered by some to be a form of metonymy.
\item[Litotes:] A figure of speech consisting of an understatement in which an affirmative is expressed by negating its opposite (e.g. \textit{not bad} to mean ‘good’).
\item[Irony:] “statements that imply a meaning in opposition to their literal meaning”.\footnote{Wikipedia.}
\end{description}
\z


The question of how metaphors work has generated an enormous body of literature, and remains a topic of controversy. For our present purposes, it is enough to recognize all of these figures of speech as patterns of reasoning that will allow a hearer to provide an extended sense when all available established senses fail to produce an acceptable interpretation of the speaker’s utterance.


\subsection{How figurative senses become established}\label{sec:5.4.2}

As mentioned above, figurative senses are not stored in the speaker/hearer’s mental lexicon; rather, they are calculated as needed, when required by the context of use. However, some figurative senses become popular, and after frequent repetition they lose the sense of freshness or novelty associated with their original use; we call such expressions “clichés”. At this stage they are remembered, rather than calculated, but are perhaps not stored in the lexicon in the same way as “normal” lexical items; they are still felt to be figurative rather than established senses. Probable examples of this type include: \textit{fishing for compliments}, \textit{sowing seeds of doubt}, \textit{at the end of the day}, \textit{burning the candle at both ends, boots on the ground, lash out,} …


At some point, these frequently used figurative senses may become lexicalized, and begin to function as established senses. For example, the original sense of \textit{grasp} is ‘to hold in the hand’; but a new sense has developed from a metaphorical use of the word to mean ‘understand’. Similar examples include \textit{freeze} ‘become ice’ > ‘remain motionless’; \textit{broadcast} ‘plant (seeds) by scattering widely’ > ‘transmit via radio or television’; and, more recently, the use of \textit{hawk} and \textit{dove} to refer to advocates of war and advocates of peace, respectively. Once this stage is reached, the hearer does not have to calculate the speaker’s intended meaning based on specific contextual or cultural factors; the intended meaning is simply selected from among the established senses already available, as with normal cases of lexical ambiguity.



When established senses develop out of metaphors, they are referred to as \textsc{conventional metaphors}, in contrast to “novel” or “creative” metaphors which are newly created. Conventional metaphors are sometimes referred to as “dead” or “frozen” metaphors, phrases which are themselves conventional metaphors expressing the intuition that the meaning of such expressions is static rather than dynamic.



Finally, in some cases the original “literal” sense of a word may fall out of use, leaving what was originally a figurative sense as the only sense of that word. This seems to be happening with the compound noun \textit{night owl}, which originally referred to a type of bird. Many current dictionaries (including the massive \textit{Random House Unabridged}) now list only the conventional metaphor sense, i.e., a person who habitually stays out late at night. (** more examples?? **)



This discussion shows how figurative senses may lead to polysemy. Earlier we noted that translation equivalents in different languages are unlikely to share the same range of polysemous senses. For example, the closest translation equivalent for \textit{grasp} in Malay is \textit{pĕgang}; but this verb never carries the sense of ‘understand’. Novel (i.e., creative) metaphors can sometimes survive and be interpretable when translated into a different language, because the general patterns of meaning extension listed in \REF{ex:5.18}, if they are not universal, are at least used across a wide range of languages. Conventional (i.e., “frozen”) metaphors, however, are much less likely to work in translation, because the specific contextual features which motivated the creative use of the metaphor need no longer be present. 


\section{“Facets” of meaning}\label{sec:5.5}

The sentences in (\ref{ex:5.19}--\ref{ex:5.22}) show examples of different uses which are possible for certain classes of words. These different uses are often cited as cases of systematic polysemy, i.e., distinct senses related by a productive rule of some kind.\footnote{See for example \citet{Pustejovsky1995}, \citet{NunbergZaenen1992}.} However, \citet{Cruse2000,Cruse2004} argues that they are best analyzed as “facets” of a single sense, by which he means “fully discrete but non-antagonistic readings of a word”.\footnote{\citet[116]{Cruse2000}.}

\settowidth\jamwidth{[\textsc{information content}]}
\ea  \label{ex:5.19}
\textit{book} \citep{Cruse2004}:\\
\ea My chemistry book makes a great doorstop.            \jambox{[\textsc{physical object}]}
\ex My chemistry book is well-organized but a bit dull.  \jambox{[\textsc{information content}]}
                       \z
\z

\ea \label{ex:5.20}
 \textit{bank} (\citealt{Cruse2000}:116; similar examples include \textit{school, university}, etc.):\\
\ea The bank in the High Street was blown up last night.  \jambox{[\textsc{premises}]}
\ex That used to be the friendliest bank in town.         \jambox{[\textsc{personnel}]}
\ex This bank was founded in 1575.                        \jambox{[\textsc{institution}]}
                       \z
\z

\ea \label{ex:5.21} \textit{Britain} (\citealt{Cruse2000}:117; \citealt{CroftCruse2004}:117):\\
\ea Britain lies under one metre of snow.                    \jambox{[\textsc{land mass}]}
\ex Britain today is mourning the death of the Royal corgi.  \jambox{[\textsc{populace}]}
\ex Britain has declared war on San Marino.                  \jambox{[\textsc{political entity}]}
                       \z
\z

\ea \label{ex:5.22}
\textit{chicken, duck}, etc. (\citealt{CroftCruse2004}:117):\\
\ea My neighbor’s chickens are noisy and smelly.  \jambox{[\textsc{animal}]}
\ex This chicken is tender and delicious.         \jambox{[\textsc{meat}]}
                       \z
\z


Cruse describes facets as “distinguishable components of a global whole”.\footnote{Croft \& \citet[116]{Cruse2004}.} The word \textit{book} for example names a complex concept which includes both the physical object (the tome) and the information which it contains (the text). In the most typical uses of the word, it is used to refer to both the object and its information content simultaneously. In contexts like those seen in \REF{ex:5.19}, however, the word can be used to refer to just one facet or the other (text or tome).



Cruse’s strongest argument against the systematic polysemy analysis is the fact that these facets are non-antagonistic; they do not give rise to zeugma effects, as illustrated in \REF{ex:5.23}. In this they are unlike normal polysemous senses, which are antagonistic. Under the systematic polysemy analysis we might derive the senses illustrated in (\ref{ex:5.19}--\ref{ex:5.22}) by a kind of metonymy, similar to that illustrated in \REF{ex:5.24}.\footnote{\citet{Nunberg1979,Nunberg1995}.} However, as the examples in \REF{ex:5.25} demonstrate, figurative senses are antagonistic with their literal counterparts. This suggests that facets are not figurative senses.


\ea \label{ex:5.23}
\ea This is a very interesting book, but it is awfully heavy to carry around. (\citealt{Cruse2004})\\
\ex My religion forbids me to eat or wear rabbit.  [\citealt{NunbergZaenen1992}]
                       \z
\z

\ea \label{ex:5.24}
\ea I’m parked out back.\\
\ex The ham sandwich at table seven left without paying.\\
\ex Yeats is widely read although he has been dead for over 50 years.\\
\ex Yeats is widely read, even though most of it is now out of print.
                       \z
\z

\ea \label{ex:5.25}
\ea[\#]{The ham sandwich at table seven was stale and left without paying.\\}
\ex[\#]{The White House needs a coat of paint but refuses to ask Congress for the money.}
                       \z
\z


We cannot pursue a detailed discussion of these issues here. It may be that some of the examples in question are best treated in one way, and some in the other. The different uses of animal names illustrated in \REF{ex:5.22}, for example, creature vs. meat, seem like good candidates for systematic polysemy, because they differ in grammatical properties (mass vs. count nouns). But the non-antagonism of the other cases seems to be a problem for the systematic polysemy analysis.


\section{Conclusion}\label{sec:5.6}

In this chapter we described several ways of identifying lexical ambiguity, based on two basic facts. First, distinct senses of a single word are “antagonistic”, and as a result only one sense is available at a time in normal usage. The incompatibility of distinct senses can be observed in puns, in zeugma effects, and in the identity requirements under ellipsis. Second, true ambiguity involves a difference in truth conditions; so sentences which contain an ambiguous word can sometimes be truly asserted under one sense of that word and denied under the other sense, in the same context. Neither of these facts applies to vagueness or indeterminacy.



Lexical ambiguity is actually quite common, but only rarely causes confusion between speaker and hearer. The hearer is normally able to identify the intended sense for an ambiguous word based on the context in which it is used. Where none of the established senses lead to a sensible interpretation in a given context, new senses can be triggered by coercion. In \chapref{sec:8} we will discuss some of the pragmatic principles which guide the hearer in working out the intended sense.



\furtherreading



\citet{Kennedy2011} provides an excellent overview of lexical ambiguity, indeterminacy, and vagueness. These issues are also addressed in \citet{Gillon1990}. \citet[ch. 3]{Cruse1986} and (\citeyear*{Cruse2000}, ch. 6) discusses many of the issues covered in this chapter, including tests for lexical ambiguity, “antagonistic” senses, polysemy vs. homonymy, and contextual modification of meaning.


\subsection*{Discussion exercises}
\paragraph*{A: Do the italicized words illustrate ambiguity, vagueness, or indeterminacy?}
\ea
\ea She spends her afternoons \textit{filing} correspondence and her fingernails.\\
\ex He spends his afternoons \textit{washing} clothes and dishes.\\
\ex He was a \textit{big} baby, even though both of his parents are \textit{small}.\\
\ex The weather wasn’t very \textit{bright}, but then neither was our tour guide.\\
\ex Donald Trump smokes \textit{expensive} cigars but drives a \textit{cheap} car.\\
\ex That boy couldn’t \textit{carry} a tune in a bucket.
                       \z
\z


\paragraph*{B: In each of the following examples, state which word is ambiguous as demonstrated by the antagonism or zeugma effect. Is it an instance of polysemy or homonymy?}

\ea
  \ea “You are free to execute your laws, and your citizens, as you see fit.”\footnote{\textit{Star Trek: The Next Generation}, via grammar.about.com}
\ex  “… and covered themselves with dust and glory.”\footnote{Mark Twain, \textit{The Adventures of Tom Sawyer}}
\ex  Arthur declined my invitation, and Susan a Latin pronoun.
\ex Susan can’t bear children.
\ex  The batteries were given out free of charge.
\ex My astrologer wants to marry a star.
\z
\z

\paragraph*{C: Figurative senses}

Identify the type of figure illustrated by the italicized words in the following passages:

\begin{enumerate}
\item Fear is the \textit{lock} and laughter the \textit{key} to your heart.\footnote{Crosby, Stills \& Nash – “Suite: Judy Blue Eyes”}
\item The \textit{White House} is concerned about terrorism.
\item She has six hungry \textit{mouths} to feed.
\item That joke is \textit{as old as the hills}.
\item It’s \textit{not the prettiest} quarter I’ve ever seen, Mr. Liddell.\footnote{Sam Mussabini in \textit{Chariots of Fire}.}
\item as \textit{pleasant and relaxed} as a coiled rattlesnake\footnote{Kurt Vonnegut in \textit{Breakfast of Champions}}
\item Headline: Korean “\textit{comfort women}” get controversial apology, compensation from Japanese government\footnote{news.com.au, December 30, 2015}
\end{enumerate}
\paragraph*{D: Semantic shift}

Identify the figures of speech that provided the source for the following historical shifts in word meaning:

\begin{stylepoints}
a) \textit{bead} (< ‘prayer’)\\
b) \textit{pastor}\\
c) \textit{drumstick} (for ‘turkey leg’)\\
d) \textit{glossa} (Greek) ‘tongue; language’\\
e) \textit{pioneer} (< Old French \textit{peon(ier)} ‘foot-soldier’; cognate: \textit{pawn})
\end{stylepoints}

\subsection*{Homework exercises}
\paragraph*{A: Lexical ambiguity}

Do the uses of \textit{strike} in the following two sentences represent distinct senses (lexical ambiguity), or just indeterminacy? Provide linguistic evidence to support your answer.

\begin{enumerate}
 \item The California Gold Rush began when James Marshall \textit{struck} gold at Sutter’s Mill.
 \item  Balaam \textit{struck} his donkey three times before it turned and spoke to him.
\end{enumerate}
 
\paragraph*{B: Dictionary entries}

Without looking at any published dictionary, draft a dictionary entry for \textit{mean}. Include the use of \textit{mean} as a noun, as an adjective, and at least three senses of \textit{mean} as a verb.

\paragraph*{C: Polysemy etc.\footnotemark{}}
\todo{not footnotes in titles}
\footnotetext{Adapted from \citet{Cruse2000}.}

How would you describe the relationship between the readings of the italicized words in the following pairs of examples? You may choose from among the following options: \textsc{polysemy, homonymy, vagueness, indeterminacy, figurative use.} If none of these terms seem appropriate, describe the sense relation in prose.
 
\ea\ea Mary ordered an \textit{omelette}.\\
\ex The \textit{omelette} at table 6 wants his coffee now.
\z\z

\ea\ea They \textit{led} the prisoner away.\\
\ex They \textit{led} him to believe that he would be freed.
\z\z

\ea I will wear a \textit{cheap} wristwatch, but I don’t want to drive a \textit{cheap} car.
\z

\ea\ea My \textit{cousin} married an actress.\\
\ex My \textit{cousin} married a policeman.
\z\z

\ea\ea Could you loan me your \textit{pen}? Mine is out of ink.\\
\ex The goats escaped from their \textit{pen} and ate up my artichokes.
\z\z


\ea\ea Wittgenstein’s \href{http://en.wikipedia.org/wiki/Philosophical_Investigations}{Philosophical Investigations} is too \textit{deep} for me.\\
\ex This river is too \textit{deep} for my Land Rover to ford.
\z\z

