\chapter{Evidentiality}\label{sec:17}

\section{Markers that indicate the speaker’s source of information}\label{sec:17.1}
The Tagalog particle \textit{daw} {\textasciitilde} \textit{raw} is used to indicate that the speaker heard the information being communicated from someone else, as illustrated in example \REF{ex:17.1}. ‘Hearsay’ markers like this are one of the most common types of \textsc{evidential} marker among the world’s languages.


\ea \label{ex:17.1}
\gll Mabuti  \textit{raw}  ang=ani.\\
good  \textsc{hearsay}  \textsc{nom}=harvest\\
\glt ‘\textit{They say that} the harvest is good.’   [\citealt{SchachterOtanes1972}:423]
\z


The term \textit{evidential} refers to a grammatical marker which indicates the speaker’s source of information. Evidentials have often been treated as a type of epistemic modality, but in this chapter we will argue that the two categories are distinct. We begin in \sectref{sec:17.2} with a brief survey of some common types of evidential systems found across languages. In \sectref{sec:17.3} we present a more careful definition of the term \textsc{evidential} and discuss the distinction between evidentiality and epistemic modality. In \sectref{sec:17.4} we discuss some of the ways in which we can distinguish evidentiality from other categories, such as tense or modality, which may tend to correlate with evidentiality. \sectref{sec:17.5} reviews a proposed distinction between two types of evidential marking. In some languages evidential markers seem to function as illocutionary (speech act) modifiers, while in other languages evidential markers seem to contribute to the propositional content of the utterance. In terms of the distinction we made in \chapref{sec:11}, the former type can be identified as contributing use-conditional meaning, while the latter can be identified as contributing truth-conditional meaning.


\section{Some common types of evidential systems}\label{sec:17.2}

As mentioned in the previous section, hearsay markers are one of the most common types of evidential marker cross-linguistically. Another common type of evidential marking is seen in languages like Cherokee, which distinguish \textsc{direct} from \textsc{indirect} knowledge. Evidentiality in Cherokee is signaled by a contrast between two different past tense forms.\footnote{\citet{Pulte1985}; Pulte uses the terms “experienced past” vs. “nonexperienced past”.} Cherokee speakers use the direct form \textit{-ʌʔi} to express what they have experienced personally, e.g. something they have seen, heard, smelled, felt, etc. They use the indirect form \textit{-eʔi} to express what they have heard from someone else; or what they have inferred based on observable evidence (e.g., seeing puddles one might say ‘It rained-\textsc{indirect}’); or what they have assumed based on prior knowledge.



Many languages which have evidential systems make only a two-way distinction, e.g. between direct vs. indirect knowledge, or between hearsay/reported information vs. other sources. However, more complex systems are not uncommon. Huallaga Quechua has three contrastive evidential categories, marked by clitic particles which (in the default pattern) attach to the verb:\footnote{If any single constituent in the sentence gets narrow focus, the evidential clitic follows the focused constituent. If not, the clitic occupies its default position after the verb.} \textit{=mi} marks “direct” knowledge (e.g. eye-witness or personal experience); \textit{=shi} marks hearsay; and \textit{=chi} marks conjecture and/or inference.\footnote{\citet{Weber1989}.} The following sentences provide a minimal contrast illustrating the use of these particles. Each of the sentences contains the same basic propositional content (\textit{You also hit me}); the choice of particle indicates how the speaker came to believe this proposition.


\ea \label{ex:17.2}

  \textbf{Huallaga Quechua evidentials} \citep[421]{Weber1989}
\ea 
\gll Qam-pis  maqa-ma-shka-nki  \textit{=mi}\\
you-also  hit-1\textsc{obj}-\textsc{perf-2subj  =direct}\\
\glt ‘You also hit me (I saw and/or felt it).’

\ex \gll Qam-pis  maqa-ma-shka-nki  \textit{=shi}\\
you-also  hit-1\textsc{obj}-\textsc{perf-2subj  =hearsay}\\
\glt ‘(Someone told me that) you also hit me (I was drunk and can’t remember).’

 \ex \gll Qam-pis  maqa-ma-shka-nki  \textit{=chi}\\
you-also  hit-1\textsc{obj}-\textsc{perf-2subj  =conject}\\
\glt ‘(I infer that) you also hit me.’\\
(I was attacked by a group of people, and I believe you were one of them).
\z \z


A few languages are reported to have five or even six grammatically distinguished evidential categories. A widely cited example of a five-category system is Tuyuca, a Tucanoan language of Colombia. Evidentiality in Tuyuca is marked by portmanteau suffixes which indicate tense and subject agreement, as well as evidential category; and these suffixes are obligatory in every finite clause in the language.\footnote{\citet{Barnes1984}.} The use of these five evidential categories is illustrated by the minimal contrasts in \REF{ex:17.3}.


\ea \label{ex:17.3}
\textbf{Tuyuca evidential system} \citep{Barnes1984}\\
\ea  \gll  díiga  apé  -wi  \\
soccer  play  -\textsc{visual}\\
\glt ‘He played soccer.’ (I saw him play.)
\ex \gll díiga  apé  -ti\\
soccer  play  -\textsc{nonvisual}\\
\glt ‘He played soccer.’ (I heard the game and him, but I didn’t see it or him.)
\ex \gll  díiga  apé  -yi\\
soccer  play  -\textsc{inference}\\
\glt ‘He played soccer.’ (I have seen evidence that he played: his distinctive shoe print on the playing field. But I did not see him play.)
\ex \gll  díiga  apé  -yigi\\
soccer  play  -\textsc{hearsay}\\
\glt ‘He played soccer.’ (I obtained the information from someone else.)
\ex \gll  díiga  apé  -hĩyi\\
soccer  play  -\textsc{assumed}\\
\glt ‘He played soccer.’ (It is reasonable to assume that he did.)
\z \z


The \textsc{visual} category (\ref{ex:17.3}a) is used for states or events which the speaker actually sees, for actions performed by the speaker, and for “timeless” knowledge which is shared by the community. The \textsc{nonvisual} category (\ref{ex:17.3}b) is used for information which the speaker perceived directly by some sense other than seeing; that is, by hearing, smell, touch, or taste. The \textsc{inference} category (\ref{ex:17.3}c), which Barnes labels “apparent”, is used for conclusions which the speaker draws based on direct evidence. The \textsc{hearsay} category (\ref{ex:17.3}d), which Barnes labels “secondhand”, is used for information which the speaker has heard from someone else. The \textsc{assumed} category (\ref{ex:17.3}e) is used for information which the speaker assumes based on background knowledge about the situation.


\section{Evidentiality and epistemic modality}\label{sec:17.3}

Having examined some examples of the kinds of distinctions that are typically found in evidential systems, let us think about what kind of meaning these grammatical markers express. \citet{Aikhenvald2004}, in her very important book on this topic, defines evidentiality as follows:


\begin{quote}
Evidentiality is a linguistic category whose primary meaning is source of information… [T]his covers the way in which information was acquired, without necessarily relating to the degree of speaker’s certainty concerning the statement or whether it is true or not… To be considered as an evidential, a morpheme has to have ‘source of information’ as its core meaning; that is, the unmarked, or default interpretation.  (\citealt{Aikhenvald2004}:3)
\end{quote}


There are several important points to be noted in this definition. First, evidentiality is a grammatical category.\footnote{cf. \citet[1]{Aikhenvald2004}.} All languages have lexical means for expressing source of information (\textit{I was told that p}; \textit{I infer that p}; \textit{apparently}; \textit{it is said}; etc.), but the term \textsc{evidential} is normally restricted to grammatical morphemes (affixes, particles, etc.). Second, an evidential marker must have source of information as its core meaning. This is significant because evidentiality often correlates with other semantic features, such as degree of certainty. Such a correlation is not surprising, since a speaker will naturally feel more certain of things he has seen with his own eyes than things he learned by hearsay. (We return below to the question of how we can know which factor represents the marker’s “core meaning”.)



It is not unusual for evidential meanings to arise as secondary functions of markers of modality, tense, etc. For example, the German modal verb \textit{sollen} ‘should’ has a secondary usage as a hearsay marker, as illustrated in \REF{ex:17.4}. This form is often cited in discussions of evidentiality; but under Aikhenvald’s strict definition of the term, it would not be classified as an evidential, because its primary function is to mark modality.\footnote{\citet[1]{Aikhenvald2004} estimates that about a quarter of the world’s languages have grammatical markers of evidentiality. In contrast, the WALS on-line database indicates that evidentiality markers are present in 57\% of the sample (237 out of 418 languages). But this is based on a broader definition of the term: WALS includes cases like German \textit{sollen}, where a modal or some other grammatical marker has a secondary evidential function.}


\ea \label{ex:17.4}
\gll Kim  \textit{soll}  einen  neuen  Job  angeboten  bekommen  haben.\\
Kim  should a  new  job  offered  get  have\\
\glt ‘Kim has supposedly been offered a new job.’  [\citealt{vonFintel2006}]
\z


A third claim implicit in Aikhenvald’s definition is that evidentiality is distinct from epistemic modality. She states this explicitly a bit later:


\begin{quote}
Evidentials may acquire secondary meanings — of reliability, probability and possibility (known as epistemic extensions), but they do not have to… Evidentiality is a category in its own right, and not a subcategory of any modality… That evidentials may have semantic extensions related to probability and speaker’s evaluation of trustworthiness of information does not make evidentiality a kind of modality. [\citealt{Aikhenvald2004}:7–8]
\end{quote}


Epistemic modality of course is the linguistic category whose primary function is to indicate the speaker’s degree of certainty concerning the proposition that is being expressed. As we have just noted, there is a close correlation between source of information and degree of certainty, and a number of authors have classified evidentiality as a kind of modality.\footnote{\citet{Palmer1986}, \citet{Frawley1992}, \citet{MatthewsonEtAl2007}, \citet{Izvorski1997}.} But Aikhenvald maintains that the two categories need to be distinguished. 



Of course, the question of whether evidentiality is a type of epistemic modality depends in part on how one defines \textit{modality}; but this is not just a terminological issue. We argued in \chapref{sec:16} that modal markers, including epistemic modals, contribute to the propositional content of an utterance. There is good evidence that evidential markers in a number of languages do not contribute to propositional content but function as illocutionary modifiers, and so must be distinct from epistemic modality. But before we review some of this evidence, it will be helpful to think about how we go about identifying a morpheme’s “primary function”.


\section{Distinguishing evidentiality from tense and modality}\label{sec:17.4}

It is not always easy to distinguish empirically between evidential markers and epistemic modals. Tense and aspect markers can also be a problem, because they too can have secondary evidential functions or associations. Perfect aspect in particular often carries an indirect evidential connotation, and indirect evidence markers frequently develop out of perfect aspect markers.\footnote{\citet{Izvorski1997}; \citet{BybeeEtAl1994}.} For example, in Iranian Azerbaijani (closely related to Turkish) the suffix \textit{-miş} is polysemous between an older perfect sense and a more recent evidential sense.\footnote{\citet{Lee2008}.} We can see that the two senses are distinct in the modern language, because they can co-occur in the same word as seen in \REF{ex:17.5}.


\ea \label{ex:17.5}
\gll zefer  qazan-miş-miş-am\\
victory  win-\textsc{perf}-\textsc{indirect-1sg}\\
\glt ‘reportedly I have won’  [Noah Lee, p.c.]
\z


So then, when we encounter a grammatical morpheme which seems to indicate source of information in at least some contexts, but has other functions as well, how can we decide what to call it? In other words, how do we determine its “primary function”? The key is to search for contexts where the expected correlation does not hold, so that the two possible analyses would make different predictions.



David \citet[421ff]{Weber1989} compares his analysis of the Huallaga Quechua evidential clitics with an alternative analysis which treats them as \textsc{validational} markers, that is, indicators of the speaker’s degree of commitment to the truth of the proposition being expressed. The choice between these two analyses is not immediately obvious, because there is a correlation between source of information and speaker’s degree of commitment. As we have noted, a speaker is likely to be more certain of knowledge gained through direct experience than of knowledge gained through hearsay or inference. In many contexts the direct evidential \textit{=mi} (which is optional) can be used to indicate certainty; and hearers may sometimes interpret the hearsay evidential \textit{=shi} as indicating uncertainty on the part of the speaker.



However, when there is a conflict between source of information and degree of commitment, it is source of information that determines the choice of clitic. For example, if someone were to say ‘My mother’s grandfather’s name was John,’ the direct evidential \textit{=mi} would be extremely unnatural, no matter how firmly the speaker believes what he is saying. The hearsay evidential \textit{=shi} must be used instead, because it is very unlikely in that culture for the speaker to have actually met his great-grandfather. Similarly, in describing cultural practices which the speaker firmly believes but has not personally experienced (e.g., ‘Having chewed coca, their strength comes to them’), the hearsay evidential is strongly preferred.



The general principle is that when we are trying to identify the meaning of a certain form, and there are two or more semantic factors that seem to correlate with the presence of that form, we need to find or create situations in which only one of those factors is possible and test whether the form would appear in such situations.


\section{Two types of evidentials}\label{sec:17.5}

In \sectref{sec:17.3} we mentioned that evidential markers in some languages do not contribute to propositional content but function as illocutionary modifiers. One of the best documented examples of this type is Cuzco Quechua as described by Martina Faller.\footnote{\citet{Faller2002,Faller2003,Faller2006}, inter alia.} Faller analyzes the evidential enclitics in Cuzco Quechua as “illocutionary modifiers which add to or modify the sincerity conditions of the act they apply to.” She notes that “they do not contribute to the main proposition expressed, can never occur in the scope of propositional operators such as negation, and can only occur in illocutionary force bearing environments.”\footnote{\citet[v]{Faller2002}.}



We present here some of her evidence for saying that the evidential enclitics do not contribute to the propositional content of the utterance, focusing on the Reportative clitic \textit{=si}. First, the evidential is always interpreted as being outside the scope of negation. In example \REF{ex:17.6}, the contribution of the Reportative evidential (‘speaker was told that p’) cannot be interpreted as part of what is being negated; so (ii) is not a possible interpretation for this sentence.


\ea \label{ex:17.6}
\gll Ines-qa  mana=s  qaynunchaw  ñaña-n-ta-chu  watuku-rqa-n.\\
Inés-\textsc{top}  not=\textsc{report}  yesterday  sister-3-\textsc{acc-neg}  visit\textsc{-past}\textsc{\textsubscript{1}}-3\\
\glt \textit{propositional content} = ‘Inés didn’t visit her sister yesterday.’\\
\textit{evidential meaning}: (i) speaker was told that Inés did not visit her sister yesterday\\
  \textit{not}:  (ii) \# speaker was not told that Inés visited her sister yesterday\\
{}[\citealt{Faller2002}, sec. 6.3.1]
\z


Second, the contribution of the Reportative evidential is not part of what can be challenged. If a speaker makes the statement in (\ref{ex:17.7}a), a hearer might challenge the truth of the statement based on the facts being reported, as in (\ref{ex:17.7}b); but it would be infelicitous to challenge the truth of the statement based on source of information, as in (\ref{ex:17.7}c). (This test is sometimes called the \textsc{assent/dissent diagnostic}.\footnote{\citet{Papafragou2006}.}) In other words, the contribution of the evidential does not seem to be part of what makes the statement true or false.


\ea \label{ex:17.7}
\ea 
\gll Ines-qa  qaynunchay  ñaña-n-ta=s  watuku-sqa.\\
Inés-\textsc{top}  yesterday  sister-\textsc{acc}=\textsc{report}  visit\textsc{-past}\textsc{\textsubscript{2}}\\
\glt \textit{propositional content} = ‘Inés visited her sister yesterday.’\\
\textit{evidential meaning}: speaker was told that Inés visited her sister yesterday
\ex \gll  Mana=n  chiqaq-chu.  Manta-n-ta-lla=n  watuku-rqa-n.\\
not=\textsc{direct}  true-\textsc{neg}  mother-3-\textsc{acc}-\textsc{limit}=\textsc{direct}  visit-\textsc{past\textsubscript{1}}-3\\
\glt ‘That’s not true. She only visited her mother.’
\ex \gll  Mana=n  chiqaq-chu.  \#Mana=n  chay-ta  willa-rqa-sunki-chu.\\
not=\textsc{direct}  true-\textsc{neg}  not=\textsc{direct}  this-\textsc{acc}  tell\textsc{-past}\textsc{\textsubscript{1}}-3S.2O-\textsc{neg}\\
\glt ‘That’s not true. \#You were not told this.’  [\citealt{Faller2002}, sec. 5.3.3]
\z \z


Third, Faller’s statement that the evidential enclitics “can only occur in illocutionary force bearing environments” means that they are restricted to main clauses or clauses which express an independent speech act. This is a characteristic feature of many illocutionary modifiers. In particular, conditional clauses are typically not the kind of environment where illocutionary modifiers can occur.\footnote{\citet{Ernst2009}, \citet{Haegeman2010a}.} Faller states that evidential enclitics cannot occur within conditional clauses, as illustrated in \REF{ex:17.8}.


\ea \label{ex:17.8}
\gll Mana(*=si)  para-sha-n-chu  chayqa  ri-sun-chis.\\
not=\textsc{report}  rain-\textsc{prog}-3-\textsc{neg}  then  go-1\textsc{fut}-\textsc{pl}\\
\glt ‘If it is not raining we will go.’  [\citealt{Faller2003}, ex. 8]
\z


The German auxiliary \textit{sollen} ‘should’, when used as a reportative or hearsay marker, behaves quite differently. For example, it is possible for \textit{sollen} to occur within a conditional clause, as illustrated in \REF{ex:17.9}.


\ea \label{ex:17.9}
F.C.B.F.A.N.: Bei uns \textit{soll} es heute schneien!!\\
‘It is \textit{said} (=predicted) to snow near us today.’\\
FAHRBACH: Also \textit{wenn es bei dir schneien soll}, dann schneit es bei mir auch.\\
‘\textit{If it said to snow near you}, then it will snow near me as well.’\\
   {}[\url{http://www.kc-forum.com/archive/index.php/t-45696}, cited in \citealt{Faller2006}]
\z


The assent/dissent diagnostic reveals another difference. German Reportative \textit{sollen}, like the Quechua Reportative evidential, allows the hearer to challenge the basic propositional content of the sentence. But in addition, it is possible to challenge the truth of a statement with \textit{sollen} based on the source of information, as illustrated in \REF{ex:17.10}.\footnote{\citet{Faller2006}.} This is impossible with the Quechua Reportative. Both of these differences are consistent with the hypothesis that German Reportative \textit{sollen} is part of the propositional content of the utterance.


\ea \label{ex:17.10}
A: Laut Polizei \textit{soll} die Gärtnerin die Juwelen gestohlen haben.\\
\glt   ‘According to the police, the gardener \textit{is said} to have stolen the jewels.’\\
B: Nein, das stimmt nicht. Das ist die Presse, die das behauptet.\\
\glt    ‘No, that’s not true. It is the press who is claiming this.’  (\citealt{Faller2006})
\z


A number of languages have evidentials which behave much like those of Cuzco Quechua. However, there are other languages in which evidentials seem to contribute to the propositional content of the utterance, like German Reportative \textit{sollen}. \citet{Murray2010} suggests that we need to recognize two different types of evidential, which we will refer to as \textsc{illocutionary evidentials} and \textsc{propositional evidentials}.\footnote{Murray uses the terms \textsc{illocutionary evidentials} vs. \textsc{epistemic evidentials}.} Illocutionary evidentials function as illocutionary operators; examples are found in Quechua, Kalaallisut, and Cheyenne. Propositional evidentials are part of the propositional content of the utterance; examples are found in German, Turkish, Bulgarian, St’át’imcets (Lillooet Salish), and Japanese.



These two types of evidentials share a number of properties in common, but Murray identifies several tests that distinguish the two classes. For example, illocutionary evidentials cannot be embedded within a conditional clause \REF{ex:17.8}, while this is possible for propositional evidentials \REF{ex:17.9}. Second, a speaker who makes a statement using a hearsay or reportative evidential of the illocutionary type is not committed to believing that the propositional content of the utterance is possibly true. So it is not a contradiction, nor is it infelicitous, for a speaker to assert something as hearsay and then deny that he believes it, as illustrated in \REF{ex:17.11}.


\ea \label{ex:17.11}
\ea  \gll Para-sha-n=si,  ichaqa  mana  crei-ni-chu.\\
rain-\textsc{prog}-3=\textsc{report}  but  not  believe-1-\textsc{neg}\\
\glt ‘It is raining (someone says), but I don’t believe it.’ [Cuzco Quechua; \citealt{Faller2002}: 194]
\ex \gll  É-hoo'kȯhó-n\.ese  naa  oha  ná-sáa-oné'séomátséstó-he-⌀.\\
3-rain-\textsc{report}.\textsc{inan.sg}  and  \textsc{contr}  1-\textsc{neg}-believe\textsubscript{INAN}-\textsc{mod\textsubscript{A}}\textsubscript{NIM}-\textsc{dir}\\
\glt ‘It’s raining, they say, but I don’t believe it.’  [Cheyenne; \citealt{Murray2010}: 58]
\z \z


A hearsay or reportative evidential of the propositional type, however, commits the speaker to believing that it is at least possible for the expressed proposition to be true. For this reason, the St’át’imcets example in \REF{ex:17.12} is infelicitous.


\ea \label{ex:17.12}
(Context: You had done some work for a company and they said they put your pay, \$200, in your bank account; but actually, they didn’t pay you at all.)\\
\gll *Um’-en-tsal-itás  ku7  i  án’was-a  xetspqíqen’kst  táola, t’u7  aoz  kw  s-7um’-en-tsál-itas  ku  stam’.\\
 give-\textsc{dir}-1sg.\textsc{obj}-3pl.\textsc{erg}  \textsc{report}  \textsc{det.pl}  two-\textsc{det}  hundred  dollar but  \textsc{neg}  \textsc{det}  \textsc{nom}-give-\textsc{dir}-1sg.\textsc{obj}-3pl.\textsc{erg}  \textsc{det}  what\\
\glt ‘They gave me \$200 [I was told], but they didn’t give me anything.’\\
   {}[\citealt{MatthewsonEtAl2007}]
\z


Third, illocutionary evidentials are always speaker-oriented. This means that they indicate the source of information of the speaker, and cannot be used to indicate the source of information of some other participant. This is illustrated in the Quechua example in \REF{ex:17.13}.


\ea \label{ex:17.13}
\gll Pilar-qa  yacha-sha-n  Marya-q  hamu-sqa-n-ta\{-n/-s/-chá\}.\\
Pilar-\textsc{top}  know-\textsc{prog}-3  Marya-\textsc{gen}  come-\textsc{past.prtcpl}-3-\textsc{acc\{-dir}/-\textsc{report}/-\textsc{conject\}}\\
\glt \textit{propositional content} = ‘Pilar knows that Marya came.’\\
\textit{evidential meaning}: (i) speaker has direct/reportative/conjectural evidence that\\
    Pilar knows that Marya came\\
\textit{not}: (ii) \#Pilar has direct/reportative/conjectural evidence that Marya came\\
     {}[\citealt{Faller2002}, ex. 184]
\z


Propositional evidentials, in contrast, can be used to indicate the source of information of some participant other than the speaker. In the St’át’imcets example in \REF{ex:17.14}, for example, the reportative evidential is interpreted as marking Lémya7’s source of information. It indicates that Lémya7’s statement was based on hearsay evidence. The speaker in \REF{ex:17.14} already had direct evidence for this information before hearing it from Lémya7.


\ea 
\gll tsut  s-Lémya7  kw  sqwemémn’ek  ku7  s-Mary,  t’u7  plán-lhkan ti7  zwát-en  —  áts’x-en-lhkan  s-Mary  áta7  tecwp-álhcw-a  inátcwas\\
say  \textsc{nom}-name  \textsc{det}  pregnant  \textsc{report  nom}-name  but  already-1sg.\textsc{subj} \textsc{dem}  know-\textsc{dir}    see-\textsc{dir}-1sg.\textsc{subj}  \textsc{nom}-name  \textsc{deic}  buy-place-\textsc{det}  yesterday\\
\glt ‘Lémya7 said that [she was told that] Mary is pregnant, but I already knew that — I had seen Mary at the store.’   [\citealt{MatthewsonEtAl2007}]
\z

A fourth difference that Murray demonstrates is that markers of tense or modality never take semantic scope over illocutionary evidential markers, whereas this is possible with propositional evidentials.\footnote{See \citet[sec. 3.4.2]{Murray2010} for examples.}



There seems to be a strong tendency for illocutionary evidential markers to be “true evidentials” in Aikhenvald’s sense, i.e., grammatical morphemes whose primary function is to mark source of information; and for propositional evidentials to be evidential uses/senses of morphemes whose primary function is something else: perfect aspect in Turkish and Bulgarian; modality in German and St’át’imcets. In terms of the distinction we made in \chapref{sec:11}, illocutionary evidentials seem to contribute use-conditional meaning, while propositional evidentials seem to contribute truth-conditional meaning.


\section{Conclusion}\label{sec:17.6}

We have suggested that a single type of meaning (source of information) can be contributed on two different levels or dimensions: truth-conditional vs. use-conditional. In \chapref{sec:18} we will argue that a similar pattern is observable with adverbial reason clauses. The conjunction \textit{because} expresses a causal relationship, but this causal relationship may either be asserted as part of the truth-conditional propositional content of the sentence, or may function as a kind of illocutionary modifier.



There is much more to be said about evidentials, but we cannot pursue the topic further here. In addition to the semantic issues introduced (all too briefly) above, the use of grammatical evidential markers interacts in interesting ways with discourse genre, world-view, first and second language acquisition, language contact, and translation, to name just a few.



\furtherreading



\citet{Aikhenvald2004} is the primary source for typological and descriptive details about the meanings and functions of evidential markers, and for discussion of the other issues mentioned in the last sentence of this chapter. \Citet{deHaan2012} provides a useful overview of the subject, while   \citet{deHaan1999,deHaan2005} discusses the relationship between evidentiality and epistemic modality.